\PassOptionsToPackage{unicode=true}{hyperref} % options for packages loaded elsewhere
\PassOptionsToPackage{hyphens}{url}
%
\documentclass[oneside,10pt,french,]{extbook} % cjns1989 - 27112019 - added the oneside option: so that the text jumps left & right when reading on a tablet/ereader
\usepackage{lmodern}
\usepackage{amssymb,amsmath}
\usepackage{ifxetex,ifluatex}
\usepackage{fixltx2e} % provides \textsubscript
\ifnum 0\ifxetex 1\fi\ifluatex 1\fi=0 % if pdftex
  \usepackage[T1]{fontenc}
  \usepackage[utf8]{inputenc}
  \usepackage{textcomp} % provides euro and other symbols
\else % if luatex or xelatex
  \usepackage{unicode-math}
  \defaultfontfeatures{Ligatures=TeX,Scale=MatchLowercase}
%   \setmainfont[]{EBGaramond-Regular}
    \setmainfont[Numbers={OldStyle,Proportional}]{EBGaramond-Regular}      % cjns1989 - 20191129 - old style numbers 
\fi
% use upquote if available, for straight quotes in verbatim environments
\IfFileExists{upquote.sty}{\usepackage{upquote}}{}
% use microtype if available
\IfFileExists{microtype.sty}{%
\usepackage[]{microtype}
\UseMicrotypeSet[protrusion]{basicmath} % disable protrusion for tt fonts
}{}
\usepackage{hyperref}
\hypersetup{
            pdftitle={SAINT-SIMON},
            pdfauthor={Mémoires XIX},
            pdfborder={0 0 0},
            breaklinks=true}
\urlstyle{same}  % don't use monospace font for urls
\usepackage[papersize={4.80 in, 6.40  in},left=.5 in,right=.5 in]{geometry}
\setlength{\emergencystretch}{3em}  % prevent overfull lines
\providecommand{\tightlist}{%
  \setlength{\itemsep}{0pt}\setlength{\parskip}{0pt}}
\setcounter{secnumdepth}{0}

% set default figure placement to htbp
\makeatletter
\def\fps@figure{htbp}
\makeatother

\usepackage{ragged2e}
\usepackage{epigraph}
\renewcommand{\textflush}{flushepinormal}

\usepackage{indentfirst}
\usepackage{relsize}

\usepackage{fancyhdr}
\pagestyle{fancy}
\fancyhf{}
\fancyhead[R]{\thepage}
\renewcommand{\headrulewidth}{0pt}
\usepackage{quoting}
\usepackage{ragged2e}

\newlength\mylen
\settowidth\mylen{...................}

\usepackage{stackengine}
\usepackage{graphicx}
\def\asterism{\par\vspace{1em}{\centering\scalebox{.9}{%
  \stackon[-0.6pt]{\bfseries*~*}{\bfseries*}}\par}\vspace{.8em}\par}

\usepackage{titlesec}
\titleformat{\chapter}[display]
  {\normalfont\bfseries\filcenter}{}{0pt}{\Large}
\titleformat{\section}[display]
  {\normalfont\bfseries\filcenter}{}{0pt}{\Large}
\titleformat{\subsection}[display]
  {\normalfont\bfseries\filcenter}{}{0pt}{\Large}

\setcounter{secnumdepth}{1}
\ifnum 0\ifxetex 1\fi\ifluatex 1\fi=0 % if pdftex
  \usepackage[shorthands=off,main=french]{babel}
\else
  % load polyglossia as late as possible as it *could* call bidi if RTL lang (e.g. Hebrew or Arabic)
%   \usepackage{polyglossia}
%   \setmainlanguage[]{french}
%   \usepackage[french]{babel} % cjns1989 - 1.43 version of polyglossia on this system does not allow disabling the autospacing feature
\fi

\title{SAINT-SIMON}
\author{Mémoires XIX}
\date{}

\begin{document}
\maketitle

\hypertarget{chapitre-i.}{%
\chapter{CHAPITRE I.}\label{chapitre-i.}}

~

{\textsc{Rang observé toujours dans l'ordre de la Toison d'or.}}
{\textsc{- Quel est l'état de capitaine général des armées d'Espagne.}}
{\textsc{- Médiannates et lansas des grands.}} {\textsc{- Appointements
des maisons royales, des capitaines généraux et des conseils.}}
{\textsc{- Explication sur les serments.}} {\textsc{- Quelles de ces
personnes n'en prêtent point\,; quelles en prêtent, et entre quelles
mains.}} {\textsc{- Buen-Retiro.}} {\textsc{- Casa del Campo.}}
{\textsc{- L'Escurial.}} {\textsc{- Aranjuez.}} {\textsc{- Le Pardo.}}
{\textsc{- La Sarçuela.}} {\textsc{- Le Pardillo.}} {\textsc{- Don
Gaspard Giron\,; sa naissance, son caractère.}} {\textsc{- Du marquis de
Villagarcias.}} {\textsc{- De Cucurani.}} {\textsc{- De Villafranca,
introducteur des ambassadeurs.}} {\textsc{- Hyghens, premier médecin du
roi d'Espagne\,; son caractère.}} {\textsc{- Hyghens m'engage à conférer
secrètement avec le duc d'Ormond\,; son caractère.}} {\textsc{-
Legendre, premier chirurgien\,; son caractère.}} {\textsc{- Ricoeur,
premier apothicaire\,; son caractère.}} {\textsc{- Marquis del Surco et
sa femme\,; leur fortune, leur caractère.}} {\textsc{- Valouse\,; sa
fortune, son caractère.}} {\textsc{- Hersent\,; son état, son
caractère.}} {\textsc{- Cardinal Borgia\,; son caractère.}} {\textsc{-
Garde et livrée.}} {\textsc{- Armendariz, lieutenant-colonel du régiment
des gardes espagnoles\,; son caractère.}} {\textsc{- Titolados.}}
{\textsc{- L'Excellence.}} {\textsc{- Comtesse d'Altamire\,; son
caractère.}} {\textsc{- Caractère de quelques \emph{señoras de honor}.
Don Domingo Guerra, confesseur de la reine\,; son caractère.}}
{\textsc{- MM. de Saint-Jean père et fils\,; leur fortune et leur
caractère.}} {\textsc{- Capitaines des gardes du corps et colonels des
régiments des gardes prêtent seuls serment entre les mains du roi
d'Espagne.}} {\textsc{- Salazar\,; sa fortune et sa réputation.}}

~

CHEVALIERS DE L'ORDRE DE LA TOISON D'OR EXISTANTS, EN AVRIL 1722.

DE CHARLES II.

L'empereur.

Le comte de Lemos.

Le prince Jacques Sobieski.

Le prince de Chimay.

Le duc de Bejar.

Le marquis de Conflans-Vatteville.

Le duc de Lorraine.

Le duc de Monteléon.

Le duc de Bavière, électeur.

DE PHILIPPE V.

Le prince des Asturies.

Le maréchal duc de Villars.

Le duc d'Orléans, régent.

Le marquis de Brancas, depuis maréchal de France.

Le duc de Noailles.

Le comte de Montijo.

Le comte de Toulouse.

Le duc de Liria.

Le duc de Berwick.

Le marquis de Béthune, depuis duc de Sully.

Le comte de Thiring, premier ministre de Bavière.

Le prince Frédéric de Nassau.

Le duc d'Albuquerque.

Le marquis, depuis maréchal d'Asfeld.

Le marquis de Villena.

Le marquis de Caylus.

Le duc de Popoli.

Le duc Lellio Caraffa.

Le marquis de Richebourg.

Le marquis Mari.

Le prince Ragotzi.

Le duc de Ruffec.

Le prince de Masseran.

Le marquis, depuis maréchal de Maulevrier.

Le duc de Bournonville.

Le marquis, depuis maréchal de La Fare.

Le duc d'Atri.

Le marquis d'Arpajon.

Le prince de Robecque.

Le marquis de Beaufremont.

Cet ordre, non plus que celui de Saint-Jacques de Calatrava et
d'Alcantara, ne souffre de rang ni de préférence que par l'ancienneté de
réception entre les chevaliers, sans exception quelconque que des tètes
couronnées, mais d'aucuns autres souverains, ni en même promotion
d'autre préférence que de l'âge, tellement que le prince des Asturies,
fils aîné de Philippe V, est le premier exemple de chevalier qui ait
précédé ses anciens, et encore à la prière du roi, son père, en plein
chapitre, accordée par les chevaliers, et sans conséquence pour tout ce
qui ne serait pas infant d'Espagne. À cet exemple, nos princes du sang,
et même légitimés, ont prétendu le même honneur, lorsqu'il y a eu depuis
des colliers envoyés en France, et des chevaliers à recevoir. Ces
princes y ont trouvé beaucoup de résistance, tellement qu'ils ne se
trouvent point aux chapitres lorsqu'il y a des chevaliers à recevoir, et
qu'eux-mêmes ont reçu le collier sans cérémonie. Je diffère à parler de
cette cérémonie de réception, et de quelques autres choses qui regardent
cet ordre, à l'occasion de la réception de mon fils aîné.

CAPITAINES GÉNÉRAUX DES ARMÉES.

Le duc d'Arcos.

Le comte de Las Torrès est devenu enfin grand d'Espagne.

Le comte d'Aguilar.

Le marquis d'Ayétone.

Le marquis de Casa-Fuerte.

Le duc de Saint-Pierre.

Don François Manriquez.

Le marquis de Bedmar.

Le marquis de Thouy.

Le marquis de Richebourg.

Le marquis, depuis maréchal de Puységur.

Le prince Pio\footnote{Le duc de Popoli, grand d'Espagne, que
  j'oubliais. (\emph{Note de Saint-Simon}.)}.

Le marquis de Seissan.

Le comte de San-Estevan de Gormaz.

Le marquis de Lede.

Ces neuf tous grands d'Espagne.

C'est tout ce qu'il existait de capitaines généraux d'armées, tandis que
j'étais en Espagne. Ces capitaines généraux sont à l'égard du militaire,
honneurs et commandements, semblables en tout à nos maréchaux de France,
et prétendent rouler d'égal avec eux. Mais ils leur sont, au fond,
totalement inférieurs, en ce qu'ils ne sont point officiers de la
couronne, qu'ils ne sont ni juges de la noblesse sur le point d'honneur,
ni supérieurs en rien à la noblesse, et qu'ils n'ont ni rang ni
honneurs, hors des fonctions militaires, sinon l'excellence, traitement
qui se borne à ce mot, dont je parlerai ailleurs.

MAISON DU ROI D'ESPAGNE LORSQUE J'Y ÉTAIS.

\emph{Majordome-major}.

Le marquis de Villena, duc d'Escalona.

\emph{Majordomes de semaine}.

Don Gaspar Giron.

Le comte de Casa-Real.

Le marquis de Villagarcias.

Le comte Cucurani.

\emph{Surnuméraires}.

Le comte Saratelli.

Le marquis d'Almodovar.

\emph{Introducteur des ambassadeurs}.

Le marquis de Villafranca.

\emph{Premier médecin}.

M. Hyghens.

\emph{Premier chirurgien}.

M. Le Gendre.

\emph{Premier apothicaire}.

M. Ricœur.

\emph{Sommelier du corps}.

Le marquis de Montalègre.

\emph{Gentilshommes de la chambre}.

Le comte de Peñeranda.

Le marquis de Montalègre, fils du sommelier.

Le duc de Bejar.

Le duc de Liria.

Le duc de Veragua.

Le comte de Maceda.

Le comte de Baños.

Le duc de Solferino.

Le comte de San-Estevan de Gormaz.

Le duc de Bournonville.

Le marquis de Santa-Cruz.

Le duc de Popoli.

Le duc del Arco.

Le duc de Monteillano.

Le duc de Gandie.

Le marquis de Cogolludo, fils aîné du duc de Medina-Coeli.

Le marquis de Los Balbazès.

Le marquis del Surco, (non grands.)

Le prince de Masseran.

Le marquis de Valouse, (non grands.)

\emph{Guardaroba}.

M. Hersent.

La grande et petite livrée du roi et de la reine d'Espagne, pages et
valets de pied, gens d'écurie et valets de peine, sont en tout les mêmes
que celles de France, même celles des garçons bleus du château et des
tapissiers.

\emph{Grand écuyer}.

Le duc del Arco.

Le duc de La Mirandole en conservait les honneurs et les appointements,
en cédant la charge qu'il avait au duc del Arco.

\emph{Premier écuyer}.

Le marquis de Valouse.

\emph{Grand aumônier}.

L'archevêque de Compostelle, par son siège, et qui effacerait le
patriarche des Indes s'il se trouvait à la cour. Mais les évêques
résident toujours dans leurs diocèses, en sorte qu'il n'est rien de plus
rare que d'en voir quelqu'un à Madrid, et toujours pour affaires
nécessaires. Les fonctions de grand aumônier sont suppléées en tout, et
sans dépendance, en absence continuelle de l'archevêque, par\,:

Le patriarche des Indes, qui est sacré \emph{in partibus} sous ce titre,
qui ne lui donne quoi que ce soit aux Indes ni ailleurs, hors de la
chapelle.

Le cardinal Borgia.

GARDE DU ROI D'ESPAGNE.

C'est Philippe V qui se l'est donnée à l'instar de la France. Ses
prédécesseurs n'avaient que la compagnie des hallebardiers, qui répond
en tout à celle de nos Cent-Suisses.

CAPITAINES DES GARDES DU CORPS.

\emph{Première compagnie, espagnole}.

Le comte de San-Estevan de Gormaz.

\emph{Deuxième compagnie, italienne}.

Le duc de Popoli.

\emph{Troisième compagnie, wallonne}.

Le duc de Bournonville.

Il n'y a point de quatrième compagnie.

\emph{Compagnie des hallebardiers}.

Le marquis de Montalègre, sommelier.

\emph{Régiment des gardes espagnoles}.

Colonel\,: le marquis d'Ayétone.

\emph{Régiment des gardes wallonnes}.

Le marquis de Richebourg.

Ces six corps, officiers, gardes, hallebardiers, soldats, drapeaux,
étendards, en tout et partout ont le pareil et tout semblable uniforme,
hommes et chevaux, que les compagnies des gardes du corps, celle des
Cent-Suisses, et les régiments des gardes françaises et suisses. Les
capitaines et les officiers des gardes du corps et des hallebardiers
portent des bâtons, comme en France, quand ils sont en quartier, et
servent de même.

GOUVERNEURS DES MAISONS ROYALES.

Le comte d'Altamire, du Buen-Retiro.

Le duc de Medina-Coeli, de la Casa del Campo.

Le père prieur de l'Escurial, de l'Escurial, d'Aranjuez.

Le duc del Arco, comme grand écuyer, est surintendant de toutes les
chasses, et gouverneur par là\,:

du Pardo\footnote{C'est ainsi que Saint-Simon écrit ce mot qu'on a
  depuis changé en celui de \emph{Prado}.},

de la Torre di Parada,

de la Sarçuela,

du Pardillo\,;

et il est personnellement gouverneur

de Balsaïm,

et de Saint-Ildephonse.

Les fonctions des charges ont été, ce me semble, suffisamment
expliquées, mais les appointements oubliés. Les voici, ils sont tous en
pistoles\,:

MAISON DU ROI.

Majordome-major

(\emph{pistoles}.) 1800

Majordomes de semaine

400

La médecine n'a rien de fixé.

Introducteur des ambassadeurs

275

Sommelier du corps

430

Gentilshommes de la chambre

90

Guarda-roba

»

Grand écuyer

900

Premier écuyer

300

Patriarche des Indes

90

Capitaines des gardes du corps

1000

Capitaines des hallebardiers

1000

Colonels des régiments des gardes

1000

MAISON DE LA REINE.

Majordome-major

1300

Majordomes de semaine

200

Camarera-mayor

800

Dames du palais

834

Señoras de honor

200

Grand écuyer

300

Premier écuyer

200

Grands officiers et autres officiers et domestiques du prince et de la
princesse des Asturies, un quart moins que ceux du roi.

Gouverneur de l'infant don Ferdinand

600

Capitaines généraux des provinces

2000

Présidents ou gouverneurs des conseils

2000

Secrétaires d'État

2000

Secrétaire de l'estampille

»

Ces conseillers d'État n'ont point d'appointements.

Nul emploi ni charge vénale en Espagne.

Il n'y a point de charge en Espagne qui réponde à notre grand prévôt ou
prévôt de l'hôtel.

Le majordome-major, en certaines choses, et le corrégidor de Madrid en
d'autres, y suppléent.

EXPLICATION DES SERMENTS.

Les trois charges chez le roi et chez la reine, reçoivent le serment de
tous ceux et celles qui sont chacun sous leurs charges.

Le patriarche aussi, et les capitaines des gardes du corps, celui des
hallebardiers, et les colonels des deux régiments des gardes.

· Le président ou gouverneur du conseil de Castille,

· Les deux majordomes-majors,

· Le capitaine des hallebardiers,

· Les gouverneurs des infants n'en prêtent point\,;

· Le sommelier du corps.

· La camarera-mayor,

· Les deux grands écuyers,

· Le patriarche des Indes le prêtent entre les mains de leur
majordome-major.

Les seuls capitaines des gardes du corps et colonels des deux régiments
des gardes entre les mains du roi.

Les présidents ou gouverneurs des conseils entre les mains de celui du
conseil de Castille.

Les conseillers et officiers de chaque conseil, entre les mains du
président ou gouverneur de leur conseil.

Les secrétaires d'État le prêtaient dans le conseil d'État.

Le secrétaire de l'estampille entre les mains du sommelier du corps.

Les conseillers d'État entre les mains du plus ancien secrétaire d'État.

Les gouverneurs des maisons royales entre les mains d'un conseiller de
la junte des bâtiments.

· Les vice-rois,

· Gouverneurs des provinces,

· Capitaines généraux des armées,

· Capitaines généraux des provinces, j'ignore s'ils prêtent serment ou
entre les mains de qui.

Pareillement le corrégidor de Madrid et {[}ceux{]} des autres villes,
comme le président ou gouverneur du conseil de Castille est leur
supérieur, je croirais que ce serait entre ses mains\footnote{Nous
  rétablissons à la place que lui avait assignée Saint-Simon le passage
  qui commence par les fonctions des charges (p. 7) jusqu'à entre ses
  mains (p.~9). On l'avait rejeté, dans les anciennes éditions, à la fin
  du chapitre.}.

Disons ici un mot de ces maisons royales, puisque l'occasion s'en
présente si naturellement, sans m'abandonner à des descriptions qui ne
sont pas de mon sujet, et qu'il faut voir dans les différents voyageurs.
Le Buen-Retiro est un vaste et magnifique palais, à une extrémité de
Madrid, dent il est séparé par un espace large d'une portée de mousquet,
et qui a un grand et fort beau pare. La cour y passait, de mon temps,
quelques mois de l'année, et s'y est \emph{fixée} depuis l'incendie du
palais de Madrid. On voit par là que c'est un gouvernement fort
agréable.

La Casa del Campo est un bâtiment fort commun, vis-à-vis la place du
palais de Madrid, le Mançanarez entre deux, et tout près dans la plaine.
Il y a un parc, quelques pièces d'eau, quelques bois, mais de ceux des
Castilles et fort peu de vrais arbres. C'est proprement une ménagerie,
mais fort mal remplie et aussi mal entretenue. Je n'ai jamais vu
personne s'y aller promener, ni Leurs Majestés Catholiques. Cela peut
faire une maison de campagne au duc de Medina-Coeli, où il peut aller en
moins de demi-heure, et fournir sa table de bien des commodités, si les
Espagnols connaissaient les tables, même les plus frugales.

J'ai dit de l'Escurial tout ce que j'en pouvais dire. Le roi est maître
d'agréer ou non l'élection du prieur, d'en mettre un, de l'ôter quand il
veut\,; et ce prieur, avec l'autorité que sa place lui donne sur ses
moines et dans le monastère, a aussi celle de gouverneur sur les
appartements de Leurs Majestés Catholiques, de leur cour et de toute
leur suite.

Pour Aranjuez, je remettrai d'en parler au petit voyage que j'y ai fait
pour le voir. Je dirai en attendant que je n'y trouvai pas le
gouverneur, chez qui pourtant je fus logé. C'était un homme du commun,
dont je n'ai pas retenu le nom, et que je n'ai jamais rencontré, ni ouï
parler de lui à personne.

Le Pardo est un bâtiment carré, fermé des quatre côtés, à peu près égaux
et assez courts, dont la cour est triste, et les appartements de Leurs
Majestés Catholiques des plus médiocres en tout\,; les autres des plus
étroits et en fort petit nombre. Il n'y a ni avant-cour ni autre
bâtiment, ni jardin, ni parc. La cour y va pourtant quelquefois, mais
avec le plus étroit nécessaire. C'est une habitation entièrement
esseulée où je ne comprends pas qu'on puisse aller, car rien dû tout n'y
appelle. Cela est au bord d'une plaine aride, peu éloigné d'une colline
au pied de laquelle on passe sur un très médiocre pont, au haut de
laquelle est un couvent de capucins, tout seul, d'où on voit tant que la
vue se peut étendre dans la plaine d'en haut et d'en bas, excepté la
Torre di Parada, qui en est assez proche. Ce n'est, en effet, qu'une
vieille tour, avec une espèce de cabaret joignant, bas et petit, où on
met des relais qui ont donné le nom di Parada à cette tour. Il y a de
Madrid au Pardo deux lieues, c'est-à-dire au moins comme de Paris à
Versailles. Le chemin est assez longtemps agréable le long du Mançanarez
en le remontant, et par ce qui fait le cours de Madrid.

La Sarçuela est un peu plus éloignée de Madrid. C'est une espèce de
petit château, fort commun en dehors et en dedans, mais qui a une sorte
de basse-cour et un jardin, mais dans un grand éloignement de toute
autre habitation. La cour n'y allait plus, mais Charles II quelquefois.

Le Pardillo est un pavillon tout seul au milieu du vaste parc de
l'Escurial, bon pour aller faire une collation, ou pour s'aller
rafraîchir une heure ou deux après la chasse dans ce vaste parc, qui a
beaucoup de fauve et de ces mauvais bois des Castilles.

De Balsaïm et de Saint-Ildephonse, je remets à en parler au voyage que
j'y ai fait. Pour varier et ne pas confondre, je placerai ici ce que je
puis dire de quelques-uns de ceux qui viennent d'être nommés. Je dis
quelques-uns, parce que tous n'en fournissent pas matière. J'ai parlé
des grands d'Espagne à chacun de leurs articles, lorsqu'il s'est trouvé
choses à en dire. Je viens maintenant à ceux qui ne le sont pas, et qui
se trouvent dans la liste précédente de la maison du roi, que j'ai tous
rangés à la suite du grand officier, grands et autres, de la charge
duquel ils dépendent, et {[}à qui ils{]} sont subordonnés.

Don Gaspard Giron, le plus ancien des majordomes du roi de semaine, fut
chargé de me recevoir, accompagner, faire servir par les officiers du
roi, convier des seigneurs à dîner chez moi, et faire les honneurs de ma
table et de ma maison, tant que je fus traité à mon arrivée, et je me
suis depuis adressé à lui quand j'ai eu besoin de quelqu'un du palais
pour ma curiosité particulière. Il était Acuña y Giron, c'est-à-dire de
même maison que le marquis de Villena, duc d'Escalona, majordome-major,
et de la branche du duc d'Ossone.

C'était un grand homme sec, noir, vieux, qui avait été bien fait et
galant, vif, quoique grave, salé en reparties et en plaisanteries, gai
et très poli, avec cela néanmoins la gravité du pays, et sentant en
toutes ses manières sa haute naissance, mais avec aisance et sans rien
de glorieux. Il faut cependant avouer que son premier aspect rappelait
tout à fait le souvenir de don Quichotte. C'était l'homme le plus rompu
à la cour, qui savait le mieux les anciennes et les nouvelles
étiquettes, les rangs, les droits, les règles, les cérémonies, les
personnages distingués ou principaux, les ressorts des fortunes et des
chutes, avec de l'esprit et de la lecture, qui tout discret qu'il fût le
rendaient d'une très aimable et utile conversation. Il avait passé sa
vie dans un emploi qui le tenait presque toujours dans le palais, où il
avait été témoin de près d'une infinité de choses importantes et
curieuses, toujours au milieu de la cour, en tous lieux, et parmi tous
les changements de ministère, plus employé qu'aucun des majordomes à
recevoir les ambassadeurs distingués, les princes et les personnes les
plus considérables qui venaient à Madrid, et que le roi voulait honorer,
M. le duc d'Orléans en particulier, au-devant duquel il fut envoyé avec
les équipages du roi, et qu'il reçut et accompagna toutes les fois qu'il
alla à Madrid. Ces fonctions continuelles lui avaient acquis une grande
familiarité avec le roi et la reine, qui se plaisaient quelquefois à
causer avec lui en particulier, et avec qui il était fort libre. Cela le
faisait compter par les courtisans les plus élevés, même par les
ministres\,; comme il passait sa vie au milieu de la cour par des
fonctions continuelles, il vivait avec tout le monde avec beaucoup
d'aisance et de familiarité. C'était un homme tout fait pour l'emploi
qu'il exerçait, et un répertoire vivant auquel le roi, les ministres,
les seigneurs avaient recours avec confiance sur les difficultés qui
survenaient sur le cérémonial, ou d'autres matières que son expérience
dans ses fonctions et dans les choses de la cour lui avaient apprises.
C'était d'ailleurs un fort honnête homme, homme d'honneur et de bien,
d'une conduite sans reproche à l'égard de la cour, et quoique assez
pauvre, désintéressé et point du tout avide de grâces. Je me suis
souvent étonné comment il était demeuré ensablé dans un emploi qui sert
de passage aux fortunes de toute espèce. Il y était si propre et si
commode au roi, aux ministres qui s'en servaient et aux
majordomes-majors pour l'exercice, de leur charge, que j'ai toujours cru
que c'est ce qui l'y avait arrêté. Je l'ai donc beaucoup fréquenté, et
j'en ai tiré des choses utiles et curieuses. Nous nous étions pris tous
deux d'amitié.

Le marquis de Villagarcias était le deuxième des majordomes. Il avait
moins d'esprit, de finesse dans l'esprit, mais un agrément, une bonté,
une politesse extrême, et un désir d'obliger toujours prêt et prévenant.
C'était aussi un homme de qualité, estimé et assez compté, qui avait été
destiné à l'ambassade de Portugal, qui n'eut pas lieu. Le duc de
Linarès, mari de la camarera-mayor de la reine douairière à Bayonne,
était mort au Mexique, dont il était vice-roi, quelque temps avant que
j'arrivasse en Espagne\,; et peu avant que j'en partisse Villagarcias
fut nommé pour lui succéder, ce qui fut pour lui une grande fortune,
dont je remarquai que toute la cour fut bien aise.

Cucurani était un Italien raffiné, appliqué, instruit, glorieux,
ambitieux, particulier, qui n'avait la confiance de personne. Il était
gendre de la nourrice de la reine, qui était aussi \emph{assafeta}, et
il espérait tout par là. Il avait de l'esprit et du manège. Depuis mon
retour, assez tôt, il obtint une ambassade dans le Nord.

Villafranca, si différent en tout du grand d'Espagne, et qui sans lui
appartenir en rien portait le même titre (j'expliquerai ce terme après),
était un vieil homme renfermé, qui ne paraissait que pour ses fonctions,
glorieux et ridicule. Je ne sais plus à quelle occasion de bonnes fêtes,
de jour de naissance ou de baptême de l'infant don Philippe, les
ambassadeurs qui étaient à Madrid allèrent ensemble complimenter le roi,
la reine, le prince et la princesse des Asturies. Les ambassadeurs
d'Angleterre, de Venise et de Hollande, Maulevrier et moi, étions avec
le nonce qui portait la parole, et ce que chacun avait amené de
principal de chez soi nous accompagnait. Arrivés au palais,
l'introducteur se fit attendre une demi-heure au delà de l'heure qu'il
avait marquée, car à ces sortes de compliments, il n'y a que
l'introducteur des ambassadeurs, à la différence de l'entrée et de la
première audience de cérémonie. Le nonce fut choqué d'attendre, et lui
en dit son avis. Sans prendre la peine de répondre, il alla gratter à la
porte du cabinet des miroirs, et nous introduisit tout de suite. En
sortant, le nonce encore plus choqué de ce procédé lui en lâcha des
lardons, auxquels l'introducteur répondit avec impertinence. Le nonce,
pour lui marquer son mépris, dédaigna de se fâcher, et avec un sourire
nous demanda ce que nous en pensions. Nous ne pûmes alors éviter d'en
dire chacun notre mot. L'introducteur, piqué, voulut se rebecquer\,; le
nonce alors se moqua de lui tout franchement, lui dit qu'il nous faisait
sentir qu'il était de méchante humeur, et le brocarda tant et si bien,
chemin faisant, que l'introducteur lui répondit enfin, après avoir assez
grommelé entre ses dents, qu'il voyait bien qu'il ferait mieux de nous
laisser faire nos visites, et nous quitta on s'en moqua de lui un peu
davantage. Nous continuâmes sans lui toute notre tournée, mais nous ne
voulûmes pas en porter de plaintes. C'était un pauvre bonhomme très
dépourvu d'esprit et de sens, fort incapable de son emploi, quoique des
plus légers, et compté pour rien par tout le monde.

Hyghens, premier médecin, était Irlandais, docteur en plusieurs
universités et en celle de Montpellier, d'où il était passé en Espagne
médecin des armées. On y fut si content de sa conduite et de sa capacité
que le roi d'Espagne le fit son premier médecin, et avait en lui
beaucoup de confiance et plus que la reine n'aurait voulu, quoiqu'elle
le traitât fort bien. Mais elle ne souffrait pas volontiers d'autres
gens que donnés de sa main pour cet intérieur si assidu et si intime, et
aurait désiré cette place à son premier médecin Servi, qui était de son
pays, et de son choix, et qui lui était entièrement livré. Elle en vint
à bout, en effet, quelques années après mon retour que Hyghens mourut.

Cet Irlandais, qui parlait parfaitement français, était un excellent
médecin qui, sans entêtement ni attachement de médecin, ne voulait que
guérir son malade avec une grande application. J'en fis une heureuse
expérience à ma petite vérole, dont les détails, qui pourraient
instruire les médecins de bonne foi, seraient ici étrangers. Son
caractère ouvert mais discret, doux mais ferme, montrait sans la plus
légère affectation une belle âme, toujours occupée du bien, sans nul
autre intérêt quelconque, quoiqu'il aimât sa famille qui était assez
nombreuse, et de plus détaché de toute ambition, voyant de très près les
intrigues, sans y vouloir jamais entrer, disait très nettement le vrai
au roi sur sa santé, et le lui disant de même et à la reine, quand l'un
ou l'autre l'en mettaient à portée sur d'autres matières, mais sans
s'avancer jamais sur aucune, et parlant toujours avec grande discrétion
et grand éloignement de nuire à personne. Aussi était-il fort aimé et
considéré. Il avait l'esprit juste, agréable, modeste, avait beaucoup de
belles-lettres et savait bien l'histoire, surtout il connaissait bien
les maîtres et la cour, et passait pour un grand et sage médecin, et
pour le seul même en Espagne qui méritât le nom de médecin. Il possédait
très bien la chirurgie et avait souvent fait d'heureuses opérations, bon
botaniste, bon artiste, connaissant bien les simples et les remèdes dont
il savait faire usage, et la composition des médicaments comme le
meilleur apothicaire et comme un bon chimiste. Tant de bonnes qualités
étaient relevées par une piété sage, éclairée et vraie, qui n'était que
pour lui, et qui n'incommodait personne que par le frein qu'elle mettait
à sa langue, plus souvent que n'auraient voulu ceux qui étaient à portée
avec lui de l'entretenir librement. Sa conversation m'a été d'un grand
secours et m'a instruit de bien des choses. Il aimait son pays, ses
compatriotes avec tendresse, et avait le plus vif attachement pour le
roi Jacques, et pour tout ce qui était de son parti. La sagesse le
retenait, à cet égard, dans les plus justes bornes, à l'extérieur\,;
mais quand il se trouvait en liberté avec des amis, ce feu de patrie lui
échappait, et bienfaisant pour tout le monde, il ne se possédait pas
d'aise quand il pouvait rendre quelque service à quelque jacobite. J'eus
tout loisir de le connaître pendant six semaines qu'il ne bougea
d'auprès de moi.

Sa candeur, sa probité, ses soins me gagnèrent, son esprit me plut, nous
prîmes grande amitié l'un pour l'autre. Je dus la sienne, à ce que je
crois, au penchant qu'il sonda et qu'il trouva en moi pour le roi
Jacques. Je le trouvai si sage et si discret que je ne me cachais point
de lui, sans toutefois lui rien dissimuler sur les liens de notre cour à
cet égard, et sur mon impuissance. Je lui expliquai même les ordres
précis que j'avais là-dessus, et d'éviter le duc d'Ormond qu'il mourait
d'envie que j'entretinsse. J'y consentis, à condition que ce serait sous
le plus grand secret, à notre retour à Madrid\,; que le duc d'Ormond se
rendrait chez lui, m'y attendrait sans pas un de ses gens dans la
maison, se tiendrait dans un cabinet séparé\,; qu'averti par Hyghens,
j'irais à l'heure marquée lui faire visite, je le trouverais seul, et
qu'après que mes gens seraient retirés, je passerais dans le cabinet où
serait le duc d'Ormond\,; qu'après la conversation, je le laisserais
dans ce cabinet et reviendrais dans la chambre de Hyghens, d'où je m'en
irais, comme ayant fini ma visite\,; que le duc d'Ormond ne se
retirerait que quelque temps après\,; qu'au palais ni ailleurs, nous ne
nous approcherions point l'un de l'autre, et que nous nous saluerions
avec la civilité que nous nous devions, mais avec froideur et
indifférence marquée. Pour le dire tout de suite, cela s'exécuta de la
sorte plusieurs fois chez Hyghens, sans que personne s'en soit jamais
aperçu, et notre froideur, si marquée ailleurs, nous donnait quelquefois
envie de rire.

Je trouvai dans le duc d'Ormond toute la grandeur d'âme que nul revers
de fortune ne pouvait altérer, la noblesse et le courage d'un grand
seigneur, la fidélité la plus à toute épreuve, et l'attachement le plus
entier au roi Jacques et à son parti, malgré les traverses qu'il en
avait essuyées, et auxquelles il était tout prêt de s'exposer de nouveau
dès qu'il pourrait en espérer le plus léger succès pour les affaires
d'un prince si malheureux. D'ailleurs, je trouvai si peu d'esprit et de
ressources que j'en fus doublement affligé pour le roi Jacques et son
parti, et pour le personnel d'un seigneur si brave, si affectionné et si
parfaitement honnête homme. Je ne lui dissimulai {[}pas{]} non plus que
j'avais fait à Hyghens les chaînes de notre cour et mon impuissance à
cet égard, de sorte que nos entretiens, où il me confia aussi ses
déplaisirs sur les méprises du roi Jacques et les divisions de son
parti, n'aboutirent qu'à des regrets communs et à des espérances bien
frêles et bien éloignées.

Le Gendre était très bon chirurgien\,; le roi l'aimait et la reine
aussi, parce qu'elle n'avait personne en main pour le remplacer. C'était
d'ailleurs un drôle hardi, souple, intéressé, qui se faisait compter, et
qui, tant qu'il pouvait se mêlait de plus que de son métier, mais
sagement et sans y paraître.

Ricœur était plus en sa place, aimé, estimé, bien, avec le roi et la
reine, capable dans son métier, obligeant, bienfaisant, fort français,
qui n'était pas sans intérêt et sans songer à ses affaires, mais sans
intéresser l'honnête homme, et qui longtemps après mon retour voyant
Hyghens mort et La Roche aussi, auxquels il était fort attaché, Servi à
la place d'Hyghens, et Le Gendre ayant l'estampille qu'avait La Roche,
obtint à toute peine de se retirer, et vint mourir en France, où il
vécut, en effet, en homme de bien et fort dans la retraite. Je n'eus
point de commerce que d'honnêteté avec ces deux derniers qui ne
pouvaient pas m'être d'un grand usage.

Le marquis del Surco était un Milanais de fortune, fin, délié, de
beaucoup d'esprit et de jugement, grand et bien fait, qui avait été à
Milan capitaine des gardes du prince de Vaudemont, et depuis, son espion
en Espagne, par conséquent impérial fort dangereux, homme de beaucoup de
manège et d'intrigue, et dont la corruption du coeur et de l'ambition
avait beaucoup profité à l'école d'un si bon maître, et si heureux en ce
genre. Un extérieur froid, mesuré cachait ses sourdes menées, toujours
bas valet de qui pouvait le plus, et ne faisant jamais sans vues le pas
en apparence le plus indifférent. Sa souplesse, son intrigue, les voiles
épais dont il savait se couvrir, une ambition en apparence tranquille,
en effet la plus active et la plus infatigable, une dévotion de
commande, une connaissance parfaite de ceux à qui il avait affaire, une
grande adresse à savoir leur plaire, les gagner, s'en servir, le porta à
la place de sous-gouverneur du prince des Asturies, et, ce qui
scandalisa toute la cour, à la clef de gentilhomme de la chambre du roi.
Sa femme, faite exprès pour lui, grande, bien faite comme lui, et de bon
air, qu'il avait bien dressée, avait aussi beaucoup d'esprit et
d'intrigue, elle était ainsi arrivée parla cabale italienne, dont je
parlerai en son temps, à être \emph{señora de honor} de la reine et
assez bien avec elle, de façon qu'il se pouvait dire qu'en gouverneur et
en sous-gouverneur du prince des Asturies, quoique chacun en son genre,
il eût été difficile de choisir deux plus insignes et plus dangereux
fripons, et plus radicalement incapables de donner la moindre éducation
à un prince, tous deux aussi malhonnêtes gens l'un que l'autre, tous
deux pleins d'art, d'esprit et de vues, mais del Surco plus encore que
le Popoli, et moins affiché que lui pour ce qu'ils étaient l'un et
l'autre. Ils se connaissaient bien tous deux, par conséquent, ne
s'aimaient ni ne s'estimaient\,; mais ils sentaient tous deux qu'il
était de leur intérêt de ne pas se brouiller et d'avoir l'air de
s'entendre, et leur intérêt était leur maître absolu. Je reçus peu de
civilités de Surco, sous prétexte de l'attachement de sa charge, mais
beaucoup de sa femme, dont les manières étaient très aimables, ce que
n'avait pas son mari, dont le dedans, à l'esprit près, et le dehors me
rappelèrent souvent M. d'O, dont del Surco avait aussi l'impertinente
importance, car pour le Saumery, il n'en avait que la corruption, et
d'ailleurs n'allait pas à la cheville du pied du Surco.

Valouse, gentilhomme d'assez bon lieu, du comtat d'Avignon, élevé page
de la petite écurie, produit par Du Mont au duc de Beauvilliers pour
être écuyer de M. le duc d'Anjou, parce qu'il était bon homme de cheval,
sage et de bonnes moeurs, suivit ce prince en Espagne, et y devint un
des fréquents exemples qu'avec de la sagesse et de la conduite on fait
fortune dans les cours sans avoir aucun esprit. Il fit son capital de
s'attacher au roi, à ses supérieurs, de ne se mêler d'aucune intrigue,
de ne donner d'ombrage à personne, d'être réservé en tout, et appliqué à
son emploi, souple à qui gouvernait, avec indifférence dans tous les
changements, appliqué à plaire au roi, et aux deux reines l'une après
l'autre, point répandu dans la cour, sous prétexte de l'assiduité de ses
fonctions\,; bien avec tout le monde, sans nulles liaisons
particulières, et inutile à tout par le non usage, de résolution prise,
de sa faveur pour qui que ce fût d'ailleurs aussi ne nuisant à personne.
Il fut bientôt majordome de semaine, puis premier écuyer, après le duc
del Arco, et totalement dans sa main, et vivant sous lui grand écuyer
comme sous son maître, dont il était fort bien traité. Il poussa enfin
longtemps après mon retour, jusqu'à être chevalier de la Toison d'or, et
mourut comme il avait vécu sans s'être marié et sans avoir amassé
beaucoup de bien, dont il ne se soucia pas.

Je l'avais connu dans la jeunesse des princes, je le retrouvai tel que
je l'avais laissé. J'en reçus toutes sortes de prévenances\,; je lui fis
aussi toutes sortes de politesses, mais sans particulier, sans liaison
qu'il ne souhaitait pas et qui m'aurait été fort inutile. Il obtint
aussi une clef de gentilhomme de la chambre, et fut préféré pour être de
service au rare défaut du marquis de Santa-Cruz et du duc del Arco, mais
cela\,: longtemps aussi depuis mon retour.

Hersent était fils d'un homme de qui j'ai parlé à l'occasion du départ
de Versailles de Philippe V. Il ressemblait à son père pour l'honneur et
la probité, mais non pour la liberté, la familiarité, la confiance du
roi, et une sorte d'autorité qu'il avait usurpée, que nul autre que les
ministres ne lui enviait, parce qu'elle était utile au bien et à tous,
et qu'il ne se méconnaissait point. Le fils n'en avait ni l'esprit ni le
crédit, ni la considération\,; quoique sur un pied d'estime, et mêlé et
fort bien avec tout le monde, en se tenant pourtant assez dans les
mesures de son état. J'en reçus toutes sortes d'attentions, mais je n'en
tirai pas grand fruit.

Le cardinal Borgia revint de Rome à Lerma, pendant ma petite vérole, du
conclave, où le cardinal Conti avait été élu. C'était un grand homme de
bonne mine, oncle paternel du duc de Gandie, et neveu d'un autre
cardinal Borgia, aussi patriarche des Indes. Son adieu au cardinal
Conti, frère du pape, le caractérisera mieux que tout ce que j'en
pourrais dire. Parmi les compliments de regrets réciproques de leur
séparation, Borgia dit à Conti que tout ce qui le consolait était
l'espérance du plaisir de le revoir bientôt, et que dans peu un autre
conclave le rappellerait à Rome. On peut juger comme le frère du pape
trouva ce compliment bien tourné. Borgia était un très bon homme, qui
n'avait pas le sens commun, et dont sa famille et le défaut de sujets
ecclésiastiques avait fait la fortune. La difficulté de la main nous
empêcha de nous visiter\,; mais force civilités au palais et partout où
nous nous rencontrions, et quelquefois des envois de compliments de l'un
chez l'autre. Son rang et sa charge lui attiraient quelque sorte de
considération\,; mais de sa personne, il était compté pour rien. Le roi
et la reine l'aimaient assez, et ne se contraignaient point de s'en
moquer.

On a vu en son lieu le temps et la façon dont le roi d'Espagne se forma
une garde, le premier de tous ses prédécesseurs, et ce qui se passa en
cette occasion. La copie de celle du roi, son grand-père, en fut si
fidèle que ce seul mot instruit de sa composition, de son service, de
son uniforme, en sorte qu'à voir cette garde on se croyait à Versailles.
Il en était de même dans les appartements à l'égard des garçons du
palais et des garçons tapissiers, quoiqu'en bien plus petit nombre que
les garçons du château et des tapissiers à Versailles, où on s'y croyait
aussi à les voir et leur service. Il en était de même pour la livrée du
roi, de la reine et de la princesse des Asturies\,; et tous les services
des compagnies des gardes du corps et des régiments des gardes, de leurs
capitaines, de leurs colonels, de leurs officiers entièrement semblables
à ceux d'ici, sinon qu'il n'y a que trois compagnies des gardes du
corps, dont les capitaines et, le guet servent par quatre mois chacun,
au lieu de trois ici, où il y a quatre compagnies.

Armendariz, lieutenant général assez distingué, était lieutenant-colonel
du régiment des gardes espagnoles. C'était un homme d'esprit, remuant,
insinuant, intrigant, impatient de\,: l'état subalterne, qui avait ses
amis et son crédit, et que le marquis d'Ayétone était importuné de
trouver assez souvent sur son chemin dans les détails et sur les grâces
à répandre dans le régiment. Mais l'extérieur était gardé entre eux, et
j'ai souvent trouvé Armendariz chez le marquis d'Ayétone, d'un air assez
libre quoique respectueux. Il était fort poli, agréable en conversation,
bien reçu partout, assez souvent chez moi. Il avait de la réputation à
la guerre\,; on prétendait qu'il ne fallait pas se lier à lui ailleurs.
Avant mon départ, il fut nommé pour succéder au marquis de Valero, sur
le point de revenir de sa vice-royauté du Pérou, qui se trouva fait duc
d'Arian et grand d'Espagne en arrivant à Madrid.

Il ne faut pas aller plus loin sans dire un mot de ce qui est connu en
Espagne sous le nom de \emph{titolados}. Ce sont les marquis et les
comtes qui ne sont point grands. La plaie française a gagné l'Espagne
sur ce point, mais d'une manière encore plus fâcheuse, en ce que ce
n'est pas simple licence comme ici, et, dès là, facile à réformer quand
il plaira au roi de le vouloir. Mais en Espagne, c'est concession du roi
en lettres-patentes enregistrées au conseil de Castille ou d'Aragon sur
une terre, et dès là érection, ou sans terre sur le simple nom de celui
que le roi veut favoriser d'un titre de marquis ou de comte, tellement
que, quelque infimes qu'ils soient en grand nombre, tels que le marchand
Robin, directeur de la conduite de Maulevrier, et le directeur de la
vente du tabac à Madrid, tous deux faits comtes peu avant mon arrivée en
Espagne, et comme quantité d'autres qui ne valent pas mieux, ces gens-là
sont véritablement marquis et comtes, et quels qu'ils soient
d'eux-mêmes, ils y sont fondés en titre qui ne peut leur être disputé,
au lieu qu'en France, qui veut se faire annoncer marquis ou comte, le
devient aussitôt pour tout le monde qui en rit, mais qui l'y appelle,
sans autre droit ni titre que l'impudence de se l'être donné à soi-même.
Ainsi en Espagne comme en France, tout est plein de marquis et de comtes
les uns de qualité, grande ou moindre, les autres, canailles ou peu s'en
faut, pour la plupart, ceux-ci, de pure usurpation de titre, ceux
d'Espagne, de concession de titre. Mais cette concession ne les mène pas
loin. Ces titres ne donnent aucun rang, et depuis qu'il n'y a plus
d'étiquette et de distinction de pièces chez le roi pour y attendre, ces
\emph{titolados} ne jouissent d'aucune distinction. Les marquis et les
comtes sont honorés et considérés de tout le monde, selon leur
naissance, leur âge, leur mérite, leurs emplois, comme le sont aussi les
gens de qualité qui n'ont point ces titres, et qu'on appelle don Diègue
un tel, etc., et ces autres marquis et comtes en détrempe sont méprisés
et plus que s'ils ne l'étaient pas, et en cela, ils font mieux que nous
ne faisons en France.

Il faut pourtant dire que ces \emph{titolados} peuvent avoir un dais
chez eux, mais toujours avec un grand portrait du roi d'Espagne dessous,
qui est la différence du dais des grands d'Espagne, qui n'ont jamais de
portrait du roi dessous, mais des ornements de broderie ou leurs armes,
ou rien du tout dans la queue, et toute unie comme il leur plaît. Ces
dais avec le portrait du roi descendent, s'il se peut, encore davantage.
Hyghens en avait un ainsi comme premier médecin, que j'y ai vu plusieurs
fois, et j'y appris qu'il était commun à d'autres fort petites charges.
Mais toutefois n'a pas un dais avec le portrait du roi, sans titre et
droit de l'avoirs mais le portrait du roi qui veut, chez soi, et comme
il veut, sans dais.

Cette matière me conduit à celle de \emph{l'Excellence}. On ne se
licencie plus de la refuser sous aucun prétexte, comme on faisait
autrefois sous prétexte de familiarité et de liberté, par des gens
fâchés de ne l'avoir pas eux-mêmes. Je ne sais comment cet abus s'est
enfin aboli\,; mais entre grands ou autres qui ont l'Excellence, il
arrive quelquefois qu'ils se tutaient et s'appellent par leurs seuls
noms de baptême, par familiarité, et non pour éviter ce qu'ils se
doivent réciproquement. L'Excellence, autrefois réservée aux grands et
aux ambassadeurs étrangers, s'est peu à peu infiniment étendue. Les fils
aînés des grands, les successeurs immédiats à une grandesse, les
vice-rois et les gouverneurs de provinces, les capitaines généraux et
les conseillers d'État, les chevaliers de la Toison d'or, ceux que le
roi nomme à une ambassade, même le cas arrivant qu'ils n'y aillent pas
(et le marquis de Villagarcias dont j'ai parlé naguère l'avait acquise
de cette sorte), à plus forte raison ceux qui ont été ambassadeurs,
enfin le gouverneur du conseil de Castille, tous ceux-là, et leurs
femmes, ont l'Excellence, tellement qu'il importe fort de savoir à qui
on parle pour ne pas offenser ceux qui l'ont à qui on ne la donnerait
pas, et peut-être davantage ceux à qui on la donnerait et à qui on ne la
devrait pas.

C'est la méprise qui m'arriva, dont je fus fiché après, mais qui aurait
pu être plus désagréable. Ce fut à Lerma, au sortir de la cérémonie du
mariage du prince et de la princesse des Asturies, à la fin de laquelle
je venais d'être déclaré grand d'Espagne de la première classe,
conjointement avec mon second fils, et l'aîné déclaré chevalier de la
Toison d'or. Je venais d'être accablé des compliments de toute la cour.
Ma journée, qui avait commencé de bon matin., était loin d'être finie,
et moi sortant de maladie, fort fatigué. Je profitai donc d'un tabouret
qui se rencontra dans une des premières salles, ayant autour de moi ce
que j'avais mené de plus considérable. Je me reposais de la sorte,
lorsqu'un jeune {[}homme{]} bien fait, un peu noir, s'en vint me faire
des compliments empressés et fort polis, avec un air de respect et de
déférence. Je crus le reconnaître parfaitement\,; je me levai, lui
répondis sur le même ton, je multipliai mes remerciements et je
l'accablai d'\emph{Excellence}. Il eut beau me témoigner sa honte de me
voir debout pour lui, je pris cela pour un raffinement de politesse, je
n'avais garde de me rasseoir, n'ayant pas d'autre siège à lui présenter,
enfin il s'en alla pour {[}me{]} laisser rasseoir. Dès qu'il fut retiré,
l'abbé de Saint-Simon me demanda quel plaisir je prenais à confondre ce
pauvre garçon qui me venait marquer son respect et sa joie, et à
l'accabler d'Excellence et de moqueries. Surpris à mon tour, je lui
demandai si je pouvais en user autrement avec le marquis de Cogolludo,
fils aîné du duc de Medina-Coeli. «\,Le marquis de Cogolludo\,! reprit
l'abbé\,; mais vous n'y songez pas, c'est le fils de
M\textsuperscript{me} de Pléneuf, dont l'embarras nous a fait pitié. À
En effet, c'était lui-même. La Fare l'avait amené avec lui, comme je
partais de Madrid pour Lerma. Je n'avais fait qu'entrevoir ce jeune
homme lorsqu'il me le présenta, et je ne lavais ni vu ni rencontré
depuis, séparé jusqu'à la veille de ce jour-là par la petite vérole. Ils
se mirent tous à rire et à se moquer de moi\,; mais ils convinrent tous
qu'il ressemblait beaucoup au marquis de Cogolludo. De lui faire des
excuses de l'avoir trop bien traité, il n'y avait pas moyen\,; de lui
laisser penser que je m'étais moqué de lui, était encore pis\,:
l'expédient fut d'en faire le conte à la Fare.

Venons maintenant à la maison de la reine d'Espagne.

MAISON DE LA REINE.

\emph{Majordome-major}.

Le marquis de Santa-Cruz.

Je ne parlerai point des trois majordomes de semaine, dont Magny en
était un.

\emph{Premier médecin}.

M. Servi.

J'ai parlé de lui il n'y a pas longtemps.

\emph{Camarera-mayor}.

La comtesse douairière d'Altamire, Angela Folch, de Cardonne et Aragon.

\emph{Dames du palais}.

La princesse de Roberque.

La princesse de Pettorano.

La duchesse de Saint-Pierre.

La comtesse de Taboada.

\emph{Señoras de honor}.

M\textsuperscript{me}s Rodrigo.

M\textsuperscript{me}s Albiville.

Carillo.

Monteher.

Nievès.

O'Calogan.

Del Surco.

Cucurani.

Riscaldalègre.

\emph{Assafeta}.

Dona Laura Piscatori, nourrice de la reine.

\emph{Confesseur}.

Don Domingo Guerra.

\emph{Grand écuyer}.

Le duc de Giovenazzo, c'est-à-dire notre prince de Cellamare.

\emph{Premier écuyer}.

Le marquis de Saint-Jean, et son fils en survivance.

La comtesse d'Altamire était fille du sixième duc de Ségorbe et de
Cardonne. Son mari mourut en 1698, étant ambassadeur d'Espagne à Rome.
Elle était mère du comte d'Altamire et du duc de Najara, et belle-mère
du comte de San-Estevan de Gormaz. On a vu ailleurs dans quelle union
elle, le marquis de Villena et le marquis de Bedmar et leurs enfants
vivaient ensemble, ce qui redoublait leur considération. Cette comtesse
d'Altamire était une des plus grandes dames d'Espagne, en tout genre,
d'une grande vertu et de beaucoup de piété. Avec un esprit qui n'était
pas supérieur, elle avait toujours su se faire respecter par sa conduite
et son maintien, et personne n'était plus compté qu'elle par la cour,
par les ministres successifs, par le roi et la reine mêmes. Elle fut
d'abord camarera-mayor, après l'expulsion de la princesse des Ursins, et
toujours également bien avec la reine, et sur un grand pied de
considération. Elle faisait fort assidûment sa charge et fort
absolument, toutefois poliment avec les dames, mais dont pas une n'eût
osé lui manquer, ni branler seulement devant elle. Elle était petite,
laide, malfaite, avait environ soixante ans et en paraissait bien
soixante et quinze. Avec cela, un air de grandeur et une gravité qui
imposait. J'allais quelquefois la voir. Elle était toujours sur un
carreau, au fond de sa chambre\,; des dames sur des carreaux ou des
sièges, comme elles voulaient\,; on me donnait un fauteuil vis-à-vis
d'elle. Je la trouvai une fois seule, elle ne savait pas un mot de
français ni moi d'espagnol, de manière que nous nous parlâmes toujours
sans nous entendre que par les gestes\,; elle en souriait parfois et moi
aussi. J'abrégeai fort cette visite.

J'ai parlé ailleurs de la princesse de Robecque, de la duchesse de
Saint-Pierre, et de la princesse de Pettorano. La comtesse de Taboada
n'était point laide, et ne manquait pas d'esprit ni de vivacité\,; j'ai
parlé de son mari et de son beau-père le comte de Maceda, grand
d'Espagne.

Parmi les \emph{señoras de honor}, il y en avait plusieurs qui avaient
de l'esprit et du mérite. La femme de Sartine, qui avait été camériste
et bien avec la reine, la devint à la fin. M\textsuperscript{me} de
Nievès, très bien avec la reine, était gouvernante de l'infante, et vint
et demeura à Paris avec elle, et s'en retourna aussi avec elle. On lui
trouva, en ce pays, de l'esprit, du sens et de la raison\,; je ne sais
si cela fut réciproque. M\textsuperscript{me} de Riscaldalègre était une
femme bien faite, qui avait beaucoup de mérite, qui était, considérée,
et qui aurait été fort propre à bien élever une princesse.
M\textsuperscript{me} d'Albiville était une Irlandaise âgée, qui
méritait aussi sa considération. Le mérite de M\textsuperscript{me} de
Cucurani était d'être fille de l'assafeta, qui était Parmesane, nourrice
de la reine, et qui toute grossière paysanne qu'elle était née et
qu'elle était encore, conservait un grand ascendant sur la reine, était
la seule qui, par l'économie des journées, pouvait chaque jour lui dire
quelque mot tête à tête, et qui avait assez d'esprit pour avoir des
vues, et les savoir conduire.. Enfin ce fut elle qui fit chasser le
cardinal Albéroni, dont on ne serait jamais venu à bout sans elle. Comme
elle était extraordinairement intéressée, il y avait des moyens sûrs de
s'en servir. D'ailleurs elle n'était point méchante. Pour son mari, ce
n'était qu'un paysan enrichi, dont on ne pouvait rien faire, et qui
n'était souffert que par l'appui de sa femme. Mais celle-ci était
redoutée et ménagée par les ministres et par toute la cour.

Don Domingo Guerra, confesseur de la reine, n'était rien ni de rien,
lorsque j'étais en Espagne. Il était frère de don Michel Guerra, de qui
je parlerai bientôt, et n'en tenait pas la moindre chose. Le plus plat
habitué, de paroisse aurait paru un aigle en comparaison de ce
confesseur. Il n'est pas de mon sujet de parler d'un peu de crédit qu'il
eut assez longtemps, depuis mon retour, qui n'en fit qu'un abbé
commandataire de Saint-Ildephonse et un évêque \emph{in partibus},
quoiqu'il l'eût enflé jusqu'à penser au cardinalat, et à se croire un
personnage, mais avec qui personne n'eut à compter.

Les deux Saint-Jean, père et fils, étaient d'espèce à donner de la
surprise de les voir premiers écuyers de la reine. Je n'ai point su par
où elle les prit en si grande amitié, qui, du temps que j'étais en
Espagne, était déjà fort marquée.

C'étaient des gens cachés, mesurés, respectueux avec tout le monde, qui
se produisaient peu, qui ne faisaient nulle montre de leur faveur, qui
ne voulaient être mal avec personne, ni liés avec aucun. Sages dans leur
conduite, ils ne donnaient aucune prise. Comme ils ne voulaient faire
que pour eux et rien pour personne, pour mieux ménager leur crédit pour
eux, éviter l'envie et cacher leurs vues, ils s'enveloppaient de
modestie et d'impuissance, et ne servaient et ne desservaient personne.
Le père avait bien commencé\,; le fils, qui avait plus d'esprit et de
montant, et longtemps depuis mon retour, on fut subitement épouvanté de
le voir tout d'un coup grand écuyer de la reine et grand d'Espagne.

J'ai expliqué avec assez de détails les fonctions de toutes ces charges
pour que je n'aie rien à y ajouter, sinon que les trois capitaines des
gardes du corps et les colonels des deux régiments des gardes prêtent
serment entre les mains du roi. Ce sont les seuls dont le roi même le
reçoit, et ces charges et ces grades sont aussi d'établissement nouveau.

On a vu plus haut de quelles personnes furent formées les maisons du
prince et de la princesse des Asturies, lorsque j'ai parlé de cet
établissement. Je n'ai donc rien à y ajouter, sinon que leurs fonctions
chez le prince et la princesse sont pareilles à celles que les mêmes
charges ont chez le roi et chez la reine. L'âge alors si tendre des
infants me dispensera de parler des personnes employées auprès d'eux.
Del Surco et Salazar, major des gardes du corps, lieutenant général et
homme d'esprit et de qualité, furent dans la suite gouverneurs chacun
d'un. Je le dis pour la singularité de cette fortune pour un homme tel
que le Surco, et pour celle du soupçon peut-être mal fondé, mais reçu
comme certain par tout le monde, que le Salazar avait empoisonné sa
femme, comme le duc de Popoli avait fait la sienne, ce qui fit dire à la
cour qu'avoir empoisonné sa femme était une condition nécessaire pour
arriver à l'honneur et à la confiance d'être gouverneur des infants.

La médiannate que paye au roi d'Espagne un grand d'Espagne pour la
première fois monte à huit mille ducats. Ses descendants en payent
quatre mille à chaque mutation. Les frais pour la première fois vont
bien à la moitié. Les \emph{lanzas} que paye tous les ans un grand
d'Espagne se montent à soixante pistoles, quand sa grandesse est placée
sur un titre de Castille.

\hypertarget{chapitre-ii.}{%
\chapter{CHAPITRE II.}\label{chapitre-ii.}}

~

{\textsc{Miraval, gouverneur du conseil de Castille\,; son caractère.}}
{\textsc{- Caractère du grand inquisiteur.}} {\textsc{- Conseils.}}
{\textsc{- Deux marquis de Campoflorido extrêmement différents à ne pas
les confondre.}} {\textsc{- Archevêque de Tolède.}} {\textsc{-
Constitution.}} {\textsc{- Inquisition.}} {\textsc{- Le nonce ni les
évêques n'ont point l'Excellence.}} {\textsc{- Premier et unique exemple
en faveur de l'archevêque de Tolède, de mon temps.}} {\textsc{-
Conseillers et conseil d'État, nuls.}} {\textsc{- Ce qu'ils étaient.}}
{\textsc{- Don Michel et don Domingo Guerra\,; leur fortune et leur
caractère.}} {\textsc{- Fortune et caractère du marquis de Grimaldo et
de sa femme.}} {\textsc{- Riperda.}} {\textsc{- Fortune et caractère du
marquis de Castellar et de sa femme.}} {\textsc{- Jalousie du P.
Daubenton {[}à l'égard{]} du P. d'Aubrusselle\,; caractère de ce
dernier.}} {\textsc{- Jésuites tous puissants, mais tous ignorants en
Espagne, et pourquoi.}} {\textsc{- Fortune et caractère du chevalier
Bourck.}} {\textsc{- Caractère et fortune du nonce Aldobrandin en
Espagne.}} {\textsc{- Caractère et fortune du colonel Stanhope,
ambassadeur d'Angleterre en Espagne.}} {\textsc{- Bragadino, ambassadeur
de Venise en Espagne.}} {\textsc{- Ambassadeur de Hollande.}} {\textsc{-
Ambassadeurs de Malte traités en sujets en Espagne.}} {\textsc{- Guzman,
envoyé de Portugal.}} {\textsc{- Caractère de Maulevrier.}} {\textsc{-
Duc d'Ormond\,; son caractère, sa situation en Espagne.}} {\textsc{-
Marquis de Rivas, jadis Ubilla\,; sa triste situation en Espagne\,; je
le visite.}}

~

Venons maintenant aux conseils que je trouvai, et que je laissai dans un
grand délabrement, pour ce qui regardait les conseils particuliers.

Le marquis de Miraval était gouverneur du conseil de Castille. C'était
un homme de médiocre naissance, qui avait été ambassadeur d'Espagne en
Hollande, et qui en fut rappelé pour occuper cette grande place dont il
n'était pas incapable. Il était doux, poli, accessible, équitable. Son
esprit toutefois n'était pas transcendant, et son inclination était
autrichienne. La cabale italienne, à laquelle il était étroitement lié,
l'avait porté par la reine à cette grande place. C'était un grand homme,
fort bien fait, qui avait l'attention polie de n'aller presque jamais en
carrosse que ses rideaux à demi tirés pour ne faire arrêter personne.

Don François Camargo, ancien évêque de Pampelune, était inquisiteur
général ou grand inquisiteur. Je n'ai jamais vu homme si maigre ni de
visage si affilé. Il ne manquait point d'esprit\,; il était doux et
modeste. On eût beaucoup gagné que l'inquisition eût été comme lui.

Le comte de Campoflorido était président du conseil des finances, où il
ne faisait rien depuis longtemps\,; une longue maladie le conduisit au
tombeau, depuis mon arrivée en Espagne\,: l'ancien de ce conseil le
gouverna pendant tout mon séjour, avec le trésorier général, desquels je
n'entendis point parler.

La présidence du conseil des Indes et de celui de la marine vaquait
pendant que j'étais en Espagne\,; les doyens obscurs de ces conseils les
conduisaient. La présidence de celui des Indes fut donnée, après mon
départ, au marquis de Valero, à son arrivée de la vice-royauté du Pérou,
avec la grandesse et le titre de duc d'Arion.

Le marquis de Bedmar était président du conseil des ordres et du conseil
de guerre. La première charge était sérieuse, donnait quelque travail,
du crédit et de la considération. L'autre était tombée à n'être plus
qu'un nom.

À l'égard du conseil d'Italie et de celui des Pays-Bas, ils étaient
tombés par le démembrement de ces pays de la domination d'Espagne, et
passés sous celle de l'empereur.

J'ai oublié d'avertir qu'il ne faut pas confondre le Campoflorido, dont
je viens de parler, avec le marquis de Campoflorido, capitaine général
du royaume de Valence, lorsque j'étais en Espagne. Celui-ci était un fin
et adroit Sicilien qui s'était acquis la protection de la reine par le
mariage de son fils avec la fille aînée de dona Laura Piscatori,
nourrice et assafeta de la reine qui, contre tous les usages d'Espagne,
le maintint quinze ou seize ans dans la place de capitaine général du
royaume de Valence qu'il gouverna, en effet, fort sagement. Il en sortit
par être fait grand d'Espagne, et vint après ici ambassadeur d'Espagne,
où chacun a pu juger de son esprit, et qu'il a été peut-être le seul bon
ambassadeur qu'on ait vu ici envoyé par l'Espagne, depuis don Patricio
Laullez.

Il y avait déjà plus d'un règne que les archevêques de Tolède,
chanceliers de Castille par leur siège, en avaient perdu toute fonction
et toute mémoire, et qu'ils étaient réduits au pur ecclésiastique, sans
plus avoir aucune autre prétention. Diego d'Astorga y Cespedes l'était
pendant que j'étais en Espagne. Né en 1666, il fut inquisiteur de
Murcie, évêque de Barcelone en décembre 1715, grand inquisiteur
d'Espagne en 1720, et en mars suivant archevêque de Tolède, en quittant
la place de grand inquisiteur, enfin cardinal par la nomination du roi
d'Espagne en novembre 1727.

On a vu ici ce que j'ai dit de ce prélat et la confiance avec laquelle
il me parla contre la constitution \emph{Unigenitus}, le despotisme des
papes et de l'inquisition en Espagne et dans tous les pays
d'inquisition, qui ne laissaient aucune autorité ni liberté aux évêques,
qu'il faisait trembler, qui étaient réduits aux simples fonctions
manuelles, et qui, bien loin d'oser juger de la foi, n'auraient pas même
hasardé de recevoir la constitution \emph{Unigenitus} sans risquer
d'être envoyés par l'inquisition pieds et poings liés, à Rome, pour
avoir osé se croire en droit de pouvoir donner une approbation à ce qui
émanait de Rome, qu'ils sont obligés de recevoir à genoux, les yeux
fermés, sans s'informer de ce que c'est, si dans cette conjoncture le
pape ne leur avait pas permis et ordonné de la recevoir\,; combien il
déplora avec moi l'anéantissement de l'épiscopat en Espagne et autres
pays d'inquisition, où ce tribunal d'une part, celui du nonce de
l'autre, avaient entièrement dépouillé les évêques, qui n'étaient plus
les ordinaires de leurs diocèses, mais de simples grands vicaires,
sacrés pour le caractère épiscopal et donner la confirmation et
l'ordination et rien de plus, destitués même des pouvoirs que les
évêques des autres pays donnent à leurs grands vicaires\,; enfin combien
il me remontra l'importance extrême que nos évêques ne tombassent pas
dans cet anéantissement, sous lequel ceux d'Espagne et de tous les pays
d'inquisition gémissaient, et combien les nôtres se devaient souvenir de
ce que c'est que d'être évêque, soutenir les droits divins de
l'épiscopat et résister avec toute la sagacité et la fermeté possible
aux ruses et aux violences de Rome, dont le but continuel est d'anéantir
partout l'épiscopat pour rendre les papes évêques seuls et uniques et
ordinaires immédiats de tous les diocèses, pour être les seuls maîtres
dans l'Église\,; et par là de revenir à la domination temporelle qu'ils
ont si longtemps essayé d'exercer partout, et de ne pouvoir enfin y être
contredits par personne de leur communion.

Ce {[}que ce{]} prélat, éclairé et si judicieux, en vénération à toute
l'Espagne par sa modestie, sa frugalité, ses moeurs, ses aumônes, sa vie
retirée et studieuse, sa douceur et son éloignement de toute ambition,
tel que les dignités le vinrent toutes chercher, sans en avoir jamais
brigué aucune, ce que ce prélat, dis-je, crut m'apprendre sur
l'esclavage et le néant de l'épiscopat dans les pays d'inquisition, et
qui met en si grande évidence le cas qu'on doit faire de l'acceptation
faite de la constitution par tous les évêques et les docteurs de ces
pays, que nos boute-feu d'ici ont tant sollicitée et tant fait retentir
pour faire accroire de force et de ruse que l'Église avait parlé, etc.,
cela même on l'a vu dans ce qui a été donné ici de M. Torcy\,; par les
dures réprimandes, et ce qu'il arriva à Aldovrandi, nonce en Espagne,
pour avoir fait accepter la constitution par des évêques, licence prise
par eux, qui fut trouvée si mauvaise à Rome, quoique à la sollicitation
d'Aldovrandi, que ce nonce en fut perdu, et eut toutes les peines qu'on
a vu à s'en relever, et que le pape, pour couvrir cet étrange excès des
évêques d'Espagne, leur commanda à tous de recevoir sa constitution
\emph{Unigenitus}, afin qu'il ne fût pas dit qu'ils eussent osé le faire
sans ses ordres précis\,; et en même temps les évêques, qui l'avaient
acceptée à la réquisition du nonce, furent fort blâmés et menacés de
Rome, comme ceux qui n'avaient osé déférer là-dessus aux instances du
nonce furent loués et approuvés.

Cet archevêque de Tolède est le premier et l'unique prélat à qui
l'Excellence ait été accordée, pour lui et pour les archevêques ses
successeurs. Aucun autre n'a ce traitement, non pas même le nonce du
pape, quoique si puissant en Espagne, et le premier de tous les
ambassadeurs, qui l'ont tous. Les nonces, comme tous les autres
archevêques et évêques d'Espagne, se contentent de la Seigneurie
illustrissime, et ne prétendent point l'Excellence, même depuis que
l'archevêque de Tolède l'a obtenue, fort peu avant que j'arrivasse en
Espagne. C'est aussi la seule distinction qu'il ait par-dessus les
autres archevêques et évêques.

CONSEILLERS D'ÉTAT.

Le duc d'Arcos, le duc de Veragua, le marquis de Bedmar, le comte
d'Aguilar, le prince de Santo-Buono, le duc de Giovenazzo, tous grands
d'Espagne, don Michel Guerra, le marquis Grimaldo, secrétaire d'État.

On l'a déjà dit, les conseillers d'État sont, ou plutôt étaient en
Espagne ce que nous appelons ici ministres d'État. Aussi était-ce le
dernier et le suprême but de la fortune et de la faveur. Mais depuis que
la princesse des Ursins eut fait quitter prise aux cardinaux
Portocarrero et d'Estrées, et à tous ceux qui avaient eu part au
testament de Charles II, qui avaient mis Philippe V sur le trône,
renfermé le roi d'Espagne avec la reine et elle, et changé toute la
forme de la cour et du gouvernement, les fonctions de conseillers d'État
tombèrent tellement en désuétude qu'il ne leur en demeura que le titre
vain et oisif, sans rang ni fonctions quelconques, et sans autre
distinction que de pouvoir aller en chaise à porteurs dans les rues de
Madrid, avec un carrosse à leur suite, et l'Excellence. Aussi fut-ce
uniquement pour donner l'Excellence à Grimaldo qu'il reçut le titre de
conseiller d'État pendant que j'étais à Madrid. Je {[}la{]} lui donnais
souvent avant qu'il l'eût par cette voie. Cela le flattait, parce qu'il
était glorieux et qu'il était peiné de travailler continuellement avec
des ambassadeurs et avec des grands et d'autres qu'il fallait bien qu'il
traitât d'Excellence, et dont il ne recevait que la Seigneurie. Il m'en
reprenait quelquefois en souriant\,; je répondais que je ne me
corrigerais point, parce que je ne pouvais me mettre dans la tête qu'il
ne l'eût pas. Nous reviendrons à lui tout à l'heure. Je passe les
grands, parce que j'en ai parlé sous leurs titres.

Don Michel Guerra était une manière de demi-ecclésiastique sans ordres,
mais qui avait des bénéfices, qui était vieux et qui n'avait jamais été
marié. C'était une des meilleures têtes d'Espagne, pour ne pas dire la
meilleure de tout ce que j'y ai connu\,; instruit, laborieux, parlant
bien et assez franchement. Aussi, quoique tout à fait hors de toutes
places, était-il fort aimé et considéré. Il était chancelier de Milan,
et à Milan lors de l'avènement de Philippe à la couronne d'Espagne\,; il
se conduisit bien dans cette conjoncture. Sa place était également
importante et considérable, et faisait compter les gouverneurs généraux
du Milanais avec elle. Il y était fort estimé et fort autorisé. Peu
après l'avènement de Philippe V à la couronne, il quitta Milan, passa
quelque temps à Paris, fut traité avec beaucoup de distinction par le
roi et les ministres, et fort accueilli des seigneurs principaux.
C'était un homme fort rompu au grand monde et aux affaires, qui ne se
trouva ni ébloui ni embarrassé parmi ce monde nouveau pour lui. Il
repassa d'ici en Espagne, après avoir vu le roi en particulier, et
conféré avec quelques-uns de nos ministres, dont il remporta l'estime et
de toute la cour. Il eut son tour à être gouverneur du conseil de
Castille, mais il ne l'accepta qu'à condition de n'être pas tenu d'en
garder le rang, s'il venait à quitter cette grande place, parce que,
disait-il, il ne prétendait pas mourir d'ennui pour y avoir passé. En
effet, il ne la conserva pas longtemps. Ce n'était pas un homme à ployer
bassement\,; et quand il l'eut quittée, il reprit, en effet, son genre
de vie accoutumé, sans aucun rang et libre dans sa taille, fort visité
et considéré, assez souvent même consulté. Je le voyais assez souvent
chez lui et chez moi. Quoiqu'il n'aimât pas les Français, il
s'entretenait fort familièrement avec moi, et, outre que sa conversation
était gaie et agréable, j'y trouvais toujours de quoi profiter et
m'instruire.

Il avait dans une forte santé une incommodité étrange sa tête se
tournait convulsivement du côté gauche. Dans l'ordinaire cela était
léger, mais presque continuel, par petites saccades. Il était déjà dans
cet état quand il passa à Paris, retournant de Milan en Espagne. Depuis,
cela avait augmenté, et la violence en était quelquefois si grande que
son menton dépassait son épaule, pour quelques instants, plusieurs fois
de suite. Je l'ai vu chez lui, le coude sur sa table\,; tenant sa tête
avec la main pour la contenir, d'autres fois au lit pour la contenir
davantage. Il m'en parlait librement, et cela n'empêchait point la
conversation. Il avait fait inutilement plusieurs remèdes en Italie et
en Espagne, et avait consulté son mal ici. Il n'avait trouvé de
soulagement considérable et long que par les bains de Barège, et il
était sur le point d'y retourner quand je partis d'Espagne.

On admirait à Madrid comment je l'avais pu si bien apprivoiser avec
moi\,; avec tout son agrément et son usage du grand monde, il avait du
rustre naturellement, et les grands emplois par lesquels il avait passé
ne l'en avaient pas corrigé. Ainsi ses propos avaient souvent une nuance
brusque, sans que lui-même le voulût, ni s'en aperçût par l'habitude. Je
sentis bien qu'il ne faisait pas grand cas du gouvernement d'Espagne, ni
beaucoup plus de celui du cardinal Dubois. Ce n'était pas matières même
à effleurer pesamment de part ni d'autre, mais qui ne laissaient pas de
se laisser entendre. Il était frère du confesseur de la reine\,; ils
logeaient ensemble\,; il le méprisait parfaitement. Don Michel était
grand, gros, noir, de fort bonne mine et la physionomie de beaucoup
d'esprit.

Le marquis de Grimaldo, secrétaire d'État des affaires étrangères, était
le seul véritable ministre. Je l'ai fait connaître plus haut par sa
figure singulière et par son caractère. C'était un homme de si peu, et
qui avait si peu de fortune, que le duc de Berwick m'a conté que la
première fois qu'il fut envoyé en Espagne\,; il lui fut présenté pour
être son secrétaire pour l'espagnol\,; qu'il ne le prit point, parce
qu'il ne savait pas un mot de français, et qu'ensuite il entra
sous-commis dans les bureaux d'Orry. Des hasards d'expéditions le firent
connaître et goûter à Orry\,; il en fit son secrétaire particulier, et
il y plut à Orry de plus en plus. Il lui donna sa confiance sur bien des
choses, le fit connaître à M\textsuperscript{me} des Ursins et à la
reine\,; il se servit peu à peu de lui pour l'envoyer porter au roi des
papiers, et en recevoir des ordres sur des affaires, quand ses
occupations lui faisaient ménager son temps. Ces messages se
multiplièrent\,; il avait la princesse des Ursins et la reine pour
lui\,; il fut donc tout à fait au gré du roi, tellement qu'Orry, à qui
son travail avec le roi n'était qu'importun, parce qu'un avec
M\textsuperscript{me} des Ursins, par conséquent maître de l'État, il
n'avait pas besoin de particuliers avec le roi pour soutenir sa
puissance et son autorité particulière, se déchargea de plus en plus de
tout le travail que Grimaldo pouvait faire pour lui avec le roi, et des
suites de ce travail, comme ordres, arrangements, etc., dont Grimaldo
faisait le détail, et lui en, rendait un compte sommaire, ce qui le tira
bientôt de la classe des premiers commis, et en fit une manière de petit
sous-ministre de confiance. Le roi s'y accoutuma si bien que la chute
d'Orry, celle de M\textsuperscript{me} des Ursins, l'ascendant que prit
la nouvelle reine sur son esprit, presque aussitôt qu'elle fut arrivée,
ne purent changer le goût que le roi avait pris pour lui, ni sa
confiance. Albéroni et la reine le chassèrent pourtant de toute affaire
et de toute entrée au palais, mais ils ne purent venir à bout de
l'exiler de Madrid.

Grimaldo, pendant la durée de son petit ministère, s'en était servi pour
se lier avec La Roche, avec les valets intérieurs et pour gagner les
bonnes grâces du duc del Arco et du marquis de Santa-Cruz, amis intimes
l'un de l'autre, l'un favori du roi, l'autre de la reine, et par leur
faveur et leurs charges dans l'intérieur du palais. Il s'était fait
aussi des amis considérables au dehors du palais, bien voulu en général
et mal voulu de personne que d'Albéroni et de ses esclaves. Plus ce
premier ministre se faisait craindre et haïr, plus on souhaitait sa
chute, plus on plaignait le malheur de Grimaldo, plus on s'intéressait à
lui. L'Arco n'avait jamais ployé sous Albéroni d'une seule ligne\,;
Albéroni n'avait pu le gagner, et n'avait osé l'attaquer. Santa-Cruz,
plus en mesure avec lui par rapport à la reine, ne l'en aimait pas
mieux. Il était comme et pourquoi je l'ai dit ailleurs, ami intime du
duc de Liria, auquel Grimaldo s'était attaché dans ses petits
commencements, parce qu'il avait cultivé la protection du duc de
Berwick, dont il avait pensé être secrétaire, et Liria et Grimaldo
furent toujours depuis dans la même liaison dans laquelle Sartine se
glissa. Santa-Cruz et l'Arco faisaient ainsi passer bien des avis de
l'intérieur à Grimaldo par Liria, quelquefois l'Arco par le même ou par
Sartine, et peu à peu il arriva bien des fois que sous quelque prétexte
de quitter la reine quelques moments, ou pendant sa confession, ou entre
le déshabillé du roi et son coucher où il n'y avait jamais que
Santa-Cruz, et l'Arco et deux valets français intérieurs, le roi faisait
entrer Grimaldo par les derrières, conduit par La Roche, et
l'entretenait d'affaires et de bien d'autres choses. La difficulté de le
voir en augmenta le désir, le goût, la confiance, tellement que la chute
d'Albéroni fit le rappel subit de Grimaldo au palais et aux affaires.

Il fut fait secrétaire d'État avec le département des affaires
étrangères, et bientôt après sans être chargé des autres départements
des secrétaires d'État, il travailla seul sur tous avec le roi, à leur
exclusion. Le roi, toujours peiné de multiplier les visages dans son
intérieur, accoutuma bientôt les autres secrétaires d'État et ceux qui
en vacance de président ou de gouverneur des conseils des Indes, des
finances, etc. {[}en faisaient les fonctions{]}, d'envoyer à Grimaldo ce
qu'ils auraient porté eux-mêmes au travail avec le roi, en sorte que
Grimaldo lui rendait compte tout seul de ces différentes affaires de
tous les départements, recevait ses ordres, et les envoyait avec les
papiers à ceux de qui il les avait reçus. On voit par cette mécanique
qu'elle rendait Grimaldo maître, ou peu s'en fallait, de toutes les
affaires, et les autres secrétaires d'État, ou conducteurs à temps des
conseils, impuissants, sans le concours de Grimaldo, par conséquent ne
voyant jamais le roi, et dès là, fort subalternes. De là vint que pas un
d'eux ne suivit plus le roi en ses voyages, qui dans Madrid ne les
voyant jamais où ils étaient tous, et ne travaillant sur les affaires de
tous les départements qu'avec le seul Grimaldo, les accoutuma bientôt à
demeurer à Madrid et à envoyer chaque jour, s'il en était besoin, ou
plusieurs fois la semaine à Grimaldo dans le lieu où le roi était, tout
ce qui avait à passer sous ses yeux, et à recevoir par Grimaldo la
réponse et les ordres du roi sur chaque affaire de chaque département.

Quoique Grimaldo fût glorieux, et qu'une situation si brillante lui fît
élever ses vues bien haut pour ennoblir et élever sa fortune, il eut
grand soin de conserver ses anciens amis, de s'en faire de nouveaux,
d'avoir un accès doux et facile pour tout le monde, d'expédier de façon
que rien ne demeurât en arrière par sa négligence, de tenir ses commis
en règle et assidus au travail, de ne les laisser maîtres de rien, et en
les traitant tous fort bien, d'empêcher qu'aucun prît ascendant sur lui.
Par cette conduite, il fit que tout le monde était content de lui, et
que, dans l'impossibilité d'espérer que le roi sortît jamais de la
prison où M\textsuperscript{me} des Ursins l'avait accoutumé, et
qu'Albéroni avait soigneusement entretenue, et à laquelle ce prince
s'était si fortement accoutumé, il n'y avait personne de la cour ni
d'ailleurs qui n'aimât mieux Grimaldo pour geôlier, et avoir affaire à
lui qu'à tout autre.

À l'égard de ceux dont il portait le travail au roi, à leur exclusion,
il adoucissait cette peine par les manières les plus polies et les plus
considérées. Il ne se mêlait immédiatement d'aucun de leurs
départements, c'est-à-dire qu'il n'écoutait point ceux qui y avait des
affaires\,: c'était à eux à s'en démêler avec les ministres naturels du
département dont étaient leurs affaires\,; et lui, il n'en entendait
parler que par l'envoi que lui faisaient ces ministres des papiers
qu'ils auraient portés devant le roi, et du compte qu'ils lui en
{[}eussent{]} rendu, s'ils eussent travaillé avec Sa Majesté.
Quelquefois alors Grimaldo écoutait ceux que ses affaires regardaient\,;
je dis quelquefois, selon que l'importance de l'affaire le demandait, ou
que la considération des personnes l'exigeait, car d'ordinaire il s'en
tenait à ce que les ministres lui envoyaient, formait son avis
là-dessus, en conformité du leur ou non, mais rapportant toujours au roi
leur avis et sur quoi ils le fondaient, accompagnait le renvoi qu'il
faisait des papiers et de la décision du roi, avec célérité et
politesse. Bien était vrai qu'il prenait plus de connaissances de
certaines affaires, mais ce n'était qu'avec beaucoup de choix pour
suffire à son propre travail, et ne se pas noyer dans celui des autres.
Malgré ces attentions, il était impossible que les autres secrétaires
d'État, etc., ne sentissent le poids de ce joug qui les séparait du roi
comme de simples commis, et qui leur donnait un censeur tête à tête avec
le roi, en lui rapportant toutes leurs affaires. J'expliquerai plus bas
cette façon de travailler, et la jalousie qui en résulta, mais qui fut
impuissante jusque longtemps après mon retour, et qui n'en mit pas les
autres ministres plus à portée du roi, trop accoutumé de si longue main
à ne travailler qu'avec un seul, toujours le même. Je me contente de
rapporter ce que j'ai vu, sans louer ni blâmer ici cette manière de
gouverner une si vaste monarchie.

Grimaldo était chancelier de l'ordre de la Toison d'or, sans en porter
sur soi ni à ses armes aucune marque. Il avait bien envie d'en devenir
chevalier, et il y parvint enfin à la longue. Par lui-même, j'ai eu lieu
de croire qu'il eût été plus modeste, mais il avait une femme qui
pouvait beaucoup sur lui, qui avait de l'esprit, des vues du monde, qui
crevait d'orgueil et d'ambition, qui ne prétendait à rien moins qu'à
voir son mari grand d'Espagne, qui ne cessait de le presser d'user de sa
faveur. Il en avait un fils et une fille fort gentils\,: c'étaient des
enfants de huit ou dix ans qui paraissaient fort bien élevés. Son frère,
l'abbé Grimaldo, fort uni avec eux, l'était parfaitement d'ambition avec
sa belle-soeur, et {[}ils{]} le poussaient de toutes leurs forces. Mais
outre que cette femme était ambitieuse pour son mari, elle était haute
et altière avec le monde, et se faisait haïr par ses airs et ses
manières, et ce fut en effet cela qui le perdit à la fin. L'abbé
Grimaldo imitait un peu sa belle-soeur dans ce dangereux défaut. Il
était craint et considéré, mais point du tout aimé, même de la plupart
des amis de son frère.

J'étais instruit de ces détails, mais des plus intérieurs par le duc de
Liria, et surtout par Sartine véritablement intéressé et attaché à
Grimaldo, et par le chevalier Bourck, dont je parlerai dans la suite. Je
voyais assez souvent M\textsuperscript{me} Grimaldo chez elle et son
beau-frère, et il est vrai qu'à travers la politesse et la bonne
réception, l'orgueil de cette femme transpirait et révoltait, non pas
moi, qui aimais son mari, et qui n'en faisais que rire en moi-même, ou
en dire tout au plus quelque petit mot, et encore rare et mesuré, à
Sartine, ou au duc de Liria. Je pense que ce fut elle qui se servit de
Sartine et de Bourck pour me pressentir sur la grandesse. Je raconte
ceci de suite, quoique après le retour de Lerma à Madrid, et pour sonder
si je voudrais y servir Grimaldo. Rappelé à sa charge de secrétaire
d'État, au moment de la chute d'Albéroni, il avait été témoin de bien
près de la rapidité de la fortune de Riperda devenu comme en un clin
d'œil premier ministre aussi absolu que le fut jamais son prédécesseur
Albéroni, et en même temps grand d'Espagne, dans le premier engouement
de ce beau traité de Vienne\footnote{Le traité, dont parle ici
  Saint-Simon et auquel on donna le nom de traité de Vienne, fut signé
  le 30 avril et le 2 mai 1725. La France et l'Angleterre, inquiètes du
  rapprochement de l'Autriche et de l'Espagne, opposèrent à cette
  alliance le traité de Hanovre (25 septembre 1725), qui réunissait
  l'Angleterre, la France et la Prusse.} qu'il y avait conclu\,: fruit
amer du renvoi de l'infante en Espagne.

Riperda\,; gentilhomme hollandais, et ambassadeur de Hollande en
Espagne, à qui il s'était attaché depuis son rappel et dont il a été
tant parlé ici, d'après Torcy, était étranger à l'Espagne, devenu une
espèce d'aventurier. Grimaldo qui, en jouant sur le mot et de sa
terminaison en \emph{o} ou en \emph{i}, avait franchement arboré les
armes pleines de Grimaldi, se prétendait être de cette maison, depuis
qu'il était secrétaire d'État, par conséquent de bien meilleure maison
que Riperda. Il n'y avait aucun Grimaldi en Espagne pour lui contester
cette prétention. Le règne de Riperda avait été court, et sa chute bien
méritée, mais affreuse. Sa gestion, à la suite de celle d'Albéroni,
avait dégoûté le roi et la reine des premiers ministres, sans les
détacher de ne travailler qu'avec un seul ministre, et ce seul ministre
fut encore Grimaldo. Il succéda donc à Riperda, non au titre ni au
pouvoir, mais au moins à l'accès unique, et à rapporter seul au roi les
affaires de tous les départements, comme il avait fait auparavant. C'en
était bien assez pour mettre la grandesse dans la tête de sa femme et de
son frère, et pour le tenter lui-même, quoique plus sage et plus
clairvoyant qu'eux.

Pour revenir sur mes pas à mon temps, le servir dans cette ambition,
n'avait rien de contraire au service ni à l'intérêt de la France\,:
c'était, au contraire, lui attacher de plus en plus l'unique ministre
qui approchât du roi et de la reine d'Espagne, et qui avait toujours
bien mérité de la France. Ces raisons et mon inclination m'y portaient
par tout ce que je devais, comme on verra bientôt, à l'amitié de
Grimaldo\,; mais je sentais aussi combien je devais éviter de me mêler
des choses purement intérieures de la cour d'Espagne, et quoique pour
l'importance et la conduite des affaires, les ministères et les dignités
n'aient rien de commun, et soient choses entièrement séparées, je
m'étais fait là-dessus à moi-même une leçon générale, quand je refusai
au P. Daubenton d'entrer dans ses vues et dans ce que le roi d'Espagne
voudrait faire par mon ministère pour faire rendre aux jésuites le
confessionnal du roi. Il était néanmoins plus que délicat d'éconduire
Sartine et Bourck sur une proposition que je sentais bien qu'ils ne me
faisaient pas d'eux-mêmes. Je pris donc le parti de leur montrer que je
la goûtais, que je me prêterais avec empressement à procurer cette
élévation à Grimaldo\,; mais tant pour lui-même que pour moi, il s'y
fallait garder de faux pas, et que c'était à lui à me conduire dans un
terrain qu'il connaissait si bien, et dont l'écorce m'était à peine
connue. Par cette réponse qui me vint sur-le-champ dans l'esprit,
j'espérai des mesures de Grimaldo, de sa crainte de se perdre en voulant
voler trop haut, de son embarras à se servir d'un étranger qui, quelque
bien qu'il fût et qu'il parût auprès de Leurs Majestés Catholiques, ne
les voyait pourtant jamais seul que par audiences, dont les occasions
désormais ne pouvaient être fréquentes, tiendraient Grimaldo en des
délais continuels qui me feraient gagner le temps de mon départ, et ne
me concilieraient pas moins sa reconnaissance de mes offres et de ma
bonne volonté. En effet, tout cela arriva comme je l'avoir prévu.

Le marquis de Castellar, secrétaire d'État de la guerre, était un grand
homme fort bien fait, avec un oeil pourtant un peu en campagne, et
jeune. Il était frère de Patino, qui était alors intendant de marine à
Cadix, qui ne vint point à Madrid de mon temps, et qui longtemps après
devint premier ministre avec plus de pouvoir qu'aucun autre, qui l'eût
été, qui se fit, à la fin, grand d'Espagne, et qui mourut dans toute
cette autorité. Il a été parlé de lui ici plus d'une fois. Ils étaient
Espagnols d'assez bon lieu, établis à Milan depuis quelques générations,
et revenus enfin en Espagne. Patino avait été jésuite. Lui et son frère
se haïssaient parfaitement, et se sont haïs toute leur vie.

Castellar aimait fort son plaisir, paraissait très rarement à la cour,
était autant qu'il pouvait dans le monde, fort paresseux avec de
l'esprit, de la capacité, une grande facilité de travail, qui expédiait
en deux heures avec justesse plus qu'un autre en sept ou huit heures. Il
portait avec la dernière impatience d'envoyer ses papiers à Grimaldo, et
de n'en recevoir que par lui les réponses et les ordres du roi.
Toutefois il fit tant qu'il parvint pendant que j'étais à Madrid, à
travailler avec le roi deux fois assez près à près, et cela fit nouvelle
et mouvement dans la cour. Grimaldo ne s'en émut pas, et il eut raison.
Castellar ne put se contenir de témoigner au roi que tout se perdait par
cette façon de faire passer toutes leurs affaires par Grimaldo, et de ne
travailler qu'avec lui. Cette représentation peut-être trop forte, et
qui put aussi être un peu aigre, déplut au roi, qui depuis ne voulut
plus travailler avec lui, et il en arriva autant à celui qui était par
\emph{interim} en premier aux finances, qu'au premier travail de
Castellar avec le roi, il y avait poullié. Ainsi Grimaldo, sans se
remuer le moins du monde, continua tranquillement à faire seul avec le
roi la besogne de tous.

Ce mauvais succès de Castellar acheva de le piquer. Sa femme n'était pas
moins haute que celle de Grimaldo, et personnellement {[}elles{]} ne se
pouvaient souffrir l'une l'autre. Le feu s'alluma donc tout à fait entre
elles et entre leurs maris. Castellar se lâcha indiscrètement sur
Grimaldo, qu'il força, malgré lui, à se fâcher. Cela fit du bruit et des
partis, mais celui de Castellar n'était rien en comparaison de celui de
Grimaldo, qui avait pour lui la faveur et la confiance privative de
toutes les affaires.

Castellar me voyait assez, sa conversation était fort agréable. On me
voyait bien avec lui et beaucoup mieux encore avec Grimaldo, et sur un
pied d'amitié et de confiance. Leurs amis me pressèrent de travailler à
les raccommoder, Sartine, Bourck, les ducs de Liria et de Veragua, le
prince de Masseran et d'autres. C'était une bonne oeuvre qui ne pouvait
qu'être bonne au service du roi et utile à tous les deux. J'aurais
réussi, si je n'avais eu affaire qu'aux deux maris, mais les deux femmes
qui voulaient se manger et périr ou culbuter le secrétaire d'État
opposé, se mirent tellement à la traverse que je m'aperçus bientôt que
je n'y gagnerais rien que de me mettre peut-être mal avec l'un ou
l'autre, tellement que je me retirai doucement de cette entremise, sans
y laisser rien du mien.

Quand ils se furent bien aboyés, ils se turent, mais ne se pardonnèrent
pas. De ce moment Castellar, à qui sa place devenait tous les jours plus
insupportable, mais qui ne pouvait la quitter pour demeurer rien, tourna
toutes ses vues sur l'ambassade de France, et m'en parla plusieurs fois.
Je lui représentai toujours que pour mon particulier, rien ne me pouvait
être plus agréable, mais qu'il prit garde à quitter le réel qu'il
tenait, et qui le pouvait devenir davantage, et plus agréable par des
choses que le temps amenait, et qu'on ne pouvait prévoir, ce que
j'accompagnais de choses flatteuses sur son mérite, sa capacité, sa
réputation, et en tout cela je lui disais vrai, et je l'entretins
toujours de la sorte sans entrer en aucun engagement\,: c'est que je
sentais combien cette ambassade serait désagréable à Grimaldo, que par
toute raison j'aimais mieux que l'autre, et que je voyais bien aussi que
la correspondance étroite, si désirable entre les deux cours, courrait
risque d'être mal servie entre un ambassadeur d'Espagne et le ministre
unique d'Espagne, et spécialement des affaires étrangères, aussi ennemis
l'un de l'autre -que l'étaient ces deux hommes.

Castellar enfin y réussit, mais longtemps après, et eut entre deux une
attaque d'apoplexie qui, d'un homme gai, léger, de la conversation la
plus fine, la plus leste, la plus aimable, mais aussi la plus solide et
la plus suivie quand cela était à propos, en fit un homme triste, pesant
jusqu'à en être lourd et massif, qui ne produisait rien, qui ne suivait
pas, qui travaillait même pour comprendre. Je m'étais fait un grand
plaisir de le revoir ici ambassadeur. À son premier aspect ma surprise
fut grande, et mon étonnement encore plus dès la première conversation.
C'était une apoplexie ambulante\,: aussi le tua-t-elle bientôt.

Il mourut à Paris, et laissa un fils à qui son oncle fit épouser
l'héritière d'une grandesse. Il était fort jeune et fort fou, du temps
que j'étais en Espagne. Il s'est depuis appliqué au service, il y a
acquis de la réputation\,; il s'est soutenu après la mort de son
oncle\,; dont il a eu aussi la grandesse. Il trouva le moyen de
s'attirer la protection de la reine\,; il eut des commandements en chef
qui l'ont conduit à être capitaine général.

J'ai parlé de La Roche et du P. Daubenton assez pour n'avoir rien à y
ajouter\,: seulement dirai-je que ce maître jésuite vieillissait et
qu'il commençait à perdre la mémoire. Je m'en aperçus dans les
conversations fréquentes que j'avais avec lui chez moi, ou au collège
impérial où il était fort bien logé. Mais cette faiblesse de mémoire me
fit découvrir plus d'une friponnerie de sa part, par lui-même, sur des
affaires où d'abord il m'avait promis merveilles, et dès le lendemain me
venait conter celles qu'il avait opérées là-dessus avec le roi, puis
quelques jours après me disait tout le contraire, oubliant ce qu'il
m'avait raconté. C'est que ce qu'il m'avait dit d'abord était une fable,
et ce qu'il me rendait après était ce qu'il avait exécuté. Je n'en fus
ni surpris ni n'en fis pas semblant. Je connaissais trop le personnage
pour m'y fier en rien, mais je ne fus pas fâché de jouir du défaut de sa
mémoire, et de m'amuser à lui en tendre des panneaux.

Mais ce qui m'importuna de lui à l'excès, fut sa jalousie du P.
d'Aubrusselle, jésuite français, demeurant aussi au collège impérial et
précepteur des infants. C'était un homme d'esprit, de savoir, fort
instruit des choses d'Espagne et de l'intérieur du palais, aimé et
estimé généralement, et d'une conversation agréable, sage, discrète,
mais toutefois instructive. Aubenton qui craignait toujours pour sa
place, et pour la confiance et l'autorité qu'elle lui donnait, se
sentait vieux et connu. L'expérience qu'il avait faite de pouvoir être
congédié, le rendait soupçonneux sur tous ceux qui lui pouvaient
succéder. Il voyait bien qu'Aubrusselle était le plus apparent et le
plus naturel\,; la bienveillance générale et la réputation qu'il avait
acquise en Espagne le blessait\,; tout lui était suspect de ce côté-là,
à tel point qu'Aubrusselle m'en avertit, me pria d'éloigner mes visites,
surtout de n'aller point chez lui les jours que j'irais voir Aubenton,
et de ne trouver pas mauvais qu'il vint peu chez moi. Je m'informai
d'ailleurs de cette jalousie, et par ce que j'en appris, je vis que le
P. d'Aubrusselle ne m'en avait pas tout dit. Il craignait encore ses
relations en France, et même à Rome, quelque vendu qu'il fût à cette
dernière cour. En un mot, tout lui faisait ombrage, et plus sa tête
vieillissait, moins il était capable de se contenir là-dessus, sans
succomber à des échappées, quelque seconde nature qu'il se fût faite de
la dissimulation la plus profonde et de la plus naturelle fausseté. Cela
fit qu'Aubrusselle et moi eûmes moins de commerce ensemble que lui et
moi n'eussions voulu.

Puisque je parle de jésuites, il faut achever ici ce qui les regarde. Je
ne les trouvai pas en Espagne moins puissants qu'ils se le sont rendus
partout ailleurs, pénétrant partout, imposant partout, et d'amour ou de
crainte se mêlant de tout. Les dominicains autrefois si puissants en
Espagne y étaient devenus de petits compagnons auprès d'eux, et dans
l'inquisition même, où les jésuites s'étaient saisis de la pluralité des
places, et des plus importantes. Mais quels pays que ceux
d'inquisition\,! Les jésuites savants partout et en tout genre de
science, ce qui ne leur est pas même disputé par leurs ennemis, les
jésuites, dis-je, sont ignorants en Espagne, mais d'une ignorance à
surprendre. Ce sont les PP. Daubenton et d'Aubrusselle qui me l'ont dit,
et plusieurs fois, qui ne pouvaient s'accoutumer en ce qu'ils en
voyaient. C'est que l'inquisition furette tout, s'alarme de tout, sévit
sur tout avec la dernière attention et cruauté. Elle éteint toute
instruction, tout fruit d'étude, toute liberté d'esprit, la plus
religieuse même et la plus mesurée. Elle veut régner et dominer sur les
esprits, elle veut régner et dominer sans mesure, encore moins sans
contradiction, et sans même de plaintes, elle veut une obéissance
aveugle sans oser réfléchir ni raisonner sur rien, par conséquent elle
abhorre toute lumière, toute science, tout usage de son esprit\,; elle
ne veut que l'ignorance, et l'ignorance la plus grossière. La stupidité
dans les chrétiens est sa qualité favorite, et celle qu'elle s'applique
le plus soigneusement d'établir partout, comme la plus sûre voie du
salut, la plus essentielle, parce qu'elle est le fondement le plus
solide de son règne et de la tranquillité de sa domination.

Le chevalier Bourck était un gentilhomme Irlandais, qui avait été
quelque temps au cardinal de Bouillon, à Rome, et qui n'aimait pas qu'on
le sût, car il était pauvre, glorieux et important. Son maître qui ne
pouvait tenir dans sa peau, et qui toujours était plein d'un inonde de
vues obliques et folles, lui reconnut de l'esprit et un esprit de manège
et d'intrigue qui, en effet, étoient le centre et la vie de Bourck, et
l'employa à des messages et à de petites négociations dans Rome et au
dehors. Il fut chargé d'une autre vers les princes d'Italie, que le
cardinal de Bouillon avait imaginée pour leur faire agréer une
augmentation de cérémonial en faveur des cardinaux. Bourck, domestique
pour son pain, parce qu'il n'en avait pas, mais blessé de l'être, tira
sur le temps, et sur la faiblesse de son maître, pour lui persuader
qu'il réussirait beaucoup mieux s'il était l'homme du sacré collège,
dont le nom imposerait bien plus aux princes avec qui il traiterait, que
s'il n'agissait qu'au nom d'un cardinal particulier, quelque
considérable qu'il fût. Bouillon, fanatique d'orgueil en tout genre, qui
s'était mis en tête cette augmentation de cérémonial, et pour le succès
duquel tout lui était bon, goûta la proposition, et obtint de la
complaisance des cardinaux, de charger Bourck de cette négociation en
leur nom, mais toutefois sans se commettre au cas qu'elle ne réussît
pas.

Ce point gagné, Bourck {[}fut{]} admis chez les principaux cardinaux,
pour recevoir leurs ordres, et voir avec eux les moyens d'agir en leur
nom, mais d'une manière secrète, et qui ne les commît point s'il ne
réussissait pas. Comme presque tous se doutaient bien qu'il échouerait,
et ne s'étaient laissé aller que par la faiblesse pour l'impétuosité du
cardinal de Bouillon qui, dans la plus haute faveur du roi, était chargé
de ses affaires à Rome, et y faisait un personnage principal, et le
premier par la splendeur de sa magnificence, Bourck, dis-je, leur
insinua que l'homme chargé par le sacré collège ne pouvait avec décence,
pour ce grand corps, être payé que par lui, et qu'il serait trop
indécent que ce même homme pût être reconnu par les princes avec qui il
traiterait pour être domestique d'un cardinal particulier. Avec cette
adresse, il se tira de sa condition, sans perdre les bonnes grâces de
son maître, et tira du sacré collège plus qu'il ne tirait du cardinal de
Bouillon.

Le voilà donc à Parme, à Modène sans éclat et sourdement\,; la
négociation traîna le plus longtemps qu'il put. Elle eût fini d'abord,
car ces princes se moquèrent de ses propositions au premier mot qu'il
leur en dit, mais Bourck voulait se faire valoir et faire durer la
commission. Elle échoua enfin, et il eut encore l'adresse de se faire
donner une petite pension par le sacré collège, dont il a toujours joui,
pour le récompenser tant de ses peines et, de ses dépenses prétendues,
que pour le dédommager de ce qu'il perdait à n'être plus au cardinal de
Bouillon. Je n'entreprendrai pas de le suivre, il me mènerait trop loin.
Je me contenterai de dire qu'il fit plusieurs voyages par inquiétude
d'esprit, et peut-être moins pour chercher fortune que chercher à se
mêler\,: car se mêler, négocier, intriguer, était son élément et sa vie.

À la fin il se fixa en Espagne, où il fut assez bien voulu de la
princesse des Ursins, dont il avait fréquenté les antichambres à Rome, à
la mode du pays. Elle lui confia même plusieurs choses, et le mit tout à
fait bien auprès du roi et de la reine qui lui parlaient souvent
familièrement, en particulier, et lui, à l'en croire, leur donnait
souvent de fort bons conseils, et à M\textsuperscript{me} des Ursins, et
leur parlait fort hardiment. Cette posture, et un naturel vif,
entreprenant, haut, souvent même audacieux et très libre, soutenu
d'esprit et de connaissances, le faisait ménager, mais craindre par les
ministres, et le mêla fort avec le monde et avec la cour où il s'était
fait des amis. L'arrivée d'une nouvelle reine, et la chute subite de
M\textsuperscript{me} des Ursins diminua fort ses accès et sa
considération. Néanmoins il se soutint, et ne laissa pas d'être encore
de quelque chose sous Albéroni. C'était un homme qui ne s'abandonnait
point, et qui savait toujours s'introduire par quelque coin. Il avait
toujours ménagé Grimaldo, en sorte qu'après le ministère d'Albéroni, il
espéra tout de la protection de Grimaldo. Mais Grimaldo, qui le
connaissait, le traita toujours avec une distinction qui l'empêcha de
s'écarter de lui, mais qui le tint toujours en panne, parce qu'en effet
ce ministre craignait son caractère, et profita de l'éloignement que la
reine avait pris de lui pour l'empêcher de se rapprocher d'elle et du
roi.

C'est dans cette situation que je le trouvai en arrivant à Madrid. On me
l'avait donné pour un homme fort attaché à la France, et dont je
pourrais tirer beaucoup de lumières. J'en tirai en effet, mais souvent
aussi bien des visions. Il était ami de plusieurs personnes distinguées,
le pays et le jacobisme l'avaient lié avec le duc de Liria, Hyghens, le
duc d'Ormond, et plusieurs autres. Il était aussi ami de Sartine, mais
tous connaissaient bien son caractère. Il était en effet fort instruit
d'événements intérieurs du palais fort curieux, et de beaucoup de choses
et d'affaires où il était entré, et d'autres où il, s'était fourré. Il
parlait bien, mais beaucoup, on pouvait dire qu'il était malade de
politique. Il y revenait toujours de quelque extrémité opposée que se
trouvât la conversation. Il possédait seul, à son avis, tous les
intérêts des différentes grandes et médiocres puissances de l'Europe, et
il en accablait sans cesse ceux qu'il fréquentait, avec un ton
d'autorité de ministre en place. Je ne laissai pas d'en tirer assez de
bonnes choses, et de m'en amuser d'ailleurs. Je dois dire aussi que je
n'en ai vu ni ouï dire rien de mauvais. Il n'était point intéressé
d'argent, et a passé toujours pour honnête homme.

Désespéré de ne pouvoir rattraper d'accès auprès du roi et de la reine,
il tourna ses pensées vers l'ambassade d'Espagne à Turin. De son premier
état à y représenter le roi d'Espagne il y avait un peu loin\,; mais on
n'épluche pas toujours ce que les ambassadeurs ont été, et je crois
qu'il se serait utilement acquitté de cette ambassade délicate. Il me
pria fort de m'y employer. J'en parlai à Grimaldo, qui me répondit en
ministre fort rompu au métier. Quoiqu'il n'oubliât rien pour me marquer
son empressement à servir Bourck, et qu'il me pressât même de tâcher de
pressentir le roi et la reine sur lui en général, sans néanmoins rien
particulariser, je sentis bien qu'il n'avait aucune envie d'employer
Bourck, ni de le mettre en aucune passe. Son caractère ferme, impérieux,
libre, arrêté à son sens, avait fait peur à tous les ministres, à ceux
même dans la confiance de qui il était entré, et qui tous le craignirent
et jugèrent le devoir écarter de tout pour n'avoir point à compter avec
lui. C'est aussi ce qui arriva en cette occasion. Je trouvai moyen de
parler de Bourck dans une audience. Comme j`évitais de traiter toute
affaire qui aurait pu me retenir en Espagne plus que je n'aurais voulu,
ces audiences se tournaient bientôt en conversation. Je reconnus de
l'éloignement dans le roi pour Bourck, et un air de secrète moquerie
dans la reine. Il ne m'en fallut pas davantage pour m'arrêter sur un
homme en faveur duquel rien ne m'engageait à me prodiguer, et auquel je
voyais tout contraire. Je rendis faiblement à Grimaldo ce qui s'était
passé là-dessus, qui sourit et n'en parut ni fâché ni surpris. À Bourck,
je ne lui dis que des choses générales, et je me gardai bien d'en
reparler depuis.

Il se lassa enfin de vains projets et d'espérances aussi vaines. Il
quitta l'Espagne peu après mon retour, et s'en vint à Paris où je le vis
assez souvent, et où il ne put s'agripper à rien. Sept ou huit mois le
lassèrent. Il s'en alla mourir à Rome entre le roi Jacques et la
princesse des Ursins, dans un âge fort avancé, après y être demeuré
quelques années à y tracasser comme il put. J'ai parlé ailleurs des
malheurs singuliers de sa famille.

Il faut dire aussi un mot des ministres étrangers qui étaient lors à
Madrid. Le nonce Aldobrandin, jeune, grand, fort bien fait, montrait un
prélat romain, c'est-à-dire un ecclésiastique qui ne l'est que pour la
fortune, sans néanmoins rien d'indécent. Il était gai, vif, plaisant,
ouvert, avec de l'esprit et beaucoup de monde, fort à travers du
meilleur de Madrid et des dames, l'air galant, familier avec le roi et
la reine, et n'aimant point du tout les Français, mais m'accablant de
recherches et de politesses. J'y répondais avec grande attention, sans
aller une ligne au delà, et je le charmais sans le convertir en lui
parlant souvent de ce que la France devait à la mémoire de Clément VIII,
et de la gloire et de la sagesse de son pontificat. Il fut cardinal au
sortir de sa nonciature, un peu plus tôt qu'il n'aurait voulu, car elle
lui valait fort gros, et il était pauvre. Quoiqu'il eût l'air fort sain,
il ne jouit pas longtemps de sa pourpre, et la France ni l'Espagne n'y
perdirent rien.

Le colonel Stanhope était ambassadeur d'Angleterre. C'est le même qui y
était depuis longtemps, en deux fois, et dont il a été tant parlé ici
dans ce qui est donné de M. de Torcy. C'était parfaitement un Anglais.
Savant et amoureux de ses livres et de l'étude des sciences abstraites,
versé dans l'histoire, fort au fait des intérêts de sa nation et des
détails de sa cour et du parlement d'Angleterre, parlant bien les
langues, sérieux, parlant peu, sans cesse aux écoutes, instruit à fond
de la cour du pays, du commerce, des intérêts généraux et particuliers
de la nation chez qui il résidait, avec cela peu répandu, aimant la
solitude, naturellement triste, rêveur, réfléchissant, une maison
honnête, une bonne table assez peu et assez mal fréquentée, poli mais
froid, fermé et je ne sais quoi de repoussant, occupé à pomper et à
parler sans rien dire, et ne laissant pas de trouver ses plaisirs au
fond ténébreux de son appartement, mais secrètement autant qu'il était
possible, et sans indécence, et ne sortant de chez lui que par raison et
point du tout par goût.

J'avais des ordres très exprès et très réitérés de le voir souvent et
avec confiance. J'en fis assez pour éviter tout reproche\,; mais j'usai
de sobriété avec un homme dont le goût particulier et de solitude m'en
offrait le moyen, et pour la confiance je m'en tins à l'écorce. C'était
un homme de beaucoup d'esprit, de conduite, de sens, mais tout en
dedans, sans rien qui attirât à lui. D'ailleurs je ne fus jamais affolé
de l'Angleterre\,; j'en laissais l'enthousiasme au cardinal Dubois, qui
le porta où il avait prétendu et qui le maintint où il était arrivé.

Stanhope avait ramassé je ne sais où un prêtre italien qu'on appelait
l'abbé Tito-Livio, qui se fourrait partout, ramassait tout, intriguait
partout. C'était un drôle d'esprit, de savoir, de fort bonne compagnie,
plaisant même avec sel et jugement, dangereux au dernier point. Il était
reçu en beaucoup d'endroits où il amusait, mais il était craint, et au
fond méprisé comme un espion qu'il était, et fort débauché. Il tâcha
fort de s'introduire chez moi, mais inutilement, sans toutefois rien qui
pût être trouvé mauvais par Stanhope. Cet ambassadeur demeura encore
longtemps en Espagne, figura depuis dans les charges et le ministère
d'Angleterre, et finit par la vice-royauté d'Irlande.

Bragadino, d'une des premières maisons de Venise, et ce n'est pas peu
dire, était ambassadeur de cette république. Lui et sa femme étaient de
fort aimables gens et d'un fort bon commerce.

L'ambassadeur de Hollande mangeait son pain et son fromage dans sa
poche. C'était un homme qu'on ne voyait et qu'on ne rencontrait jamais.

L'ambassadeur de Malte était un chevalier espagnol, qui, avec le
caractère et les immunités d'ambassadeur, ne jouissait d'aucun des
honneurs de la cour qui y sont attachés, parce que Malte a été donnée à
la Religion\footnote{C'est-à-dire à l'ordre religieux de Malte.} comme
un fief de Sicile dont les rois d'Espagne avaient toujours été en
possession, quoique alors Philippe V n'y fût plus. J'ai vu cet
ambassadeur avoir une audience en cérémonie, en présence de tous les
grands avertis, et moi comme les autres, car les ambassadeurs ne se
trouvent point à ces fonctions, le roi debout, sous son dais, couvert,
les grands couverts, appuyés à la muraille, les gens de qualité
vis-à-vis, découverts. L'ambassadeur de Malte ne se couvrit point,
complimenta le roi d'Espagne, et lui présenta de fort beaux faucons de
la part du grand maître \emph{et} de la Religion. Comme c'était une
espèce d'hommage, je m'informai si cet ambassadeur ne se couvrait point
en arrivant en sa première audience de cérémonie. Il me fut répondu que
non, et qu'elles se passaient toutes comme celle que je voyais, excepté
les faucons. Ce qui me surprit le plus, c'est que les grands ne se
découvrirent pas un seul moment, et il se retira comme il était entré,
le roi et tous les grands présents et couverts.

Un Guzman était envoyé de Portugal qui voyait fort le monde, vivait fort
noblement et se faisait aimer et estimer. Il me donna un grand,
magnifique et excellent repas la veille de mon départ, avec toute sorte
d'aisance et de politesse.

Après avoir différé, et parlé de tous les ministres étrangers, il faut
enfin venir à M. de Maulevrier. De ma vie je ne Pavais vu qu'à Madrid,
ni n'avais eu occasion de rien direct ni indirect à son égard, ni avec
personne qui lui touchât en rien. Le seul des siens que j'avais vu et
connu était l'abbé de Maulevrier, son oncle, aumônier du feu roi, dont
il a été parlé ici quelquefois, et avec lequel j'avais toujours été fort
bien. J'ignore donc en quoi je pus déplaire à un homme entièrement
inconnu, et qui sans mon consentement n'aurait pas eu l'honneur de
recevoir le caractère d'ambassadeur du roi. Dès Paris, je savais qu'il
avait trouvé fort mauvais que je vinsse en Espagne, et comme je l'ai
déjà dit, qu'on n'eût pas choisi le duc de Villeroy ou La Feuillade. Je
résolus d'ignorer cette impertinence, et de vivre avec lui comme si
j'eusse été content de lui. Je trouvai un homme fort respectueux, fort
silencieux, fort réservé, et je m'aperçus bientôt qu'il n'y avait rien
dans cette épaisse bouteille que de l'humeur, de la grossièreté et des
sottises. Je ne sais où l'abbé Dubois avait pris un animal si mal
peigné.

Il l'avait fait accompagner par un marchand, devenu petit financier, qui
s'appelait Robin, et qui en portait tout à fait la mine. C'était pour le
diriger dans les affaires du commerce, mais il se trouva qu'il le
dirigeait dans toutes, et que sans Robin aucune n'eût marché. Aussi
Robin, qui avait de l'esprit et du sens, ayant envie d'être dépêché au
roi pour lui porter son contrat de mariage, je n'osai priver Maulevrier
de son mentor, quoiqu'ils m'en priassent tous deux. Je me contentai de
mander le refus au cardinal Dubois sans m'expliquer de la raison. Le
cardinal ne fut pas si réservé dans sa réponse à cet article. Il me
remercia de l'avoir refusé, et ajouta plaisamment que Robin était
l'Apollon sans lequel Maulevrier ne pouvait faire des vers. Peu de jours
après mon arrivée, je l'allai voir en cérémonie. Je ne sais si ce fut
ignorance ou panneau, il voulut donner la main à mes enfants. Je m'en
aperçus assez tôt pour l'empêcher.

Sa bêtise l'avait mis à merveille avec Grimaldo, parce que sans autre
façon, il lui montrait toutes les dépêches qu'il recevait de la cour.
Rien n'était plus commode au ministre d'Espagne. J'en avertis le
cardinal Dubois, mais sans aucun commentaire, qui me manda qu'il n'était
pas à le savoir, et que tout le remède qu'il y avait trouvé, c'était
d'être fort attentif à ne rien écrire à Maulevrier que Grimaldo ne pût
voir.

J'ai expliqué ailleurs la conduite qu'il eut avec moi à la signature du
contrat de mariage. Si je m'amusais à marquer toutes ses sottises, je
serais bien long et bien ennuyeux. Malgré tout cela, je lui montrai
toujours le même visage, et à son caractère les mômes égards. Il venait
presque tous les jours chez moi le plus librement du monde et très
souvent dîner, fort souvent aussi au palais ensemble. Le monde qui avait
ou vu ou su ce qui s'était passé à la signature du contrat de mariage,
et qui le haïssait et le méprisait, admirait ou mon tranquille mépris ou
ma patience. Comme j'avais résolu de ne me point fâcher, et surtout de
ne point divertir le monde à nos dépens, je tournais toujours ce qu'on
me disait de lui en plaisanterie, et disais qu'il était le meilleur
homme du monde.

Sa grossièreté, son humeur et sa bêtise lui avaient acquis une haine peu
commune et générale. Il me voyait personne, et disait franchement au
palais, à tous ces seigneurs, qu'il aimait mieux être tout seul que voir
des Espagnols. Cette brutalité qu'ils m'ont tous rapportée, qu'il leur
répétait souvent, est inconcevable. Il blâmait devant eux leurs moeurs,
leurs coutumes, leurs manières, leur disait qu'elles étaient ridicules,
n'en approuvait aucune, et même ce qu'il y avait de plus beau, édifices,
fêtes, etc., il le trouvait vilain, et se plaisait à le leur dire,
jusque-là qu'il n'avait pas honte de leur témoigner nettement et souvent
qu'il ne pouvait souffrir l'Espagne ni les Espagnols. La plupart des
seigneurs lui tournaient le dos au palais\,: je l'y trouvais isolé, seul
au milieu de la cour.

Quoique ces brutalités me revinssent de toutes parts, je les aurais
crues exagérées, sans une des plus fortes dont je fus témoin et bien
honteux. C'était à Lerma, la veille du mariage, et la première fois que
je fis la révérence au roi et à la reine après ma petite vérole.
J'attendais, pour avoir cet honneur, dans une petite pièce devant leur
appartement intérieur avec Maulevrier et cinq ou six grands d'Espagne,
avec lesquels je causais. Un homme était dans la même pièce, au haut
d'une fort longue échelle, qui rattachait une tapisserie. Tout d'un coup
voilà Maulevrier qui se met à dire en faisant la grimace\,:
«\,Voyez-vous cet animal là-haut, combien il est maladroit\,; aussi
est-ce un Espagnol.\,» Et tout de suite à dire des injures. Moi, bien
étonné, à rompre les chiens, et ces seigneurs à me regarder. Pour tout
cela, Maulevrier ne démordit point. «\,B\ldots. d'Espagnol, dit-il, je
voudrais te voir tomber de là-haut pour ta peine, et te rompre le cou\,;
tu le mériterais bien, j'en donnerais deux pistoles.\,» Véritablement je
fus si effarouché, que je n'eus pas le mot à dire pour détourner ces
beaux propos\,: «\,Eh le sot b\ldots.. d'Espagnol\,! Eh le sot\,! eh le
maladroit\,! mais voyez donc comme il est gauche.\,» J'écoutai tout
comme ne sachant plus ce que j'entendais ni où j'étais. Ces seigneurs, à
force d'excès, s'en mirent à rire et à me dire\,: «\,M. le marquis de
Maulevrier nous loue toujours. J'eusse voulu être en mon village. Ce mot
n'arrêta point Maulevrier\,; il soutint son dire. Enfin je fus appelé
pour entrer où étaient le roi et la reine. Je pense qu'après les avoir
quittés, ces seigneurs ne tinrent pas longue compagnie à cet ambassadeur
si bien appris\,; outre qu'avec la haine, cette rusticité lui concilia
le mépris, et sa vie mesquine en table nulle, et en équipages pauvres et
courts, l'acheva. Il me donna pourtant une fois et même deux un assez
grand et bon repas.

Il s'en fallait bien que je me crusse à portée de lui parler d'adoucir
et de modérer ses manières. Quelque peu d'intérêt que je prisse en lui,
je ne pouvais me détacher de celui de la nation et de ce déshonneur du
choix d'un pareil ministre. Je n'en parlai point non plus à son
conducteur Robin, que je jugeai bien qui sentait les mêmes choses, et
qu'il ne pouvait retenir cette étrange humeur. J'ignore quel mérite il
avait à la guerre, ni comment il ensorcela M. le prince de Conti de se
piquer d'honneur d'arracher pour lui un bâton de maréchal de France. Ce
que je sais, c'est, que ce fut à l'étonnement général, pour n'en pas
dire davantage.

Le duc d'Ormond était à Madrid sur un grand pied de considération de
tout le monde et des ministres. Il en était fort visité et tenait une
table abondante et délicate, où il y avait toujours quelques seigneurs
et beaucoup d'officiers. Il tirait gros du roi d'Espagne. Il allait
presque tous les jours au palais où il était fort accueilli, et je ne
l'ai point vu à portée du roi et de la reine qu'ils ne lui parlassent,
et quelquefois même en s'arrêtant à lui avec un air de considération et
de bonté. Il portait publiquement la Jarretière et le nom de duc
d'Ormond. Il ne se trouvait point où on se couvrait\,; mais d'ailleurs
il était traité en tout et partout comme les grands. Il était petit,
gros, engoncé, et toutefois de la grâce à tout, et l'air d'un fort grand
seigneur, avec beaucoup de politesse et de noblesse. Il était fort
attaché à la religion anglicane, et refusa constamment les
établissements solides qui lui furent souvent offerts en Espagne pour la
quitter.

Ubilla, ou le marquis de Rivas, secrétaire de la dépêche universelle
sous Charles II, qui eut tant de part à son testament qu'il écrivit sous
ce prince, avait eu le sort commun à tous ceux à qui Philippe V avait
obligation de sa couronne, que la princesse des Ursins fit chasser. Il
languissait depuis obscurément et avec peu de bien, dans le conseil de
Castille, où on lui avait donné une place, comme dans un vieux sérail\,;
et, avec les années et l'infortune, il vivait fort seul, fort abandonné,
se présentant rarement, toujours très inutilement, au palais où il était
fort peu accueilli. Louville m'avait conseillé à Paris de rendre une
visite à cet illustre malheureux, comme chose fort convenable au service
qu'il avait rendu à la France. Je m'en souvins au retour de Lerma, et,
quoique je n'eusse pas ouï parler de lui, je l'allai voir avec plus de
suite que je n'avais coutume de mener dans mes visites. Jamais homme si
surpris ni si aise, et je le fus beaucoup de lui avoir fait tant de
plaisir. C'était un petit homme mince, et sur l'âge, dont la mine
n'imposait pas, mais plein d'esprit, de sens et de mémoire, et avec qui
je me serais extrêmement plu et instruit, s'il avait parlé moins
difficilement François. Il se montra avec moi fort mesuré sur sa
disgrâce, à laquelle pourtant on sentait qu'il n'était pas accoutumé. Ce
n'était pas comme nos ministres renvoyés, dont les restes enrichiraient
plusieurs seigneurs et les logeraient magnifiquement à la ville et à la
campagne. Celui-ci, qui avait exercé plusieurs années une charge qui
comprend les quatre charges de nos secrétaires d'État\footnote{Les
  quatre secrétaires d'État de l'ancienne monarchie étaient ceux de la
  guerre, des affaires étrangères, de la marine et de la maison du roi.
  Ils se partageaient l'administration des provinces. Le ministère de
  l'intérieur n'a été établi qu'à l'époque de la révolution. Le
  chancelier avait la surveillance de l'administration de la justice, de
  l'imprimerie et de la librairie. Les finances dépendaient du
  contrôleur général, qui, depuis 1661, avait remplacé le surintendant
  des finances.}, était logé plus que médiocrement, presque sans
meubles, et les plus simples, avec fort peu de valets. Il revint me voir
et me fit présent d'un beau livre espagnol qu'il avait composé des
voyages et des campagnes de Philippe V. Cette visite me fit honneur à
Madrid, et ne déplut pas aux ministres.

\hypertarget{chapitre-iii.}{%
\chapter{CHAPITRE III.}\label{chapitre-iii.}}

~

{\textsc{Situation de la cour d'Espagne.}} {\textsc{- Goût et conduite
de la reine.}} {\textsc{- Elle hait les Espagnols, qui la haïssent
publiquement.}} {\textsc{- Cabale nationale à la cour d'Espagne.}}
{\textsc{- Fortune de Caylus.}} {\textsc{- Importance de la mécanique
journalière.}} {\textsc{- Plan de la reine arrivant à Madrid.}}
{\textsc{- Sa conduite.}} {\textsc{- Fortune d'Albéroni\,; son règne, sa
chute.}} {\textsc{- Vie journalière du roi et de la reine d'Espagne.}}
{\textsc{- Déjeuner.}} {\textsc{- Prière.}} {\textsc{- Travail avec
Grimaldo.}} {\textsc{- Lever.}} {\textsc{- Toilette.}} {\textsc{- Heures
des audiences particulières des seigneurs et des ministres étrangers.}}
{\textsc{- De l'audience publique et sa description.}} {\textsc{- De
l'audience du conseil de Castille.}} {\textsc{- Des audiences publiques
des ambassadeurs et de la couverture des grands.}} {\textsc{- La messe
et confession et communion.}} {\textsc{- Dîner.}} {\textsc{- Sortie et
rentrée de la chasse.}} {\textsc{- Collation, et travail de Grimaldo.}}
{\textsc{- Temps de la confession de la reine\,; sa contrainte.}}
{\textsc{- Souper et coucher.}} {\textsc{- Voyages.}} {\textsc{- La
reine présente à toutes les audiences particulières des ministres
étrangers et des sujets.}} {\textsc{- Raisons de l'explication du détail
des journées.}} {\textsc{- Jalousie réciproque du roi et de la reine.}}
{\textsc{- Difficulté extrême de la voir en particulier, et de tout
commerce d'affaires avec elle seule.}} {\textsc{- Caractère de Philippe
V.}} {\textsc{- Éducation et sentiments de la reine d'Espagne pour sa
famille et pour son pays.}} {\textsc{- Fortune de Scotti.}} {\textsc{-
Caractère, vie, vues, art, manèges, conduite, pouvoir, contrainte de la
reine d'Espagne.}} {\textsc{- Extinction par la princesse des Ursins des
étiquettes\,; des conseils où le roi se trouvait\,; des fonctions des
charges principales, qui a toujours duré depuis.}} {\textsc{- Oubli
réparé d'une fonction du grand et du premier écuyer.}}

~

Outre les inimitiés particulières et les divisions que l'ambition et les
différents intérêts forment et entretiennent toujours dans les cours, il
y en avait de nationales dans celle de Madrid. La reine était d'un poids
très principal dans les affaires de toute espèce, dans les choix, dans
les grâces. Si elle n'était pas sûre de l'inclusion, elle l'était au
moins de l'exclusion. Le comment on l'expliquera bientôt, et son crédit
certain et invulnérable était universellement reconnu au dedans et au
dehors. Elle était Italienne, Albéroni l'était aussi\,; tous deux
régnèrent conjointement comme avait fait la feue reine avec la princesse
des Ursins, {[}et ils{]} avaient tous attiré des Italiens à la cour et
dans le service militaire. Les besoins de ménager la nation espagnole,
et la reconnaissance due à sa fidélité singulière dans les revers les
plus désespérés, et les signalés services qui avaient par deux fois
remis sa couronne sur la tête de Philippe V, avaient duré presque
jusqu'à la mort de cette reine, qui n'avait cessé de s'attacher les
Espagnols par le solide et par le charme de ses manières, qui l'en avait
fait pour ainsi dire adorer. Après sa mort le roi, enfermé dans l'hôtel
de Medina-Cœli avec la princesse des Ursins, n'y voyait qu'elle dans
tous les moments de la journée, et par-ci par-là quelques-unes des sept
ou huit personnes qu'elle avait choisies pour se relayer les unes les
autres, à toute autre exception, pour accompagner le roi à la chasse et
à la promenade, desquelles elle était parfaitement assurée. Les dangers
étaient passés, elle gouvernait seule, en plein et publiquement, sans
contradiction de personne.

Le traitement d'Altesse qu'on a vu ailleurs qu'elle avait fait donner au
duc de Vendôme et à elle, avait mis les Espagnols au désespoir contre
elle, et leur haine éclatait de toutes parts, maigre toute sa puissance.
La nécessité des ménagements était passée avec la guerre\,; elle tenait
le roi au point de ne craindre rien, pas môme le feu roi qu'elle
offensa, et qui la perdit. Elle rendit donc aux Espagnols haine pour
haine\,; mais toute-puissante de sa part. Le second mariage du roi
d'Espagne fut son ouvrage\,; personne en Espagne ni ailleurs n'en douta,
elle en était même bien aise. Mais la conséquence fut que ce second
mariage ne fut pas du goût des Espagnols, et pour d'autres raisons
encore peu agréable à l'État, à la maison au personnel de la nouvelle
reine, au point que la chute si précipitée de la princesse des Ursins
par l'arrivée de cette reine, ne put la réconcilier avec les Espagnols,
beaucoup moins les Espagnols avec elle, à qui elle ne pardonna jamais
leur éloignement de son mariage. On a vu ailleurs comment elle s'empara
du roi d'Espagne, tout en arrivant, et par elle, et avec elle bientôt
après Albéroni. Entre son introduction et le comble de sa puissance, il
y eut assez d'intervalle pour laisser aux Espagnols la liberté de se
répandre sur un champignon poussé de si bas par une main qui leur était
déjà odieuse. Ce fut bien pis pour les sentiments quand le poids du joug
les empêcha de parler. Ils s'exhalèrent, à la vérité, à sa chute, mais
cette chute môme était l'ouvrage de la reine, qui n'en demeurait que
plus absolue et plus régnante. Ainsi ils ne l'en aimèrent pas mieux, ni
elle eux, jusque-là qu'elle dédaigna de profiter d'une conjoncture si
favorable pour se les rapprocher. Aussi est-il incroyable jusqu'où alla
cette réciproque aversion. Quand elle sortait avec le roi pour aller à
l'Atocha ou à la chasse, le peuple criait sans cesse, ainsi que les
bourgeois, dans leurs boutiques\,: \emph{Viva et Rey la Savoyana, y la
Savoyana}\footnote{Il s'agit de Louise-Marie-Gabrielle de Savoie,
  première femme de Philippe V\,; elle était morte, comme on l'a vu plus
  haut, en 1714.}\emph{\,!} et répétait sans cesse \emph{la Savoyana} à
gorge déployée, qui est la feue reine, pour qu'on ne s'y méprit pas,
sans qu'aucune voix criât jamais\,: \emph{Viva la Reina\,!} La reine
faisait semblant de mépriser cela, mais elle rageait en elle-même, on le
voyait, elle ne pouvait s'y accoutumer. Aussi disait-elle fort
librement, et me l'a dit à moi plus d'une fois\,: «\,Les Espagnols ne
m'aiment pas, mais je les hais bien aussi,\,» avec un air de pique et de
colère. Ce n'était pas qu'il n'y en eût quelques-uns, mais en plus que
très petit nombre qu'elle aimait, comme Santa-Cruz, la comtesse
d'Altamire, Montijo, et quelque peu d'autres, et quelques-uns encore
qu'elle traitait bien à cause de leurs places, de leur état, même
familièrement, et avec un air de bonté, comme le duc del Arco, à cause
du goût du roi. Par la même raison du roi, et par la conjoncture
d'alors, elle traitait bien les François, mais au fond elle ne les
aimait pas.

Son goût était déclaré pour les Italiens, qui se rassemblaient entre eux
en cabale contre les Espagnols, sous la protection de la reine. Les
Flamands s'accrochaient à eux pour plaire à la reine et par ancienne
aversion de leur nation pour l'espagnole, et ce qu'il, y avait
d'Irlandais aussi en officiers et en señoras de honor, et en
caméristes., quoique le duc d'Ormond et le marquis de Lede, auxquels
chacune des deux nations se ralliait, se maintinssent bien avec la
reine, et avec les Espagnols. Des Espagnols aussi, mais en petit nombre,
se joignaient à la cabale italienne, comme Montijo, tout jeune qu'il
était encore, comme Miraval, gouverneur du conseil de Castille, ami
intime du duc de Popoli, et quelques autres, ou pour des vues de
fortune, ou par avoir secrètement la maison d'Autriche dans le coeur.

Les Espagnols payaient de haine, de hauteur, de mépris, et ne
détestaient rien tant au monde que les Italiens, et après eux les
Flamands. Ils souffraient les Irlandais, et la considération du roi, qui
aimait fort les François, les retenait à leur égard. Ce qui faisait
encore cette différence, c'est qu'ils trouvaient beaucoup de seigneurs
en leur chemin des deux premières nations pour les fortunes, les
distinctions, les charges et les grandes places, ce qui ne se
rencontrait pas dans les deux autres où il n'y avait personne à pouvoir
s'égaler à eux\,; et d'ailleurs les Français établis à demeure n'étoient
rien pour le nombre. Caylus était le seul qui pointât vers la fortune\,;
il était militaire plus que courtisan, et point marié. Toutefois il
avait la Toison, et visait à être capitaine général d'une province et
d'armée. Il y arriva en effet, et longtemps depuis mon retour, à la
grandesse et à la vice-royauté du Pérou. Mais ce n'était qu'un seul
homme. À l'égard du duc de Liria, il avait su se maintenir avec les uns
et les autres, et il en était regardé comme naturel Espagnol, à cause de
sa femme héritière en Espagne\,; car tous ces seigneurs italiens et
flamands n'avaient que leurs titres, leurs charges et leurs emplois, et
pas un pouce de terre, au lieu que le Liria n'avait ni terres, ni
espérances, ni établissement qu'en Espagne.

Ces deux cabale, l'espagnole sur son palier, l'étrangère sous la
bannière de la reine, n'éclataient ni ne se montraient au dehors, mais
en dessous se guettaient sans cesse, et par leur haine, leur envie, leur
jalousie, faisaient des mouvements intérieurs. La reine, à la vie
qu'elle menait, ne pouvait pas toujours être avertie, et tout le menu
lui échappait, parce que tous les secrétaires d'État et tous les membres
des conseils et des juntes, pour ce qui en subsistait, étaient tous
Espagnols, et par ce encore que les grands seigneurs espagnols ne
laissaient pas de trouver des accès auprès du roi, quelque enfermé qu'il
fût, et qui, au fond, les considérait et donnait dans son coeur et dans
son goût une grande préférence aux Espagnols sur toute autre nation,
excepté la française, mais sur laquelle il tenait son goût de fort
court, en considération des Espagnols\,; laquelle considération était
bien connue à la reine, et la contraignait beaucoup et souvent. Toutes
ces choses invisibles en détail au gros du monde, même de la cour, était
un spectacle fort intéressant, ou fort amusant et curieux, polir qui
était au fait des personnages de l'intérieur du palais et des
événements.

Ceci conduit naturellement à donner la mécanique extérieure du
journalier du roi et de la reine d'Espagne, parce que rien n'influe tant
sur le grand et le petit que cette mécanique des souverains. C'est ce
qu'une expérience continuelle apprend à ceux qui sont initiés dans
l'intérieur par la faveur ou par les affaires, et à ceux des dehors
assez en confiance avec ces initiés pour qu'ils leur parlent librement.
Je dirai, en passant, par l'expérience que j'ai faite vingt ans durant,
et plus en l'une et en l'autre manière, que cette connaissance est une
des meilleures clefs de toutes les autres, et qu'elle manque toujours
aux histoires, souvent aux mémoires, dont les plus intéressants et les
plus instructifs le seraient bien davantage s'ils avaient moins négligé
cette partie, que qui n'en connaît pas le prix, regarde comme une
bagatelle indigne d'entrer dans un récit. Toutefois suis-je bien assuré
qu'il n'est point de ministre d'État, de favori, de ce peu de gens qui
de tous étages se trouvent initiés dans l'intérieur des souverains par
le service nécessaire de leurs emplois ou de leurs charges, qui ne soit
en tout de mon sentiment là-dessus.

La reine, arrivant en Espagne, ne songea qu'à remplir seule auprès du
roi le vide qu'y laissait l'expulsion qu'elle venait de faire de la
princesse des Ursins\,; et le roi, impatient par tempérament d'avoir une
épouse, retenu qu'il était par sa conscience de trouver ailleurs, lui
donna là-dessus tout le jeu qu'elle pouvait désirer\,; mais accoutumé au
tête-à-tête continuel, tout au plus au tiers, la reine n'eut pas à
choisir. Son peu de connaissance lui fit bientôt admettre entre eux deux
Albéroni, qui était le seul homme qu'elle connût, et qui, uni de même
intérêt qu'elle par être Parmesan et ambitieux, était son conseil unique
depuis leur départ de Parme, et le seul qu'elle pût avoir en Espagne, au
moins dans les commencements. Il devint donc bientôt avec le roi et
cette reine ce que M\textsuperscript{me} des Ursins avait été avec
l'autre reine, avec la différence du sexe, qui en ôta le ridicule, et
qui le rendit capable du nom comme du pouvoir de premier ministre, et
enfin de la dignité de cardinal. Pour arriver à ces grandes choses, il
suivit le plan dont la princesse des Ursins s'était si bien trouvée, et
dont les gens avisés qui peuvent tout sur les rois font tous, d'une
façon ou d'une autre, un usage si utile pour eux, mais si détestable
pour leurs maîtres et si pernicieux pour leurs États, leurs sujets, leur
gouvernement. Albéroni n'eut, pour cela, rien à faire qu'à suivre le
goût funeste que le roi avait pris pour la prison où
M\textsuperscript{me} des Ursins avait su le renfermer peu à peu avec la
reine, puis avec elle seule lorsqu'il devint veuf. La nouvelle reine et
Albéroni suivirent la même route\,; ils renfermèrent le roi entre eux
deux seuls et le rendirent inaccessible à tout le reste de la nature.
Albéroni chassé, la reine lassée d'avoir été si longtemps prisonnière,
victime de sa propre ambition et de celle de cet Italien, tenta
plusieurs fois d'élargir son esclavage, sans jamais y avoir pu réussir.
L'habitude du roi était trop enracinée\,; elle avait passé en lui en une
seconde nature, et la reine désespéra bientôt d'adoucir ses fers. Voici
donc quelle était leur vie en tous lieux, en tout temps, en toute
saison.

Le roi et la reine n'eurent jamais qu'un seul et même appartement et
qu'un lit, tel que je l'ai décrit, lorsque je fus admis avec Maulevrier
à les y voir, lorsque nous leur portâmes la nouvelle du départ de Paris
de la future princesse des Asturies. Fièvres, maladie telle qu'elle pût
être de part et d'autre, couches enfin, jamais une seule nuit de
séparation\,; et la feue reine, pourrie d'écrouelles, le roi ne découcha
d'avec elle que peu de jours avant sa mort. Sur les neuf heures du matin
le rideau était tiré par l'assafeta, suivie d'un seul valet intérieur
français portant un couvert et une écuelle qui était pleine d'un
chaudeau. Hyghens, dans la convalescence de ma petite vérole, m'expliqua
ce que c'est, et m'en fit faire un lui-même pour m'en faire goûter.
C'est une mixtion légère de bouillon, de lait, de vin qui domine, d'un
ou deux jaunes d'oeufs, de sucre, de cannelle et d'un peu de girofle.
Cela est blanc, a le goût très fort avec un mélange de douceur. Je n'en
ferais pas volontiers mon mets, mais il est pourtant vrai que cela n'est
pas désagréable. On y met, quand on veut, des croûtes de pain et
quelquefois grillées, et alors c'est une espèce de potage, autrement
cela s'avale comme un bouillon\,; et pour l'ordinaire, cette dernière
façon de le prendre était, celle du roi d'Espagne. Cela est onctueux,
mais fort chaud, et un restaurant singulièrement bon à réparer la nuit
passée, et à préparer la prochaine.

Pendant que le roi faisait ce court déjeuner, l'assafeta apportait à la
reine de quoi travailler en tapisserie, passait des manteaux de lit à
Leurs Majestés, et mettait sur le lit partie des papiers qui se
trouvaient sur les sièges prochains, puis se retirait avec le valet et
ce qu'il avait apporté. Leurs Majestés faisaient alors leurs prières du
matin. Grimaldo, sûr de l'heure, mais qui de plus était averti dans sa
cavachuela au palais, montait chez Leurs Majestés, et entrait.
Quelquefois ils lui faisaient signe d'attendre en entrant, puis
l'appelaient quand leur prière était finie, car il n'y avait personne
autre, et la chambre du lit était fort petite. Là Grimaldo étalait ses
papiers, tirait de sa poche une écritoire et travaillait avec le roi et
la reine, que sa tapisserie n'empêchait pas de dire son avis. Ce travail
durait plus ou moins, selon les affaires ou quelque conversation.
Grimaldo, en sortant avec ses papiers, trouvait la pièce joignante vide,
et un valet dans celle d'après, qui, le voyant passer, entrait dans la
pièce vide, la traversait et avertissait l'assafeta, qui, sur-le-champ,
venait présenter au roi ses mules et sa robe de chambre, qui tout de
suite passait seul la pièce vide, et entrait dans un cabinet où il
s'habillait, servi par trois valets français intérieurs, toujours les
mêmes, et par le duc del Arco ou le marquis de Santa-Cruz, et souvent
par tous les deux, sans que jamais qui que ce soit autre entrât à ce
lever. Lorsqu'il était tout à fait à sa fin, un de ces valets allait
appeler le P. Daubenton dans le salon des miroirs, qui venait trouver le
roi dans ce cabinet, d'où, sur-le-champ, les valets susdits emportaient
à la fois les débris du lever, et ne rentraient plus. Si le roi faisait
un signe de la tête à ces deux seigneurs, après la sortie des valets,
ils sortaient aussi\,; mais cela n'arrivait que quelquefois, et ils
restaient se tenant vers la porte, et le roi parlait dans la fenêtre au
P. Daubenton.

La reine, dès que le roi était passé à son lever, se chaussait seule
avec l'assafeta, qui lui donnait sa robe de chambre. C'était le seul
moment où elle pouvait parler seule à la reine et la reine à elle\,;
mais ce moment allait au plus et non toujours à un demi-quart d'heure.
Plus long, le roi l'aurait su, et aurait voulu savoir ce qui l'aurait
allongé. La reine passait cette pièce vide, et entrait dans un beau et
grand cabinet où sa toilette l'attendait. La camarera-mayor, deux dames
du palais, deux señoras de honor tour à tour par semaine, et les
caméristes étaient autour, quelquefois quelques dames du palais ou
quelque señora de honor, qui n'étaient pas en semaine, mais rarement.
Quand le roi avait fini avec le P. Daubenton, et d'ordinaire cela était
court, il allait à la toilette de la reine, suivi des deux seigneurs,
qui, pendant sa conversation avec le P. Daubenton, l'attendaient à la
porte du cabinet, soit en dedans, soit en dehors. Les infants venaient
aussi à la toilette où il n'entrait avec eux que leurs gouverneurs et,
depuis le mariage du prince des Asturies, la princesse des Asturies, le
duc de Popoli et la duchesse de Monteillano, quelquefois une dame du
palais aussi de la princesse. Le cardinal Borgia avait cette privance,
et s'en servait souvent. Le marquis de Villena l'avait aussi, mais fâché
d'être réduit à celle-là, et privé de toutes celles que de droit lui
donnait sa charge, il n'en usait presque jamais. La chasse, les voyages,
les beaux habits du roi et des infants étaient la matière de la
conversation. Par-ci, par-là, quelque petit avis de réprimande de la
reine à ses dames sur l'assiduité de leur service, ou sur leurs
commerces, ou sûr la dévotion, car elle les tenait fort de court pour ne
pas voir grand monde et sur le choix de leur commerce\,; et pour être
bien avec elle, il fallait moucher souvent, n'être pas trop longtemps en
couche ni souvent incommodée, surtout faire ses dévotions tous les huit
jours. Souvent aussi le cardinal Borgia défrayait la toilette par les
plaisanteries qu'on lui faisait, et auxquelles il donnait lieu. Cette
toilette durait bien trois quarts d'heure, le roi debout, et tout ce qui
y était.

Tandis qu'on en sortait, le roi venait entrebâiller la porte du salon
des miroirs dans le salon qui est entre celui-là et le salon des grands,
où la cour se rassemblait, et là donnait l'ordre à ceux qui, en très
petit nombre, avaient à le prendre, puis allait retrouver la reine dans
cette pièce que j'ai tout à l'heure `appelée si souvent vide. C'était là
l'heure des audiences particulières des ministres étrangers et des
seigneurs ou autres sujets qui l'obtenaient. Ministres étrangers et
sujets s'adressaient à La Roche pour la demander. Il prenait l'ordre du
roi, les faisait avertir, et les introduisait l'un après l'autre, sans
demeurer avec eux dans le salon des miroirs où le roi la donnait
toujours.

Une fois la semaine, le lundi, il y avait audience publique, qui est une
pratique qu'on ne peut trop louer, quand on ne la corrompt pas. Le roi,
au lieu d'entrebâiller la porte dont je viens de parler, l'ouvrait,
donnait l'ordre sur le pas de la porte, et tout de suite traversait tous
ses appartements au milieu de sa cour, ces jours-là assez nombreuse,
jusqu'à la pièce de l'audience publique des ambassadeurs et de la
couverture des grands. Tous s'y rangent comme en ces occasions, dont
j'ai décrit l'assiette et la cérémonie ailleurs. Mais en celle-ci le roi
s'assied dans un fauteuil avec une table, une écritoire et du papier à
sa droite. Il se couvre et tous les grands. Alors La Roche, qui a une
liste à la main, ouvre la porte opposée à celle par où le roi et sa cour
est entrée, et appelle à haute voix le premier qui se trouve sur la
liste. Celui-là entre, fait au roi une profonde révérence en entrant,
une au milieu, puis se met à genoux devant le roi, excepté les prêtres
qui ôtent leur calotte, et font une génuflexion en abordant le roi et en
se retirant, et parlent debout, mais baissés. C'est le roi qui à leur
génuflexion les fait relever\,: tout autre demeure et parle à genoux,
jusqu'à ce qu'il se retire. On parle au roi tant qu'on veut, de qui on
veut et comme on veut, et on lui donne par écrit ce qu'on veut. Mais les
Espagnols ne ressemblent en rien aux François\,; ils sont mesurés,
discrets, respectueux, courts. Celui-là ayant fini, se relève, baise la
main au roi, fait une profonde révérence, et se retire, sans en faire
d'autres, par où il est entré. Alors La Roche appelle le second, et
ainsi tant qu'il y en a.

Lorsque quelqu'un veut parler au roi tête à tête, et qu'il est bien
connu, cela ne se refuse point, et après avoir été appelé, La Roche se
tourne sans bouger vers les grands, et dit du même ton qu'il a appelé\,:
«\,C'est une audience secrète.\,» Alors les grands se découvrent,
passent promptement devant le roi avec une révérence, et se retirent par
la porte par où ils sont entrés, dans la pièce voisine. Le capitaine des
gardes tient cette porte, la tête un peu en dehors pour voir toujours le
roi et celui qui lui parle, qui est seul dans la pièce où il ne reste
personne que le roi et lui. Dès qu'il se lève, le capitaine des gardes
le voit, rentre et tous aussi comme ils étaient sortis, et se remettent
où ils étaient. Je n'ai point vu d'audience publique sans audiences
secrètes, et quelquefois deux et trois. Dans le peu que je fus à Madrid
avant le mariage, les grands me prièrent de m'y trouver comme duc et
ayant les mêmes honneurs qu'eux, et j'y fus. Au retour du mariage, j'y
eus double droit, comme duc et pair de France et comme grand d'Espagne.
Mon second fils s'y trouva aussi avec moi, après sa couverture. Quand
tout est fini, on reconduit le roi comme on l'avait accompagné. Venant
et retournant dans le palais, en quelque temps ou occasion que ce fût,
le roi ne se couvrait jamais. C'était aussi le temps des audiences
publiques des ambassadeurs et de la couverture des grands.

Cette même heure est aussi celle où le conseil de Castille vient au
palais rendre compte au roi des jugements qu'il a rendus dans la
semaine. Je crois avoir expliqué ce qui s'y passe et comment\,: ainsi je
ne le répéterai pas. Ce temps, avec le court travail qui le suit, dans
une des autres pièces, entre le roi et le gouverneur du conseil de
Castille dure au plus une heure et demie, mais rarement, et l'audience
publique rarement trois quarts d'heure. Ce sont des temps d'autant plus
précieux pour la reine qu'elle n'avait que ceux-là dans la semaine,
encore quand le roi était au palais ou au Retiro\,; car hors de Madrid,
il n'y avait jamais d'audience du conseil de Castille ni d'audience
publique. Ainsi à l'Escurial, à Balsaïm de mon temps, à Saint-Ildephonse
depuis, au Pardo, à Aranjuez, la reine n'avait exactement et précisément
à elle que le temps de sa chaussure en sortant du lit.

J'oubliais d'ajouter que tout ce qui n'est pas ce qu'on appelait
autrefois en France, mais non à présent, gens de qualité ou militaire
fort distingué, vont tous à ces audiences publiques. Il s'y amasse des
placets et des mémoires que le roi reçoit et jette à mesure sur la
table, et que La Roche porte après lui dans l'appartement intérieur\,;
mais il y en a toujours quelques-uns que le roi mettait dans sa poche ou
emportait dans sa main. C'est ce qu'étaient nos placets dans l'origine,
qui sont tombés, comme on les voit, et comme je ne les ai jamais vus
autrement que pendant la régence.

Le roi rentré tout droit auprès de la reine, ou après s'être amusé avec
elle seule, s'il n'y a point d'audience, allait à la messe avec elle, ce
même intérieur de la toilette, et le capitaine des gardes en quartier de
plus. Le chemin se faisait tout dans l'intérieur jusque dans la tribune,
dans laquelle il y avait un autel, où on leur disait la messe, et où ils
communiaient tous deux ensemble et jamais séparément, et ordinairement
tous les huit jours, et alors ils y entendaient une seconde messe. Quand
le roi se confessait, c'était après son lever, avant d'aller à la
toilette de la reine. S'il était jour de tenir chapelle, c'était à la
même heure\,; la reine allait par l'intérieur dans la tribune, et le roi
avec sa cour à travers les appartements. Le marquis de Santa-Cruz et le
duc del Arco avaient tant d'assiduité qu'ils n'allaient guère ni à la
tribune ni aux chapelles, mais quelquefois le marquis de Villena à la
tribune, quand il n'y avait pas chapelle, et qu'il voulait parler au
roi, comme sa charge, toute mutilée qu'elle était, l'y obligeait assez
souvent.

Au retour de la messe, ou fort peu après, on servait le dîner. J'en ai
expliqué les différents services des dames de la reine. Nul n'y entrait
que ce qui entrait à la toilette. Le dîner était toujours de chez la
reine, ainsi que le souper, et cela partout, mais le roi et la reine
avaient chacun leurs plats\,; le roi peu, la reine beaucoup\,: c'est
qu'elle aimait à manger et qu'elle mangeait de tout, et le roi toujours
des mêmes choses. Un potage uni, des chapons, poulets, pigeons bouillis
et rôtis, et toujours une longe de veau rôtie\,; ni fruit, ni salade, ni
fromage, rarement quelque pâtisserie, jamais maigre, souvent des neufs
ou frais ou en diverses façons, et ne buvait que du vin de Champagne,
ainsi que la reine. Le dîner fini, ils priaient Dieu ensemble. S'il
arrivait quelque chose de pressé, Grimaldo venait leur en rendre un
compte sommaire.

Environ une heure après le dîner, ils sortaient par un endroit public de
l'appartement, mais court, et par un petit escalier allaient monter en
carrosse, et au retour revenaient par le même chemin. Les seigneurs qui
fréquentaient un peu familièrement la cour se trouvaient, tantôt les
uns, tantôt les autres, à ce passage, ou lés suivaient à leurs
carrosses. Très souvent je les voyais à ces passages allant ou revenant.
La reine y disait toujours quelque mot honnête à ce qui s'y trouvait. Je
parlerai ailleurs de la chasse, toujours la même, où ils allaient tous
les jours, et du Mail et de l'Atocha, certains dimanches ou fêtes qu'ils
y allaient sans cérémonie.

Au retour de la chasse le roi donnait l'ordre en rentrant. S'ils
n'avaient pas fait collation dans leur carrosse, ils la faisaient en
arrivant. C'était, pour le roi, un morceau de pain, un grand biscuit, de
l'eau et du vin\,; et pour la reine, de la pâtisserie et des fruits,
dans la saison\,; quelquefois du fromage. Le prince et la princesse des
Asturies, et les infants, suivis comme à la toilette, les attendaient
dans l'appartement intérieur. Cette compagnie se retirait en moins de
demi-quart d'heure. Grimaldo montait et travaillait, et ordinairement
longtemps\,; c'était le temps du vrai travail. Quand la reine avait à se
confesser, c'était là l'heure. Outre ce qui regardait la confession,
elle et son confesseur n'avaient pas le temps de se parler. Le cabinet
où elle était avec lui était contigu à la pièce où était le roi qui,
quand il trouvait la confession trop longue, venait ouvrir la porte et
l'appelait. Grimaldo sorti, ils se mettaient ensemble en prières, ou
quelquefois en lectures spirituelles jusqu'au souper. Il était en tout
servi comme le dîner. Il y avait à l'un et à l'autre beaucoup plus de
plats à la française qu'à l'espagnole ni même qu'à l'italienne.

Après souper, la conversation ou la prière tête à tête les conduisait à
l'heure du coucher, où tout se passait comme au lever, excepté qu'à la
toilette de la reine le prince, ni la princesse des Asturies, ni les
infants, ni le cardinal Borgia n'y allaient point. Enfin Leurs Majestés
Catholiques n'avaient jamais partout que la même garde-robe, et leurs
deux chaises percées étaient à côté l'une de l'autre dans toutes leurs
maisons.

Ces journées si uniformes étaient les mêmes dans tous les lieux et même
dans les voyages, et le même tête-à-tête partout. Les journées de voyage
étaient si petites que le temps qui se donnait à la chasse de tous les
jours suffisait pour aller d'un lieu dans un autre, et tout le reste se
passait dans les maisons où Leurs Majestés Catholiques logeaient sur la
route tout comme si elles étaient dans leur palais. Je parle ici du
voyage de Lerma et de ceux qui se sont faits depuis mon retour. À
l'égard de ceux de l'Escurial, de Balsaïm, d'Aranjuez, tous à peu près
de la même longueur, mais trop courts pour coucher en chemin, tout
s'avançait peu à peu dans la matinée l'un sur l'autre d'une heure. Le
départ était au sortir de table, et l'arrivée quelque temps avant
l'heure de souper. En carrosse, soit pour la chasse, soit en voyage,
toujours Leurs majestés tête à tête dans un grand carrosse de la reine à
sept glaces, et la housse de velours rouge clouée comme ici.

Pour ne rien omettre, il faut ajouter que la reine avait encore à elle
seule les premières et dernières audiences de cérémonie des
ambassadeurs, et les couvertures des grands. Mais comme ces ambassadeurs
et ces grands allaient toujours de chez le roi immédiatement chez elle,
elle s'y préparait, en les attendant, au milieu de ses dames et des
autres dames qui n'avaient que ces occasions de venir au palais, et de
lui faire leur cour\,; car pour les bals publics et les comédies, il n'y
en avait point au palais sans des occasions extraordinaires et fort
rares.

À l'égard des audiences particulières des ministres étrangers ou des
seigneurs, elles ne se donnaient jamais qu'en présence de la reine, soit
qu'elle y demeurât à côté du roi, soit qu'elle se retirât un peu à
l'écart dans la même pièce. Aussi n'arrivait-il guère que ceux qui
avaient ces audiences laissassent écarter la reine. On connaissait quel
était son pouvoir sur le roi, et son influence dans toutes les affaires
et les grâces, et ils étaient bien certains que si la reine s'était
écartée, lorsqu'ils parlaient au roi, ils étaient cependant bien
examinés par la reine, et qu'ils n'étaient pas plus tôt retirés, qu'elle
apprenait du roi tout ce qu'ils lui avaient dit, et ce qu'il leur avait
répondu, qui n'était jamais rien de précis sur quoi que ce fût, parce
qu'il voulait toujours avoir le temps de consulter la reine et Grimaldo.

Si ce détail des journées paraît long et petit, c'est qu'il est
incroyable à qui ne l'a vu dans sa précision et son unisson, toujours et
partout les mêmes. C'est qu'un tête-à-tête jour et nuit si continuel, et
si momentanément et rarement interrompu, semble avec raison
insoutenable. C'est l'influence entière que ce tête-à-tête insupportable
portait sur toutes les affaires de l'État et sur celles des
particuliers, c'est la démonstration nécessaire de ne pouvoir jamais,
quel que l'on fût, parler au roi sans la reine, ni pareillement à la
reine sans le roi, dont tous deux avaient réciproquement une jalousie
extrême l'un à l'égard de l'autre\,; c'est enfin ce qui rendait
l'assafeta si nécessaire pour faire passer à la reine seule ce qu'on
voulait dans le moment de sa chaussure, et dans les temps de l'audience
publique et de l'audience du conseil de Castille, qui n'était jamais que
dans Madrid, et qui étaient les seuls où la reine pouvait parler à
quelqu'un du dehors, qui en prenant bien juste ses mesures pouvait être
secrètement introduit par l'assafeta en lieu où la reine pût venir.
C'est à quoi elle-même ne se jouait guère, dans la frayeur de la
découverte et des suites. Mais au moins pouvait-elle dans ces courts,
rares et précieux moments, recevoir et lire des lettres et des mémoires,
et en écrire elle-même. Mais on peut juger avec quelle précipitation et
avec quel soin de ne garder aucun papier.

Philippe V n'était pas né avec des lumières supérieures, ni avec rien de
ce qu'on appelle de l'imagination. Il était froid, silencieux, triste,
sobre, touché d'aucun plaisir que de la chasse, craignant le monde, se
craignant soi-même, produisant peu, solitaire et enfermé par goût et par
habitude, rarement touché d'autrui, du bon sens néanmoins et droit, et
comprenant assez bien les choses, opiniâtre quand il s'y mettait, et
souvent alors sans pouvoir être ramené, et néanmoins parfaitement facile
à être entraîné et gouverné.

Il sentait peu. Dans ses campagnes, il se laissait mettre où on le
plaçait, sous un feu vif sans en être ébranlé le moins du monde, et s'y
amusant à examiner si quelqu'un avait peur. À couvert et en éloignement
du danger tout de même, sans penser que sa gloire en pouvait souffrir.
En tout, il aimait à faire la guerre, avec la même indifférence d'y
aller ou de n'y aller pas, et présent ou absent, laissait tout faire aux
généraux sans y mettre rien du sien. Il était extrêmement glorieux, ne
pouvait souffrir de résistance dans aucune de ses entreprises\,; et ce
qui me fit juger qu'il aimait les louanges, c'est que la reine le louait
sans cesse et jusqu'à sa figure, et à me demander un jour à la fin d'une
audience, qui s'était tournée en conversation, si je ne le trouvais pas
fort beau et plus beau que tout ce que je connaissais. Sa piété n'était
que coutume, scrupules, frayeurs, petites observances, sans connaître du
tout la religion, le pape une divinité quand il ne le choquait pas,
enfin la douce écorce des jésuites pour lesquels il était passionné.
Quoique sa santé frit très bonne, il se tâtait toujours, il craignait
toujours pour elle. Un médecin tel que celui que Louis XI enrichit tant
à la fin de sa vie, un maître Coctier, aurait fait auprès de lui un
riche et puissant personnage\,: heureusement le sien était solidement
homme de bien et d'honneur, et celui qui lui succéda depuis tout à la
reine et tenu de court par elle.

Philippe V avait moins de peine à bien parler que de paresse et de
défiance de lui-même. C'est ce qui le rendait si retenu et si rare à
entrer le moins du monde dans la conversation, qu'il laissait tenir à la
reine avec ce qui les suivait au Mail ou dans les audiences
particulières, et qu'il la laissait aussi parler aux uns et aux autres
en passant, sans presque jamais leur rien dire\,: d'ailleurs c'était
l'homme du monde qui remarquait mieux les défauts et les ridicules, et
qui en faisait un conte le mieux dit et le plus plaisant. J'en dirai
peut-être bientôt quelque chose. On a vu avec quelle dignité et quelle
justesse il me répondit à mon audience solennelle, et avec quel
discernement de paroles et de ton sur l'un et l'autre mariage, et cela
seul montre bien qu'il savait s'énoncer parfaitement, mais qu'il n'en
voulait presque jamais prendre la peine. À la fin, je l'avais un peu
apprivoisé, et, dans mes audiences qui se tournaient toujours en
conversation, je l'ai plusieurs fois ouï parler et raisonner bien\,;
mais où il y avait du monde, ordinairement il ne me disait qu'un mot qui
était une question courte ou quelque chose de semblable, et n'entrait
jamais dans aucune conversation.

Il était bon, facile à servir, familier avec l'intérieur, quelquefois
même au dehors avec quelques seigneurs. L'amour de la France lui sortait
de partout. Il conservait une grande reconnaissance et vénération pour
le feu roi, et de la tendresse pour feu Monseigneur, surtout pour feu
Mgr le Dauphin, son frère, de la perte du quel il ne pouvait se
consoler. Je ne lui ai rien remarqué sur pas un autre de la famille
royale que pour le roi, et {[}il{]} ne s'est jamais informé à moi de qui
que ce soit de la cour que de la seule duchesse de Beauvilliers, et avec
amitié.

On a peine à comprendre ses scrupules sur sa couronne, et de les
concilier avec cet esprit de retour, en cas de malheur, à la couronne de
ses pères, à laquelle il avait si solennellement renoncé, et plus d'une
fois. C'est qu'il ne pouvait s'ôter de la tête la force des
renonciations de la reine en épousant le feu roi, et de toutes les
précautions possibles dont on les avait affermies, et en même temps il
ne pouvait comprendre que Charles II eût été en droit et en pouvoir de
disposer par son testament d'une monarchie dont il n'était
qu'usufruitier, et non pas propriétaire, comme l'est un particulier de
ses acquêts dont il est libre de disposer. Voilà sur quoi le P.
Daubenton avait eu sans cesse à le combattre\,; il se croyait
usurpateur. Dans cette pensée, il nourrissait cet esprit de retour en
France, et par en préférer la couronne et le séjour, et peut-être plus
encore pour finir ses scrupules en abandonnant l'Espagne. On ne peut pas
se cacher que tout cela ne fût fort mal arrangé dans sa tête, mais le
fait est que cela l'était ainsi, et que l'impossibilité seule s'est
opposée à un abandon auquel il croyait être obligé, et qui eut une part
très principale en l'abdication qu'il fit et qu'il méditait dès avant
que j'allasse en Espagne, quoiqu'il laissât sa couronne à son fils.
C'était bien la même usurpation à ses yeux, mais enfin ne pouvant
là-dessus ce qu'il eût voulu par scrupule, il se contentait au moins en
faisant de soi ce qu'il pouvait en l'abdiquant. Ce fut encore ce qui lui
fit tant de peine à la reprendre à la mort de son fils, malgré l'ennui
qu'il avait essuyé, et le dépit, fréquent de n'être pas assez consulté,
et ses avis suivis par son fils et par ses ministres. On peut bien
croire que ce prince ne m'a jamais parlé de cette délicate matière, mais
je n'en ai pas été moins bien informé d'ailleurs. Pour entre Grimaldo et
moi, il ne s'est jamais dit une seule parole qui pût y avoir le moindre
rapport.

La reine n'avait pas moins de désir d'abandonner l'Espagne qu'elle
haïssait, et de venir régner en France, si malheur y fût arrivé, où elle
espérait mener une vie moins enfermée et bien plus agréable. Cela s'est
bien vu d'elle surtout et de son Albéroni, dans les morceaux d'affaires
étrangères que j'ai donnés ici de M. de Torcy.

Parmi tout ce que je viens de dire, il ne laisse pas d'être très vrai
que Philippe V était peu peiné des guerres qu'il faisait, qu'il aimait
les entreprises, et que sa passion était d'être respecté et redouté, et
de figurer grandement en Europe.

La reine avait été élevée fort durement dans un grenier du palais de
Parme, par la duchesse sa mère, qui ne lui avait pas laissé voir le
jour, et qui depuis la conclusion de son prodigieux mariage ne l'avait
laissé voir que le moins qu'elle avait pu, et jamais que sous ses yeux.
Cette extrême sévérité n'avait pas réussi auprès de la reine, dont le
mariage ne réconcilia pas son tueur avec une mère, soeur de
l'impératrice, veuve de l'empereur Léopold, et Autrichienne elle-même
jusque dans les moelles. Ainsi il ne resta entre la fille et la mère que
des dehors de bienséance, souvent assaisonnés d'aigreur. Il n'en était
pas de même entre la reine et le duc de Parme, frère et successeur de
son père, et second mari de sa mère. Ce prince l'avait toujours traitée
avec amitié et considération, et tâché d'adoucir à son égard l'humeur
farouche de sa mère. Aussi la reine aima toujours tendrement le duc de
Parme, dont elle porta sans cesse les intérêts et même les désirs avec
la plus grande chaleur\,; et le crédit de ce prince auprès d'elle était
le plus sûr et le plus fort qu'on y pût employer.

Elle aimait, protégeait et avançait tant qu'il lui était possible les
Parmesans\,; elle avait un faible pour eux bien connu d'Albéroni, et
qu'il redoutait sur toutes choses, comme on l'a vu dans ce qui a été
donné ici de M. de Torcy. Scotti, d'une des premières maisons de Parme,
car il y a d'autres Scotti qui n'en sont pas, et qui sont peu de chose,
était venu à Madrid chargé des affaires du duc de Parme, lorsque
Albéroni s'en défit et devint premier ministre. Scotti était toujours
demeuré â Madrid sous la protection de la reine, qui se moquait de lui
la première, et qui une fois ou deux me laissa très bien entendre le peu
de cas qu'elle en faisait, en quoi elle était imitée de toute la cour,
qui néanmoins lui témoignait des égards à cause de l'affection sans
estime de la reine. En effet, c'était un grand et gros homme, fort
lourd, dont l'épaisseur se montrait en tout ce qu'il disait et
faisait\,; bon homme et honnête homme d'ailleurs, mais parfaitement
incapable. Personne n'en était si persuadé que la reine, mais il était
Parmesan et d'une des premières maisons sujettes du duc de Parme, et
cela lui suffit pour faire à la longue et faute de concurrents du même
pays, la haute fortune où il est à la fin parvenu par la bienveillance
de la reine, sans néanmoins qu'elle ait jamais fait de lui le moindre
cas. Elle l'a fait gouverneur du dernier des infants, lui a valu la
Toison d'or, enfin la grandesse, et pour couronner tout, après l'avoir
extrêmement enrichi, de fort pauvre qu'il était, l'ordre du
Saint-Esprit.

Après l'explication préalable sur la tendresse de la reine pour son
oncle et pour sa patrie, et sa façon d'être avec la duchesse sa mère, il
faut venir à quelque chose de plus particulier. Cette princesse était
née avec beaucoup d'esprit et avec toutes les grâces naturelles que
l'esprit savait gouverner. Le sens, la réflexion, la conduite, savaient
se servir de son esprit et l'employer à propos, et tirer de ses grâces
tout le parti possible. Oui l'a connue est toujours dans le dernier
étonnement comment l'esprit et le sens ont pu suppléer autant qu'ils ont
fait en elle à la connaissance du monde, des affaires et des personnes,
dont le grenier de Parme et le perpétuel tête-à-tête d'Espagne l'ont
toujours empêchée de pouvoir s'instruire véritablement. Aussi ne peut-on
disconvenir de la perspicacité qui était en elle, qui lui faisait saisir
du vrai côté tout ce qu'elle pouvait apercevoir en gens et en choses, et
ce don singulier aurait eu en elle toute sa perfection si l'humeur ne
s'en fût jamais mêlée\,; mais elle en avait, et il faut avouer qu'à la
vie qu'elle menait on en aurait eu à moins. Elle sentait ses talents et
ses forces, mais sans cette fatuité d'étalage et d'orgueil qui les
affaiblit et les rend ridicules. Son courant était simple, uni, même
avec une gaieté naturelle qui étincelait à travers la gêne éternelle de
sa vie\,; et quoique avec l'humeur, et quelquefois l'aigreur que cette
contrainte sans relâche lui donnait, c'était une femme qui ne prétendait
à rien plus dans le courant ordinaire, et qui y était véritablement
charmante.

Arrivée en Espagne, sûre d'en chasser d'abord la princesse des Ursins,
et avec le projet de la remplacer dans le gouvernement, elle le saisit
d'abord et s'en empara si bien, ainsi que de l'esprit du roi, qu'elle
disposa bientôt de l'un et de l'autre. Sur les affaires, rien ne lui
pouvait être caché. Le roi ne travaillait jamais qu'en sa présence. Tout
ce qu'il voyait seul, elle le lisait et en raisonnait avec lui. Elle
était toujours présente à toutes les audiences particulières qu'il
donnait, soit à ses sujets, soit aux ministres étrangers, comme on l'a
déjà expliqué ci-dessus, en sorte que rien ne pouvait lui échapper du
côté des affaires ni des grâces. De celui du roi, ce tête-à-tête éternel
que jour et nuit elle avait avec lui, lui donnait tout lieu de le
connaître, et, pour ainsi dire, de le savoir par coeur. Elle voyait donc
à revers les temps des insinuations préparatoires, leurs succès, les
résistances, lorsqu'il s'en trouvait, leurs causes et les façons de les
exténuer, les moyens de ployer pour revenir après, ceux de tenir ferme
et d'emporter de force. Tous ces manèges lui étaient nécessaires,
quelque crédit qu'elle eût\,; et si on l'ose dire, le tempérament du roi
était pour elle la pièce la plus forte, et elle y avait quelquefois
recours. Alors les refus nocturnes excitaient des tempêtes. Le roi
criait et menaçait, par-ci, par-là passait outre\,; elle tenait ferme,
pleurait et quelquefois se défendait. Le matin tout était en orage\,; le
très petit et intime intérieur agissait envers l'un et envers l'autre
sans pénétrer souvent ce qui l'avait excité. La paix se consommait la
nuit suivante, et il était rare que ce ne fût à l'avantage de la reine
qui emportait sur le roi ce qu'elle avait voulu.

Il arriva une querelle de cette sorte pendant que j'étais à Madrid, qui
fut même poussée fort loin. J'en fus instruit par le chevalier Bourck et
par Sartine qui l'étaient eux-mêmes par l'assafeta, et dans un détail
que je n'ai pas oublié, mais que je ne rendrai pas. Ils me voulurent
persuader de m'en mêler, et que l'assafeta les avait chargés de m'en
presser. Je me mis à rire et les assurai que je me garderais bien de
suivre ce conseil, et même de laisser apercevoir à personne que j'eusse
la moindre connaissance de ce qu'ils venaient de me raconter.

Ainsi la vie de la reine était également contrainte et agitée au delà de
tout ce qui s'en peut imaginer\,; et quelque grand que fût son pouvoir,
elle le devait à tant d'art, de souplesses, de manèges, de patience, que
ce n'est point trop dire, quelque étendu qu'il fût, qu'elle {[}le{]}
payait beaucoup trop chèrement. Mais elle était si vive, si active, si
décidée, si arrêtée, si véhémente dans ses volontés, et ses intérêts lui
étoient si chers et lui paraissaient si grands, que rien ne lui coûtait
pour arriver où elle tendait. Son premier objet fut de se mettre à
couvert par tous les moyens possibles du dénuement et de {[}la{]}
tristesse de la vie d'une reine d'Espagne, veuve, et de ce qui lui
pourrait arriver de la part du fils et successeur du roi, qui n'était
pas le sien.

D'autres objets ne tardèrent pas à se joindre à celui-là, et à le rendre
moins difficile. Elle eut plusieurs princes, et dès lors elle tourna
toutes ses pensées à en faire un souverain indépendant pendant la vie du
roi, chez qui, après sa mort, elle pût se retirer et commander. Pour
arriver à ce but que jour et nuit elle méditait, il fallait tourner les
affaires de manière à le faciliter, se faire des créatures, et leur
procurer des places dont les fonctions et l'autorité la pussent aider.
Ce fut aussi à quoi elle se tourna tout entière, et ce fut par les
ouvertures vraies ou fausses que l'adroit Albéroni sut lui présenter
qu'il se rendit tout à fait maître de son esprit, ce que ses successeurs
Riperda et Patiño imitèrent depuis avec le même succès pour eux-mêmes.

Dans l'entre-deux d'Albéroni et de Riperda que j'étais à Madrid, et que
Grimaldo était le seul qui travaillait avec le roi, elle n'avait point
de secours, parce que les impressions qu'Albéroni lui avait données
contre Grimaldo subsistaient dans son esprit, de façon qu'elle ne
pouvait lui confier son secret et se servir de lui. Ce secret toutefois
était pénétré. Albéroni en furie de sa chute ne le lui avait pas
gardé\,; mais elle se flattait qu'un premier ministre chassé, et de la
réputation que celui-là s'était si justement acquise partout, au dedans
et dehors, n'en serait pas cru à ses discours pleins de rage et de fiel.
Mais elle était étrangement embarrassée, abandonnée ainsi à sa seule
conduite. C'était aussi ce qui l'attachait plus fortement à la cabale
italienne, et qui, par cela même, donnait aux Italiens plus de force, de
vigueur et de crédit. Elle se piquait d'avoir beaucoup d'égards pour le
prince, et la princesse des Asturies, et de marquer des soins et de
l'amitié aux enfants de la feue reine, ce qui changea bien quelque temps
après mon retour ici. Enfin ces desseins de souveraineté pour ses
enfants qui, du temps même d'Albéroni, étaient publics par tout ce qui
s'était proposé et même traité là-dessus, malgré tout ce secret que la
reine voulait encore prétendre, ont été le pivot constant sur lequel ont
roulé depuis toutes les affaires avec l'Espagne, ou qui y ont eu un
rapport.

Mais ce qui les gâta sans cesse, et à tous égards, fut la contrainte
continuelle des ministres étrangers et de ceux du roi d'Espagne, dont
les premiers ne pouvaient lui parler, ni les autres travailler avec lui
qu'en présence de la reine. Quoiqu'en usage de tout voir et de tout
entendre, elle ne pouvait en avoir assez appris par là pour discerner
avec justesse ce qui l'éloignait ou l'approchait de son but, ou ce qui y
était étranger et indifférent, de sorte que ses méprises traversaient
les propositions, les plans, les avis les plus raisonnables, et en
soutenaient de tout contraires avec une âcreté qui imposait absolument
aux ministres espagnols, et qui faisait perdre terre aux ministres
étrangers, parce qu'ils sentaient bien que rien ne pouvait réussir
malgré elle.

Rien aussi n'a été plus funeste à l'Espagne que cette forcenerie
d'établissements souverains pour les fils de la reine, et que cette
impossibilité de traiter de rien qu'avec le roi et la reine ensemble.
Elle avait une telle peur de tout ce qui pouvait croiser ses projets, et
avait une teinture d'affaires si superficielle, que tout ce qui se
proposait lui était suspect dès qu'il n'entrait pas dans son sens. Dès
lors, elle le bargait, et si quelquefois on la faisait revenir, ce ne
pouvait être qu'avec des circuits, des ménagements, des longueurs qui
gâtaient et bien souvent perdaient les affaires, en faisant manquer de
précieuses occasions. Que si on eût pu l'entretenir seule avec un peu de
loisir, elle avait de l'esprit et du sens de reste pour bien entendre et
discuter avec jugement, et on aurait été en état de la combattre avec
succès, ce qui était impossible, le roi présent, parce qu'elle avait
tant de peur qu'il ne prit les impressions qu'on lui présentait\,; et
qui lui entraient à elle dans la tête, comme l'éloignant de son but,
qu'elle ne laissait lieu à aucune explication, et barrait tout, et
jusqu'à des choses qui facilitaient ses vues, parce qu'elle n'en
comprenait pas d'abord les suites et les conséquences, tellement que les
ministres espagnols demeuraient tout courts dans la crainte de s'attirer
sa disgrâce et de perdre leurs places, et les ministres étrangers
enrayaient aussi dans la certitude de l'inutilité de pousser plus avant.
C'est ce qui a fait un tort extrême et continuel aux affaires d'Espagne.

À l'égard dés choses intérieures d'Espagne et des grâces, elle n'était
pars toujours maîtresse de les faire tourner comme elle voulait, surtout
les grâces, quoiqu'elle en emportât la plus nombreuse partie. Mais pour
l'exclusion, elle ne la manquait guère, quand elle la voulait donner, et
à force d'exclusions, elle arrivait quelquefois à faire tomber la grâce
sur qui elle ne l'avait pu d'abord. Rien n'égalait la finesse et le tour
qu'elle savait donner aux choses, et les adresses avec lesquelles elle
savait prendre le roi, et peu à peu l'affecter de ses goûts à elle et de
ses aversions. Rarement allait-elle de front, mais par des préparations
éloignées, des contours et retours qu'elle poussait ou retenait à la
boussole de l'air des réponses, de l'humeur du roi qu'elle avait eu tout
le temps de connaître sans s'y pouvoir tromper. Ses louanges, ses
flatteries, ses complaisances étaient continuelles\,; jamais l'ennui,
jamais la pesanteur du fardeau ne se laissait apercevoir. Dans ce qui
était étranger à ses projets, le roi avait toujours raison, quoi qu'il
pût dire ou vouloir, et allait sans cesse au devant de tout ce qui
pouvait lui plaire, avec un air si naturel qu'il semblait que ce fût son
goût à elle-même.

La chaîne toutefois était ` si fortement tendue qu'elle ne quittait
jamais le côté gauche du roi. Je l'ai vue plusieurs fois au Mail,
emportée des instants par un récit ou par la conversation, marcher un
peu plus lentement que le roi et se trouver à quatre ou cinq pas en
arrière, le roi se retourner, elle à l'instant même regagner son côté en
deux sauts, et y continuer la conversation ou le récit commencé avec le
peu de seigneurs qui la suivaient, et qui comme elle, et moi avec eux,
regagnaient promptement aussi ce si peu de terrain qu'on avait laissé
perdre. Je parlerai du Mail à part tout â l'heure.

On voit aisément, par le détail des journées du roi et de la reine
d'Espagne, qu'il ne restait pas même vestige des anciennes étiquettes de
cette cour, qu'elle était tombée à rien, et que les seigneurs n'avaient
plus que des instants de passage à pouvoir se montrer, mais qu'il n'y en
avait plus aucun pour les dames, de conseil et de travail qu'avec un
seul ministre, et que presque toutes les charges de la cour étaient
anéanties, ainsi que la distinction des pièces par degrés de dignité, où
chacun connaissait et se tenait dans sa mesure, et attendait avec ses
pareils à voir le roi. La charge de sommelier du corps, l'une des trois
charges par excellence, et celle des gentilshommes de la chambre, sans
autorité et sans fonction quelconque, n'étaient plus que des noms vains,
et leurs clefs une montre entièrement inutile. Aussi plusieurs d'eux ne
venaient guère au palais, et quoique le marquis de Montalègre, sommelier
du corps, fût aussi capitaine des hallebardiers, rien n'était plus rare
que de l'y rencontrer. Il ne restait au majordome-major que
l'honorifique de cette grande charge, encore borné à sa place auprès du
roi, ou aux chapelles à la tête des grands, et l'autorité `sur les
provisions de bois, de charbon, des caves et des cuisines\,; ces
dernières encore fort diminuées, parce que le roi mangeait toujours de
chez la reine, et jamais de chez lui\,; et il lui restait encore
quelques débris à l'égard des ordres pour les fêtes, encore assez
bornés, quelques rares cérémonies, et sur lés logements dans les
voyages, ce qui était encore plus rare, enfin sur là réception des
ambassadeurs et des autres étrangers distingués à qui le roi en voulait
faire. Les majordomes de semaine étaient sous lui dans les mêmes
privations. Le grand écuyer, seul des trois charges, n'avait presque
rien perdu, parce que toutes ses fonctions n'étaient que dans le dehors,
et le premier écuyer de même. Le patriarche des Indes non plus, dont les
fonctions ne s'étendaient que sur la chapelle, et à dire le
\emph{benedicite} ou les \emph{garces} quand sans contrainte il se
trouvait au dîner du roi. Le capitaine des hallebardiers n'avait jamais
eu de fonction personnelle, comme a ici le capitaine des Cent-Suisses,
sinon de prendre l'ordre, quand sans contrainte il se trouve quand le
roi le donne. Les capitaines des gardes du corps et leurs compagnies, et
les deux colonels des régiments des gardes, créés en même temps, eurent
toujours le même service qu'ils ont ici.

Ce fut la princesse des Ursins qui peu {[}à peu{]} abolit les conseils
où le roi assistait, les étiquettes du palais et les fonctions des
charges, pour tenir le roi enfermé avec la feue reine et elle, et ôter
tout moyen de lui pouvoir parler et d'en approcher, et pareillement aux
dames, à l'égard de la reine. Aussi prit-elle toujours bien garde au
choix qu'elle faisait des dames du palais, des \emph{señoras de honor}
et des caméristes, et ces deux dernières classes elle les avait remplies
tant qu'elle avait pu d'Irlandaises et d'autres étrangères. Depuis
M\textsuperscript{me} des Ursins, l'enfermerie du roi et {[}de{]} la
nouvelle reine continua également, et les étiquettes et les charges ne
se relevèrent plus. La camarera-mayor qui lui succéda, n'eut plus aucun
particulier avec la reine, toujours enfermée avec le roi, et fut réduite
comme le majordome-major de la reine à la toilette et aux repas.

Mais puisque je reparle ici des charges, je crois devoir réparer un
oubli que je crois m'être échappé sur le grand et sur le premier écuyer.
C'est que dès que le roi est dehors, s'il mange sur l'herbe ou dans un
village, non pas en voyage, mais chasse ou promenade, s'il boit même
seulement un coup, s'il veut se laver les mains, s'il prend un manteau
ou un surtout, ou le quitte\,; si même il change de chemise, et par
conséquent se déshabille et se rhabille, le grand écuyer le sert et le
premier écuyer, et celui-là ôte au sommelier du corps toutes ses
fonctions, même en sa présence, et celui-ci de même aux gentilshommes de
la chambre, non au sommelier, ce qui fait que le sommelier et les
gentilshommes de la chambre ne sont pas curieux de suivre le roi dehors.

Parlons maintenant de la chasse, de l'Atocha et du Mail.

\hypertarget{chapitre-iv.}{%
\chapter{CHAPITRE IV.}\label{chapitre-iv.}}

~

{\textsc{Chasse.}} {\textsc{- L'Atoche.}} {\textsc{- Impudence
monacale.}} {\textsc{- Le Mail.}} {\textsc{- Vie ordinaire de Madrid.}}
{\textsc{- Recao\,; ce que c'est.}} {\textsc{- Usage dans les visites.}}
{\textsc{- Vie des gens employés dans les affaires.}} {\textsc{-
Politesse et dignité des Espagnols.}} {\textsc{- Mesures pour la
grandesse et la Toison.}} {\textsc{- Lettres de M. le duc d'Orléans au
roi d'Espagne, et du cardinal Dubois à Grimaldo pour ma grandesse, d'une
telle faiblesse, que Grimaldo ne voulut pas remettre au roi celle de M.
le duc d'Orléans, ni lui parler de celle du cardinal Dubois.}}

~

La chasse était le plaisir du roi de tous les jours, et il fallait qu'il
fût celui de la reine. Mais cette chasse était toujours la même. Leurs
Majestés Catholiques me firent l'honneur, fort singulier, de m'ordonner
de m'y trouver une fois, et j'y allai dans mon carrosse. Ainsi je l'ai
bien vue, et qui en a vu une les a vues toutes. Les bêtes noires et
rousses ne se rencontrent point dans les plaines. Il faut donc les
chercher vers les montagnes, et ces pays sont trop âpres pour y courre
le cerf, le sanglier et d'autres bêtes comme on fait ici et ailleurs.
Les plaines mêmes sont si sèches, si dures, si pleines de crevasses
profondes, qu'on n'aperçait que de dessus le bord, que les meilleurs
chiens courants ou lévriers seraient bientôt rendus après les lièvres,
et auraient les pieds écorchés, même estropiés pour longtemps.
D'ailleurs tout y est si plein d'herbes fortes que les chiens courants
ne tireraient pas grand secours de leur nez. Tirer en volant, il y avait
longtemps que le roi avait quitté cette chasse, et qu'il ne montait,
plus à cheval\,; ainsi les chasses se bornaient à des battues.

Le duc del Arco, qui par sa charge de grand écuyer avait l'intendance de
toutes les chasses, choisissait le lieu où le roi et la reine devaient
aller. On y dressait deux grandes feuillées, adossées l'une à l'autre,
presque fermées, avec force espèce de fenêtres larges et ouvertes
presque à hauteur d'appui. Le roi, la reine, le capitaine des gardes en
quartier et le grand écuyer, et quatre chargeurs de fusils, étaient
seuls dans la première, avec une vingtaine de fusils et de quoi les
charger. Dans l'autre feuillée, le jour que je fus à la chasse, étaient
le prince des Asturies venu dans son carrosse à part avec le duc de
Popoli et le marquis del Surco, aussi dans cette feuillée le marquis de
Santa-Cruz, le duc Giovenazzo, majordome-major et grand-écuyer de la
reine, Valouse, deux ou trois officiers des gardes du corps et moi,
force fusils, et quelques hommes pour les charger. Une seule dame du
palais de jour suivait tour à tour la reine, dans un autre carrosse,
toute seule, duquel elle ne sortait point, et y portait pour sa
consolation un livre et quelque ouvrage, car personne de la suite n'en
approchait. Leurs Majestés et cette suite faisaient le chemin à toutes
jambes, avec des relais de gardes et de chevaux de carrosse, parce qu'il
y avait au moins trois ou quatre lieues à faire, qui valent au moins le
double de celles de Paris à Versailles. On mettait pied à terre aux
feuillées, et aussitôt on emmenait les carrosses, la pauvre damé du
palais et tous les chevaux hors de toute vue, fort loin, de peur que ces
équipages n'effarouchassent les animaux.

Deux, trois, quatre cents paysans commandés avaient fait dès la nuit des
enceintes, et des huées dès le grand matin, au loin pour effrayer les
animaux, les faire lever, les rassembler autant qu'il était possible, et
les pousser doucement du côté des feuillées. Dans ces feuillées, il ne
fallait pas remuer ni parler le moins du monde, ni qu'il y eût aucun
habit voyant, et chacun y demeurait debout, en silence. Cela dura bien
une heure et demie d'attente, et ne me parut pas fort amusant. Enfin
nous entendîmes de loin de grandes huées, et bientôt après nous vîmes
des troupes d'animaux passer à reprise à la portée et à demi-portée de
fusil de nous, et tout aussitôt le roi et la reine faire beau feu. Ce
plaisir ou cette espèce de boucherie dura plus de demi-heure à voir
passer, tuer, estropier cerfs, biches, chevreuils, sangliers, lièvres,
loups, blaireaux, renards, fouines sans nombre. Il fallait laisser tirer
le roi et la reine qui, assez souvent, permettaient au grand écuyer et
au capitaine des gardes de tirer\,; et comme nous ne savions de quelle
main partait le feu, il fallait attendre que celui de la feuillée du roi
se fût tu, puis laisser tirer le prince, qui souvent n'avait plus sur
quoi, et nous encore moins. Je tuai pourtant un renard, à la vérité un
peu plus tôt qu'il n'était à propos, dont un peu honteux, je fis des
excuses au prince des Asturies, qui s'en mit à rire et la compagnie
aussi, moi après à leur exemple, et tout cela fort poliment. À mesure
que les paysans s'approchent et se resserrent, la chasse s'avance, et
elle finit quand ils viennent tout près des feuillées, huant toujours,
parce qu'il n'y a plus rien derrière eux. Alors les équipages
reviennent, et les deux feuillées sortent et se joignent, on apporte les
bêtes tuées devant le roi. On les charge après derrière les carrosses.
Pendant tout cela, la conversation se fait, qui roule sur la chasse. On
emporta ce jour-là une douzaine de bêtes et plus, et quelques lièvres,
renards et fouines. La nuit nous prit peu après être partis des
feuillées. Voilà le plaisir de Leurs Majestés Catholiques tous les jours
ouvriers. Les paysans employés sont payés, et le roi leur fait donner
encore quelque chose assez souvent, en montant en carrosse.

Notre-Dame d'Atocha ou l'Atoche, comme on l'appelle le plus
ordinairement pour abréger, est une image miraculeuse de la sainte
Vierge, dans la riche chapelle d'une église, d'ailleurs assez ordinaire,
d'un vaste et superbe couvent de dominicains hors de Madrid, mais à
moins d'une portée de fusil des dernières maisons, et joignant le bout
du parc du palais du Buen-Retiro, qui enferme aussi un beau et grand
monastère de hiéronymites, dont l'église sert de chapelle à ce palais,
d'où on y va, à couvert, de partout, ainsi que dans le monastère.
L'Atoche est tellement la grande dévotion de Madrid, et de toute la
Castille, que c'est devant cette image que s'offrent les voeux, les
prières, les remerciements publics pour les nécessités et les
prospérités du royaume, et dans les cas de maladie périlleuse du roi et
de sa guérison. Le roi n'entreprend jamais de vrai voyage, et cela
depuis un temps immémorial, qu'il n'aille en cérémonie faire ses prières
devant cette image, ce qui ne s'appelle point autrement qu'aller prendre
congé de Notre-Dame d'Atocha, et y va de même dès qu'il est de retour.
Les richesses de cette image en or, en pierreries, en dentelles, en
étoffes somptueuses, sont prodigieuses. C'est toujours une des plus
grandes et des plus riches dames qui a le titre de sa dame d'atours, et
c'est un honneur fort recherché, quoique très cher, car il lui en coûte
quarante mille livres et quelquefois cinquante mille tous les ans pour
la fournir de dentelles et d'étoffes qui reviennent bientôt au profit du
couvent. Je ne m'arrêterai pas aux réflexions sur ces dévotions. La
duchesse d'Albe, qu'on a vue à Paris ambassadrice d'Espagne\,; l'était
alors. Je ne sais qui lui succéda dans cet emploi. Elle mourut peu de
jours après mon arrivée à Madrid.

Il y a plusieurs jours, dimanches ou fêtes, quelquefois même des jours
ouvriers de fêtes non fêtées, où il y a sur le soir un salut à l'Atoche,
qui est fort fréquenté, et où le roi et la reine allaient souvent sans
cérémonie par les de hors de Madrid, et sans entrer dans l'église ni
dans le couvent. Il y a au dehors un médiocre corps de logis sans cour.
On monte en dedans une quinzaine de marches, et on trouve trois pièces
dont celle du milieu est la plus grande. Une longue tribune règne sur
l'église dans laquelle on entre des deux secondes pièces. Celle du roi
est séparée dans la même longueur par une cloison\,; la famille royale
et le service le plus indispensable s'y met\,; dans l'autre toute leur
suite\,; ce qui est en charge médiocre demeure dans la pièce du milieu,
et le bas domestique dans celle d'entrée, desquels tous va qui veut dans
l'église\,; en sorte que dans la tribune de la suite, il n'y entre
qu'elle et le peu de seigneurs principaux courtisans, qui, les uns ou
les autres y viennent faire leur cour, dont la plupart même ne sont pas
dans cet usage. J'y allais presque toujours attendre Leurs Majestés un
moment avant qu'elles arrivassent. Je n'y ai jamais vu qu'une douzaine,
toujours les mêmes, de ceux qui n'y étaient pas obligés par leurs
fonctions, et jamais plus de trois ou quatre à la fois. Les dames du
palais et les señoras de honor y suivaient la reine, plusieurs, mais non
pas toutes, et si la reine allait de là au Mail, il n'en restait qu'une
dame du palais\,; toutes les autres dames et la camarera-mayor s'en
retournaient. Trois ou quatre dominicains, des premiers du couvent, y
recevaient Leurs Majestés et les voyaient partir, qui leur disaient
toujours quelque chose en s'arrêtant à eux, et à ceux qu'elles
trouvaient dans ces pièces, avant d'entrer dans la tribune et en en
sortant.

Je ne vis jamais moines si gros, si grands, si grossiers, si rogues.
L'orgueil leur sortait par les yeux et de toute leur contenance. La
présence de Leurs Majestés ne l'affaiblissait point, même en leur
parlant\,; je dis pour l'air, les manières, le ton, car ils ne parlaient
qu'espagnol que je n'entendais pas. Ce qui me surprit, à n'en pas croire
mes yeux la première fois que je le vis, fut l'arrogance et
l'effronterie jusqu'à la brutalité avec laquelle ces maîtres moines
poussaient leurs coudes dans le nez de ces dames, et dans celui de la
camarera-mayor comme des autres, qui, toutes à ce signal, leur faisaient
une profonde révérence, baisaient humblement leurs manches, redoublaient
après leurs révérences, sans que le moine branlât le moins du monde, qui
rarement après leur disait quelque mot d'un air audacieux, et sans
marquer la civilité la plus légère, à quoi, lorsque cela arrivait, ces
dames répondaient le plus respectueusement du monde, à leur ton et à
toute leur contenance. J'ai vu quelquefois quelque seigneur leur baiser
aussi la manche, mais comme à la dérobée, d'un air honteux et pressé,
mais jamais les moines la présenter à pas un d'eux. Quoique cette rare
cérémonie se renouvelât toutes les fois que le roi allait à l'Atoche,
elle me surprit toujours, et je ne pus m'y accoutumer.

La tribune donnait également en face de la chapelle de Notre-Dame et du
grand autel\,; le saint-sacrement était dans le tabernacle de l'un et de
l'autre, et si alors il était exposé, ce qui n'arrivait pas toujours,
c'était à l'autel de Notre-Dame, très magnifiquement et avec une
infinité de lumière. Il l'était fort haut\,; et pour donner la
bénédiction il descendait et remontait après par une machine cachée
derrière l'autel. Cela me parut un peu machine d'opéra bien déplacée.
Quand le saint-sacrement n'était pas exposé, il n'y avait point de
bénédiction. Les moines chantaient dans leur choeur, qu'on ne pouvait
voir, les litanies de la Vierge et d'autres prières d'un ton lent,
triste et très lugubre, et cela durait demi-heure ou trois quarts
d'heure. Ce salut était très commode pour voir Leurs Majestés et leur
faire sa cour.

De l'Atoche il était fort ordinaire que le roi entrât dans le pare du
Retira, et il y était suivi par les mêmes qui s'étaient trouvés au
salut. On mettait pied à terre au Mail, beau, large, extrêmement long.
Le roi y jouait avec le grand et le premier écuyer, le marquis de
Santa-Cruz ou quelque autre seigneur, et y jouait toujours trois tours
complets d'aller et venir, la reine toujours à son côté, et quand il le
fallait {[}elle{]} changeait de place pour être toujours à sa gauche. Ce
Mail était extrêmement agréable par les charmes qu'elle y répandait. Il
n'y avait que des seigneurs dans le Mail, et la dame du palais qui la
suivait\,; tout le reste se tenait des deux côtés sans y entrer. On
suivait le roi et la reine qui faisait la conversation avec les uns et
les autres, avec une aimable familiarité, et amusait de temps en temps
le roi par les plaisanteries qu'elle faisait, dont Valouse
s'embarrassait fort ordinairement et en augmentait la gaieté. Elle
attaquait fort aussi le duc del Arco, prenait plaisir à le mettre aux
mains avec Santa-Cruz, et faisait en sorte qu'ils s'en disaient souvent
de bonnes. Le grand écuyer ne laissait pas de se rebecquer quelquefois
contre la reine, librement, et plaisamment quelquefois. Si quelque
joueur faisait une pirouette ou quelque mauvais coup, c'était de rire et
de lui tomber sur le corps, en sorte que ce temps du Mail paraissait
toujours trop court. Le roi, toujours grave, souriait\,; quelquefois un
mot tout court et rare. Il jouait très bien et de bonne grâce, et la
reine l'admirait fort. À la fin du dernier tour, les carrosses venaient
au bout du Mail, et on s'en retournait. De la mi-février à la mi-avril
on laissait reposer et repeupler les animaux\,; il n'y avait point de
chasse, et le Mail allongé d'un peu de promenade, dans le même parc
quelquefois, en remplissait un peu le vide, presque tous les jours.

La vie de Madrid était de deux sortes pour les personnes sans
occupation\,: celle des Espagnols et celle des étrangers, je dis
étrangers établis en Espagne. Les Espagnols ne mangeaient point,
paressaient chez eux, et avaient entre eux peu de commerce, encore moins
avec les étrangers\,; quelques conversations, par espèce de sociétés de
cinq ou six chez l'un d'eux, mais à porte ouverte, s'il y venait de
hasard quelque autre. J'en ai trouvé quelquefois en faisant des visites.
Ils demeuraient là trois heures ensemble à causer, presque jamais à
jouer. On leur apportait du chocolat, des biscuits, de la mousse de
sucre, des eaux glacées, le tout à la main. Les dames espagnoles
vivaient de même entre elles. Dans les beaux jours le cours était assez
fréquenté dans la belle rue, qui conduit au Retiro, ou en bas sous des
arbres entre quelques fontaines, le long du Mançanarez. Ils voyaient et
rarement les étrangers en visite, et ne se mêlaient point avec eux. À
l'égard de ceux-ci, hommes et femmes mangeaient et vivaient à la
française, en liberté, et se rassemblaient fort entre eux en diverses
maisons. La cour montrait quelquefois que cela n'était pas de son goût,
et s'en lassa à la fin, parce qu'il n'en était autre chose. De paroisses
ni d'office canonical, c'est ce qui ne se fréquentait point\,; mais des
saluts, des processions, et la messe basse dans les couvents. On
rencontre par les rues beaucoup moins de prêtres et de moines qu'à
Paris, quoique Madrid soit plein de couvents des deux sexes.

L'usage est que les dames envoient de loin à loin savoir des nouvelles
des seigneurs fort distingués. Cela s'appelle un
\emph{recao}\footnote{On dit ordinairement \emph{recado}, mot qui
  signifie \emph{compliment que l'on fait faire à quelqu'un}.}\emph{\,;}
et le même usage veut que le lendemain, au moins très peu après, celui
qui a reçu ce \emph{recao} aille en remercier la dame. Cela m'est
souvent arrivé, et souvent aussi je trouvais la dame seule. Je voyais
souvent, indépendamment des \emph{recao}, la comtesse de Lemos et la
duchesse douairière d'Ossone\,: la première, soeur du duc de
Medina-Sidonia, l'autre, fille du dernier connétable de Castille\,;
toutes deux magnifiquement logées et superbement meublées. Cette
dernière aimait fort M. le duc d'Orléans qui l'avait beaucoup vue à
Madrid. Il me l'avait fort recommandée, et m'avait chargé de lui faire
ses compliments. Elle avait chez elle une salle d'opéra complète, moins
large, un peu moins longue, mais bien autrement belle que celle de
Paris, et singulièrement commode pour les communications des loges de
l'amphithéâtre et du parterre. Ces deux dames n'auraient point paru
désagréables ici, parlaient bien François, et avaient, surtout la
dernière, une conversation extrêmement agréable, et toutes deux l'air de
très grandes dames, ainsi qu'elles l'étaient en effet. Je voyais aussi
plusieurs autres dames.

La première que je visitai en arrivant à Madrid fut la marquise de
Grimaldo. On ne m'avait point averti de la façon de recevoir en usage
pour les dames. Je la trouvai au fond d'un cabinet en face de la porte,
avec quelque compagnie d'hommes et de femmes, des deux côtés. Elle se
leva dès qu'elle me vit entrer, mais sans démarrer d'un pas, et
s'inclina, lorsque j'approchai, comme font les religieuses, qui est leur
révérence. Quand je me retirai, elle en fit autant, sans avancer d'une
ligne, ni aucune excuse de ce qu'elle n'en faisait pas davantage\,:
c'est l'usage du pays. Pour les hommes, ils viennent plus ou moins loin
au-devant, et reconduisent de même suivant les conditions des gens, car
tout est réglé et certain, et néanmoins n'ôte pas l'importunité des
compliments. De part et d'autre on s'en fait bien plus qu'ici pour
empêcher ou pour prolonger la conduite. Chacun des deux sait bien
jusqu'où elle doit aller, que rien ne l'abrégera ni ne l'étendra, que
tout ce qui se dit de part et d'autre est parfaitement inutile, que l'un
serait blâmé, l'autre justement offensé si la conduite ne
s'accomplissait pas en entier telle qu'elle doit être. Tout cela
n'empêche point qu'on ne s'arrête à tout moment, et que ces compliments
ne durent la moitié du temps de la visite\,; cela est insupportable (on
parle ici des visites de cérémonie). Mais quand la familiarité est
établie, on vit ensemble à peu près comme on fait ici. En aucun cas les
femmes ne vont voir les hommes\,; mais elles vont chezeux lorsqu'elles
en sont priées pour une musique ou un bal ou un feu d'artifice ou
quelque chose de semblable. Et si alors, outre les rafraîchissements, il
y a un souper, elles se mettent à table et mangent avec la compagnie.

Les gens employés sont tout à fait séquestrés du commerce, et dispensés
de faire des visites, hors certains cas particuliers, ou de gens fort
distingués. J'en excepte les visites de cérémonie, aux ambassadeurs, et
autres telles personnes, par exemple cardinaux, voyageurs distingués que
le roi fait recevoir par un de ses majordomes, un vice-roi ou un général
d'armée de retour, ou celui qui revient d'une des premières ambassades.
Mais ces visites ne se redoublent pas sans nécessité d'affaires, si
l'amitié ou une considération supérieure n'y donne occasion. Aussi ne
les va-t-on guère voir que pour affaires, ou occasions semblables, et
leur rendre leurs visites, excepté leurs amis particuliers ou leurs
familiers. Ces derniers les voient quelquefois chez eux, mais pas
toujours, jamais les autres, quand ce sont des secrétaires d'État, parce
qu'ils ne sont chez eux que pour le moment du dîner, et le soir pour
celui du souper, après lequel ils se retirent avec leurs femmes et leurs
enfants, jusqu'à ce qu'ils se couchent.

Leurs journées se passent chacun dans leur \emph{cavachuela}, et c'est
où on les va trouver. De la cour du palais on voit des portes à
rez-de-chaussée. On y descend plusieurs marches, au bas desquelles on
entre en des lieux spacieux, bas, voûtés, dont la plupart n'ont point de
fenêtres. Ces lieux sont remplis de longues tables et d'autres petites,
autour desquelles un grand nombre de commis écrivent et travaillent sans
se dire un seul mot. Les petites sont pour les commis principaux qui
chacun travaillent seuls sur leurs tables. Ces tables ont des lumières
d'espace en espace assez pour éclairer dessus, mais qui laissent ces
lieux fort obscurs. Au bout de ces espèces de caves est une manière de
cabinet un peu orné, qui a des fenêtres sur le Mançanarez et sur la
campagne, avec un bureau pour travailler, des armoires, quelques tables
et quelques sièges. C'est la cavachuela particulière du secrétaire
d'État, où il se tient toute la journées et où on le trouve toujours.

Celle de Grimaldo était gaie par la vue de deux fenêtres, assez petite,
et voûtée comme les autres, dont il n'était séparé que par la porte\,;
en sorte qu'il n'avait qu'à sonner, un commis entrait et il donnait ses
ordres sans attendre et sans interrompre son travail\,; et comme il
était toujours dans sa cavachuela, les commis demeuraient aussi
assidûment dans les leurs, sous les yeux des premiers commis, et n'en
sortaient, pour dîner et le soir pour se retirer, qu'en même temps que
le secrétaire d'État qui les voyait, en passant, et les y retrouvait en
venant de dîner. Que le roi fût au palais ou hors de Madrid, même des
temps considérables, c'était toujours la même assiduité dans les
cavachuela. Grimaldo, qui suivait toujours le roi, demeura à Madrid
pendant un voyage de Balsaïm de huit ou dix jours. J'eus affaire à lui
pendant cette absence\,; je dirai ailleurs de quoi il s'agissait. Je le
trouvai dans sa cavachuela, comme si le roi eût été dans le palais.
Grimaldo ne laissait pas de venir assez souvent chez moi, même sans
aucune affaire et d'y venir dîner familièrement aussi, sans prier,
amenant ou amené par le duc de Liria ou le prince de Masseran, ou le
marquis de Lede, ou quelque autre de ses amis, quelquefois le duc del
Arco, quelque dimanche que ce seigneur en avait le temps. Si on
proposait de mener cette vie à nos secrétaires d'État, même à leurs
commis, ils seraient bien étonnés, et je pense aussi bien indignés.

À l'égard de ceux qui étaient des différents conseils qui subsistaient,
on les voyait chez eux lorsqu'on y avait affaire\,; ils y travaillaient,
et les cavachuela n'étaient que pour les secrétaires d'État et leurs
commis. Il faut dire ici que rien n'égale la civilité, la politesse
noble et la prévenance attentive des Espagnols, lorsqu'on le mérite par
les manières qu'on a avec eux\,; comme il n'y a personne aussi nulle
part qui se sente davantage, et qui le fasse mieux et plus
dédaigneusement sentir, quand ils ont lieu de croire qu'on n'en use pas
à leur égard comme on doit. Je dis quand ils ont lieu, car ils sont par
grandeur éloignés de la pointille et de la vétille, et passent aisément
mille choses aux étrangers qui ignorent et qui n'ont point l'air de
gloire et de prétendre. C'est ce que Maulevrier et moi avons sans cesse
expérimenté d'eux, depuis le plus grand seigneur jusqu'aux moindres
personnes, mais en deux manières en tout extrêmement différentes.

Il est temps enfin de reprendre le fil que tant de descriptions et
d'explications peu connues jusqu'à présent, mais curieuses, ont
interrompu. On a vu en son ordre le motif qui m'avait fait souhaiter
l'ambassade d'Espagne\,: c'était la grandesse pour mon second fils et
brancher ainsi ma maison. Ce qui ne m'eût jamais conduit en Espagne,
mais concomitance que je ne voulais pas négliger sans en faire de
principal, était une Toison d'or pour mon fils aîné, afin qu'il
remportât de ce voyage un agrément qui, à son âge, était une décoration.
J'étais parti de Paris en toute liberté de m'aider de tout ce que je
pourrais à ces égards, et avec promesse de la demande expresse de la
grandesse au roi d'Espagne par M. le duc d'Orléans, d'y interposer même
le nom du roi, et des lettres les plus fortes du cardinal Dubois au
marquis de Grimaldo et au P. Daubenton. J'en parlai à l'un et à l'autre
une fois à Madrid, au milieu du tourbillon d'affaires, de cérémonial et
des réjouissances, et j'en avais été reçu à souhait. Sur tout ce qui
n'était point constitution les jésuites se louaient de moi, et ils en
avaient très bien informé le P. Daubenton. Ils avaient encore à compter
avec moi pour longtemps, suivant, toutes apparences. Au fond peu leur
importait d'un grand d'Espagne François\,; mais il ne leur était pas
indifférent que j'eusse lieu de croire qu'ils eussent contribué à me
faire obtenir ce que je désirais.

Grimaldo était droit et vrai\,; il s'affectionna à moi de très bonne foi
il m'en donna toutes sortes de preuves, dès ce premier séjour à Madrid,
comme j'en ai rapporté quelques-unes. Il voyait aussi une union des deux
cours par des mariages qui pouvaient influer sur les ministres. Son seul
point d'appui était le roi d'Espagne pour se maintenir dans le poste
unique qu'il occupait, si brillant et si envié. Il ne pouvait pas faire
de fondement solide sur la reine, comme on l'a vu ci-devant. Il voulait
donc s'appuyer de la France, tout au moins ne l'avoir pas contraire, et
il connaissait parfaitement la duplicité et les caprices du cardinal
Dubois. La cour d'Espagne, de tout temps si attentive sur M. le duc
d'Orléans, par tout ce qui s'était passé du temps de la princesse des
Ursins, et depuis pendant la régence, n'ignorait pas la confiance intime
et non interrompue que de tout temps ce prince avait en moi, ni ma façon
d'être avec lui. Ces sortes d'objets se grossissent de loin plus que
d'autres, et le choix qui avait été fait de moi pour cette singulière
ambassade y confirmait encore. Grimaldo put donc penser à s'assurer de
mon amitié et de mes services auprès de M. le duc d'Orléans dans les
occasions fortuites\,; et je ne crois pas me tromper en lui prêtant
cette politique pour me favoriser sur une grâce, au fond assez
naturelle, qui, par l'occasion unique de me la faire, ne tirait à nulle
conséquence, et qui\,; à son égard particulier, n'avait aucun
inconvénient.

Je m'ouvris aussi à Sartine, que mes égards pour lui si opposés aux
brutalités qu'il essuyait souvent de Maulevrier, et les bons offices que
je tâchais de lui rendre auprès de M. le duc d'Orléans et du cardinal
Dubois, m'avaient entièrement dévoué. On a vu qu'il était ami
particulier et familier de Grimaldo, et je me servis utilement de ce
canal pour faire passer à ce ministre ce qu'il eût été moins convenable
de lui dire moi-même. Je touchai encore un mot de cette grandesse et de
la Toison au P. Daubenton, la veille qu'il partit pour Lerma, et fis
pressentir en même temps Grimaldo sur la Toison par Sartine, et l'un et
l'autre avec succès.

Je regardais l'instant de la célébration du mariage comme l'époque
d'obtenir ce que je désirais, et je considérais que, étant passée sans
avoir obtenu, tout se refroidirait et deviendrait incertain et fort
désagréable. Je n'avais rien oublié dans ce court et premier séjour à
Madrid pour y plaire à tout le monde, et j'ose dire que j'y avais
d'autant mieux réussi, que j'avais tâché de donner du poids et du mérite
ma politesse, en gardant tout le milieu possible aux degrés et aux
mesures qu'elle devait avoir, à l'égard de chacun, sans prostitution et
sans avarice, et c'est ce qui me fit hâter de connaître tout ce que je
pus de la naissance, des dignités, des emplois, des alliances, de la
réputation, pour y proportionner ma façon de me conduire avec tant de
diverses personnes.

Mais il fallait le véhicule de la demande de M. le duc d'Orléans et des
lettres du cardinal Dubois. Je ne doutais pas de la volonté du régent,
mais beaucoup de celle de son ministre, et on a vu avec combien de
raison. Ces lettres, qui devaient au plus tard arriver à Madrid en même
temps que moi, se faisaient attendre inutilement d'ordinaire en
ordinaire. Ce qui redoublait mon impatience était que je les lisais
d'avance, et que je voulais avoir le temps de réfléchir et de me tourner
pour en tirer, malgré elles, tout le secours que je pourrais. Je
comptais parfaitement sur toute l'écorce d'empressement du cardinal
Dubois, qui, avec sa fausseté et sa mauvaise volonté, n'enfanterait que
des demi-choses, souvent plus nuisibles que rien du tout, et qui, ne
pouvant empêcher M. le duc d'Orléans d'écrire au roi d'Espagne, se
chargerait de faire la lettre, et la ferait au plus faible et au plus
mal, sans que M. le duc d'Orléans, livré à lui, sans appui contre lui,
moi absent, osât y rien changer. Cette opinion que j'eus toujours de ces
lettres fut ce qui me porta le plus à fortifier mes batteries en
Espagne, tant auprès du ministre et du confesseur qu'auprès de Leurs
Majestés Catholiques et de toute leur cour, pour me rendre assez
agréable au roi et à la reine pour leur inspirer le penchant de me faire
ces grâces\,; et à leur cour, sinon le désir, du moins une véritable
approbation qui pût revenir à leurs oreilles, et fortifier ce penchant
que je tâchais muettement de leur faire naître, d'autant qu'il était
difficile qu'on ne pensât à la cour, et par conséquent qu'il ne s'y
parlât, d'une grandesse pour moi dans une occasion si faite exprès, pour
ainsi dire, et à toutes les bontés et toutes les distinctions que
l'emploi dont j'étais honoré auprès de Leurs Majestés Catholiques
attirait sur moi de leur part.

Peu de jours avant d'aller à Lerma, je reçus des lettres du cardinal
Dubois sur mon affaire. Rien de plus vif ni de plus empressé, jusqu'à me
donner des conseils pour parvenir à mon but, et à me presser de l'aviser
de tout ce en quoi il y pourrait contribuer, et m'assurant que les
lettres de M. le duc d'Orléans et les siennes arriveraient à temps. À
travers le parfum de tant de fleurs, l'odeur du faux perçait par sa
nature. J'y avais compté, j'avais fait tout ce que la sagesse et la
mesure la plus honnête m'avait permis pour y suppléer. Je pris pour bon
toutes les merveilles que le cardinal m'écrivait, et je partis pour
Lerma bien résolu de cultiver de plus en plus mon affaire sans me
reposer sur les lettres qu'on me promettait, mais dans le dessein d'en
tirer tout le parti que je pourrais.

En arrivant à mon quartier, près de Lerma, je tombai malade, comme on
l'a vu ailleurs, et la petite vérole m'y retint quarante jours en exil.
Le roi et la reine, non contents de m'avoir envoyé M. Hyghens, comme je
l'ai dit ailleurs, pour ne me point quitter jour et nuit, voulurent être
informés deux fois par jour de mes nouvelles, et quand je fus mieux, me
firent témoigner sans cesse mille bontés, en quoi toute la cour les
imita. Je rends d'autant plus librement hommage à des bontés si
continuelles et si marquées, que je n'ai jamais pensé à les devoir qu'au
personnage que j'avais l'honneur de représenter, et dans des moments si
agréables. Pendant ce long intervalle, l'abbé de Saint-Simon entretint
commerce avec le cardinal Dubois, d'autant plus aisément que je n'avais
voulu me charger que de très peu d'affaires, et d'aucunes qui eussent
des queues capables de me retenir en Espagne plus que je n'aurais voulu.
En même temps il n'oublia pas d'entretenir aussi commerce avec le
marquis de Grimaldo et avec Sartine qui vint à Lerma, et de suivre mon
affaire.

Ces lettres tant promises se firent attendre jusque vers la fin de ma
quarantaine. À la fin elles arrivèrent, mais telles que je les avais
prévues. Le cardinal Dubois ne s'expliquait à Grimaldo que par contours
et circonlocutions\,; et si une phrase témoignait de l'empressement et
du désir, la suivante la détruisait par un air de respect et de
ménagement, protestant de ne vouloir que ce que le roi d'Espagne
voudrait lui-même, avec tous les assaisonnements nécessaires pour
anéantir ses offices sous le voile de ne pas se proposer de le presser
en rien, ni de l'importuner d'aucune chose. Il en disait autant à
Grimaldo pour lui, de sorte que ce bégaiement par écrit sentait fort le
galimatias d'un homme qui n'avait nulle envie de me servir, mais qui,
n'osant aussi manquer à sa promesse, mettait tout son esprit à tortiller
et à énerver le peu qu'il ne pouvait s'empêcher de dire. Cette lettre
n'était que pour Grimaldo, comme celle de M. le duc d'Orléans n'était
que pour le roi d'Espagne. Celle-ci fut encore plus faible que l'autre.
C'était comme un dessin au crayon que la pluie aurait presque effacé, et
où il ne paraissait plus d'ensemble.. Elle osait à peine mettre le doigt
sur la lettre, et se confondait aussitôt en respects, en retenue, en
mesure, à ne vouloir et à ne se proposer là-dessus que ce qui serait le
plus du goût du roi d'Espagne\,; en un mot, qui se retirait beaucoup
plus qu'elle ne s'avançait, et qui ne présentait qu'une sorte de manière
d'acquit, qui ne se pouvait refuser, mais dont le succès était fort
indifférent. Il est aisé de comprendre que ces lettres me déplurent
beaucoup. Quoique j'y eusse prévu toute la malice du cardinal Dubois, je
la trouvai au delà et bien plus à découvert que je ne l'avais imaginé.

Telles qu'elles fussent, si fallut-il s'en servir. L'abbé de Saint-Simon
écrivit à Grimaldo et à Sartine, et les envoya à ce dernier pour
remettre sa lettre et celles de la cour à Grimaldo, car je n'osais
encore écrire moi-même dans le ménagement qu'il fallait garder pour le
mauvais air. Sartine, à qui je n'avais pas fait confidence, encore moins
à Grimaldo, de la faiblesse à laquelle je m'attendais de ces
recommandations, tombèrent dans la dernière surprise à leur lecture. Ils
raisonnèrent ensemble, ils s'indignèrent, ils cherchèrent des biais pour
fortifier ce qui en avait tant de besoin\,; mais ces biais ne se
trouvant point, ils se consultèrent, et Grimaldo prit un parti hardi qui
m'étonna au dernier point, et qui aussi me mit fort en peine. Il conclut
que ces lettres me nuiraient sûrement plus qu'elles ne me serviraient\,;
qu'il fallait les supprimer, n'en jamais parler au roi d'Espagne, le
confirmer dans la pensée qu'il ferait, en m'accordant ces grâces, un
plaisir d'autant plus grand à M. le duc d'Orléans qu'il voyait jusqu'où
allait sa retenue de ne lui en point parler, et la mienne de ne point
les lui faire demander par Son Altesse Royale, quoiqu'il y eût tout lieu
de s'y attendre\,; tirer de là toute la force qu'auraient eue les
lettres, si leur style en avait eu\,; et qu'avec ce qu'il saurait y
mettre du sien, il me répondait de la grandesse et de la Toison, sans
faire mention aucune des lettres de M. le duc d'Orléans au roi
d'Espagne, et du cardinal Dubois à lui. Sartine, par son ordre, le fit
savoir à l'abbé de Saint-Simon, qui me le rendit\,; et après en avoir
raisonné ensemble avec Hyghens, qui connaissait le terrain aussi bien
qu'eux, et qui s'était vraiment livré à moi, je m'abandonnai aveuglément
à la conduite et à l'amitié de Grimaldo, dont on verra bientôt le plein
succès.

En racontant ici la façon très singulière par laquelle mon affaire
réussit, je suis bien éloigné d'en soustraire à M. le duc d'Orléans
toute la reconnaissance. S'il ne m'avait pas confié le double mariage, à
l'insu de Dubois et malgré le secret qu'il lui avait demandé précisément
pour moi, et cela dès qu'ils furent conclus, je n'aurais pas été à
portée de lui demander l'ambassade. Je la lui demandai sur-le-champ, en
lui en déclarant le seul but, qui était la grandesse pour mon second
fils, et sur-le-champ il me l'accorda, et me l'accorda pour ce but, et
pour m'aider de sa recommandation à y parvenir, et sous le dernier
secret, par rapport au dépit qu'en aurait Dubois, et se donner du temps
pour se tourner avec lui et lui faire avaler la pilule. Si je n'avais
pas eu l'ambassade de la sorte, elle m'aurait sûrement échappé, et alors
tombait de soi-même toute idée de grandesse, dont il n'y aurait plus eu,
ni occasion, ni raison, ni moyen. L'amitié et la confiance de ce prince
prévalut donc à l'ensorcellement que son misérable précepteur avait jeté
sur lui\,; et s'il céda depuis aux fourbes, aux manèges, aux folies que
Dubois employa dans la suite de cette ambassade pour me perdre et me
ruiner, et pour me faire manquer le seul objet qui m'avait fait la
désirer, il ne s'en faut prendre qu'à sa scélératesse, et à la
déplorable faiblesse de M. le duc d'Orléans, qui m'ont causé bien de
fâcheux embarras, et m'ont fait bien du mal, mais qui ont fait bien pis
à l'État et au prince lui-même. C'est par cette triste, mais trop vraie
réflexion que je finirai cette année 1721.

\hypertarget{chapitre-v.}{%
\chapter{CHAPITRE V.}\label{chapitre-v.}}

1722

~

{\textsc{Année 1722.}} {\textsc{- Échange des princesses (9 janvier).}}
{\textsc{- Usurpation des Rohan.}} {\textsc{- Ruses, artifices, manèges
du prince de Rohan, tous inutiles auprès du marquis de Santa-Cruz, qui
le force à céder sur ses chimères dans l'acte espagnol, dont j'ai la
copie authentique et légalisée.}} {\textsc{- Présents du roi aux
Espagnols, pitoyables.}} {\textsc{- Grands d'Espagne, espagnols, n'en
prennent point la qualité dans leurs titres, et pourquoi.}} {\textsc{-
Avances singulières que le cardinal de Rohan me fait faire de Rome\,;
leur motif.}} {\textsc{- Sottise énorme du cardinal de Rohan partant de
Rome.}} {\textsc{- Échange des princesses dans l'île des Faisans.}}
{\textsc{- Présents et prostitution de rang de la reine douairière
d'Espagne, à qui je procure un payement sur ce qui lui était dû.}}
{\textsc{- Je vais faire la révérence à Leurs Majestés Catholiques.}}
{\textsc{- Matière de cette audience.}} {\textsc{- Conte singulièrement
plaisant par où elle finit.}} {\textsc{- Le roi, la reine et le prince
des Asturies vont, comme à la suite du duc del Arco, voir la princesse à
Cogollos.}} {\textsc{- Je vais saluer la princesse à Cogollos, puis à
Lerma, à son arrivée.}} {\textsc{- Chapelle.}} {\textsc{- J'y précède
tranquillement le nonce, sans faire semblant de rien.}} {\textsc{- Rare
et plaisante ignorance du cardinal Borgia, qui célèbre le mariage, dont
la cérémonie extérieure est différente en Espagne.}} {\textsc{-
Célébration du mariage, l'après-dîner du 20 janvier.}} {\textsc{- Je
suis fait grand d'Espagne de la première classe, conjointement avec un
de mes fils à mon choix, pour en jouir actuellement l'un et l'autre\,;
et la Toison donnée à l'aîné, sans choix.}} {\textsc{- Je donne à
l'instant la grandesse au cadet.}} {\textsc{- Remerciement.}} {\textsc{-
Compliments de toute la cour.}} {\textsc{- Je me propose, sans en avoir
aucun ordre et contre tout exemple en Espagne, de rendre public le
coucher des noces du prince et de la princesse des Asturies\,; et je
l'exécute, et je l'obtiens.}} {\textsc{- Bonté et distinction sans
exemple du roi d'Espagne pour moi et pour mon fils aîné au bal, dont je
m'excuse par ménagement pour les seigneurs espagnols.}} {\textsc{-
Mesures que je prends pour éviter que le coucher public ne choque les
Espagnols.}} {\textsc{- Vin et huile détestablement faits en Espagne,
mais admirablement chez les seigneurs.}} {\textsc{- Jambons de cochons
nourris de vipères, singulièrement excellents.}} {\textsc{- Évêques
debout au bal, en rochet et camail.}} {\textsc{- Cardinal Borgia n'y
paraît point.}} {\textsc{- Vélation\,; ce que c'est.}} {\textsc{- J'y
précède encore le nonce, sans faire semblant de rien.}} {\textsc{-
Maulevrier n'y paraît point, parti furtivement dès le matin de son
quartier pour Madrid, qui en est fort blâmé.}} {\textsc{- Conduite
réciproque entre lui et moi pendant les jours du mariage.}} {\textsc{-
Étrange conduite et prétentions de La Fare.}} {\textsc{- Ma conduite à
cet égard.}}

~

L'année 1722 commença par l'échange des princesses, futures épouses du
roi et du prince des Asturies, dans l'île des Faisans, de la petite
rivière de Bidassoa qui sépare les deux royaumes, où on avait construit
une maison de bois à cet effet, mais toute simple en comparaison de
celle qui, au même endroit, avait servi en 1659 aux célèbres conférences
du cardinal Mazarin avec don Louis de Haro, premiers ministres de France
et d'Espagne, à la signature de la paix dite des Pyrénées, et depuis à
l'entrevue du roi et de la reine sa mère avec le roi d'Espagne Philippe
IV, frère de la reine-mère.

J'avais prévu toute la coupable complaisance du cardinal Dubois pour les
folles chimères des Rohan, et que le prince de Rohan n'avait voulu être
chargé de l'échange de la part du roi que pour les fortifier de ce qu'il
se proposait d'y usurper. Le rang de la maison de Rohan, acquis ou
arraché pièce à pièce, ne remontait pas plus haut que le dernier règne.
Il était sans titre et sans prétexte que la volonté du feu roi. Il n'y
avait eu jamais de reconnaissance de la qualité de prince\,; car on a vu
en son lieu que le feu roi avait mis ordre à ce que sa signature
d'honneur, apposée aux contrats de mariage, n'autorisât en rien les
titres que chacun y prenait. Le temps n'était pas venu pour le cardinal
Dubois de se moquer des promesses qu'il avait faites au cardinal de
Rohan en l'envoyant à Rome presser son chapeau, et bien auparavant pour
se servir de lui à cet usage par son crédit et ses amis. Un homme de la
condition et du caractère de Dubois fait aisément litière de ce qui ne
lui coûte rien et de ce qui lui est même momentanément utile. Il
dominait en plein le régent, et ce prince aimait à tout brouiller, et à
favoriser les divisions et le désordre. Le cardinal Dubois, à mesure
qu'il était monté, s'était défait des emplois subalternes qui lui
avaient servi de degrés. Ainsi, dès qu'il fut secrétaire d'État, il
produisit le médecin, son frère, et lui céda sa charge de secrétaire du
cabinet du roi ayant la plume. Ce fut lui qui, en cette qualité, fut
chargé de faire pour la France les actes nécessaires à l'échange, comme
La Roche, secrétaire du cabinet du roi d'Espagne, ayant l'estampille, le
fut pour l'Espagne. Je n'eus donc pas peine à comprendre que Dubois
aurait ordre du cardinal son frère de faire en cette occasion tout ce
qui plairait au prince de Rohan, et ne pouvant parer l'usurpation que je
prévoyais, je voulus du moins empêcher qu'elle ne fût complète.

Je prévins donc à Madrid le duc de Liria sur l'\emph{Altesse} que le
prince de Rohan ne manquerait pas de se faire donner par Dubois dans
l'acte de l'échange, et sûrement de s'en faire un titre pour le
prétendre dans l'acte espagnol. Liria sentit comme moi toutes les
raisons de l'empêcher, et de les bien expliquer et inculquer au marquis
de Santa-Cruz, grand d'Espagne, et parfaitement espagnol, son ami
particulier. À la première mention, Santa-Cruz monta aux nues. Je lui en
parlai après, et il me promit bien de tenir le prince de Rohan si roide
et si ferme qu'il ne lui laisserait rien passer. Le duc de Liria fut
chargé de porter les présents du roi d'Espagne à sa future belle-fille
au lieu de l'échange. Je le sus avant le départ de Madrid, et je lui
rafraîchis tout ce que je lui avais dit sur les prétentions que le
prince de Rohan allait produire\,; et, outre que le marquis de
Santa-Cruz était bien résolu de ne les pas souffrir, le duc de Liria me
promit de le tenir de près, et d'avoir, à cet égard, toute la vigilance
possible.

Dès qu'on fut arrivé des deux côtés au lieu de l'échange, c'est-à-dire à
la dernière couchée des deux royaumes pour n'avoir plus qu'à passer dans
l'île pour la cérémonie\,; quand tout serait convenu, il fut d'abord
question de tout régler. L'acte en soi de l'échange, ni les qualités du
prince de Rohan, et du marquis de Santa-Cruz ne firent point de
difficulté, qualités dont je dirai un mot ensuite. Il n'y en eut point
même de la part du marquis de Santa-Cruz sur ces mots de l'acte
français, signés par un secrétaire du cabinet du roi\,: \emph{conduite
par le très excellent seigneur Son Altesse} le prince de Rohan\,; ce
n'était pas à Santa-Cruz à régler l'acte français. Mais quand de cet
acte le prince de Rohan voulut se faire un titre pour avoir l'Altesse
dans l'acte espagnol, le marquis de Santa-Cruz le rejeta avec tant de
hauteur, et une fermeté si décidée, que le prince de Rohan eut recours à
des \emph{mezzo termine}, devenus malheureusement chez nous si à la
mode. Il proposa de ne point prendre d'Altesse dans l'acte français si
Santa-Cruz se contentait de ne point prendre d'Excellence dans l'acte
espagnol, en sorte que tous deux éviteraient entièrement toute
qualification. Cela fut rejeté avec la même hauteur. Déchu de cet
expédient, Rohan fit dire à Santa-Cruz qu'en lui passant l'Altesse, il
la lui passerait aussi, s'il là voulait prendre, et que de cette façon
tout serait accommodé avec un grand avantage pour Santa-Cruz.
Santa-Cruz, avec son rire moqueur, répondit que Rohan et lui n'étaient
pas princes, et qu'il serait plaisant qu'ils imaginassent se faire
princes l'un l'autre, de leur seule autorité, en se passant mutuellement
l'Altesse, qui n'appartenait ni à l'un ni, à l'autre, et se moqua de la
proposition avec beaucoup de mépris. Le prince de Rohan, qui avait
compté l'attraper en l'éblouissant de l'Altesse, se trouva extrêmement
embarrassé et mortifié. Enfin, en sautant le bâton, il crut en retenir
un bout par une proposition spécieuse qui revint à la première c'était
de se contenter respectivement de leurs noms et de celui de leurs
emplois, sans nulle Altesse ni Excellence, ni Excellentissime Seigneur.
Mais cela fut encore refusé et traité de réchauffé.

Enfin, à bout de voie, Rohan se réduisit à une dernière ressource, dont
il espéra que le fond secret échapperait à Santa-Cruz. Ce fut que le
prince de Rohan ne prendrait ni Altesse, ni Excellence, ni
Excellentissime Seigneur, et qu'il consentirait que Santa-Cruz prit
l'Excellence et l'Excellentissime Seigneur. Mais ce prince par les appas
de sa mère avait affaire à un homme trop avisé pour donner dans ce
panneau. Santa-Cruz lui manda qu'il était las de tant de fantaisies, qui
retardaient l'échange depuis deux jours et le voyage des princesses, et
dont la plus longue durée, par des prétentions si déplacées, devenait
indécente par le retardement\,; qu'en deux paroles ils étaient tous deux
grands de leur pays, et dans la même commission, chacun de la part de
son maître, par conséquent égaux de tous points\,; par conséquent qu'il
ne souffrirait pas la plus légère ombre de différence entre eux deux
dans l'acte espagnol\,; qu'il lui déclarait donc qu'il y prendrait
l'Excellence et l'Excellentissime Seigneur, qui est le traitement de
tout temps établi pour les grands d'Espagne\,; que les ducs de France
ayant, depuis Philippe V, l'égalité avec eux, et les grands d'Espagne
l'égalité avec les ducs de France, il prétendait qu'il prît également
comme lui l'Excellence et l'Excellentissime Seigneur dans l'acte
espagnol\,; que c'était son dernier mot\,; qu'il n'écouterait plus
aucune sorte de proposition à cet égard\,; qu'il le priait de lui
envoyer sur-le-champ sa dernière résolution, sur laquelle il préparerait
tout pour achever la cérémonie de l'échange, ou il ferait partir, dès le
lendemain matin, l'infante pour aller attendre, en lieu plus commode que
celui où elle était, les ordres de Madrid, où il allait dépêcher un
courrier.

Cette réponse si précise accabla le prince Rohan. Il n'osa se commettre
à l'éclat qui le menaçait\,; il craignit la colère du roi et de la reine
d'Espagne, et qu'il ne lui en coûtât l'Altesse dans l'acte François. Il
céda donc tout court, et se consola par ce titre escroqué pour la
première fois dans un acte signé par un homme du roi. C'est de la sorte
que se bâtissent les titres de princes de nos gentilshommes français,
pièce à pièce, suivant le temps et les occasions qu'ils épient et qu'ils
saisissent aux cheveux.

L'échange se fit enfin le 9 janvier de cette année 1722\,; et après les
compliments réciproques et les présents du roi aux Espagnols, chaque
princesse et sa suite continua son voyage. Je passai une heure à Lerma
chez Santa-Cruz, et le Liria en troisième, où ils me contèrent tout ce
que je viens d'écrire, mais bien plus en détail, avec force gausseries
de Santa-Cruz sur la \emph{princerie}. Il se retint sur les présents.
Mais il ne put s'empêcher de me montrer le sien en souriant, ni moi d'en
hausser les épaules, sans nous parler d'un autre langage. En effet, ces
pierreries en petit nombre étaient pitoyables. Sur celui du personnage
principal de l'échange on peut juger de ce que furent les autres. Les
Espagnols s'en moquaient tout haut, et j'en mourais de honte. Ce n'était
pas l'occasion d'épargner cinquante mille écus qui répandus sur tous les
présents, les auraient rendus dignes du monarque qui les faisait. Mais
la canaille en retient toujours quelque coin, dans quelque élévation que
l'aveugle fortune la pousse.

Dès que nous fûmes de retour à Madrid, je priai La Roche de vouloir
bien, pour ma curiosité, m'expédier une copie des deux actes, l'un
français, l'autre espagnol, de l'échange, et de les signer pour les
certifier véritables. Il les expédia et signa, et me les envoya, et ils
sont actuellement sous mes yeux, dans le second portefeuille de mon
ambassade, à l'heure que j'écris. Je connais les mensonges et les
assertions hardies des gens à prétentions, et j'ai voulu avoir et
conserver un titre paré de l'Excellence, et non Altesse, du prince de
Rohan, dans l'acte espagnol, et de son égalité en tout et partout avec
le marquis de Santa-Cruz, malgré ses prétentions, ses diverses
propositions, ses artifices et ses ruses.

J'ai réservé un mot à dire sur les autres titres, ou pour mieux dire
qualités, qu'ils prirent et qui n'avaient point de difficulté. Les
grands espagnols ne prennent jamais dans leurs titres la qualité de
grands d'Espagne. S'il s'en trouve quelques-uns, ce n'est que bien peu,
et depuis Philippe V, à l'exemple des François. La raison de ne la point
prendre n'est qu'une rodomontade espagnole. Ils prétendent que leurs
noms doivent être si connus que leur grandesse ne peut manquer de l'être
en même temps qu'on entend leurs noms. Mais le fond est le même qui leur
fait cacher leur ancienneté avec tant de soin. En prenant la qualité de
grands d'Espagne, les actes d'eux ou de leurs pères feraient foi du
temps qu'ils auraient commencé à la prendre, et mettrait en évidence ce
qu'ils veulent soustraire à la connaissance, et c'est la vraie raison,
cachée sous la rodomontade, qui leur fait omettre la qualité qui fait
leur essence et leur rang, tandis qu'ils n'omettent aucune de leurs
charges, de leurs emplois, même de leurs commanderies dans les ordres de
Saint-Jacques de Calatrava, etc., qui sont communes à la plus petite
noblesse et à leurs propres domestiques actuels avec eux. Le prince de
Rohan, si désireux d'être duc et pair, malgré sa \emph{princerie}, et
dont l'habile mère disait qu'il n'y avait de solide que cette dignité,
qui ne se pouvait ôter comme les honneurs de prince, qui dépendaient
toujours d'un trait de plume, et qu'elle ne serait point contente
qu'elle n'en vit son fils revêtu\,; le prince de Rohan, dis-je, ravi d'y
être enfin parvenu, mais après la mort de sa mère, par la voie qu'on a
vue ici alors, voulut, sûr du vrai et du solide, y faire surnager sa
\emph{princerie}, comme je l'ai expliqué alors. Voyant donc Santa-Cruz
ne prendre point la qualité de grand d'Espagne, et prendre les autres
qu'il avait, {[}il{]} n'eut pas de peine à s'y conformer et à saisir
ainsi un air de négligence pour une chose qu'il avait si fortement
passionnée, et qu'il était si aise d'avoir mise dans sa maison et dans
sa branche.

Puisque les Rohan se trouvent sous ma plume, encore un petit mot sur le
cardinal, frère du prince de Rohan. Lui et son frère étaient les gens du
monde avec qui, de tout temps, j'avais eu le moins de commerce. Sans
division marquée, tout m'en avait toujours éloigné. Nos sociétés avaient
toujours été très différentes du temps du feu roi, et toujours depuis,
jusque-là même que le hasard ne nous faisait point nous rencontrer.
J'étais de la sorte avec eux lorsque le cardinal s'en alla à Rome. Il
n'y fut pas plutôt arrivé que les lettres que je recevais toutes les
semaines, comme je l'ai dit ailleurs, du cardinal Gualterio, ne furent
remplies que des éloges que le cardinal de Rohan lui faisait de moi, et
du désir extrême qu'il avait de pouvoir mériter quelque part en mon
amitié.

On ne peut être plus étonné que je le fus d'avances si fortes, si
continuelles, et auxquelles rien n'avait donné lieu. Je connaissais
assez le cardinal de Rohan pour être bien sûr que de pareilles démarches
ne pouvaient être fondées que sur des vues qu'il pouvait craindre que je
ne traversasse\,; et par cette raison, mes réponses polies et froides ne
furent pas faites de manière à entretenir ces compliments\,; mais ils
persévérèrent toutes les semaines, s'échauffèrent de plus en plus,
jusque-là que Gualterio s'entremit pour m'engager d'amitié avec le
cardinal de Rohan. Gualterio était trop sage et trop mesuré pour se
porter à cela de lui-même, et par les compliments directs qu'il ajoutait
du cardinal de Rohan pour moi, qui l'en chargeait en même temps, je ne
pus pas douter que ce ne fût lui qui faisait agir notre ami commun. Plus
les efforts redoublaient à découvert, plus ils m'étaient suspects. Mais,
venus jusqu'à ce point, ils m'embarrassaient, parce que je ne voulais
point de liaisons, encore moins d'engagements d'amitié avec un homme
dont les intérêts, les engagements, la conduite, se trouvaient en
opposition si entière avec les miens, et qu'il n'était pas possible de
ne pas répondre à tant d'empressement d'une façon convenable à la
naissance, à la dignité, et au personnage que faisait le cardinal de
Rohan. Je fis donc ce que je pus pour accorder toutes ces choses\,; mais
comme je n'ai jamais pu trahir mes sentiments, je crois que j'en vins
mal à bout, car après que cela eut duré pendant tout son séjour à Rome,
tout tomba dès qu'il en fut parti, sans que jamais il en ait été mention
depuis, et comme de chose non avenue. Le fait était que le cardinal
Dubois lui avait donné sa parole que, devenu cardinal par son secours,
il le ferait entrer dans le conseil en arrivant de Rome, et incontinent
après déclarer premier ministre. Le cardinal de Rohan, également dupe du
fripon et de sa propre ambition, donna en plein dans ce panneau dont un
enfant se serait gardé, parce qu'il était plus qu'évident que si le
cardinal Dubois se trouvait en pouvoir de faire un premier ministre, il
ne préférerait personne à lui-même, et se le ferait aux dépens de
quelque parole qu'il eût pu donner, dont, même sur les moindres choses,
il n'était rien moins qu'esclave.

Cette réflexion si naturelle n'atteignit point le cardinal de Rohan.
Persuadé, par son ambition, de la bonne foi d'un homme qui n'en eut
jamais aucune, il ne pensa qu'à ranger les obstacles qu'il pourrait
rencontrer. Il crut aisément qu'un premier ministre ne serait pas de mon
goût, bien moins encore un premier ministre cardinal, et qui se
prétendait prince. C'est ce qui l'engagea à toutes les avances, les
flatteries, les fadeurs dont il me fit accabler pendant tout son séjour
à Rome par le cardinal Gualterio, qu'il abandonna tout court quand il en
fut parti, parce qu'il en sentit apparemment l'inutilité. C'est aussi ce
qui précipita son retour le plus promptement qu'il lui fut possible,
après l'élection du pape, pour me gagner de la main, tandis que j'étais
encore en Espagne, et avant mon retour se faire bombarder premier
ministre. Il fut même assez imprudent et assez entraîné par la certitude
qu'il se figura là-dessus pour en faire part au pape, en prenant congé
de lui, et le dire franchement à plusieurs cardinaux et à d'autres, en
sorte qu'il en laissa le bruit répandu et tout commun à Rome. Porté sur
les ailes d'une si ferme et si douce espérance, il arriva à Paris le 28
décembre 1721.

Tout enfin étant réglé et prêt pour l'échange, l'infante partit le 9
janvier d'Oyarson, et M\textsuperscript{lle} de Montpensier de
Saint-Jean de Luz, avec chacune tout leur accompagnement\,; et
{[}elles{]} se trouvèrent en même temps vis-à-vis l'île des Faisans, où
elles entrèrent en même temps. Elles n'y demeurèrent que ce qu'il
fallait pour les compliments réciproques et les choses nécessaires pour
l'échange, et en sortirent en même temps\,: l'infante menée par le
prince de Rohan, et M\textsuperscript{lle} de Montpensier par le marquis
de Santa-Cruz. Elles couchèrent\,: l'une à Saint-Jean de Luz, l'autre à
Oyarson\,; et poursuivirent le lendemain leur voyage. La pauvre reine
douairière d'Espagne s'épuisa pour elles en présents magnifiques de
pierreries et de bijoux, à leur passage à Bayonne\,; et par une
prostitution de flatterie qu'elle apprenait de ses extrêmes besoins elle
voulut traiter M\textsuperscript{lle} de Montpensier en princesse des
Asturies, et comme si elle eût déjà été mariée. Elle lui donna un
fauteuil et la visita chez elle. Pendant la séance du fauteuil, les
duchesses passèrent dans un autre endroit avec la camarera-mayor de la
reine. Je me servis de tout ce que cette pauvre reine avait fait pour
toucher le roi et la reine d'Espagne pour lui procurer quelques secours
sur ce qui lui était dû, qui était fort considérable et fort en arrière,
et j'en obtins enfin un payement assez gros, mais ce fut tout, et je ne
pus en obtenir depuis. Bayonne passé, le prince de Rohan, dont la
magnificence avait été sans table et momentanée, prit la poste et gagna
Paris, où il rendit compte de ce qui s'était passé, et de ce qu'il avait
vu ou voulu voir de l'infante. Le marquis de Santa-Cruz dépêcha
quelqu'un à Lerma, et ne vint qu'avec M\textsuperscript{lle} de
Montpensier, qui se trouva seule entre les mains des Espagnols, sans
aucune dame, ni femme ni domestique français, dont aucun, sans
exception, ne passa la Bidassoa, comme on en était sagement convenu.
Pour l'infante, elle fut uniquement suivie de donna Maria de Nieves, sa
gouvernante, qui, à cause de son petit âge, devait passer quelques
années en France auprès d'elle, et qui avait toute la confiance de la
reine sa mère. Ces gouvernantes d'infants et d'infantes, pour le dire en
passant, n'approchent point de la volée des gouvernantes des enfants de
France, et sont prises d'entre les señoras de honor, ou parmi des femmes
de cet étage. Pour les infants cadets, leurs gouverneurs ne sont pas
plus relevés, hors des circonstances de nécessité ou de faveur, comme il
est arrivé dans la suite aux fils de la reine. Mais, à l'égard du prince
aisé et successeur, leurs gouverneurs sont toujours des seigneurs fort
distingués.

Tandis que M\textsuperscript{lle} de Montpensier continuait son voyage,
la quarantaine de mon exil s'avançait aussi, et finit justement deux
jours avant son arrivée à Lerma. Les bontés de Leurs Majestés
Catholiques redoublaient pour moi, et les soins et les attentions
obligeantes de toute leur cour, qui peu à peu s'était rendue fort
nombreuse, et tellement que le roi, ne pouvant vivre à Lerma, aussi
retiré qu'il avait accoutumé d'être à Madrid et dans ses maisons de
plaisance, où personne ne le suivait au delà du pur nécessaire, voulut
aller à Ventosilla, petit château et bourg à quelques lieues de Lerma,
avec la reine et le plus court indispensable, d'où il ne revint à Lerma
que le 15 janvier pour l'arrivée de la princesse. Ils avaient eu la
bonté de me faire dire plusieurs fois qu'ils voulaient me voir dès le
lendemain de ma quarantaine finie. Et moi, qui savais la crainte que le
roi avait de la petite vérole, je résistai jusqu'à un commandement
absolu, auquel il fallut obéir, quoique fort rouge, à quoi le grand
froid contribuait beaucoup, quelques drogues qu'on m'eût fait employer
pour me dérougir. J'allai donc pour la première fois à Lerma faire la
révérence et tous mes remerciements à Leurs Majestés Catholiques, le
matin du 19 janvier.

Après les compliments et les propos qui suivirent sur ma petite vérole,
les soins et la capacité de M. Hyghens, etc., je parlai de la promotion
que l'empereur s'avisait de faire de chevaliers de la Toison d'or, au
nombre de laquelle était le fils aîné du duc de Lorraine, qui préparait
une grande pompe pour en donner le collier à ce prince au nom de
l'empereur. J'avais reçu un ordre exprès de traiter expressément cette
matière dans ma première audience, que la petite vérole avait retardée
jusqu'alors\,; de tâcher d'empêcher le roi d'Espagne de montrer trop de
ressentiment de cette entreprise, pour ne pas troubler la négociation
qui s'ouvrait au congrès de Cambrai, et où cette prétention sur la
Toison devait être discutée et réglée en faveur de l'Espagne, par les
mesures que la France avait prises là-dessus\,; en même temps de faire
sentir la partialité si publique du duc de Lorraine pour l'empereur, si
promptement après avoir obtenu des réponses de la générosité de Leurs
Majestés Catholiques pleines d'espérance sur la grâce qu'il lui avait
demandée de vouloir bien consentir à ce qu'il fût compris dans la paix
générale, et que son accession y fût reçue\,; et j'étais chargé de
porter le roi d'Espagne à lui faire acheter désormais ce consentement
par beaucoup de délais, et de fatiguer longuement l'inquiétude et la
patience de ce prince là-dessus. Je m'acquittai donc de cette commission
dans les termes qui m'étaient prescrits, et je n'eus pas grand'peine à
réussir dans les deux points qu'elle renfermait.

Le roi et la reine me parurent piqués de l'entreprise de l'empereur,
qu'il ne pouvait fonder que sur sa souveraineté des Pays-Bas, où les
premières promotions de la Toison s'étaient faites. Mais Philippe le Bon
avait institué cet ordre comme duc de Bourgogne, et non comme seigneur
des Pays-Bas. Il est vrai qu'à ce titre cet ordre aurait dû suivre le
duché de Bourgogne, et le roi par conséquent en être devenu grand
maître. Mais nos rois, ne s'en étant jamais souciés, ayant leurs propres
ordres institués par eux, et n'ayant pas voulu embarrasser la cession du
duché de Bourgogne, que Louis XI saisit et occupa à la mort de Charles,
dernier duc de Bourgogne, sur son héritière, comme fief masculin et
première pairie de France, réversible de droit par sa nature à la
couronne, faute d'hoirs mâles, {[}nos rois{]} n'avaient jamais montré de
prétention sur la grande maîtrise de l'ordre de la Toison qu'ils
n'avaient point contestée. Les rois d'Espagne s'en étaient mis en
possession comme issus de l'héritière du dernier duc de Bourgogne,
auxquels Philippe V ayant succédé, il devait, par conséquent, succéder
aussi à la grande maîtrise de cet ordre, à laquelle même personne ne lui
avait formé aucune difficulté là-dessus à la paix d'Utrecht, qui était
prise pour base du traité à achever entre l'empereur et le roi
d'Espagne, duquel, en attendant, il était tacitement reconnu pour tel.
Néanmoins, Leurs Majestés Catholiques n'eurent pas de peine à vouloir
bien mépriser extérieurement cette entreprise, pourvu que justice leur
en fût faite à Cambrai et que la France s'engageât de plus en plus à
leur y faire céder l'ordre de la Toison.

À l'égard du duc de Lorraine, ils me témoignèrent qu'ils n'avaient pas
besoin de cette épreuve pour savoir à quoi s'en tenir sur l'attachement
sans bornes du duc de Lorraine, à l'exemple de ses pères, pour la maison
d'Autriche, et de sa préférence pour les intérêts et les volontés de
l'empereur sur toute autre considération\,; en même temps que, sans
montrer faire plus de cas qu'il ne convenait d'un si petit prince, il
était bon de le faire languir incertainement et longuement sur
l'agrément qu'il recherchait d'être compris dans la paix générale, et de
lui faire doucement sentir, aux occasions qui s'en pouvaient présenter,
le peu de considération qu'il méritait des deux couronnes. L'audience se
tourna ensuite en conversation.

Ils me firent l'honneur de me parler du cardinal Borgia, arrivé de Rome
à Lerma depuis peu de jours, et de ce qu'il leur avait conté de ce
pays-là. Dans le cours de cette conversation sur Rome, le roi se mit à
rire, regarda la reine, et me dit qu'il leur avait conté la plus
plaisante chose du monde. Je souris, comme pour lui demander quoi, sans
oser rien dire. Il regarda encore la reine et lui dit\,: «\,Cela n'est
pas trop bien à dire\,;» puis\,: «\,Lui dirons-nous\,? --- Pourquoi non,
répondit la reine. --- Mais, me dit le roi, c'est donc à condition que
vous n'en parlerez à qui que ce soit, sans exception.\,» Je le promis,
et j'ai tenu exactement parole. J'en parle ici pour la première fois,
après la mort du roi d'Espagne, et de ceux que cela regardait\,; et je
le laisserai apprendre à qui lira ces Mémoires, si jamais après moi
quelqu'un leur fait voir le jour. Alors il n'y aurait plus personne que
cette histoire puisse intéresser par rapport à celui qu'elle regarde.

Le roi me fit donc l'honneur de me conter que le cardinal Borgia lui
avait dit que le cardinal de Rohan, avec toute sa magnificence et les
agréments de ses manières flatteuses, remportait peu de crédit et de
réputation de Rome, où ses fatuités et le soin de sa beauté, quoique à
son âge, avait été jusqu'à se baigner souvent dans du lait pour se
rendre la peau plus douce et plus belle\,; que, quelque secret qu'il y
eût apporté, la chose avait été sue avec certitude, et avait indigné les
dévots, et attiré le mépris et les railleries des autres. Et là-dessus
le roi et la reine à commenter, et eux et moi à rire de tout notre
coeur, car le roi fit ce conte le mieux et le plus plaisamment du monde,
et les commentaires aussi. Je les assurai que je n'en étais point
scandalisé, parce que je connaissais depuis longtemps quel était ce Père
de l'Église. Je n'en dis pas davantage, le terrain n'était pas propre à
faire mention de la constitution que le P. Daubenton avait fabriquée
tête à tête avec le cardinal Fabroni, créature fidèle des jésuites et
maître audacieux de Clément XI, et par eux affichée et publiée à son
insu, et sans la lui avoir montrée, comme je l'ai raconté ici en son
temps, et dont le cardinal de Rohan a su tirer tant de grands partis
pour soi et pour les siens.

Cette audience se termina par toutes les bontés possibles de Leurs
Majestés Catholiques. J'eus aussi tout lieu de me louer extrêmement de
l'empressement de toute la cour, et de tout genre, à me témoigner la
joie de me revoir en bonne santé après une si dangereuse maladie.
J'allai après faire ma cour un moment au prince des Asturies, qui me
reçut avec les mêmes bontés qu'avaient fait le roi et la reine, qui tous
trois me parurent fort aises de l'arrivée de la princesse, et fort
impatients de la voir. En effet, étant retourné dîner en mon quartier,
j'appris que Leurs Majestés Catholiques, avec le prince des Asturies,
étaient montés avec des habits communs, et sans aucune sorte
d'accompagnement, dans un carrosse de suite du duc del Arco, qui allait
de leur part complimenter la princesse à Cogollos, lieu assez mauvais à
quatre lieues de Lerma, qui en font huit comme celles de Paris à
Versailles, ou elle devait arriver de bonne heure, ce même jour 29
janvier. Le duc del Arco la trouva arrivée. Il dit le mot à l'oreille au
marquis de Santa-Cruz pour qu'il avertit la duchesse de Monteillano et
les dames de se contenir\,; puis, introduit chez la princesse, il lui
fit son compliment, qu'il allongea exprès pour donner à sa royale suite
le temps de la bien considérer. Ensuite il lui demanda la permission de
lui présenter une dame et deux cavaliers de sa suite qui avaient un
grand empressement de lui rendre leurs respects. Une dame, venue avec
deux hommes à la suite d'un troisième, gâta tout le mystère. La
princesse se douta de la qualité de ces suivants, se jeta à leurs mains
pour les baiser et en fut aussitôt embrassée. La visite se passa en
beaucoup d'amitiés d'une part, de respects et de reconnaissance de
l'autre\,; et au bout d'un quart d'heure Leurs Majestés remontèrent en
carrosse et arrivèrent fort tard à Lerma.

J'étais convenu avec Maulevrier, qui, ce même jour était revenu avec
moi, de Lerma, dîner chez moi, qu'il s'y rendrait droit le lendemain
matin de son quartier, à une lieue du mien, entre six et sept heures du
matin, pour partir ensemble avec tous mes carrosses et nos suites pour
aller saluer la princesse à Cogollos. C'était huit lieues à faire,
c'est-à-dire seize de ce pays-ci, aller et venir. Il fallait avoir le
temps de manger un morceau chez moi au retour, et nous trouver à Lerma
pour l'arrivée de la princesse. Nous partîmes donc ensemble à sept
heures précises, et les mules nous menèrent grand train. Nous fûmes
introduits chez la princesse, qui achevait de s'habiller. Nous lui fûmes
présentés, puis je lui présentai le comte de Céreste, mes enfants, le
comte de Larges et M. de Saint-Simon. La duchesse de Monteillano, les
autres dames, Santa-Cruz aussi, firent tout ce qu'ils purent pour que la
princesse nous dît quelque mot, sans avoir pu y parvenir. Ils y
suppléèrent par toutes les civilités possibles. Nous n'avions pas de
temps à perdre\,; moins d'un quart d'heure acheva ce devoir, et nous
revînmes chez moi manger un morceau à la hâte, qui fut servi à
l'instant, et nous nous en allâmes aussitôt après à Lerma, dont bien
nous prit, car nous n'y attendîmes pas une demi-heure.

Dès que j'y fus arrivé, je montai chez le marquis de Grimaldo, quoique
je l'eusse vu chez lui la veille. Sa chambre était au bout d'une très
grande salle où on avait fait un retranchement pour servir de chapelle.
J'avais affaire encore une fois au nonce. Je craignais qu'il ne se
souvînt de ce qui s'était passé à la signature, et je ne voulais pas
donner prise au cardinal Dubois. Je ne vis donc qu'imparfaitement la
réception de la princesse, au-devant de laquelle le roi, la reine, qui
logeaient en bas, et le prince, se précipitèrent, pour ainsi dire,
presque jusqu'à la descente du carrosse, et je remontai vite à la
chapelle, que j'avais déjà reconnue allant chez Grimaldo.

Le prie-Dieu du roi était placé vis-à-vis de l'autel, à peu de distance
des marches, précisément comme le prie-Dieu du roi à Versailles, mais
plus près de l'autel, avec deux carreaux à côté l'un de l'autre. La
chapelle était vide de courtisans. Je me mis à côté du carreau du roi, à
droite tout au bord en dehors du tapis, et je m'amusai là mieux que je
ne m'y étais attendu. Le cardinal Borgia, pontificalement revêtu, était
au coin de l'épître, le visage tourné à moi, apprenant sa leçon entre
deux aumôniers en surplis, qui lui tenaient un grand livre ouvert devant
lui. Le bon prélat n'y savait lire\,; il s'efforçait, lisait tout haut
et de travers. Les aumôniers le reprenaient, il se fâchait et les
grondait, recommençait, était repris de nouveau, et se courrouçait de
plus en plus, jusqu'à se tourner à eux et à leur secouer le surplis. Je
riais tant que je pouvais, car il ne s'apercevait de rien, tant il était
occupé et empêtré de sa leçon. Les mariages en Espagne se font
l'après-dînée, et commencent à la porte de l'église comme les baptêmes.
Le roi, la reine, le prince et la princesse y arrivèrent avec toute la
cour, et {[}le roi{]} fut annoncé tout haut. «\,Qu'ils attendent,
s'écria le cardinal en colère, je ne suis pas prêt.\,» Ils s'arrêtèrent,
en effet, et le cardinal continua sa leçon, plus rouge que sa calotte et
toujours furibond. Enfin il s'en alla à la porte où cela dura assez
longtemps. La curiosité m'aurait fait suivre, sans la raison de
conserver mon poste. J'y perdis du divertissement, car je vis arriver le
roi et la reine à leur prie-Dieu riant et se parlant, et toute la cour
riant aussi. Le nonce arrivant à moi me marqua sa surprise par gestes,
et répétant\,: «\,Signor, signor\,!» et moi, qui avais résolu de n'y
rien comprendre, je lui montrai le cardinal en riant, et lui reprochai
de ne l'avoir pas mieux instruit pour l'honneur du sacré collège. Le
nonce entendait bien le français et l'écorchait fort mal. Cette
plaisanterie et l'air ingénu dont je la faisais, sans faire semblant des
démonstrations du nonce, fit si heureusement diversion qu'il ne fut plus
question d'autre chose, d'autant plus que le cardinal y donna lieu de
plus en plus en continuant la cérémonie, pendant laquelle il ne savait
ni où il en était, ni ce qu'il faisait, repris et montré à tous moments
par ses aumôniers, et lui bouffant contre eux, en sorte que le roi ni la
reine ne purent se contenir, ni personne de ce qui en fut témoin. Je ne
voyais que le dos du prince et de la princesse à genoux, sur chacun un
carreau, entre le prie-Dieu et l'autel, et le cardinal en face qui
faisait des grimaces du dernier embarras. Heureusement je n'eus là
affaire qu'au nonce, le majordome-major du roi s'étant placé à côté de
son fils, capitaine des gardes en quartier, au bord de la queue du tapis
du prie-Dieu. Les grands étaient en foule autour, et tout ce qu'il y
avait de gens considérables, et le reste remplissait toute la chapelle à
ne se pouvoir remuer.

Parmi ce divertissement que ce pauvre cardinal donnait à tout ce qui le
voyait, je remarquai un contentement extrême dans le roi et la reine de
voir accomplir ce mariage. La cérémonie finie, qui ne fut pas bien
longue, pendant laquelle personne ne se mit à genoux que le roi et la
reine, et où il le fallut, les deux mariés, Leurs Majestés Catholiques
se levèrent et se retirèrent vers le coin gauche du bas de leur drap de
pied, et se parlèrent bas peut-être l'espace d'un bon \emph{credo},
après quoi la reine demeura où elle était, et le roi vint à moi qui
étais à la place où j'avais toujours été pendant la cérémonie. Le roi
arrivé à moi me fit l'honneur de me dire\,: «\, Monsieur, je suis si
content de vous en toutes manières\,; et de celle en particulier dont
vous vous êtes acquitté de votre ambassade auprès de moi, que je veux
vous donner des marques de ma satisfaction, de mon estime et de mon
amitié. Je vous fais grand d'Espagne de la première classe, vous et en
même temps celui de vos deux fils que vous voudrez choisir pour être
grand d'Espagne et en jouir en même temps que vous\,; et je fais votre
fils aîné chevalier de la Toison d'or.\,» Aussitôt je lui embrassai les
genoux, et je tâchai de lui témoigner ma reconnaissance et mon désir
extrême de me rendre digne des grâces qu'il daignait répandre sur moi,
par mon attachement, mes très humbles services et mon plus profond
respect. Puis je lui baisai la main, et me tournai pour faire appeler
mes enfants, qui furent quelques moments à être avertis et à venir
jusqu'à moi, {[}moments{]} que j'employai en remerciements redoublés.
Dès qu'ils approchèrent, j'appelai le cadet, et lui dis d'embrasser les
genoux du roi qui nous comblait de grâces, et qui le faisait grand
d'Espagne avec moi. Il baisa la main du roi, en se relevant, qui lui dit
qu'il était fort aise de ce qu'il venait de faire. Je lui présentai
après l'aîné pour le remercier de la Toison, et qui se baissa fort bas
seulement et lui baisa la main. Dès que cela fut fait, le roi alla vers
la reine, où je le suivis avec mes enfants. Je me baissai fort bas
devant la reine\,; je lui fis mon remerciement particulier, puis lui
présentai mes enfants, le cadet le premier, l'aîné après. La reine nous
reçut avec beaucoup de bonté et nous dit mille choses obligeantes, puis
se mit en marche avec le roi, suivis du prince qui donnait la main la
princesse, que nous saluâmes en passant, et {[}ils{]} retournèrent dans
leur appartement. Je voulus les suivre, mais je fus comme enlevé par la
foule qui s'empressa autour de moi à me faire des compliments. J'eus
grande attention à répondre à chacun le plus convenablement, et à tous
le plus poliment qu'il me fut possible\,; et quoique je ne m'attendisse
à rien moins, qu'à recevoir ces grâces dans ce moment, et que je n'eusse
qu'une certitude vague par. Grimaldo, et de lui-même et indéfinie pour
le temps, il me parut depuis que toute cette nombreuse cour fut contente
de moi.

J'affectai fort de témoigner aux grands d'Espagne que j'avais toute ma
vie eu une si haute idée de leur dignité, qu'encore que j'eusse
l'honneur d'être revêtu de la première du royaume de France, je me
trouvais fort honoré de l'être de la leur. Je n'en dissimulai pas ma
joie, ni combien j'étais sensible au bonheur de mon second fils, pour
lequel je leur demandai leurs bontés. Je n'oubliai pas aussi de
témoigner aux chevaliers de la Toison combien j'étais touché de
l'honneur que mon fils aîné recevait, et moi avec lui de sa promotion à
ce noble et grand ordre, et je tâchai de n'oublier rien de tout ce qui
pouvait le plus leur marquer l'estime que je faisais des Espagnols, et
des dignités et des honneurs de l'Espagne, et répondre le mieux à
l'empressement, pour ne pas dire à l'accablement de leurs compliments à
tous, ainsi que ma reconnaissance pour les bontés et les grâces que je
recevais de Leurs Majestés Catholiques. Mes enfants que la foule qui
fondait sans cesse sur nous sépara bientôt de moi, firent de leur mieux
de leur côté\,; et cela dura plus d'une heure dans sa force, et
longtemps après de ceux de moindre qualité qui n'avaient pu nous
approcher plus tôt\,; et je tâchai, suivant leurs degrés, de ne pas
moins bien {[}les{]} recevoir et {[}leur{]} répondre que j'avais fait
aux autres. Je ne me contentai pas d'avoir vu Grimaldo dans cette foule.
Dès que je fus un peu débarrassé, je remontai chez lui et lui fis les
remerciements que je lui devais avec grande effusion de coeur. J'étais,
en effet, au comble de nia joie de me voir arrivé au seul but qui
m'avait fait désirer l'ambassade en Espagne, et je le lui devais presque
entièrement.

Revenons maintenant un moment sur nos pas pour reprendre de suite ce que
j'ai omis, pour ne le pas interrompre. La modestie et la gravité des
Espagnols ne leur permet pas de voir coucher des mariés\,: le souper de
noce fini, il se fait un peu de conversation, assez courte, et chacun se
retire chez soi, même les plus proches parents, hommes et femmes de tout
âge, après quoi les mariés se déshabillent chacun en son particulier, et
se couchent sans témoins que le peu de gens nécessaires à les servir,
tout comme s'ils étaient mariés depuis longtemps. Je n'ignorais pas
cette coutume et je n'avais reçu aucun ordre là-dessus. Néanmoins,
prévenu des nôtres, je ne pouvais regarder comme bien solide un mariage
qui ne serait point suivi de consommation au moins présumée.

On était convenu, à cause de l'âge et de la délicatesse du prince des
Asturies, qu'il n'habiterait avec la princesse que lorsque Leurs
Majestés Catholiques le jugeraient à propos, et on comptait que ce ne
serait d'un an, tout au moins. Je témoignai ma peine là-dessus au
marquis de Grimaldo, à Lerma\,; je n'y gagnai rien\,; il était Espagnol,
et il ne fit que tâcher de me rassurer sur une chose où il ne voyait pas
qu'il se pût rien changer. Outre que je n'eus que quelques moments avec
lui, je crus ne devoir pas insister, et lui laisser, au contraire,
croire que je me tenais pour battu, de peur que s'il apercevait plus
d'opiniâtreté, et que j'en voulusse parler au roi et à la reine, il ne
me gagnât de la main à l'instant, et, les prévint à maintenir la coutume
établie, et qui, jusqu'alors, n'avait jamais été enfreinte\,; mais
résolu à part moi de n'en pas demeurer là, puisque, au pis aller, je ne
réussirais pas, et ma tentative demeurerait ignorée. Ainsi dans
l'audience que j'eus à Lerma, et que j'ai racontée après avoir fini ce
qui regardait la Toison de l'empereur et le duc de Lorraine, je me mis à
parler du mariage, et de l'un à l'autre, de la consommation, en
approuvant fort le délai que demandait l'âge et la délicatesse du
prince. De là je vins à la joie que recevrait M. le duc d'Orléans d'en
apprendre la célébration\,; et je me mis à les flatter sur l'extrême
honneur qu'il recevrait de ce grand mariage, de sa sensibilité
là-dessus, et plus, s'il se pouvait encore, d'un gage si précieux et si
certain du véritable retour de l'honneur des bonnes grâces de Leurs
Majestés Catholiques, que j'étais témoin qu'il avait toujours si
passionnément désiré. Je fis là une pause pour voir l'effet de ce
discours\,; et comme il me parut répondre au dessein qui me l'avait fait
tenir, je m'enhardis à ajouter que plus cet honneur était grand et si
justement cher à M. le duc d'Orléans, plus il était envié de toute
l'Europe et des Français mal intentionnés pour le régent, et plus la
solidité du mariage lui était importante\,: que je n'ignorais pas les
usages sages et modestes de l'Espagne, mais que je n'en étais pas moins
persuadé qu'ils se pouvaient enfreindre en faveur d'un objet aussi grand
que l'était le dernier degré de solidité dans un cas aussi singulier, et
que je regarderais comme le comble des grâces de Leurs Majestés pour M.
le duc d'Orléans, et de la certitude de ce retour si précieux, si cher
et si passionné pour lui, de l'honneur de leur amitié, en même temps la
marque la plus éclatante de l'intime et indissoluble union des deux
branches royales, et des deux couronnes à la face de toute l'Europe, si
Leurs Majestés voulaient permettre qu'il en fût usé dans ce mariage,
comme Sa Majesté avait été elle-même témoin qu'il en avait été usé au
mariage de Mgr le duc de Bourgogne, qui ne fut que si longtemps après
avec M\textsuperscript{me} la duchesse de Bourgogne.

Le roi et la reine me laissèrent tout dire sans m'interrompre. Je le
pris à bon augure. Ils se regardèrent, puis le roi lui dit\,: «\,Qu'en
dites-vous\,? --- Mais vous-même, monsieur,\,» répondit-elle. Là-dessus,
je repris la parole, et leur dis que je ne voulais point les tromper\,;
que je leur avouais que je n'avoir aucun ordre là-dessus\,; que cette
matière n'avait été traitée avec moi, ni de bouche avant mon départ, ni
par écrit dans mes instructions, ni depuis mon départ de Paris dans
aucune dépêche\,; que ce que je prenais la liberté de leur représenter
là-dessus, venait uniquement de moi et de mes réflexions et qu'en cela
je croyais ne parler pas moins avec l'attachement d'un vrai serviteur
des deux couronnes, en vrai Français, en bon Espagnol, qu'en serviteur
de M. le duc d'Orléans, par l'effet qui en résulterait dans les deux
monarchies et dans toute l'Europe\,; qu'on y désespérerait alors de
pouvoir opérer des conjonctures qui pussent faire regarder de bon oeil
ce mariage comme possible à séparer, et par conséquent à travailler
profondément et à tout ce qui pourrait y conduire\,; enfin que toute
l'Europe conjurée pour rompre l'union des deux couronnes, dont la durée
intime opérerait nécessairement toute la grandeur et la puissance, telle
que la même union des deux branches de la maison d'Autriche l'a opérée
en sa faveur, abandonnerait enfin le dessein d'y attenter de nouveau, le
regardant comme impossible, après avoir vu l'Espagne si attachée à ses
usages, y contrevenir pour la première fois, uniquement pour donner à ce
mariage le dernier degré d'indissolubilité, selon l'opinion de toutes
les nations, encore que, selon la sienne, il ne lui en manquât aucune
sans cette formalité.

Ces raisons emportèrent Leurs Majestés Catholiques\,; elles se
regardèrent encore, se dirent quelques mots bas, puis le roi me dit\,:
«\,Mais si nous consentions à ce que vous proposez, comment
entendriez-vous faire\,?» Je répondis que rien n'était plus aisé et plus
simple\,; que Sa Majesté en avait vu le modèle au mariage de Mgr le duc
de Bourgogne\,; mais qu'il était inutile de laisser entrevoir la
résolution qui en serait prise avant le temps de l'exécution, pour
éviter les discours de gens ennemis de toute nouveauté, et qui n'en
verraient pas d'abord les raisons si solides et si importantes\,; que
supposé que Leurs Majestés voulussent bien embrasser un parti qui
paraissait si nécessaire, il suffirait d'en faire doucement répandre la
résolution dans le grand bal qui devait précéder le coucher, où le
spectacle d'un lieu si public arrêterait les raisonnements, et où la
chose serait sue à temps de retenir les spectateurs après le bal, par le
désir de faire leur cour, et par la curiosité d'être témoins de chose
pour eux si nouvelle\,; que pour l'exécution, Leurs Majestés seules,
avec le pur nécessaire, assisteraient au déshabiller, les verraient
mettre au lit, feraient placer aux deux côtés du chevet le duc de Popoli
près du prince, la duchesse de Monteillano près de la princesse, et tous
les rideaux entièrement ouverts des trois côtés du lit\,; feraient
ouvrir les deux battants de la porte, et entrer toute la cour, et la
foule s'approcher du lit, laisser bien remplir la chambre de tout ce
qu'elle pourrait contenir\,; avoir la patience d'un quart d'heure pour
satisfaire pleinement la vue de chacun\,; puis faire fermer les rideaux
en présence de la foule et la congédier, pendant quoi le duc de Popoli
et la duchesse de Monteillano auraient soin de se glisser sous les
rideaux, et de ne pas perdre un instant le prince et la princesse de
vue, et la foule sortie des antichambres jusqu'au dernier, faire lever
le prince et {[}le{]} conduire dans son appartement.

Le roi et la reine approuvèrent tout ce plan, et après quelque peu de
conversation et de raisonnements là-dessus, me promirent de le faire
exécuter de la sorte, et je leur en fis tous mes très humbles
remerciements. J'eus tout lieu de juger que mes raisons les avaient
frappés, par la facilité avec laquelle ils s'y rendirent, et que la
chose même, toute nouvelle et singulière qu'elle fût en Espagne, ne leur
déplaisait pas, parce que ce fut après tous ces propos, et m'avoir
promis l'exécution, que Leurs Majestés se mirent sur le cardinal Borgia,
sur Rome, et qu'elles finirent par me raconter cette ridicule histoire
du cardinal de Rohan, qui les divertit tant et moi aussi, que j'ai déjà
rapportée. Je sortis donc de l'audience fort content, et m'en retournai
dîner à mon quartier sans retourner chez Grimaldo que j'avais vu
auparavant, et qui m'aurait pu faire des difficultés que je voulais
d'autant plus éviter que je savais qu'il ne verrait le roi ni la reine
de toute cette journée, parce qu'ils allaient à la messe quand je sortis
d'auprès de Leurs Majestés, dîner tout de suite et monter en carrosse
pour suivre, comme je l'ai dit, le duc del Arco à Cogollos, d'où ils ne
pouvaient revenir que fort tard, comme ils firent.

Le lendemain après le mariage, et que je fus un peu libre de foule et de
compliments, je montai avec mes enfants chez Grimaldo. À moitié du
degré, je fus atteint par un des trois domestiques intérieurs français,
qui me cherchait et qui me dit que le roi lui avait ordonné de me dire
qu'il y aurait au bal une embrasure de fenêtre où je trouverais un
tabouret pour le nonce, un pour moi, un autre pour Maulevrier, et un
quatrième que Sa Majesté avait expressément commandé pour mon fils aîné
qui relevait d'une seconde maladie qu'il avait eue dans mon quartier
pendant ma petite vérole. Je fus fort touché d'une attention du roi si
pleine de bonté\,; mais j'en sentis en même temps toute la distinction
de mon fils, ni duc, ni grand, assis où nul duc, ni grand ne s'assied
point, que les trois par charges, que j'ai expliqués ailleurs, et traité
comme les ambassadeurs. Je compris à l'instant combien cet honneur
singulier pourrait faire de peine aux grands et blesser même les
Espagnols. Je répondis donc avec tous les respects et les remerciements
possibles, que je suppliais le roi de me permettre de renvoyer mon fils
aîné avant le bal, parce que sa santé était encore si faible qu'il avait
besoin de ce repos, après la fatigue de toute cette journée, et j'évitai
de la sorte -un honneur qui aurait pu `donner lieu à du mécontentement.
J'achevai ensuite de monter l'escalier et d'aller chez le marquis de
Grimaldo.

Mes remerciements faits, je renvoyai mes enfants, puis je dis à Grimaldo
que n'ayant pas eu le temps de le voir depuis mon audience de la veille,
je venais l'informer de ce qui s'y était passé, quoiqu'il le sût sans
doute, si l'embarras de ces journées si remplies lui avait laissé le
loisir de voir Leurs Majestés. Je lui déduisis ce qui avait regardé
l'empereur, la Toison et le duc de Lorraine\,; puis j'ajoutai que mes
réflexions sur l'importance du coucher public m'affectant toujours,
nonobstant ce qu'il m'avait répondu là-dessus, je n'avais pu me tenir
d'en parler au roi et à la reine, et je lui dis toutes les mêmes choses
que je leur avais représentées. Soit que ce ministre fit semblant
d'ignorer ce qu'il savait, soit qu'en effet l'embarras de ces journées
si pleines eût empêché son travail avec Leurs Majestés, je vis se
peindre une curiosité extrême dans ses yeux et dans sa physionomie\,; et
{[}lui{]} m'interrompre plusieurs fois pour m'en demander le succès.
Avant que de le satisfaire, je voulus lui déduire toutes mes raisons
pour tâcher de le persuader au moins sur une chose accordée, et je finis
par lui dire qu'elle l'était, et lui témoigner combien M. le duc
d'Orléans y serait sensible, et à quel point j'étais moi-même touché de
la complaisance de Leurs Majestés. Grimaldo, en habile homme, peut-être
y entra-t-il aussi de l'amitié pour moi, prit la chose de fort bonne
grâce. Il me dit que ce qui abondait ne nuisait point\,; mais que la
cour serait bien surprise. Je l'avertis que cela ne se saurait qu'au
bal, et après un peu d'entretien, je le quittai. Je voulais éviter
l'improbation des Espagnols, et je crus ne pouvoir mieux m'y prendre
qu'en mettant de mon côté le marquis de Villena, Espagnol au dernier
point, et qui, par son âge, sa charge de majordome-major, et plus encore
par sa considération personnelle et le respect universel qu'on lui
portait, arrêterait tout par son approbation, si je pouvais la tirer de
lui.

Je l'avais toujours singulièrement cultivé dans le peu de temps que
j'avais eu à le pouvoir faire, et il y avait continuellement répondu
avec toute sorte d'attention, même d'amitié, jusqu'à m'être venu voir à
Villahalmanzo, avant que j'eusse pu aller à Lerma. J'allai donc chez lui
au sortir de chez Grimaldo, et lui dis que je venais lui faire une
confidence, bien fâché que les occupations de ces deux journées ne
m'eussent pas permis de le consulter auparavant, comme je le voulais. De
là je lui expliquai toutes mes raisons pour le coucher public, et ma
peine de ce qui y pouvait blesser les Espagnols. Je m'étendis
flatteusement sur ce dernier point, et j'ajoutai qu'après le combat qui
s'était passé en moi-même entre cette considération et l'importance de
donner le dernier degré de solidité au mariage, j'avais estimé que mon
devoir et l'intérêt des deux couronnes devait prévaloir. Il me laissa
tout exposer, puis me répondit que ces raisons étaient, en effet, très
fortes\,; que les usages des différents pays n'étaient pas des lois qui
ne dussent pas céder à des considérations aussi importantes\,; que pour
lui, il n'y voyait aucun inconvénient, et qu'il ne croyait pas, non
plus, que personne y en pût trouver, quand les raisons d'innover, pour
cette fois, seraient connues et pesées. Cette réponse, faite de bonne
grâce par un seigneur d'un si grand poids, me mit fort à mon aise. Je le
lui témoignai, et après lui avoir fait entendre la manière de
l'exécution convenue par Leurs Majestés, je le suppliai de vouloir bien
s'expliquer au sortir du bal, un peu publiquement, de la même manière
qu'il venait de le faire avec moi, pour disposer le gros du monde à
penser de même et l'entraîner par l'autorité de son suffrage. Je le
flattai là-dessus, comme il le méritait. Il me promit très honnêtement
de s'expliquer comme je le désirais. Il me tint exactement parole, et le
succès en fut tel que personne n'osa se montrer scandalisé d'une
nouveauté si grande et si peu attendue, qui alors, ni depuis, ne reçut
aucun blâme de personne.

Content au dernier point de ces précautions, j'allai souper avec tous
les Français de marque chez le duc del Arco, qui nous avait invités, où
plusieurs des plus distingués de la cour se trouvèrent. Le souper fut à
l'espagnole, mais une oille\footnote{On appelle \emph{oille}, une espèce
  de potage, où entrent des viandes et des légumes de diverses sortes.
  Ce mot vient de l'espagnol \emph{olla}.} excellente suppléa à d'autres
mets auxquels nous étions peu accoutumés, avec d'excellent vin de la
Manche. Le vin et l'huile que les seigneurs font faire chez eux, pour
eux, sont admirables, et condamnent bien la paresse publique qui des
thèmes crus en fait dont on ne peut pas seulement souffrir l'odeur. On y
servit aussi de petits jambons vermeils, fort rares en Espagne même, qui
ne se font que chez le duc d'Arcos\footnote{Saint-Simon a écrit le duc
  d'Arcos\,; mais il faut lire le duc del Arco.} et deux autres
seigneurs, de cochons renfermés dans des espèces de petits parcs,
remplis de halliers où tout fourmille de vipères, dont ces cochons se
nourrissent uniquement. Ces jambons ont un parfum admirable, et un goût
si relevé et si vivifiant qu'on en est surpris, et qu'il est impossible
de manger rien de si exquis. Le souper fut long, abondant, plein de joie
et de politesse, bien et magnifiquement servi. En sortant de table nous
passâmes tous dans les appartements du roi, où tout était déjà prêt pour
le bal.

Toute la cour était en partie arrivée, le reste suivit incontinent.
L'attente après fut courte. Leurs Majestés et Leurs Altesses parurent
bientôt, et la reine ouvrit le bal avec le prince des Asturies. Ce bal
fut disposé comme celui de Madrid, que j'ai décrit. Ainsi je me
dispenserai de la répétition. Le nonce, Maulevrier et moi le vîmes de
l'embrasure d'une fenêtre, de dessus nos tabourets. Mais je n'eus pas
grand repos sur le mien, tant on me fit danser de menuets et de
contredanses. J'avais un habit d'une extrême pesanteur, les mouvements
continuels de cette journée et de la veille m'avaient extrêmement
fatigué\,; mais c'était la fête du mariage, je venais d'obtenir au delà
de ce que j'avais pu y désirer\,; par là, c'était aussi ma fête
particulière, j'aurais eu mauvaise grâce de rien refuser. Ce bal fut
fort gai sans déroger en rien à la majesté et à la dignité. Il dura
jusque vers deux heures après minuit. Le nonce seul assis, avec
Maulevrier et moi, car nul autre ambassadeur ne parut à Lerma\,; le duc
d'Abrantès, évêque de Cuença debout, ainsi qu'un autre évêque voisin\,;
deux évêques \emph{in partibus} suffragants de Tolède\,; et le grand
inquisiteur, qui avaient assisté sans fonctions au mariage, furent au
bal tout du long en rochet et camail, leur bonnet à la main. L'évêque
diocésain de Burgos, exilé pour son attachement fort marqué à l'archiduc
et à la maison d'Autriche, ne put s'y montrer, et le cardinal Borgia n'y
put être par ses prétentions. On sut au bal qu'il y aurait coucher
public. Il ne m'en parut que de la surprise, mais nul mécontentement.
Personne ne s'en alla après le bal\,: on attendit pour voir ce coucher.

Au sortir du bal, tout le monde suivit le roi et la reine dans
l'appartement de la princesse, et attendit dans les antichambres. Il
n'entra dans la chambre que le service nécessaire. J'y fus appelé. La
toilette fut courte\,; Leurs Majestés et le prince extrêmement gais.
Tout se passa comme j'ai expliqué qu'il avait été résolu, et je regagnai
Villahalmanzo, et mon lit dont j'avais un extrême besoin.

Ce ne fut pas pour y demeurer longtemps. Le lendemain, 21 janvier, il
fallut me trouver de bonne heure à Lerma pour la cérémonie de la
Vélation. C'est qu'en Espagne où on marie l'après-dînée ou le soir, la
noce entend le lendemain la messe du mariage qui n'a pu se dire la
veille, pendant laquelle se font les cérémonies extérieures, et où les
mariés sont mis sous le poêle. J'allai, en arrivant, chez le marquis de
Grimaldo, puis tout de suite prendre mon poste de la veille, où bientôt
après toute la cour arriva.

Le prie-Dieu du roi et les carreaux en avant pour le prince et la
princesse étaient disposés comme la veille, et le cardinal Borgia tout
revêtu, étudiant encore sa leçon avec ses aumôniers, n'avait plus que sa
chasuble à prendre, ce qu'il fit, dès que le roi et la reine entrèrent,
suivis du prince qui donnait la main à la princesse. Le nonce, qui vint
en même temps, me fit civilité, se mit auprès de moi, du côté de
l'autel, comme la veille, et m'y parut tout accoutumé. Maulevrier, qui
au mariage était un peu derrière moi, du côté d'en bas, ne parut point,
et nous sûmes après, car il ne m'en avait pas ouvert la bouche, qu'il
était parti ce même matin de son quartier pour retourner à Madrid. Le
cardinal dit la messe basse où il ne me parut guère plus habile qu'aux
cérémonies, et se barbouilla fort encore en celles qui restaient à
faire. La messe finie, j'accompagnai Leurs Majestés chez elles, qui
s'amusèrent avec moi des embarras du cardinal. Et comme elles
rentraient, je leur demandai la permission de prendre congé d'elles au
sortir du dîner, parce qu'elles partaient le lendemain pour Madrid.
J'oublie de marquer que le poêle fut tenu par deux aumôniers du roi
qu'on appelle en Espagne sommeliers de courtine. Je crois que ce nom
leur vient de ce que, jusqu'à Philippe V, tout l'enfoncement où on place
le prie-Dieu du roi, lorsqu'il tient chapelle, et dont j'ai décrit la
séance ailleurs, était tout enfermé de rideaux, qu'on appelle en
espagnol \emph{cortinas}, et que la fonction des aumôniers du roi était
de relever un peu le rideau, lorsque cela était nécessaire, pour
recevoir l'encens, baiser l'Évangile, etc.

J'allai avec nos Français d'élite dîner chez le duc del Arco, en grande
et illustre compagnie, où nous étions invités, et où le repas fut
magnifique comme la veille. Je ne m'y oubliai pas encore à l'oille ni
aux jambons de vipères. Les Espagnols étaient toujours ravis de voir un
Français s'accommoder du safran, surtout d'en trouver toujours chez moi
en plusieurs mets, et de m'en voir manger avec plaisir. Pour dans le
pain et dans la salière, où ils en mettent volontiers, je ne pus pousser
jusque-là mon goût ni ma complaisance. Le dîner fut long et gai.

La surprise de l'absence de Maulevrier fit à demi bas le tour de la
table, et fut d'autant plus blâmée qu'il n'était pas aimé. Je fus sobre
sur cet article, mais on n'en dit pas moins. Je ne lui avais point parlé
de mes réflexions sur le coucher public. Je gardais avec lui l'extérieur
le plus exact, mais j'avais lieu de me dispenser des consultations et
des confidences. Je ne lui dis que vers le milieu du bal que toute la
cour serait admise à voir les deux époux au lit, mais crûment, comme une
nouvelle, sans le plus léger détail. Il m'en parut étonné à l'excès,
puis tout renfrogné me demanda comment une chose si étrange et si
nouvelle en Espagne avait pu être résolue. Je lui répondis simplement
que Leurs Majestés l'avaient jugé à propos ainsi, et tout de suite je me
mis à parler sur le bal et sur la danse. Du reste du bal et du soir, il
ne me parla presque plus, et toujours d'un air chagrin. Ce n'en fut
qu'une dose ajoutée de plus. Ma grandesse et l'éclat des compliments et
de l'applaudissement public le hérissa tellement qu'il ne put se
contenir, jusque-là que les courtisans se divertirent à lui en parler,
quelques-uns même à lui en faire compliment comme d'une chose agréable à
la France, pour l'embarrasser et s'en attirer des réponses sèches et
brusques. Ils l'appelaient le chat fâché et se moquaient de lui\,; à
moi-même il ne put s'empêcher de m'en faire un compliment sur ce même
ton, et fort court, que je pris pour bon, avec tous les remerciements
possibles. Il n'eut pas même la patience de les écouter jusqu'au bout,
et s'en alla d'un autre côté. À mes enfants à peine leur dit-il un mot
brusque en passant. Le coucher public, qu'il n'apprit que comme je viens
de le rapporter, le courrouça apparemment encore. Il s'en dépita par
s'en aller le lendemain sans m'en dire un mot ni à personne, et manquer
ainsi de propos délibéré une fonction où le caractère dont il était
honoré l'obligeait d'être présent.

Un autre homme parut aussi fort mécontent, et me surprit au dernier
point. Ce fut La Fare, à qui le roi d'Espagne donna la Toison, en même
temps qu'à mon fils aîné. Qui eût dit à son père que ce fils aurait la
Toison, jamais il n'aurait pu le croire. Toutefois me voyant fait grand
d'Espagne, et conjointement avec mon second fils, cet homme si fort du
monde, doux, poli, gai, en reçut les compliments avec un sec, un court,
un air, un ton qu'il ne pouvait avoir emprunté que de Maulevrier. Il se
méconnut assez pour m'en faire ses plaintes. Quel qu'en fût mon
étonnement, je ne crus pas devoir le lui témoigner, mais le traiter en
malade, avec complaisance\,; ainsi {[}je{]} tâchais là, comme depuis à
Madrid, de le porter à des manières qui ne dégoûtassent ni le roi ni sa
cour, et qui ne lui fermassent pas les voies de ce qu'il désirait, mais
que je savais bien qu'il était hors de portée d'obtenir. Il se servit
tant qu'il put, et très mal à propos, du nom du régent et du cardinal
Dubois, auprès de Grimaldo, et même avec d'autres seigneurs, familiers
chez moi, qui après, riaient et haussaient les épaules, et m'exhortaient
de tâcher à le faire rentrer en lui-même.

Cette ambition lui tourna tellement la tête, qu'il se mit à hasarder des
propos comme s'il était ambassadeur de M. le duc d'Orléans, et à le
prétendre. En me pressant sur sa grandesse, il me lâcha quelques traits
de cette prétention que je ne pus lui passer comme le reste. La
grandesse était une chimère personnelle, mais l'appuyer de cette
prétention d'ambassade portait sur M. le duc d'Orléans. Je lui remontrai
donc que quelque grand prince que fût M. le duc d'Orléans, par sa
naissance et par sa régence, il ne laissait pas d'être sujet du roi,
dont la qualité ne comportait pas d'envoyer en son nom des ambassadeurs,
pas même des envoyés ayant le caractère et les honneurs qu'ont les
envoyés des souverains\,; qu'il n'avait qu'à voir son instruction et son
titre, où je m'assurais qu'il ne trouverait rien qui pût favoriser cette
idée\,; que de plus connaissant M. le duc d'Orléans autant qu'il le
connaissait, et le cardinal Dubois aussi, il devait craindre que cette
prétention leur revint, qu'ils trouveraient sûrement extrêmement
mauvaise, et qui donnerait lieu à ses ennemis d'en profiter dès à
présent dans le public, et dans la suite auprès du roi, en accusant M.
le duc d'Orléans de vouloir déjà trancher du souverain, dans
l'impatience de le devenir en effet, par des malheurs qu'on ne pouvait
assez craindre\,; ce qui donnerait un nouveau cours aux horreurs tant
débitées et si souvent renouvelées. Mais les vérités les plus palpables
ne trouvent point d'entrée dans un esprit prévenu et que l'ambition
aveugle.

La Fare se mit à pester contre la faiblesse de M. le duc d'Orléans, qui
ne se souciait point de sa grandeur, et me voulut persuader que mon
attachement pour lui y devait suppléer en cette occasion. Je me tus, car
que répondre à une pareille folie\,? et ce silence lui persuada que je
ne voulais pas qu'il fût ambassadeur ni grand d'Espagne comme je
l'étais. Pour grand, j'en aurais été bien étonné. C'eût été donner à un
gentilhomme chargé des remerciements de M. le duc d'Orléans ce qui se
pouvait donner de plus grand, et la même chose, pour ne parler ici que
des caractères, qui était donnée à l'ambassadeur extraordinaire du roi
venu pour faire la demande de l'infante et en signer le contrat de
mariage. Mais quelque étrange que cela eût été, je me serais bien gardé
de mettre le moindre obstacle à la fortune d'un gentilhomme, comme, par
cette même raison, il n'avait tenu qu'à moi d'empêcher Maulevrier d'être
ambassadeur, et je n'avais pas voulu le faire, quoique je ne l'eusse
jamais vu, et que je connusse la naissance des Andrault pour bien plus
légère encore que celle de La Fare. Par cette même raison, j'aurais
trouvé aussi fort bon que ce dernier fût ambassadeur de M. le duc
d'Orléans, ou même en eût usurpé le traitement, si ce n'avait pas été
une folie, une chose impossible, et d'ailleurs une chimère que M. le duc
d'Orléans aurait fort désapprouvée, et qui lui aurait été en effet très
préjudiciable.

Je n'oubliai pas à représenter à La Fare que feu Monsieur, fils, frère,
gendre, beau-père et beau-frère de rois, n'avait jamais eu d'envoyés
nulle part, tels qu'ont les souverains, mais dépêché seulement en
Espagne, en Angleterre, etc., des personnes distinguées de sa cour pour
faire ses compliments aux rois et aux princes, aux occasions qui s'en
sont présentées, comme lui-même l'était actuellement par M. le duc
d'Orléans, son fils. Mais nulle raison ne put prendre sur La Fare. Il se
persuada que mon intérêt m'empêchait de le servir et de le faire
réussir, de manière qu'il me bouda longtemps, et me vit assez peu. Cette
folie d'ambassade, jusqu'à des plaintes de n'avoir pas été reçu et de
n'être pas traité avec les honneurs qui lui étaient dus, commençaient à
être fort sues\footnote{L'irrégularité de cette phrase s'explique, parce
  que Saint-Simon y fait accorder le verbe et le participe avec le mot
  \emph{plaines}.}, dont Grimaldo ne me cacha pas qu'il était fort
scandalisé\,; j'en craignis donc le contre-coup en France, et de
recevoir des reproches de mon silence et de ma tolérance là-dessus. Pour
la tolérance, je n'avais rien à y faire\,; mais pour le silence, je le
rompis. J'en écrivis donc un petit mot au cardinal Dubois, mais court et
fort en douceur. Il ne m'y répondit pas de même sur La Pare, et lui
écrivit de façon qu'il n'osa plus parler de caractère. Je crois que
cette lettre ne m'accommoda pas avec lui.

Cette conduite avec moi, à qui il avait toute l'obligation de cet
agréable voyage, et de la Toison qu'il lui valait, m'engagea à en écrire
à Belle-Ile, à la prière duquel j'avais demandé La Fare à M. le duc
d'Orléans pour aller de sa part en Espagne. Je lui parlai au long de sa
chimère d'ambassade, et ce que j'avais tu au cardinal Dubois de la
grandesse qu'il voulait\,; enfin de sa conduite avec moi. Belle-Ile
avait trop d'esprit et de sens pour ne pas voir et sentir tout ce que
c'était que ce procédé et ces chimères, et me le manda franchement, et
qu'il en écrivait de même à La Fare sur tout ce qui me regardait. Ce ne
fut pourtant que tout à la fin de mon séjour en Espagne que La Fare
reprit peu à peu ses véritables errements avec moi, et depuis notre
retour en France nous avons été amis. Il a bien su depuis pousser sa
fortune, et par de bien des sortes de chemins, toutefois pourtant sans
intéresser son honneur. Il est étonnant combien l'ambition ouvre
l'esprit le plus médiocre, et combien il est des gens à qui tout
réussit, dont on ne se douterait jamais. J'ai voulu raconter toute cette
aventure de suite. Retournons chez le duc del Arco d'où nous sommes
partis.

\hypertarget{chapitre-vi.}{%
\chapter{CHAPITRE VI.}\label{chapitre-vi.}}

1722

~

{\textsc{Ma conduite en France sur les grâces reçues en Espagne.}}
{\textsc{- Parrains de mes deux fils.}} {\textsc{- Princesse des
Asturies fort incommodée.}} {\textsc{- Inquiétude du roi et de la reine,
qui me commandent de la voir tous les jours, contre tout usage en
Espagne.}} {\textsc{- Ils me confient les causes secrètes de leurs
alarmes, sur lesquelles je les rassure.}} {\textsc{- Couverture de mon
second fils.}} {\textsc{- Le cordon bleu donné au duc d'Ossone.}}
{\textsc{- Je prouve à M. le duc d'Orléans qu'il pouvait et qu'il devait
faire lui-même le duc d'Ossone chevalier de l'ordre, et lui propose sept
ou huit colliers pour l'Espagne, lors de la grande promotion, dont un
pour Grimaldo.}} {\textsc{- L'ordre offert au cardinal Albane et refusé
par lui.}} {\textsc{- Office au cardinal Gualterio, à qui le feu roi
l'avait promis.}} {\textsc{- Chavigny en Espagne, mal reçu\,; son
caractère.}} {\textsc{- Chavigny à Madrid.}} {\textsc{- Sa mission, et
de qui.}} {\textsc{- Vision du duc de Parme la plus inepte sur Castro et
Ronciglione.}} {\textsc{- Fausseté puante de Chavigny sur le duc de
Parme.}} {\textsc{- Chavigny chargé par le duc de Parme de proposer le
passage actuel de l'infant don Carlos à Parme avec six mille hommes,
dont le duc de Parme aurait le commandement, les subsides, et
l'administration du jeune prince.}} {\textsc{- Chavigny sans ordre ni
aucune réponse du cardinal Dubois sur le passage de don Carlos en
Italie\,; sans lettre de créance ni instruction du cardinal Dubois pour
la cour d'Espagne.}} {\textsc{- Ordre de lui seulement d'y servir le duc
de Parme, mais sans y entrer en trop de détails sur Castro et
Ronciglione.}} {\textsc{- Tableau de la cour intérieure d'Espagne.}}
{\textsc{- Chavigny se montre à Pecquet vouloir un établissement actuel
à don Carlos en Italie.}} {\textsc{- Multiplicité à la fois des
ministres de France à Madrid publiquement odieuse et suspecte à la cour
d'Espagne.}} {\textsc{- Dangers et absurdité du passage actuel de don
Carlos en Italie, sans aucun fruit à en pouvoir espérer.}} {\textsc{-
Chimère ridicule de l'indult.}} {\textsc{- Mon embarras du silence
opiniâtre du cardinal Dubois sur le projet du passage de don Carlos en
Italie.}} {\textsc{- Mesures que je prends en France et en Espagne pour
faire échouer la proposition du passage de don Carlos en Italie, qui
réussissent.}} {\textsc{- Je mène Chavigny au marquis de Grimaldo, et le
présente au roi et à la reine d'Espagne, desquels il est extrêmement mal
reçu.}} {\textsc{- Il échoue sur les deux affaires qu'il me dit l'avoir
amené à Madrid.}}

~

Après dîner et un peu de conversation, j'allai chez le roi et la reine
qui m'avaient permis d'aller prendre congé d'eux. Je renouvelai mes
remerciements sur le coucher public, que je leur dis, comme il était
vrai, n'avoir été désapprouvé de personne, et les miens ensuite sur les
grâces que je venais de recevoir, qui furent tous reçus avec beaucoup de
bonté. Je pris congé jusqu'à Madrid. J'allai de là prendre congé du
prince des Asturies et dire adieu au marquis de Villena, de chez qui je
retournai en mon quartier faire mes dépêches et écrire quantité de
lettres à famille et à amis, pour leur donner part des grâces que je
venais de recevoir. J'en eus tant à faire que j'y donnai tout le
lendemain 22 que la cour partait de Lerma, et je ne partis avec tout ce
qui était avec moi que le lendemain 23 janvier. L'embarras n'était pas
médiocre de mander à M. le duc d'Orléans et au cardinal Dubois que je
m'étais passé de leurs lettres, et que sans ce secours j'avais si
promptement et si agréablement reçu toutes les grâces de Leurs Majestés
Catholiques dont j'avais à rendre compte au régent et à son ministre.
J'étais peu en peine de M. le duc d'Orléans dont la légèreté et
l'incurie sur les petites choses, et trop souvent sur les grandes, me
rassurait sur le peu d'impression qu'il en recevrait. Mais il n'en était
pas de même du cardinal Dubois, qui n'avait fait les deux lettres de
cette étrange faiblesse que dans l'espérance de me faire manquer le but
qui m'avait fait demander et obtenir à son insu l'ambassade d'Espagne,
qui serait d'autant plus piqué que j'y fusse arrivé malgré lui, et qui
n'oublierait rien pour aigrir s'il pouvait M. le duc d'Orléans
là-dessus. Je pris donc le parti d'écrire à ce prince une lettre
désinvolte et courte là-dessus, suivant son goût, mais pleine de toute
la reconnaissance que je devais à sa volonté, au cardinal Dubois un
verbiage où je me répandis avec profusion en reconnaissance, et où je
lui fis accroire que ce n'était qu'à ces deux lettres non présentées,
mais toutefois lues par Grimaldo à Leurs Majestés Catholiques, que je
devais les grâces que j'en avais reçues, dès le jour même qu'elles
avaient été informées par cette lecture du désir de son Altesse Royale
et des siens. L'affaire était faite\,; comme que ce fût, je lui en
donnais l'honneur. Faute de pouvoir pis, il prit le tout en bonne part,
me félicita, et se donna pour fort aise d'avoir si heureusement
travaillé en ma faveur.

Pour l'y confirmer en même temps, je lui avais demandé des lettres de
remerciements de ces grâces de M. le duc d'Orléans au roi d'Espagne, et
de lui au marquis de Grimaldo et au P. Daubenton. Comme il ne s'agissait
plus de me les procurer, mais d'en remercier comme de l'accomplissement
d'un ouvrage qu'il lui plaisait de s'approprier, j'eus ces lettres en
réponse des miennes, dont le style animé était bien différent de la
langueur de celui des deux lettres de prétendue demande, qu'il m'avait
fait attendre si longtemps, et qui, de l'avis de Grimaldo, restèrent
dans mes portefeuilles. Sa réponse à moi, glissant sur la retenue des
deux lettres, fut le compliment de conjouissance le plus vif du succès
de ce qu'il m'insinuait doucement être son ouvrage\,; et la lettre qu'il
fit de M. le duc d'Orléans se ressentit du même style. Je tenais mon
affaire et j'en fus content.

Je rendis compte au cardinal, en lui mandant les grâces que je venais de
recevoir, qu'il fallait un parrain pour la couverture de mon second
fils, et pour la Toison de l'aîné, et des raisons qui me les avaient
fait choisir. Au sortir de table à Lerma, de chez le duc del Arco, je le
priai de vouloir faire cet honneur à mon second fils, et il l'accepta de
façon à me persuader qu'il s'en trouvait flatté, et en même temps je
priai le duc de Liria de vouloir bien l'être de l'aîné pour la Toison\,:
je ne pouvais moins pour lui. Il se réputait Français\,; il était fils
aîné du duc de Berwick, que M. le duc d'Orléans aimait et estimait. Il
était ami particulier de Grimaldo\,; il m'avait donné tous les siens,
facilité une infinité de choses\,; il n'y avait sortes d'avances, de
prévenances, d'amitiés, de services que je n'en eusse reçus. Pour le duc
del Arco, M. le duc d'Orléans m'en avait toujours paru content. Il était
favori du roi, était grand d'Espagne de sa main, possédait une des trois
grandes charges, était aimé et estimé et dans la première considération.
J'en avais d'ailleurs reçu toutes sortes de politesses, et il était de
ceux qui venaient manger familièrement chez moi, sans prier', surtout le
soir, quand il en avait le temps. Je crus même que ce choix plaisait au
roi d'Espagne, et ne pourrait que me faire honneur. Ces deux parrains
furent fort approuvés en Espagne et pareillement de M. le duc d'Orléans
et du cardinal Dubois.

Enfin j'écrivis au roi une lettre à part, outre celle d'affaires, pour
le remercier des grâces que sa protection venait de me procurer, parce
que, tout enfant qu'il fût encore, tout lui devait être rapporté. Je
dépêchai un officier de bon lieu du régiment de Saint-Simon infanterie
pour porter avec ces lettres le compte que je rendais du détail du
mariage, en considération duquel je demandais pour lui une croix de
Saint-Louis, la commission de capitaine et une gratification. On verra
plus bas que ce n'est pas sans raison que je rapporte ici ces
bagatelles. Mon courrier partit quelques heures avant moi de mon
quartier de Villahalmanzo et fit diligence. Je suivis la route que la
cour avait prise par des montagnes où jamais voiture n'avait passé. Les
Espagnols sont les premiers ouvriers du monde pour accommoder de pareils
chemins\,; mais c'est sans solidité, et bientôt après il n'y paraît
plus. La cour fut cinq jours en chemin jusqu'à Madrid. J'y arrivai un
jour avant elle.

La princesse des Asturies se trouva incommodée sur la fin du voyage. Il
lui parut des rougeurs sur le visage qui se tournèrent en érésipèle, et
il s'y joignit un peu de fièvre. J'allai au palais, dès que la cour fut
arrivée, où je trouvai Leurs Majestés alarmées. Je tâchai de les
rassurer sur ce que la princesse avait eu la rougeole et la
petite-vérole, et qu'il n'était pas surprenant qu'elle se ressentit de
la fatigue d'un si long voyage et d'un changement de vie tel qu'il lui
arrivait. Mes raisons ne persuadèrent point, et le lendemain, je trouvai
leur inquiétude augmentée. Ce contretemps les contraria fort. Les fêtes
préparées furent suspendues, et le grand bal déjà tout rangé dans le
salon des grands demeura longtemps en cet état. La reine me demanda si
j'avais vu la princesse\,; je répondis que j'avais été savoir de ses
nouvelles à la porte de son appartement. Mais elle m'ordonna de la voir
et le roi aussi.

Rien n'est plus opposé aux usages d'Espagne, où un homme, même très
proche parent, ne voit jamais une femme au lit. Des raisons essentielles
m'avaient fait obtenir qu'on n'y eût point d'égard au coucher des noces,
mais je n'en trouvais point ici pour les violer de nouveau, et d'une
façon encore qui m'était personnelle, et dont la distinction choquerait
les Espagnols contre la vanité à laquelle ils l'attribueraient. Je m'en
excusai donc le plus qu'il me fut possible, sans pouvoir faire changer
Leurs Majestés là-dessus. Les trois jours suivants ils me demandèrent si
j'avais vu la princesse. J'eus beau tergiverser, ils savaient que je ne
l'avais pas vue, et que la duchesse de Monteillano, venue me parler à la
porte de la chambre, n'avait pu me persuader d'y entrer. Ils m'en
grondèrent l'un et l'autre, et me dirent qu'ils voulaient que je visse
en quel état elle était, les remèdes et les soins qu'on lui donnait. Le
roi y allait une ou deux fois par jour, et la reine bien plus souvent,
et ne dédaignait pas de lui présenter elle-même ses bouillons et ce
qu'elle avait à prendre. Je les assurai l'un et l'autre que, si ce
n'était que pour {[}que{]} je pusse rendre compte à M. le duc d'Orléans
de leurs bontés et de leurs soins pour la princesse, j'en étais si bien
informé et dans un si grand repos que je n'avais aucun besoin de la voir
pour témoigner à M. le duc d'Orléans, et le persuader qu'elle était
mieux entre leurs mains qu'entre les siennes. Enfin le troisième jour
ils se fâchèrent tout de bon, me dirent que j'étais bien opiniâtre,
qu'en un mot, ils voulaient être obéis, et qu'ils m'ordonnaient
expressément et bien sérieusement de la voir tous les jours. Il ne me
resta donc plus qu'à obéir.

J'entrai dès le lendemain chez la princesse, auprès du lit de laquelle
je fus conduit par la duchesse de Monteillano. L'érésipèle me parut fort
étendu et fort enflammé. Ces dames me dirent qu'il avait gagné la gorge
et le cou, et que la fièvre, quoique médiocre, subsistait toujours. On
me la fit regarder avec une bougie, quoi que je pusse dire pour
l'empêcher, et on me dit le régime et les remèdes qu'on employait.
J'allai de là chez le roi et la reine qui me faisaient entrer tous les
jours en tiers avec eux, depuis le retour de Lerma, pour me parler de la
princesse, de chez laquelle je leur dis d'abord que j'en sortais. Cela
leur fit prendre un air serein. Ils se hâtèrent de me demander comment
je la trouvais. Après un peu de conversation sur le mal et les
remèdes\,: «\,Vous ne savez pas tout, me dit le roi, il faut vous
l'apprendre. Il y a deux glandes fort gonflées à la gorge, et voilà ce
qui nous inquiète tant, car nous ne savons qu'en penser.\,» Dans
l'instant je sentis ce que cela signifiait. Je lui répondis que je
comprenais ce qu'il me faisait l'honneur de me faire entendre, et assez
pour pouvoir lui répondre que son inquiétude était sans fondement\,; que
je ne pouvais lui dissimuler que la vie de M. le duc d'Orléans n'eût été
licencieuse, mais que je pouvais l'assurer très fermement qu'elle avait
toujours été sans mauvaises suites\,; que sa santé avait toujours été
constante et sans soupçon\,; qu'il n'avait jamais cessé un seul jour de
paraître dans son état ordinaire\,; que j'avais vécu sans cesse dans une
si grande privance avec lui qu'il eût été tout à fait impossible que la
plus légère mauvaise suite de ses plaisirs m'eût échappé, et que
néanmoins je pouvais jurer à Leurs Majestés que jamais je ne m'étais
aperçu d'aucune\,; qu'enfin M\textsuperscript{me} la duchesse d'Orléans
avait toujours joui de la santé la plus égale et la plus parfaite,
rempli chaque jour chez le roi, chez elle, et partout, les devoirs de
son rang en public, et qu'aucun de tous ses enfants n'avait donné lieu
par sa santé au plus léger soupçon de cette nature.

Pendant ce discours, je remarquai dans le roi et la reine une attention
extraordinaire à me regarder, à m'écouter, à me pénétrer, et sur la fin
un air de contentement fort marqué. Tous deux me dirent que je les
soulageais beaucoup de leur donner de si fortes assurances, bien
persuadés que je ne les voudrais pas tromper. Après un peu de
conversation là-dessus le roi me dit qu'à cette inquiétude, que je
calmais, en succédait une autre qui faisait d'autant plus d'impression
sur lui que le mal dont la feue reine son épouse était morte avait
commencé par ces sortes de glandes, et s'était, longtemps après, déclaré
en écrouelles, dont aucun remède n'avait pu venir à bout. Je lui fis
observer que, suivant ce qui nous en avait été rapporté en France, ces
glandes n'avaient paru qu'à la suite d'un goitre qu'elle avait apporté
de son pays, où le voisinage des Alpes les rend si ordinaires, et dont
M\textsuperscript{me} la duchesse de Bourgogne sa soeur, n'était pas
exempte\,; qu'en la princesse il n'y avait rien de pareil, ni dans pas
un de ceux dont elle tirait sa naissance\,; qu'il y avait donc tout lieu
de croire que ces glandes ne s'étaient engorgées que de l'humeur de
l'érésipèle si voisine, et de ne pas douter qu'elles ne se guérissent
avec la cause qui les avait fait enfler. La conversation, qui fut
extrêmement longue, finit par m'ordonner de nouveau et bien précisément
de voir tous les jours la princesse, eux ensuite, et me prier de rendre
un compte exact à M. le duc d'Orléans de leur inquiétude et de leurs
soins, sans toutefois lui laisser rien sentir des ouvertures que leur
confiance en moi les avait engagés à me faire sur les deux origines,
qu'ils avaient appréhendées, du gonflement de ces glandes, qui devaient
demeurer à moi tout seul.

Deux jours après néanmoins, ayant l'honneur d'être en tiers avec eux au
sortir de chez la princesse, je m'aperçus que leur inquiétude subsistait
plus qu'ils ne voulaient me la montrer. Raisonnant avec moi sur cette
maladie et sur ces glandes qui ne diminuaient point encore, et sur les
remèdes qu'on y faisait, ils me dirent qu'ils avaient commandé à Hyghens
d'en écrire un détail fort circonstancié à Chirac, premier médecin de M.
le duc d'Orléans, et de le consulter, comme ayant plus de connaissance
du tempérament de la princesse, sur quoi ils souhaitaient beaucoup que
Chirac, mettant à part les compliments et les lieux communs trop
ordinaires entre médecins, mandât son avis de bonne foi et sans détour à
Hyghens. Cela m'engagea à en écrire en conformité au cardinal Dubois, en
rendant compte à M. le duc d'Orléans et à lui de l'inquiétude, des soins
et des attentions infinies de Leurs Majestés Catholiques pour la
princesse, sans toutefois leur en toucher le véritable motif, sinon à M.
le duc d'Orléans, de ma main, et à lui seul. C'était l'affaire de
Hyghens avec Chirac, s'il trouvait à propos de toucher cette corde.

Tant que la princesse fut malade, je ne pus omettre d'y aller tous les
jours, et chez Leurs Majestés ensuite, sans que jamais elle me dit un
seul mot, quoique ses dames et le princes des Asturies que j'y trouvais
souvent, fissent tout ce qu'ils pouvaient pour m'en attirer quelque
parole. Quand les glandes commencèrent à se dissiper et l'érésipèle à
diminuer, je me contentai d'attendre Leurs Majestés au retour de leur
chasse, et de leur dire un mot en passant.

La couverture de mon second fils se fit le 1er février, jour pour jour,
précisément quatre-vingt-sept ans depuis la réception de mon père au
parlement, comme duc et pair de France. Elle excita une légère
altercation entre le duc del Arco qui, comme parrain, en prit le jour du
roi et en fit avertir les grands, et le marquis de Villena, qui, comme
majordome-major, prétendait que c'était à lui à le faire. J'ai donné
ailleurs la description de cette belle cérémonie pour chacune des trois
classes. Je me contenterai donc de dire ici que le duc del Arco, qui
n'allait que dans les carrosses du roi comme grand écuyer, dans lesquels
il ne pouvait donner la main à personne, sans exception, eut la
politesse de venir prendre le marquis de Ruffec et moi dans son propre
carrosse, avec ses livrées, suivi de celui du duc d'Albe, oncle paternel
de celui qui est mort ambassadeur d'Espagne à Paris, et son héritier,
qu'il avait prié de lui aider dans cette cérémonie, comme le parrain en
prie toujours un grand. Quoique mon fils et moi pussions faire ou dire,
il n'y eut jamais moyen de les faire monter en carrosse avant lui, ni de
les empêcher de se mettre tous deux sur le devant du carrosse. On ne
saurait ajouter à la politesse et à l'attention avec laquelle ils
s'acquittèrent de la fonction qu'ils avaient bien voulu accepter, soit
pour convier à dîner chez moi, en attendant que le roi arrivât dans la
pièce de l'audience où la cérémonie s'allait faire, soit chez moi à y
faire les honneurs, plus et mieux que moi. Je fus extrêmement flatté de
voir un si grand nombre de grands d'Espagne et d'autres seigneurs à
cette couverture, où on m'assura n'en avoir jamais tant vu en aucune, et
au retour chez moi, nous nous trouvâmes quarante-cinq à table, ou
grands, ou de ce qu'il y avait d'ailleurs de plus distingué, avec
d'autres tables qui se trouvèrent aussi, mais plus médiocrement
remplies. J'allai et revins du palais avec le même cortège de suite, de
livrées et de carrosses qu'à ma première audience de cérémonie pour la
demande de l'infante, et je sus que cette parité de pompe fut sensible
aux Espagnols.

Après la cérémonie il y eut chapelle, où j'eus le plaisir de voir mon
second fils sur le banc des grands, de celui des ambassadeurs où
j'étais\,: comme la grandesse était la même et commune entre mon second
fils et moi, je crus devoir me contenter de sa couverture\,; et ne point
faire la mienne. De quelque sotte brutalité qu'en eût usé Maulevrier en
cette occasion de grandesse, je considérai assez le caractère dont il
était revêtu pour l'emporter sur le mépris de sa personne. Je le priai
au festin de la grandesse, car les ambassadeurs n'assistent point aux
couvertures. Il s'en excusa fort grossièrement. Cela ne me rebuta point,
et quoique accablé de visites à recevoir et à rendre, car il faut aller
deux fois chez chaque grand, une pour le prier de se trouver à la
couverture, une autre pour les inviter et leurs fils aînés au repas,
j'allai avec mon second fils chez Maulevrier qui se résolut enfin d'y
venir, et qui y fit d'autant plus triste et méchante figure, que tout ce
qui s'y trouva voulut par un air de gaieté et de liberté peu ordinaire à
la nation, me témoigner prendre part à ma satisfaction, et aussi à la
chère, car il y fut bu et mangé plus qu'on ne fait ici en de pareils
repas. Il me fallut après retourner chez tous les grands avec mon fils,
et chez les autres personnes distinguées qui avaient dîné chez moi ce
jour-là.

J'appris par une lettre du 27 janvier, du cardinal Dubois, le cordon
bleu donné au duc d'Ossone, et la manière dont cela s'était fait, à
laquelle je reviendrai tout à l'heure. J'allai aussitôt attendre le
retour de la chasse, et je suivis Leurs Majestés dans leur appartement
de retraite. Je leur rendis compte de ce qui venait d'être fait pour M.
le duc d'Ossone. Je leur en relevai la singularité, et je leur fis
remarquer qu'on ne savait ni qu'on ne pouvait savoir alors à Paris les
grâces dont il avait plu à Leurs Majestés de me combler. Elles me
parurent extrêmement sensibles à cette marque de considération qu'elles
recevaient en la personne de leur ambassadeur, et me chargèrent de le
témoigner à M. le duc d'Orléans. Le duc d'Ossone avait pris auparavant
son audience de congé\,; mais il demeurait à Paris où il donnait de
belles fêtes en attendant l'arrivée de l'infante.

On s'était franchement moqué de M. le duc d'Orléans et de son cardinal
ministre sur le cordon bleu du duc d'Ossone. Le maître méprisait ces
choses-là qu'il traitait de bagatelles, et le valet n'était pas né, et
n'avait pas même vécu à en savoir là-dessus davantage. La vieille cour
abattue par les découvertes sur elles, sur le duc et la duchesse du
Maine, sur Cellamare, et par le lit de justice des Tuileries, reprenait
peu à peu vigueur à mesure que le parlement relevait la crête et que la
majorité approchait. D'espagnole passionnée qu'elle s'était montrée,
elle était devenue ennemie de l'Espagne depuis la réconciliation de M.
le duc d'Orléans, et n'avait vu qu'avec désespoir le double mariage qui
l'avait immédiatement suivie. Étourdie du coup, elle ne pouvait
supporter le resserrement de ces liens par les bienfaits réciproquement
répandus sur les ambassadeurs, sans de nouveaux dépits. Elle chercha
donc à affaiblir ce que M. le duc d'Orléans se proposa pour le duc
d'Ossone, et du même coup à l'arrêter tout court sur la promotion qui
suit toujours le sacre, et lui persuada aisément que n'y ayant point de
grand maître de l'ordre du Saint-Esprit, parce que le roi, qui n'avait
pu faire encore sa première communion, n'en avait pas reçu le collier,
et portait l'ordre par le droit de sa naissance, sans en être chevalier,
on ne pouvait faire aucun chevalier de l'ordre. Cette raison, si elle
avait mérité ce nom, militait pour l'exclusion de la promotion du
lendemain du sacre, parce que le temps n'y aurait pas permis du jour au
lendemain de nommer les chevaliers en chapitre, à eux de faire leurs
preuves, à un second chapitre, de les recevoir, et d'être arrivés à
Reims avec leurs habits tout faits, le tout en moins de douze heures, à
compter de la fin du festin royal\,; et si le sacre se faisait avant la
majorité, nécessité de l'attendre, pour que le grand maître de l'ordre
pût faire la promotion par lui-même.

Ces bluettes aveuglèrent le cardinal Dubois, et M. le duc d'Orléans eut
plus tôt fait de s'en laisser éblouir que d'y faire la plus légère
réflexion, de sorte que lui et le, cardinal Dubois eurent recours à
leurs \emph{mezzo termine} si favoris, et crurent faire merveilles et un
grand coup d'autorité d'envoyer le cordon bleu au duc d'Ossone, avec
permission de porter dès lors les marques de l'ordre qu'il prit
sur-le-champ, en attendant que le roi fût en état de l'en faire
chevalier. Mais la réponse à ces deux prétendus obstacles était bien
aisée. Henri IV au siège de Rouen, huguenot encore, par conséquent, tout
roi qu'il était, incapable d'être chevalier du Saint-Esprit, et même de
le porter, et d'en être \emph{fonctionnellement} grand maître, expédia
une commission au premier maréchal de Biron pour tenir le chapitre de
l'ordre, et le donner au baron de Biron son fils, devenu depuis duc et
pair et maréchal de France, et qui eut enfin la tête coupée, et d'y
donner en même temps le cordon bleu à Renaud de Beaune, archevêque de
Bourges, depuis de Sens, comme grand aumônier de France, dont Henri IV
venait de lui donner la charge, qu'il venait d'ôter à Jacques Amyot,
évêque d'Auxerre, passionné ligueur. Voilà qui est sans réplique pour
faire des chevaliers de l'ordre sans qu'il y ait de grand maître, et la
cérémonie s'en fit dans l'église paroissiale du faubourg Darnetal de
Rouen\footnote{Darnetal n'est pas, à proprement parler, un faubourg de
  Rouen. C'est une petite ville située à trois ou quatre kilomètres de
  Rouen, et remarquable par ses établissements industriels.} dont le roi
était maître. À l'égard d'un roi, non seulement point grand maître de
l'ordre, mais de plus mineur, Louis XIII, né à Fontainebleau dans le
cabinet de l'ovale, le jeudi 17 septembre. 1601, sur les onze heures du
soir, sacré à Reims le dimanche 17 octobre 1610, n'était ni grand maître
de l'ordre ni majeur, et toutefois il fit le prince de Condé chevalier
de l'ordre le lendemain, de son sacre\,; tellement que de quatre rois
immédiats prédécesseurs du roi, deux seulement, dont l'instituteur de
l'ordre est le premier, et l'autre est le feu roi, étaient majeurs et
sacrés quand ils ont fait des chevaliers de l'ordre\,; et deux autres,
l'un huguenot, par conséquent ni sacré, ni grand maître, ni même portant
l'ordre, l'autre sacré, mineur, ont fait des chevaliers de l'ordre, l'un
par commission, étant hors d'état de les faire lui-même, l'autre le
lendemain de son sacre et sous la régence de la reine sa mère.
Qu'auraient pu répondre à cela ces messieurs de la vieille cour\,? Mais
quoique trivial et moderne, le cardinal n'en savait pas tant, et le
régent ne prenait pas la peine d'y penser un moment, et de se rappeler
ces exemples décisifs.

Quoique chose faite, je ne laissai pas de leur mander ce que j'en
pensais, et qu'ils s'étaient laissé prendre grossièrement pour dupes.
Mais je me gardai bien de dire à personne en Espagne que cela se pouvait
et devait faire autrement, et que la régente sous Louis XIII nomma et
fit faire M. le Prince chevalier de l'ordre. Cela est clair par
conséquent que M. le duc d'Orléans régent avait le même pouvoir. Je leur
rendis compte du très bon effet et de la joie que cette distinction
accordée au duc d'Ossone avait faits dans toute la cour d'Espagne, et
j'en pris occasion de leur représenter combien il était du service de M.
le duc d'Orléans de réserver sept ou huit colliers, qui étaient presque
tous vacants, quand il ferait la promotion entière, et de les envoyer
sans destination au roi d'Espagne pour les donner à qui il lui plairait,
excepté le marquis de Grimaldo, dont les services et le constant
attachement à l'union des deux couronnes méritait la distinction d'être
nommé par le roi uniquement, sur quoi je leur remis devant les yeux la
conduite des rois d'Espagne de la maison d'Autriche, qui envoyaient aux
empereurs toutes les Toisons qu'ils voulaient pour leur cour, et encore
que cet ordre ne soit que de la moitié en nombre de celui du
Saint-Esprit, et leur rappelai aussi le grand nombre de Toisons données
à la France, auquel le petit nombre de colliers du Saint-Esprit accordés
à l'Espagne ne pouvait se comparer, infiniment moins aux grandesses
françaises, qui ne peuvent recevoir d'équivalent. Cette épargne de
colliers à l'Espagne pour les prostituer ici à des gens qui, sous le feu
roi, auraient couru avec incertitude après un cordon rouge, et s'en
seraient crus comblés, n'est pas une des moindres fautes, à tous égards,
en laquelle on s'est si opiniâtrement affermi depuis. Je fis, en même
temps, un reproche à M. le duc d'Orléans d'un dégoût {[}que{]} la
sottise du cardinal Dubois, que je ne nommais point, venait de lui faire
essuyer.

À propos de la résolution prise de donner le cordon bleu au duc
d'Ossone, ce prince, qui croyait si peu avoir le pouvoir de faire des
chevaliers de l'ordre, l'envoya au cardinal Albane. C'était une
reconnaissance du cardinal Dubois pour son chapeau, auquel le cardinal
Albane, entraîné par les lettres pressantes du cardinal de Rohan,
s'était montré favorable, et une galanterie qu'il voulait faire à tout
le parti de la constitution. Il en fut comme des exemples d'Henri IV et
de Louis XIII cités ci-dessus. Dubois, petit compagnon alors, ignorait,
et son maître avait oublié, que le feu roi ayant voulu donner l'ordre au
cardinal Ottobon, protecteur des affaires de France, et brouillé, et
comme proscrit parla république de Venise pour avoir accepté cette
protection, comme on l'a vu ici en son lieu, le refusa tout net, et
répondit qu'encore qu'il eût pris un attachement déclaré pour la France
par cette protection, elle n'était pas incompatible avec rien de ce
qu'il était, mais que le cordon bleu, qui n'était presque jamais que
pour les cardinaux français, ne lui paraissait pouvoir convenir avec sa
charge de vice-chancelier de l'Église, ni avec ce qu'il était d'ailleurs
dans le sacré collège\,: il voulait dire son ancienneté qui touchait au
décanat, sa qualité de neveu d'Alexandre VIII, qui le mettait à la tête
des créatures de son oncle, enfin sa nation\,; et le feu roi eut le
dégoût d'en être refusé. Albane, Italien, camerlingue et chef des
nombreuses créatures de Clément XI son oncle, eut les mêmes raisons. Il
n'avait pas été pressenti auparavant\,; Dubois, qui ne doutait de rien,
ne s'en était pas donné la peine, tellement que le refus tout plat fut
public et l'ordre renvoyé.

Je ne faisais pas cette leçon\,; mais mon reproche fut que Son Altesse
Royale ne pouvait ignorer la promesse publique et réitérée du feu roi au
cardinal Gualterio de la première place de cardinal qui vaquerait dans
l'ordre\,; que ses services et son attachement si marqué, et qui lui
avaient coûté tant de dégoûts depuis son retour à Rome, méritaient à
tant de titres, et non pas le dégoût nouveau, qu'il n'avait jamais
mérité de M. le duc d'Orléans le moins du monde, de se voir oublié et
envoyer l'ordre à un autre cardinal si inférieur à lui, pour ne pas dire
plus, en mérites à l'égard de la France. Mais Dubois gouvernait seul et
en plein. Les grandes et les petites choses dépendaient entièrement de
lui, et M. le duc d'Orléans tranquillement le laissait faire. J'en
écrivis en même temps au cardinal Dubois, et je lui représentai que
l'estime et l'amitié si marquée du cardinal de Rohan pour le cardinal
Gualterio ne pourrait pas être insensible à. une si grande
mortification.

En arrivant de Lerma à Madrid, j'avais reçu une lettre du cardinal
Dubois qui, après des raisonnements sur l'état incertain de la santé du
grand-duc, et de ce qui pouvait se passer en Italie en conséquence, me
mandait que Chavigny, envoyé du roi à Gênes, était si fort au fait de
toutes ces affaires-là qu'il pourrait bien lui envoyer faire un tour en
Espagne et me le recommandait très fortement.

Ce Chavigny était le même Chavignard, fils d'un procureur de Beaune, en
Bourgogne, qui trompa feu de Soubise, et se fit présenter par lui au feu
roi avec son frère, comme ses parents, et de la maison de
Chauvigny-le-Roy, ancienne, illustre, éteinte depuis longtemps, obtint
un guidon de gendarmerie aussitôt, et son frère une abbaye. Ils
obtinrent aussi des gratifications et des distinctions par les jésuites
qui étaient leurs dupes, ou qui feignaient de l'être, et par M. de
Soubise, à l'ombre duquel ils se fourrèrent partout où ils purent. Enfin
reconnus pour ce qu'ils étaient et pour avoir changé leur nom de
Chavignard en celui de Chauvigny, le roi les dépouilla de ses grâces et
les chassa du royaume. Ils errèrent longtemps où ils purent, sous le nom
de Chavigny, pour ne s'écarter que le moins qu'ils purent du beau nom
qu'ils avaient usurpé\,; et quoique si châtiés et si déshonorés,
l'ambition et l'impudence leur étaient si naturelles que ni l'une ni
l'autre ne put en être affaiblie, et qu'ils ne cessèrent, en cédant à la
fortune, de chercher sans cesse à se raccrocher. J'en ai parlé ici, dans
le temps de leur aventure\,; mais j'ai cru en devoir rafraîchir la
mémoire en cet endroit.

En courant le pays, ils se firent nouvellistes, espèce de gens dont les
personnes en place ne manquent pas, tous aventuriers, gens de rien et la
plupart fripons, dont il m'en est passé plusieurs par les mains.
Chavigny avait beaucoup d'esprit, d'art, de ruse, de manège, un esprit
tout tourné à l'intrigue, à l'application, à l'instruction, avec tout ce
qu'il fallait pour en tirer parti\,: une douceur, une flatterie fine,
mais basse, un entregent merveilleux, et le tact très fin pour
reconnaître son monde, s'insinuer doucement, à pas comptés, et juger
très sainement de lâcher ou de retenir la bride, éloquent, bien disant,
avec une surface de réserve et de modestie, maître absolu de ses paroles
et de leur choix, et toujours examinant son homme jusqu'au fond de
l'âme, tandis qu'il tenait la sienne sous les enveloppes les plus
épaisses, toutefois puant le faux de fort loin. Personne plus
respectueux en apparence, plus doux, plus simple, en effet plus double,
plus intéressé, plus effronté, plus insolent et hardi au dernier point,
quand il croyait pouvoir l'être. Ces talents rassemblés, qui font une
espèce de scélérat très méprisable, mais fort dangereux, font aussi un
homme dont quelquefois on peut se servir utilement. Torcy en jugea
ainsi. De bas nouvelliste, il s'en fit une manière de correspondant, et
prétendit s'en être bien trouvé en Hollande et à Utrecht, où néanmoins
il n'osait fréquenter nos ambassadeurs, mais se fourrait chez les
ministres des autres puissances, en subalterne tout à fait, mais dont il
savait tirer des lumières par leurs bureaux, où il se familiarisait, en
leur en laissant tirer de lui qu'il leur présentait comme des hameçons.

Son frère n'en savait pas moins que lui\,; mais son humeur naturellement
haute et rustre le rendait moins souple, moins ployant, moins propre à
s'insinuer et à abuser longtemps de suite. Toutefois ils s'entendaient
et s'aidaient merveilleusement. Ces manèges obscurs, hors de France et
tout à fait à l'insu du feu roi, durèrent jusqu'à sa mort. Elle leur
donna bientôt la hardiesse de revenir en France, où trouvant Torcy hors
de place et seulement conservant les postes et une place dans le conseil
de récence, ils continuèrent à lui faire leur cour pour s'en faire un
patron dans le cabinet du régent, avec qui le secret des postes le
tenait dans un commerce important et intime, mais un patron qui ne
pouvait que les aider. Ils n'osaient pourtant se produire au grand jour,
mais ils frappaient doucement à plusieurs portes pour essayer où ils
pourraient entrer.

Comme ils avaient le nez bon, ils avisèrent bientôt que l'abbé Dubois
serait leur vrai fait, s'ils se pouvaient insinuer auprès de lui, et
que, fait comme il était et comme était aussi M. le duc d'Orléans, il y
aurait bien du malheur si l'espèce de disgrâce où il était lors ne se
changeait bientôt en une confiance qui le mènerait loin, et dont
eux-mêmes pourraient profiter\,; ils cherchèrent donc par où
l'approcher. La fréquentation qu'ils avaient eue en Hollande avec les
Anglais les introduisit auprès de Stairs\,; ils y firent leur cour à
Rémond qui n'en bougeait. Il faut se souvenir de ce qui a été expliqué
ici des premiers temps de la régence, des liaisons, des vues et des
manèges de l'abbé Dubois pour se raccrocher auprès de son maître et
s'ouvrir un chemin à ce qu'il devint depuis. Rémond, peu accoutumé aux
applaudissements et aux respects, fut enchanté de ceux qu'il trouva dans
les deux frères. À son tour, il fut charmé de leur esprit et de leurs
lumières. Il les présenta à Canillac à qui ils prostituèrent tout leur
encens. Lui et Rémond en parlèrent à l'abbé Dubois. Rémond fit que
Stairs les lui vanta aussi\,; il les voulut voir. Jamais deux hommes si
faits exprès l'un pour l'autre que Dubois et Chavigny, si ce n'est que
celui-ci en savait bien plus que l'autre, avait la tête froide et
capable de plusieurs affaires à la fois. Dubois le reconnut bientôt pour
un homme qui lui serait utile, et dont la délicatesse ne
s'effaroucherait de rien. Il l'employa donc en de petites choses quand
lui-même commença à poindre\,; en de plus grandes, à mesure qu'il
avança\,; et en fit enfin son confident dans le soulagement dont il eut
besoin dans ses négociations avec l'Angleterre. Parvenu au chapeau et à
la toute-puissance, et n'ayant plus besoin de ce second à Londres ni à
Hanovre, il l'envoya à Gènes rôder et découvrir en Italie, et enfin
exécuter une commission secrète en Espagne.

Au premier mot que je dis de sa prochaine arrivée au marquis de
Grimaldo, il fit un cri qui m'étonna, il rougit, se mit en colère\,:
«\,Comment, monsieur, me dit-il, dans le moment de la réconciliation
personnelle de M. le duc d'Orléans, dans le moment des deux mariages qui
en sont le sceau, et de l'union la plus intime des deux couronnes et des
deux branches royales, nous envoyer Chavigny, si publiquement déshonoré
qu'il n'est personne en Europe qui ignore une telle aventure\,! Que veut
dire votre cardinal Dubois par un tel négociateur\,? N'est-ce pas
afficher qu'il veut nous tromper que de l'envoyer ici chargé de quelque
chose\,?» Il en dit tant, et plus sur le cardinal, et se déboutonna
pleinement sur l'opinion qu'il avait de lui. Je le laissai tout dire, et
je ne pus disconvenir avec lui que Chavigny ne portait pas une
réputation qui pût concilier la confiance. Mais enfin je lui dis que le
cardinal en avait fait son confident personnel, qu'il l'envoyait sans
m'en avoir rien mandé auparavant\,; que tout ce qu'il m'en marquait
était qu'il l'avait choisi comme étant parfaitement instruit de ce qui
se passait en Italie, en particulier à l'occasion de l'état incertain de
la santé du grand-duc, et que je n'en savais pas davantage.

Grimaldo tout bouffant me répondit qu'ils en savaient autant que lui, et
que si le cardinal l'en croyait si instruit, il n'avait qu'à lui en
faire faire un mémoire et le leur envoyer, et non pas un fripon aussi
connu que cet homme-là, auquel il n'y avait pas même moyen de parler. Je
le laissai encore s'exhaler tant qu'il voulut, puis, le ramenant
doucement peu à peu, je lui dis que si fallait-il bien pourtant qu'il le
vit, quand ce ne serait que pour voir ce qu'il voudrait dire. Grimaldo
me répliqua que quand il pourrait se résoudre à le voir, il m'assurait
bien que le roi ne permettrait pas qu'il se présentât devant lui. Je lui
représentai qu'en convenant avec lui du mauvais air du choix, le régent
aurait droit de se plaindre qu'on ne voulût pas entendre en Espagne
celui qu'il y envoyait\,; et que le roi d'Espagne, dans la position si
heureuse où la France et le régent se trouvaient avec Sa Majesté
Catholique, elle en usât à l'égard de Chavigny comme on fait tout au
plus au moment d'une rupture résolue. Grimaldo me répliqua avec dépit\,:
«\,Et pourquoi nous envoyer un coquin décrié partout\,? n'est-ce pas
tout ce qu'ils pourraient faire dans une rupture\,? que veulent-ils que
nous pensions de ce beau choix et si unique à faire\,? Quelle confiance
prétendent-ils que nous lui donnions\,? Il faut qu'ils nous croient
stupides, et qu'ils aient pour nous le dernier mépris. Mais nous le leur
rendrons bien aussi, et nous leur renverrons leur fripon tout comme il
sera venu. Cela leur apprendra du moins à ne nous plus envoyer des
fripons reconnus, déshonorés par tout le monde, et s'ils nous veulent
tromper, du moins de ne l'afficher pas d'avance, et de nous envoyer des
fripons qui aient du moins la figure de gens ordinaires\,!» Comme je vis
que je ne ferais que l'opiniâtrer davantage, je me retirai, en le priant
du moins d'y penser.

Je retournai le voir le lendemain, et je lui demandai en riant de quelle
humeur il était ce jour-là. Il me fit mille politesses et mille amitiés,
sur lesquelles je pris thème de lui dire qu'il ne me pouvait arriver
rien de plus fâcheux que l'exécution de ce qu'il m'avait dit la
veille\,; qu'il connaissait les fougues du cardinal Dubois\,; qu'il
avait vu, par le délai si affecté de m'envoyer la lettre du roi pour
l'infante, qu'il avait eu dessein de me jeter dans l'embarras dont
j'avais été forcé de lui faire la confidence, et dont il avait eu la
bonté de me tirer\,; qu'il avait vu encore par la faiblesse de sa lettre
à lui, et de celle qu'il avait faite de M. le duc d'Orléans pour le roi
d'Espagne, le peu d'envie qu'il avait que j'obtinsse les grâces de
Leurs. Majestés Catholiques auxquelles lui avait eu toute la part, et
avait voulu supprimer ces lettres, qui l'étaient demeurées en effet,
comme plus nuisibles qu'utiles\,; que j'en aurais bien d'autres à lui
apprendre pour lui faire voir quel était le cardinal Dubois à mon
égard\,; que si Chavigny n'était point écouté, si le roi d'Espagne lui
faisait l'affront de ne vouloir pas permettre que j'eusse l'honneur de
le lui présenter, le cardinal, qui pouvait tout sur M. le duc d'Orléans,
ferait qu'il s'en prendrait à moi, l'imputerait à la jalousie du secret
de ce dont Chavigny était porteur, publierait et persuaderait que je
sacrifiais l'honneur du régent et de la France, l'union et la
réconciliation si récente des deux cours à ma vanité personnelle, et que
traité comme je l'étais en Espagne, on ne pouvait douter que Chavigny
n'y eût été très bien reçu et très bien traité si je l'avais voulu\,;
que je ne serais pas dans le cabinet de M. le duc d'Orléans pour imposer
au cardinal, comme il m'arrivait souvent, ni pour me défendre\,;
qu'enfin j'espérais de son amitié à lui, jointe aux autres
considérations que je lui avais représentées la veille, qu'il ne
voudrait pas me faire échouer au port.

Je lui parlai si bien, ou il avait si bien réfléchi sur ce refus,
qu'enfin il me promit de voir Chavigny et de faire ce qu'il pourrait
pour que je le pusse présenter au roi d'Espagne, sans toutefois me
répondre de venir à bout de ce dernier point. Ce fut tout en arrivant de
Lerma que j'eus ces deux conversations avec lui. Il était arrivé
incommodé et enrhumé, la fièvre s'y joignit après, et il fut sept ou
huit jours sans voir personne, ni sortir de son logis.

Le 16 février Chavigny arriva et me vint voir le lendemain matin. Après
des propos généraux où il déploya toute sa souplesse, ses respects et
son bien-dire, il m'apprit qu'il venait avec une lettre de créance du
duc de Parme, qui comprenant bien l'impossibilité de retirer des mains
du pape le duché de Castro et la principauté de Ronciglione, et toute la
difficulté d'en retirer l'équivalent en terres, il se restreignait à lui
en demander un qui serait aisé, si l'Espagne voulait bien y contribuer
en se joignant à lui pour demander au pape un indult sur le clergé des
Indes, dont le duc de Parme toucherait l'argent à la décharge du
saint-siège, jusqu'à parfait dédommagement. Avec sa manière hésitante et
volontairement enveloppée, il ne laissa pas de me dire, quoique non
clairement, que le cardinal Dubois approuvait fort cet expédient, et je
sentis qu'il y entrait fort pour sortir par là de l'engagement où il
s'était mis avec ce prince pour lui procurer cette restitution.

Ce qui me surprit fut l'aveu de Chavigny, vrai ou supposé, de n'avoir
point de lettres de créance du cardinal Dubois, avec l'air d'un assez
grand embarras, sur quoi je me divertis à lui dire que la confiance de
ce ministre en lui était si généralement connue qu'il n'avait qu'à se
présenter pour obtenir la même des ministres avec qui il pourrait avoir
à traiter. Il se mit après sur les louanges du duc de Parme sagesse,
capacité, considération dans toute l'Italie\,; sur tout, et plus que
tout, il me vanta son attachement de tous les temps pour la France, qui
l'avait exposé à tous les mauvais traitements de l'empereur. Je lui
demandai en bon ignorant comment il s'était comporté dans l'affaire du
double mariage. Chavigny me répondit sans hésiter que tout avait passé
par lui, qu'il y avait fait merveilles, qu'il y avait eu la principale
part. Je pris cela pour fort bon, et tout comme il me le donna, mais il
ne se doutait pas que j'en savais là-dessus autant ou plus que lui.

Lorsque M. le duc d'Orléans me confia pour la première fois les
mariages, avant même que l'affaire fût entièrement achevée, il me dit en
même temps que tout se faisait à l'insu du duc de Parme\,; qu'un secret
profond lui cacherait cette affaire par les deux cours, jusqu'à ce
qu'elle fût entièrement parachevée\,; que M. de Parme était le promoteur
et le principal instrument des mariages des infants d'Espagne avec les
archiduchesses dont il avait toute la négociation. Lorsque les mariages
furent faits, M. le duc d'Orléans me dit qu'ils étaient tombés sur la
tête du duc de Parme comme une bombe\,; qu'il en était au désespoir. Et
quand après le cardinal Dubois et, moi fûmes, comme je l'ai raconté en
son lieu, replâtrés, et que nous fûmes à portée de parler d'affaires et
de mon ambassade prochaine, je lui parlai du duc de Parme, sans lui
laisser rien sentir de ce que M. le duc d'Orléans m'en avait dit, et il
m'en rapporta les mêmes choses précisément que j'en avais apprises du
régent. Ce souvenir, que je ne pouvais avoir que très présent en
Espagne, me confirma de plus en plus dans l'opinion que j'avais de
Chavigny, et de me bien garder de lui en laisser flairer l'odeur la plus
légère. De là, il me battit la campagne avec force bourre, à travers
laquelle il s'étendit, mais fort en général, sur la nécessité de
l'établissement de l'infant don Carlos en Italie, sur les bonnes choses
qu'il y aurait à faire en cette partie de l'Europe, sur le respect où le
double mariage y allait retenir l'empereur à l'égard des deux couronnes,
sur sa faiblesse par faute d'argent. Il finit par me dire qu'il avait un
plein pouvoir de M. de Parme si étendu qu'il lui soumettait son ministre
à Madrid, et lui permettait même d'agir contre l'instruction qu'il lui
avait donnée, s'il le jugeait à propos\,; enfin que ce prince comptait
tellement sur l'amitié et la protection du cardinal Dubois qu'il l'avait
chargé de suivre en tous les ordres de ce ministre sur ce qui le
regardait.

Le soir du même jour, tout tard, Pecquet me vint apprendre que Chavigny
l'avait vu et lui avait dit qu'il arrivait à Madrid pour une commission
qui serait fort agréable, qu'il s'agissait de faire passer don Carlos
actuellement en Italie, de le confier au duc de Parme, de l'accompagner
de six mille hommes dont M. de Parme aurait le commandement, ainsi que
l'administration du jeune prince.

Chavigny me revint voir le lendemain matin, et après la répétition de
plusieurs choses de sa première conversation, et force bourre, pendant
quoi j'étais fort attentif à ne lui pas laisser apercevoir que je susse
la moindre chose sur don Carlos, il m'en parla lui-même avec ses
enveloppes accoutumées. Il me dit que M. de Parme désirait fort d'avoir
dès à présent ce petit prince auprès de lui\,; qu'en ce cas il lui
faudrait donner six mille hommes pour sa garde\,; que l'un et l'autre
rendraient le duc de Parme fort considérable en Italie, et lui
donneraient un maniement de subsides qui l'accommoderait fort, et
l'administration du jeune prince. Je lui fis quelques légères objections
pour l'exciter à parler. Il me dit qu'il était vrai que ce passage
n'était peut-être pas bien nécessaire à l'âge de l'infant, que néanmoins
sa présence en Italie pourrait contenir les partis qui se formaient
parmi les Florentins pour se remettre en république après la mort du
grand prince de Toscane, et encouragerait ceux qui voulaient un
souverain\,; mais qu'au fond ce passage actuel é toit sans aucun
inconvénient. Il me dit cela d'un air simple, comme si en effet il
s'agissait d'une chose indifférente. Je lui répondis, avec la même
apparente indifférence, que je n'en savais pas assez pour voir les
avantages et les inconvénients de ce projet qu'il m'assura, en passant,
être fort du goût de la cour d'Espagne. J'ajoutai que je croyais que par
caractère, et par capacité également démontrée par le double mariage et
par les affaires du nord, le cardinal Dubois devait être la boussole sur
laquelle uniquement on se devait régler\,; qu'il avait si profondément
le système de l'Europe dans la tête, et l'art de combiner et d'en tirer
les plus grands avantages, que c'était de lui et de ses lumières qu'on
devait attendre les ordres pour s'y conformer entièrement.

Là-dessus Chavigny me dit, avec un air d'ingénuité plaintive, que
c'était là tout ce qui faisait son embarras\,; qu'il y avait dix mois
que cette affaire de don Carlos se traitait\,; qu'il en avait souvent
écrit au cardinal Dubois, sans en avoir jamais reçu là-dessus aucune
réponse\,; qu'il s'était contenté de lui écrire sur l'affaire de Castro
et de Ronciglione, de lui prescrire de se rendre à Madrid pour y donner
un compte général des affaires d'Italie, sans entrer même en beaucoup de
détails là-dessus avec la cour d'Espagne, et d'agir pour M. de Parme
suivant qu'il lui ordonnerait touchant Castro et Ronciglione. Je me mis
à sourire, et je lui dis que, si M. le cardinal ne s'expliquait pas sur
l'affaire du passage, j'en suspendrais aussi mon jugement, ce qui me
serait d'autant plus aisé que je n'avais plus que peu de jours à
demeurer à Madrid. Il me répondit, en reprenant son air de plainte,
qu'il n'avait pas seulement d'instruction ni de lettres de créance du
cardinal Dubois pour la cour d'Espagne\,; puis, reprenant un air plus
satisfait, il ajouta tout de suite que cette façon était aussi plus
simple entre deux cours aussi étroitement unies que l'étaient celles de
France et d'Espagne. Il fallait que Chavigny me crût bien neuf pour
tâter de cette sottise. Je ne pus m'empêcher de lui répondre, mais en
riant en moi-même, que ce qui constituait le ministre était moins sa
lettre de créance que celle qu'on lui voulait bien donner, et les
affaires qu'on traitait avec lui. Et comme le cardinal Dubois me l'avait
extrêmement recommandé, et que j'avais vaincu la répugnance du marquis
de Grimaldo, je crus lui devoir offrir de le mener chez ce ministre dès
qu'il serait visible, et au roi d'Espagne, comme un homme de la
confiance du cardinal Dubois avec lequel on pouvait traiter, ce qu'il
accepta avec beaucoup de satisfaction et de remerciements.

De ces deux conversations, avec ce que dans l'entre-deux j'avais appris
de Pecquet, je compris aisément que la mission apparente de Chavigny,
quoique effective, était l'affaire de Castro et de Ronciglione\,; mais
que ce qui l'amenait véritablement à Madrid était le passage actuel de
don Carlos en Italie. Ce qui me confirma encore dans cette persuasion
fut que j'appris deux jours après qu'on armait six vaisseaux de guerre
et quatre frégates à Barcelone, pour être prêts à la fin de mai, même
avec beaucoup d'indiscrétion, c'est-à-dire à grands frais et avec
beaucoup de bruit.

Avant que d'expliquer mon sentiment sur la mission de Chavigny, et ce
que je crus devoir faire en conséquence, il faut expliquer l'état
d'alors de la cour d'Espagne, des cabales de laquelle je n'ai donné
qu'un simple crayon jusqu'à présent.

Le P. Daubenton avait très certainement été le seul confident avec le
cardinal Albéroni de l'entreprise méditée sur Naples, et faite ensuite
en Sicile. Ils se craignaient et se ménageaient réciproquement\,; et le
jésuite, qui ne voulait pas hasarder de perdre sa place une seconde
fois, qui seule le pouvait conduire au chapeau où il tendait sourdement
de toutes ses forces, tremblait intérieurement devant Albéroni, qui le
sentait et en profitait pour s'en servir comme il lui convenait, sans
s'aimer le moins du monde\,: c'est ce qu'on a vu répandu en mille
endroits de ce que j'ai donné de M. de Torcy sur les affaires
étrangères. Tous deux haïssaient Grimaldo, pour lequel ils craignaient
l'affection et le goût du roi. Quoiqu'ils l'eussent chassé des affaires
et du palais, et quoi qu'on eût fait, depuis les changements de
ministère, pour réunir le P. Daubenton et Grimaldo, jamais le confesseur
ne put lui pardonner le mal qu'il lui avait fait, en sorte qu'il n'y eut
jamais entre eux que des apparences très superficielles. Castellar,
secrétaire d'État de la guerre, et très capable de cet emploi, était au
désespoir que les troupes ne fussent point payées, de les voir
journellement se détruire, et les officiers qui étaient dans l'étendue
de la couronne d'Aragon réduits à se faire nourrir par charité dans les
monastères\,; que tous les projets qu'il avait présentés pour y remédier
fussent toujours remis à un examen qui ne se faisait point\,; et tout
cela je le savais de lui-même. Il accusait Grimaldo de soutenir le
marquis de Campoflorido, ministre en chef des finances, malade depuis
deux ans, hors d'état de donner ordre à rien, et qui mourut avant mon
départ de Madrid, à qui pourtant toutes les choses qui regardaient les
finances étaient renvoyées, qui demeuraient toutes et tombaient dans la
dernière confusion, sans que le roi d'Espagne y fît autre chose
qu'attendre sa guérison, ni voulût, même par intérim, prendre aucun
parti là-dessus.

Castellar, qui m'avait fait ces mêmes plaintes, mais sans me parler de
Grimaldo, avait désiré d'être remis en union avec lui, qui s'était
altérée entre eux. On y avait travaillé utilement, et on fut surpris
que, dans le temps que Grimaldo s'y prêtait le plus, Castellar, de
propos délibéré, se retira tout d'un coup, et mit les choses en beaucoup
plus mauvais état qu'elles n'avaient été auparavant. L'époque de cette
conduite bizarre de Castellar fut {[}celle{]} du voyage de Lerma\,; et
la maladie qui, au retour, retint Grimaldo près de quinze jours au lit,
sans sortir ni voir personne, fut attribuée par gens bien instruits à
deux chagrins violents que ce ministre essuya en arrivant de ce voyage.
Dans ce même temps, Castellar était souvent enfermé avec le P.
Daubenton, entrait chez lui par une porte de derrière, {[}et{]} en
sortait bien avant dans la nuit. Le confesseur était étroitement uni
avec Miraval, gouverneur du conseil de Castille. Le lien de cette union
était qu'Aubenton faisait, depuis quelque temps, renvoyer toutes les
affaires par le roi d'Espagne aux consultes, c'est-à-dire aux conseils
et aux tribunaux, en quoi le confesseur trouvait parfaitement son
compte, parce que tout était à la cour d'Espagne affaire de conscience,
et que, sur le renvoi ou la réponse des différentes consultes que le roi
lui renvoyait toujours, la vraie décision en demeurait au jésuite tout
seul, qu'il montrait comme sienne à qui elle était favorable, et comme
venant des conseils et des tribunaux à qui elle était contraire. D'un
autre côté Miraval était dans la liaison la plus intime avec le duc de
Popoli jusque-là que, contre la dignité de sa place de gouverneur du
conseil de Castille, inviolablement conservée jusqu'alors, et dont
Miraval était lui-même fort jaloux, il allait souvent chez le duc de
Popoli au palais, et demeurait fort longtemps tête à tête avec lui dans
sa chambre.

De tous les Italiens Popoli était le plus dangereux par son esprit et
par sa haine pour la France. Il était l'âme de la cabale italienne qui
se réunissait toute à lui, laquelle détestait la France et l'union.
Cellamare, qui portait le nom de duc de Giovenazzo depuis la mort de son
père, était revenu deux jours avant mon arrivée de Galice, où il
commandait, sans apparence d'y retourner, ni qu'on y renvoyât personne
en sa place, et faisait sa charge de grand écuyer de la reine, avec qui
il était fort bien. Le prince Pio était aussi de retour de Catalogne où
il commandait, et préférait à ce bel emploi la charge, sans fonctions,
de grand écuyer de la princesse des Asturies, qui n'avait point
d'écurie, servie par celles de Leurs Majestés. Tout cela montrait qu'on
rassemblait à Madrid les principaux seigneurs italiens pour les
consulter sur les affaires d'Italie, comme le duc de Popoli le fut sur
l'entreprise de Naples dont il fournit tous les mémoires. Castellar ne
pouvait avoir si brusquement changé sur sa réconciliation avec Grimaldo
sans avoir subitement pris d'autres vues et s'être assuré d'autres
ressources, qui ne pouvaient être autres que le confesseur et les
Italiens, et se mettre bien avec la reine en flattant son ignorance des
affaires et son ambition sur le passage de don Carlos, qui d'ailleurs
convenait si bien à Castellar, parce que cela forçait le roi d'Espagne à
mettre enfin ordre à ses troupes et à ses finances, à quoi il buttait
pour sa caisse militaire. Et comme il était très vrai que le désordre
des finances ne venait que par faute d'administration, parce que le
fonds en était très bon, et pour ainsi dire sans dettes, Castellar
aurait vu avec plaisir quelque rupture en Italie, qui n'aurait pu
qu'augmenter le crédit et l'autorité de sa charge. C'était là le désir
suprême de la cabale italienne, tant pour se mêler d'affaires et
acquérir de la considération et du crédit, que dans le désir et
l'espérance toujours subsistante, pour raccrocher une partie de leurs
biens d'Italie, d'essayer, contre toute raison, quelque restitution au
roi d'Espagne de ce que l'empereur lui détenait, dont, au pis aller, le
mauvais succès ne pouvait rendre à cet égard leur condition pire.

Cette vision, quelque insensée qu'elle fût, méritait d'autant plus
d'être considérée qu'il était arrivé à Chavigny de lâcher un grand mot à
Pecquet, dans une seconde conversation qu'il eut avec lui, et dont
Pecquet me rendit compte incontinent après. Raisonnant ensemble de ce
passage actuel de don Carlos en Italie, Pecquet lui dit que c'était
l'envoyer bien matin pour une succession si éloignée, à quoi Chavigny
répondit avec sa tranquille et balbutiante douceur\,: «\,Il faudrait
quelque chose de présent, quelque chose de présent.\,» Or ce quelque
chose de présent ne pouvait s'arracher que par la force, et je découvris
en même temps que le duc de Popoli avait été consulté, comme il l'avait
été sur l'entreprise de Naples. Outre cet objet de la cabale italienne
qui vient d'être expliqué, elle avait encore celui de brouiller les deux
couronnes, ce qu'elle prévoyait facile si elle pouvait parvenir à faire
attacher quelque chose en Italie, par la difficulté des secours
militaires, et bien autant par l'impossibilité de satisfaire toutes les
volontés de la reine, dont les Italiens se sauraient bien prévaloir pour
faire naître des brouilleries continuelles avec notre cour, qui n'en
ferait jamais assez à son gré, ni au leur, devenus maîtres de son esprit
en flattant et entretenant son ambition. Le duc de Bournonville, déjà
uni avec la cabale italienne, dès avant sa nomination à l'ambassade de
France, de laquelle je parlerai ensuite, ne bougeait plus d'avec les
Italiens, particulièrement d'avec Popoli et Giovenazzo, au premier
desquels il faisait bassement sa cour. Ils furent tous deux embarrassés,
jusqu'à en être déconcertés d'avoir été rencontrés par l'abbé de
Saint-Simon à la promenade, tête à tête.

Le roi et la reine d'Espagne, leurs deux confesseurs, les deux
secrétaires d'État principaux ne se cachaient point du dégoût et des
soupçons qu'ils concevaient du nombre de ministres dont la France se
servait en leur cour, disaient hautement et nettement qu'ils ne savaient
en qui se fier\,; que quand on voulait agir de bonne foi, il ne fallait
qu'un canal. Le P. Daubenton s'expliqua même que cette conduite de la
France lui faisait prendre le parti de se mettre à quartier de tout, et
de ne se mêler de quoi que ce fût\,; et je m'aperçus très bien qu'il
s'était tenu parole avec moi-même. Je sus qu'il avait conseillé la même
conduite à d'autres, et à Castellar à diverses reprises. Quoique cette
multiplicité si peu décente fût très propre à produire cet effet, il put
très bien être aussi une suite de la liaison du confesseur avec
Castellar et Miraval et avec les Italiens. Castellar, qui m'avait
infiniment recherché, et fort entretenu avant et depuis Lerma, s'en
était retiré tout à coup, et ne me témoignait plus que de la politesse
quand nous nous rencontrions\,; je ne laissai pas de le prier deux fois
à dîner chez moi dans ce temps-là, où il venait auparavant fort
librement de lui-même.

Enfin un dernier objet, mais vif, de cette cabale italienne, était de
perdre radicalement Grimaldo et par haine personnelle et comme obstacle
à leurs projets, desquels il était très éloigné par principes d'État et
encore par aversion d'eux comme de ses ennemis\,; par mêmes principes
d'État très favorables à la France, entièrement dévoué à l'union, seul
vraiment au fait des affaires étrangères, fort Espagnol et tout à eux,
et comme eux tous dans l'aversion active et passive des es Italiens.

Après l'exposition fidèle de ce tableau de la cour d'Espagne alors, j e
viens à celle de ce que j e conçus des deux points dont Chavigny m'avait
entretenu, comme du sujet de son arrivée à Madrid.

Je ne vis aucune sorte de bien à espérer du passage actuel de don Carlos
en Italie. Ce n'était qu'un enfant dépaysé dont la présence ne pouvait
hâter la succession qu'on espérait pour lui, qui dépendait de la vie des
possesseurs doubles dans chacun des États de Parme et de Toscane, et il
me parut qu'un tel déplacement, sans aucun fruit qui en dût
naturellement résulter, devait pour le moins être mis au rang des choses
inutiles, et par cela seul destitué de convenance et de sagesse, sans
compter la dignité.

À l'égard des inconvénients, ils me parurent infinis. Hasarder pour rien
la santé d'un enfant de cinq ou six ans\,; l'accompagner nécessairement
de personnes qui voudraient considération et profit, qui par conséquent
donneraient jalousie aux principaux du pays\,; et si on le livrait entre
les mains des Parmesans, comme une fille qu'on marie en pays étrangers,
ces Parmesans mêmes voudraient tirer considération et profit de leurs
places auprès du petit prince, et donneraient aux autres Parmesans la
même jalousie. L'enfant venant à croître, en serait gouverné, excité par
eux à vouloir se mêler des affaires pour y avoir part eux-mêmes.

Le prince, croissant toujours, s'ennuierait de son état de pupille, et
n'ayant pas un pouce de terre à lui, ne pourrait être autre chose, d'où
résulteraient des cabales et des brouilleries qui feraient également
repentir les possesseurs et leur futur héritier de se trouver ensemble,
dont les suites ne pourraient être que très fâcheuses, et peut-être
devenir ruineuses à tous. Cette situation pourrait durer nombre d'années
de la maturité du prince, parce que le frère et successeur direct du duc
de Parme n'avait lorsque quarante-deux ans, et le grand prince de
Toscane, successeur direct du grand-duc son père, n'en avait que
cinquante-trois\,; que si par l'événement le grand prince de Toscane ou
le duc de Parme, beaucoup plus jeune que la duchesse de Parme, venaient
à perdre leurs épouses, que l'amour si naturel de leur maison et d'avoir
postérité les engageât à se remarier, ou seulement que le prince de
Parme, qui n'était point marié, s'avisât de prendre une femme, quelle
pourrait devenir alors la situation de don Carlos\,?

Je considérai que ce prince était de droit petit-fils de France, et par
accident fils de France, en rang et en traitement, fils du roi
d'Espagne, cousin germain du roi et son futur beau-frère. Nos simples
princes du sang jouissent depuis longtemps par toute l'Europe d'un rang
plus distingué que nulle autre maison régnante. MM. les princes de Conti
trouvèrent des électeurs à Vienne et en Hongrie sur lesquels ils
conservèrent toujours la supériorité, dans une sorte d'égalité qui ne
les empêchait pas de les précéder sans, embarras ni difficulté.
Néanmoins l'électeur de Bavière, qui en était un, sut, depuis son union
avec la France, usurper d'abord, puis se faire donner des distinctions
jusqu'alors inouïes et jamais prétendues sur les premiers sujets du roi
et par ce {[}sur{]} les généraux en chef de ses armées, d'où il résulta
que ce même électeur, qui s'était toujours contenté d'un tabouret devant
le prince d'Orange, devenu roi d'Angleterre, assis dans un fauteuil,
venu à Paris, obtint l'\emph{incognito} de la complaisance du feu roi,
d'en être reçu debout, sans aucun siège pour l'un ni pour l'autre,
toutes les fois qu'il le vit, et que le roi souffrit l'énormité de sa
prétention de la main chez Monseigneur, puisqu'il consentit qu'il ne le
verrait que dans les jardins de Meudon, sans entrer dans le château, et
qu'ils montassent tous deux dans la même calèche en même instant, chacun
par sa portière, ce qui n'avait jamais été prétendu par aucun souverain,
même sans être incognito, quoique dans le même temps l'électeur de
Cologne, son frère, mais plus raisonnable, incognito aussi, mais vêtu en
évêque, ne prétendit rien de semblable, et vit debout le roi dans un
fauteuil, après souper, avec sa famille, plus d'une fois, où
véritablement Monseigneur et Mgrs ses fils étaient debout aussi, et les
princesses sur des tabourets.

À l'égard de Monseigneur, il le vit à Meudon, y dîna avec lui, vis-à-vis
de lui au bas bout, avec les dames et les courtisans, tous sur des
sièges à dos, faits pour la table, comme à l'ordinaire, et suivit
toujours Monseigneur, se reculant même aux portes, qui lui montra toute
la maison, puis les jardins, où l'électeur ne fit aucune difficulté de
monter dans la calèche de Monseigneur toujours après lui. De ces
variations on pouvait conclure quels seraient les embarras du cérémonial
entre don Carlos, le duc de Parme lui-même, les autres princes d'Italie,
les cardinaux et les autres principaux grands, desquels tous il faudrait
continuellement encourir la haine pour des points de cérémonial, ou
laisser flétrir en sa personne la dignité de sa naissance et celle des
deux couronnes.

Rien ne m'avait été plus recommandé en partant que d'écarter toutes les
idées de la cour d'Espagne sur l'Italie, particulièrement sur tout ce
qui pouvait de près ou de loin tendre à quelque entreprise et -à quelque
rupture de ce côté-là. Rien n'y pouvait pourtant conduire d'une façon
plus directe que ce passage actuel de don Carlos avec des troupes.
C'était réveiller toute l'Europe sur un projet dont elle s'embarrassait
peu, tandis qu'il paraissait éloigné au point où il l'était par sa
nature, mais qui aurait tout à coup changé de face dès qu'on aurait vu
paraître don Carlos armé en Italie. Il aurait fallu payer et entretenir
ces troupes, et ce n'eût pas été aux dépens du duc de Parme. Quand bien
même ce prince eût pu consentir de soudoyer ces troupes de l'argent qui
lui serait accordé par le pape, et par le roi d'Espagne, de l'indult sur
le clergé des Indes pour le payement de Castro et de Ronciglione, indult
néanmoins qui était une chimère, on aurait dû s'attendre que l'Espagne,
sur les sujets de laquelle ces sommes seraient tirées, nous aurait
demandé de contribuer de notre part. L'empereur, qui ne verrait point
cet événement sans une jalousie extrême, pourrait prétendre de s'y
opposer par la voie des armes, comme à une chose qui, n'ayant point
d'apparence par l'éloignement naturel de ces successions, le menacerait
d'une manière effective. Mais par impossible, prenant la chose avec plus
de modération, il pourrait prendre une autre voie qui, à la fin, ne
conduirait pas moins à la rupture\,: il dirait que les États de Parme et
de Toscane sont menacés d'invasion, tout au moins d'oppression\,;
qu'encore que le duc de Parme y consentît pour le sien, lui empereur
n'était pas moins obligé de protéger ses feudataires. Il prétendrait
garder les places de ces États\,; il y trouverait toute sorte de
facilité pour celui de Toscane\,; et pour six mille hommes que nous
aurions en Italie, il y en aurait le nombre que bon lui semblerait, avec
toute la facilité que lui donnent les États qu'il possède en Italie, et
que lui présente le passage par le Tyrol de ce qu'il y voudrait envoyer
d'Allemagne. Le roi de Sardaigne, qui gardait si étroitement ses
frontières dans la crainte de la peste, aurait ce prétexte pour nous
refuser tout passage, et les Suisses pareillement, qui n'auraient osé
choquer l'empereur. Nous serions donc par là, et l'Espagne par sa
situation naturelle, à ne pouvoir secourir don Carlos tant de recrues
que de troupes d'augmentation, sinon par mer, dont les transports sont
infiniment ruineux, et dont l'Espagne a peu de moyens, et de vaisseaux
encore moins. Alors l'Angleterre avec ses flottes deviendrait maîtresse
des secours. Quelque bien que nous fussions avec elle, il ne faudrait
pas se flatter qu'un prince d'Allemagne, tel que de son estoc était le
roi d'Angleterre, résistât aux mouvements de l'empereur dans le point le
plus sensible, tel que lui était l'Italie. Il faudrait de plus compter
que la jalousie de se conserver le port Mahon et Gibraltar, que les
Anglais ont usurpé dans le sein de l'Espagne, lui ferait embrasser
ardemment cette cause de l'empereur, dans la crainte que l'établissement
d'une branche d'Espagne en Italie ne le forçât enfin à la restitution.
Une entreprise si prématurée pour du présent en Italie à don Carlos,
n'aurait pas manqué d'échauffer les esprits de toutes parts, jusqu'à
produire une guerre où bientôt après la France n'aurait pu éviter
d'entrer. Et comme il s'y agirait de fiefs de l'Empire\,; que le roi de
Pologne avait marié le prince électoral de Saxe, son fils, à une
archiduchesse\,; que l'électeur de Bavière recherchait passionnément
l'autre archiduchesse pour le sien, ces deux princes, les plus
considérables de l'Empire, regarderaient d'un oeil de propriété les
États héréditaires de l'Empire, tellement qu'avec le concours certain du
roi d'Angleterre, électeur d'Hanovre, cette guerre deviendrait aisément
une guerre de l'Empire. Or, en quelque disette d'argent que pût être
l'empereur, il n'est jamais si puissant ni si riche que lorsqu'il a une
guerre de l'Empire. Ses prétentions sur nos bords du Rhin, môme sur les
trois évêchés', et qu'il n'abandonnera jamais, ses difficultés
subsistantes avec lui pour les limites entre ses Pays-Bas et les nôtres,
lui fourniraient bientôt des prétextes de porter la guerre sur ces deux
frontières, et je ne voyais point que nous fussions en état de la bien
soutenir par nous-mêmes ni par nos alliances. Je sentais le triste état
de nos finances, et je voyais le désordre de celles d'Espagne. Notre
épuisement d'hommes se présentait à moi, et je le trouvais encore plus
grand en Espagne. Notre peste, par surcroît de malheur, détruisait
encore les hommes, et les finances aussi par l'interruption du commerce.
Nous touchions au congrès de Cambrai, que cette guerre aurait dissipé ou
tourné contre nous\,; et, pour ne rien oublier, le roi, majeur dans un
an, à qui on ne manquerait pas de peindre cette entreprise avec les
couleurs les plus noires.

Toutes ces raisons mises d'un côté, l'inutilité indécente du passage de
don Carlos actuellement, même de bien longtemps, de l'autre, et avant
l'ouverture de la première des deux successions, me fit conclure que si
j'étais du conseil de l'empereur, je ne désirerais rien davantage qu'une
telle entreprise si fort à contre-temps, qui ne pouvait mériter que le
nom d'une folle équipée, qui n'aurait pu que lui procurer une
augmentation de grandeur en Italie et en Europe, une grande jalousie et
l'épuisement aux deux couronnes, et tout au moins faire échouer
l'établissement de don Carlos en Italie. Que si, au contraire, je
m'étais trouvé à la tête du conseil du roi ou de l'Espagne, je n'aurais
songé qu'à éteindre l'inquiétude causée par la nouvelle réunion des deux
branches royales et des deux couronnes par la plus profonde apparence
d'inaction, de prétentions, de désirs\,; qu'à éviter tout ce qui
pourrait entraîner le plus petit engagement\,; qu'à terminer utilement
le congrès de Cambrai pour nous procurer une situation stable, paisible,
assurée avec tous nos voisins\,; entretenir une longue et profonde
paix\,; éteindre toute crainte et tous soupçons, quelque légers qu'ils
puissent être\,; étreindre soigneusement l'union des deux couronnes\,;
profiter continuellement mais doucement et sans éclat des avantages de
son commerce\,; acquérir au roi la confiance, et, s'il était possible,
la dictature de l'Europe, et se faire de plus en plus aimer et
considérer, par assoupir les différends, étrangers à nous, des grandes
et des petites puissances\,; n'oublier rien pendant ce grand repos pour
réparer les finances\,; faire respirer les peuples, les laisser
multiplier, croître, devenir robustes et féconds, par leur laisser les
moyens de se nourrir, et de fournir utilement à l'agriculture et aux
autres travaux\,; réparer soigneusement et augmenter doucement notre
marine, ou, pour mieux dire, la créer peu à peu de nouveau\,; ne point
perdre de vue le grand événement, quoique très apparemment très éloigné,
de la mort de l'empereur, sans enfants mâles, ni la faute énorme de la
guerre qui fut terminée par la paix de Ryswick, qui ligua toute l'Europe
contre la France, et que cette paix faite depuis deux ans n'avait pas
encore assez séparée pour ne s'être pas incontinent rassemblée dès
qu'elle vit la France résolue à profiter du testament de Charles II et
du voeu unanime de tous les Espagnols, quoique si affaiblie d'hommes et
d'argent, et n'avait pas eu le temps de respirer depuis la fin de cette
dernière guerre, qui avait duré dix ans contre toute l'Europe\,; enfin
se mettre en état, à force de sagesse au dehors et de soins continuels
au dedans, de pouvoir bien profiter de l'ouverture des successions
auxquelles don Carlos était appelé du consentement de toute l'Europe, en
faire un grand prince en Italie, capable d'y tenir de court la puissance
de la maison d'Autriche, et si elle venait à s'éteindre tôt ou tard, se
trouver en force et en moyens de profiter grandement de sa chute.

Pour l'affaire de Castro et de Ronciglione, elle était si chimérique
qu'il suffira de raconter ici qu'ayant rencontré le P. Daubenton au
palais, qui, d'un air instruit de tout, me demanda si Chavigny m'avait
dit le sujet de son voyage, je ne jugeai pas à propos de lui parler
d'autre chose que de l'indult, sur quoi le bon père se prenant à rire me
répondit qu'il était assez plaisant de payer et de retirer ses dettes
sur le fonds d'autrui, et riant encore plus fort, ajouta qu'il ne savait
pas si cette voie accommoderait fort le roi et ses sujets. Je me mis à
rire aussi, et je l'assurai que je laisserais cette fusée à démêler à
qui en était chargé. Il me demanda ensuite avec quelque empressement si
je ne savais rien de plus. Quoiqu'il pût être que Chavigny lui eût
confié qu'il m'en avait parlé, j'aimai mieux me tenir fermé qu'entrer en
affaire avec un homme dont les liaisons, ci-dessus expliquées, le
jetaient très vraisemblablement dans une opinion toute différente de
celle que j'avais prise, et dont je ne le ferais pas revenir, parce que
les meilleures raisons échouent toujours contre celle des intérêts
personnels et des cabales, et que, de plus, j'ignorais les sentiments du
cardinal Dubois là-dessus. J'en sortis donc par lui dire que les fêtes
du carnaval et les fonctions des premiers jours de carême ne m'avaient
permis d'entretenir Chavigny qu'à la hâte.

Cette ignorance où j'étais de ce que le cardinal Dubois pensait sur ce
passage de don Carlos en Italie, et sur cet étrange présent qu'il
faudrait à ce prince que Chavigny avait lâché à Pecquet, m'embarrassa
beaucoup. Dubois et Chavigny étaient si faux, si doubles, si consommés
fripons et si parfaitement connus pour l'être qu'il n'y avait personne
qui ajoutât la moindre foi en leurs discours\,; par-dessus cela, si
sordidement intéressés, si ambitieux, si étrangement personnels, si
profonds en leurs vues et leurs allures, si fort méprisant tout autre
intérêt que le leur particulier, si excellemment impudents, et si
étroitement liés de confiance par leur commune scélératesse, à laquelle
tous moyens étaient bons, quels qu'ils pussent être, et si accoutumés
aux voies les plus tortueuses que les serpents ne pouvaient être d'un
plus dangereux ni d'un plus difficile commerce. Je ne pouvais donc
allier ces deux choses si opposées\,: l'une que Chavigny fût venu en
Espagne sans lettres de créance du cardinal Dubois\,; l'autre que,
chargé de deux affaires par le duc de Parme, il n'eût d'ordre du
cardinal que sur la première, et encore faible, et que sur l'autre, qui
était si importante, non seulement il n'en eût point, mais que depuis
dix mois qu'elle se tramait, et que Chavigny lui en écrivait, il n'en
eût pas reçu là-dessus un seul mot de réponse.

Cette affection me semblait étrange, encore plus l'aveu très volontaire
que Chavigny m'en faisait\,; et que, malgré un silence si opiniâtre, il
osât mettre sur le tapis une affaire de cette conséquence, lui si
mesuré, si froid, si circonspect, et si fort au fait de l'incomparable
jalousie d'autorité du cardinal Dubois qui ne souffrait pas qu'une
affaire de la plus petite bagatelle se traitât sans sa participation. Je
soupçonnai donc là-dessus un jeu joué entre le maître et le valet que
celui-ci savait bien ce qu'il faisait, et que l'autre avait ses raisons
de le faire agir ainsi sans y vouloir paraître. Mais de pénétrer les
raisons d'un homme qui n'agissait que par intérêt personnel, auquel il
rapportait et soumettait sans bornes les plus grands intérêts de l'État,
très souvent encore par fougue ou par caprice, c'était ce qu'il n'était
pas possible de découvrir. Je n'osai donc hasarder de lui écrire de
cette affaire. Il ne m'en avait écrit en aucune sorte, et son confident
Chavigny se plaignait gratuitement à moi de n'en avoir pu tirer un seul
mot de réponse là-dessus. Je n'avais donc aucun compte à rendre de ce
dont je n'étais point chargé, et que je pouvoir ignorer\,; mais la chose
me parut tellement importante que je ne pus pour cela m'en tenir quitte.

J'avais laissé Belle-Ile, ami intime de Le Blanc, duquel le cardinal
Dubois se servait en toutes choses, en usage d'aller tous les soirs avec
Le Blanc passer une heure chez le cardinal seuls avec lui, à parler de
toutes sortes d'affaires. Mon fils aîné devait s'en retourner
incessamment à Paris. Par lui, je fis à Belle-Ile une ample dépêche de
tout ce que je viens d'expliquer et de raconter. Je le priai de la
communiquer à Le Blanc, et de voir ensemble ce qu'ils pourraient faire
pour empêcher l'exécution d'un projet, dont l'absurdité était la moins
mauvaise partie. En même temps je fis prier Grimaldo par Sartine que je
le pusse voir dès qu'il serait en état d'entendre un peu parler
d'affaire qui pressait, et que ce fût même avant de recommencer d'aller
travailler au palais. Il le fit en effet de très bonne grâce, et c'est
la seule fois que je l'aie vu dans sa maison à Madrid. Je lui appris
tout ce que j'avais su de Chavigny, et il me parut que je lui faisais
grand plaisir. Il admira autant que moi ce manège apparent de silence
obstiné du cardinal avec Chavigny sur le passage de don Carlos, et
l'apparente témérité de cet intime confident de la traiter à Madrid sans
ordre, instruction, ni lettre de créance.

Grimaldo n'avait pas besoin de cette touche pour former son opinion sur
tous les deux. Nous continuâmes à nous déboutonner ensemble sur l'un et
sur l'autre. De là je lui représentai au long tout ce que je viens
d'expliquer de l'absurdité et des dangers de ce prématuré passage\,;
surtout je ne lui laissai pas ignorer le mot de Chavigny, échappé à
Pecquet, d'établissement présent pour don Carlos et lui, et lui en
exposai toutes les conséquences. Grimaldo ne feignit point de s'ouvrir
entièrement avec moi là-dessus et fut totalement de mon sentiment. Il me
donna ensuite une plus grande marque de confiance, quoiqu'en me parlant
plus obscurément de sa crainte d'un si funeste projet, mais qui pouvait
flatter et éblouir\,; et comme j'étais au fait des intérêts, des
liaisons, des cabales que j'ai ici rapportées, son discours, tout
mesuré, tout enveloppé là-dessus, me fit sentir que j'étais parfaitement
informé. Il me remercia de cette visite comme d'un service essentiel que
je lui avais rendu pour le mettre au fait de ce que Chavigny lui
proposerait, et le mettre en état de prévenir, et s'il le pouvait, de
prémunir Leurs Majestés Catholiques là-dessus, et de les garantir du
précipice. Il me rassura sur l'armement de Barcelone, qu'il me répondit
être fait pour l'Amérique. Il fut encore quelques jours sans pouvoir
aller au palais.

Pour achever cette matière de suite, Grimaldo me dit qu'il avait
heureusement prévenu le roi et la reine, leur avait expliqué les
embarras, puis les dangers où les jetterait ce passage qui, au mieux
aller, ne pouvait apporter aucun fruit\,; et {[}qu'il avait{]} si bien
combattu les raisons, dont il pouvait bien être que quelques gens se
fussent déjà servis auprès d'eux, qu'il espérait tout à fait les
maintenir dans la négative\,; d'autant plus qu'il les avait trouvés si
choqués de l'arrivée de Chavigny, dont ils savaient les aventures et
connaissaient la réputation, qu'il avait eu toutes les peines du monde à
gagner sur le roi et la reine de ne pas trouver mauvais que je le leur
présentasse, parce que je ne pouvais m'en dispenser sans me faire une
affaire fâcheuse avec le cardinal Dubois qui me l'avait très
particulièrement recommandé.

Le lendemain de cette conversation, je menai Chavigny au marquis de
Grimaldo qui le reçut fort civilement, mais fort froidement\,; et le
soir, comme Leurs Majestés Catholiques revenaient de la chasse, je le
leur présentai à la porte de leur appartement intérieur. En effet le roi
passa sans s'arrêter et sans tourner la tête vers lui, ni par conséquent
vers moi qui le présentais, et sans dire un seul mot. La reine me dit
quelque chose, pour me parler seulement et sans aucun rapport à Chavigny
qu'elle ne regarda pas non plus. Quoique j'eusse lieu de m'attendre à
une assez mauvaise réception, celle-ci la fut tellement et si marquée
que j'en demeurai confondu. Chavigny, avec toute sa douce et timide
effronterie, ne laissa pas d'en être embarrassé. Comme cela se passa en
public, la cour et la ville en discoururent. Chavigny se garda bien de
m'en parler, et moi à lui\,; mais il m'en parut mortifié pendant
plusieurs jours. Cette présentation faite, il marcha par lui-même et je
ne m'en mêlai plus. Il mangeait très souvent chez moi\,; j'en fus quitte
pour des civilités et pour prendre pour bon le peu qu'il s'avisait
quelquefois de me dire, ce qui n'allait à rien, et sans m'entremettre de
la moindre chose. Il ne trouva pas mieux son compte avec Grimaldo sur
l'indult que sur le passage. Ce ministre se moqua bien avec moi de cette
vision du duc de Parme, et n'en rit pas moins qu'avait fait le P.
Daubenton. Chavigny échoua donc sur l'affaire de l'indult et sur celle
du passage de don Carlos en Italie. Il demeura néanmoins deux mois après
moi à Madrid, soit que la cabale italienne l'y retînt dans l'espérance
de faire enfin goûter ce projet à la reine, ou que le cardinal Dubois
l'eût chargé de choses qui passaient Maulevrier, et qui ne sont point
venues à ma connaissance, mais dont il n'a résulté aucun effet qui ait
été aperçu.

\hypertarget{chapitre-vii.}{%
\chapter{CHAPITRE VII.}\label{chapitre-vii.}}

1722

~

{\textsc{Le duc de Bournonville, nommé à l'ambassade de France, en est
exclus.}} {\textsc{- Je tente en vain d'obtenir la restitution de
l'honneur des bonnes grâces de Leurs Majestés Catholiques au duc de
Berwick.}} {\textsc{- Je tente en vain d'obtenir la grandesse pour le
duc de Saint-Aignan.}} {\textsc{- Conduite étrange de la princesse des
Asturies à l'égard de Leurs Majestés Catholiques.}} {\textsc{- Bal de
l'intérieur du palais.}} {\textsc{- La Pérégrine, perle incomparable.}}
{\textsc{- Illuminations\,; feux d'artifice admirables.}} {\textsc{-
Leurs Majestés Catholiques en cérémonie à l'Atoche.}} {\textsc{- Raison
qui me fait abstenir d'y aller.}} {\textsc{- Fête de la course des
flambeaux.}} {\textsc{- Fête d'un combat naval.}}

~

Une autre affaire m'occupait en même temps. On avait su avant mon départ
de Paris que le duc de Bournonville briguait fort à Madrid l'ambassade
de France, dont Laullez avait fini son temps\,; et le cardinal Dubois
qui ne voulait point absolument du duc de Bournonville, m'avait fort
recommandé de n'oublier rien pour l'y traverser. J'eus si peu de temps
entre mon arrivée à Madrid et le départ pour Lerma, et ce temps si
occupé d'affaires, de fêtes, de cérémonial, de fonctions et de visites
infinies que je n'eus pas celui d'entamer rien sur cette ambassade, dont
je comptai avoir tout loisir à Lerma. Mais en arrivant au quartier que
je devais occuper, je tombai malade le jour même, et la petite vérole,
qui se déclara, me mit pour quarante jours hors de moyen de sortir de
mon village. Pendant ce temps-là le duc de Bournonville bien averti de
Paris, et qui me craignait fort pour son ambassade, intrigua si bien,
qu'il se la fit donner et déclarer. Je reçus à Villahalmanzo une lettre
du cardinal Dubois, dès qu'il eut appris cette nouvelle, pleine de
regrets sur la lacune de ma petite vérole et de ma séparation de la
cour, qui eût, à ce qu'il me disait, paré ce choix. De là, s'étendant
sur le caractère du duc de Bournonville, sur ses liaisons intimes avec
le duc de Noailles, et c'était là le principal point du cardinal, car la
maréchale de Noailles et lui étaient enfants des deux frères, le
cardinal se lamentait des inconvénients qui résulteraient sûrement de
cette ambassade, et pour les cabales de la cour, et contre l'union si
nécessaire des deux couronnes, que le duc de Bournonville et le duc de
Noailles sacrifiaient à leurs vues et à leurs intérêts particuliers.
Enfin il m'avançait que l'usage constant entre les grandes couronnes
était de faire pressentir celle où il fallait un ambassadeur sur la
personne qu'on pensait à y envoyer, afin de ne lui pas donner un
ministre désagréable\,; à plus forte raison l'Espagne devait ce
ménagement à la France, dans la position actuelle où les deux couronnes
se trouvaient si heureusement ensemble. Il m'exhortait à faire valoir
cette raison et de tâcher à faire révoquer une disposition si peu propre
à entretenir l'amitié et l'union si désirable entre les deux branches
royales et entre les deux cours. Il était vrai que la maréchale de
Noailles, qui aimait fort sa maison, et en général à obliger, avait pris
soin, tant qu'elle avait pu, de ce cousin germain qui était un
arrière-cadet sans bien, et que le duc de Noailles l'ayant trouvé fort
homogène à lui, ils s'étaient intimement liés depuis fort longtemps.
Depuis que le duc de Noailles avait perdu l'administration des finances,
quoique comblé en même temps des plus grandes grâces pour lui en adoucir
l'amertume, il n'avait pu digérer la perte de ce grand emploi. Il
s'était éloigné de ceux à qui il s'en prenait et de ceux qui lui avaient
succédé. Dubois et d'Argenson étaient dans la plus grande liaison et ne
s'éloignèrent pas moins du duc de Noailles. Ils ne songèrent qu'à le
rendre suspect et à l'écarter de M. le duc d'Orléans, dont la confiance
pour lui, tant qu'il avait eu les finances, leur était fâcheuse, dans la
crainte des retours, tellement que cette liaison si étroite, formée à
l'entrée de la régence, entre l'abbé Dubois, le duc de Noailles,
Canillac et Stairs, formée avec tant d'art et de soin par Dubois pour
s'ouvrir un chemin à la fortune, de délaissé qu'il était alors de M. le
duc d'Orléans, et la liaison particulière du duc de Noailles avec lui
pour s'en servir contre moi, et pour lui-même, lorsque Dubois, à leur
aide, serait revenu sur l'eau\,; cette union se refroidit à mesure que
Dubois sentit fortifier ses ailes et se changea en éloignement, quoique
caché, depuis la perte de l'administration des finances. Outre ces
raisons et celles du caractère du duc de Bournonville, que je crois
avoir suffisamment expliquées ici en plus d'un endroit, le cardinal en
avait une autre plus secrète et plus personnelle qu'il n'est pas temps
de développer et qui m'était encore inconnue. Ce n'était pas une petite
affaire que d'empêcher que l'ambassade de Bournonville eût lieu. Sa
déclaration était pour le roi d'Espagne un engagement public\,: la
rétracter était un affront à un homme qui, à la vérité, ne fut jamais à
ces choses-là près, mais qui par sa dignité, sa naissance, sa charge, et
la Toison qu'il portait, méritait plus d'égards. Je ne laissai pas de
l'entreprendre, tant pour ne pas déplaire au cardinal Dubois, en choses
qui m'étaient aussi indifférentes, que parce qu'en effet je ne pouvais
que tout craindre pour l'union des deux cours d'un homme du caractère de
Bournonville, asservi à Popoli, à Miraval, à toute la cabale italienne
si ennemie de la France et de l'union, conduit par le duc de Noailles de
même caractère que lui, et à qui tout serait bon pour rentrer en
danse\,; enfin d'un homme haï et craint par le cardinal Dubois qui ne
pourrait traiter qu'avec lui. Je représentai donc ce dernier
inconvénient à Grimaldo. Je lui demandai quel choix on pouvait faire
entre se servir d'un canal qui devait être plus que suspect en Espagne à
tout ce qui en aimait les vrais intérêts, la grandeur et l'union avec la
France, odieux à celui avec qui il aurait uniquement à traiter, et qui
était le maître de toutes les affaires, ou faire une peine à un seigneur
à qui on pouvait trouver d'autres emplois capables de le dédommager de
celui où il était personnellement impossible qu'il pût réussir. Je lui
parlais plus librement par l'amitié et la confiance qui s'était établie
entre lui et moi, et plus hardiment par la connaissance que j'avais des
cabales de cette cour, et que Grimaldo n'ignorait pas combien
Bournonville était engagé avec ses ennemis. Je lui expliquai la
situation où le cardinal Dubois était avec le duc de Noailles, et les
intimes et anciennes liaisons de parenté, d'amitié, d'homogénéité qui
étaient entre les ducs de Noailles et de Bournonville, et ce que la
maréchale de Noailles était et dans sa famille et dans le monde\,; en un
mot, que s'il voulait humeurs, caprices, brouilleries, dégoûts
réciproques entre les deux cours, leur désunion certaine, il serait
servi par un tel ambassadeur, avec lequel tout cela serait infaillible,
tandis que les deux cours ne recevraient que satisfaction réciproque,
intelligence, union de plus en plus resserrée dans le désir qu'elles en
avaient l'une et l'autre en envoyant ambassadeur quel que ce fût, pourvu
que ce fût un homme d'honneur, droit, de nulle cabale, uniquement
attaché aux intérêts de l'Espagne, et à bien servir dans son emploi.
Grimaldo goûta mes raisons, mais l'embarras fut d'en persuader assez
Leurs Majestés Catholiques pour entraîner la reine, qui méprisait
Bournonville comme faisaient tous ceux qui le connaissaient, mais qui
avait les plus fortes protections auprès d'elles, à l'abandonner à cet
affront. Je répondis que si Bournonville avait un grain de sens, il
serait le premier à demander d'être déchargé d'une ambassade où il ne
pourrait jamais réussir, à voir que le cardinal Dubois mettrait toute
son industrie à faire retomber sur lui par l'Espagne même tous les
fâcheux succès de ses négociations, à sentir que ce qui réussirait en
toutes autres mains romprait entre les siennes, et qu'en prétextant
santé, dépense, affaires, il pouvait remettre l'ambassade sans affront.
Je donnai courage à Grimaldo\,; je lui dis qu'il n'y avait qu'à
continuer Laullez qui servait l'Espagne à son gré, et qui était
extrêmement agréable à notre cour, prétexter qu'il avait entamé des
affaires qu'il n'était pas à. propos de changer de mains, et se donner
ainsi tout le temps nécessaire de lui choisir un successeur qui lui
ressemblât, et qui marchât sur ses mêmes errements. Enfin Grimaldo,
convaincu de mes raisons, peut-être des siennes personnelles qui se
trouvaient couvertes par les miennes, me promit merveilles et me les
tint. Bournonville, qui m'accablait de souplesses et de bassesses, ne
fut pas assez sage pour refuser. Il insista toujours, comptant sur la
publicité de sa déclaration et sur le crédit de sa cabale. Il en fut la
dupe, et ses Italiens avec lui, qui en furent outrés de dépit. Pour lui,
il sentit le coup, et parut comme un condamné, mais il ne m'en fit que
mieux, et me conjura sans cesse de détruire à mon retour lés préventions
qu'on avait prises contre lui, et d'obtenir la permission du régent et
du cardinal Dubois d'aller en France se justifier auprès d'eux. Il me
faisait parler par tous ses amis, me raccrochait partout et me désolait
en plaidoyers qui ne finissaient point. Cela dura jusqu'à la veille de
mon départ, que je le trouvai tout tard qui m'attendait à mon carrosse,
dans la cour du Retiro, où il me demanda une dernière audience, et quoi
que je pusse faire m'y promena près de deux heures.

Si j'eus le bonheur de réussir en ces deux affaires, j'eus le malheur
d'échouer en deux autres, dont la seconde surtout ne me tenait pas moins
au coeur qu'avait fait la grandesse de mon second fils, seule cause de
mon voyage en Espagne, et d'en avoir désiré et obtenu l'ambassade.

Sur la première il faut se souvenir que lorsque le cardinal Dubois
embarqua M. le duc d'Orléans à faire si follement la guerre à l'Espagne
pour faire sa cour aux Anglais, et obtenir son chapeau, le duc de
Berwick accepta sans balancer le commandement de l'armée de Guipuscoa,
prit des places et brûla la marine d'Espagne au Ferrol, qui était le
grand objet des Anglais, ce que le roi d'Espagne, qui l'avait comblé lui
et son fils aîné de bienfaits, ne put jamais lui pardonner. C'était ce
pardon que le cardinal Dubois avait extraordinairement à coeur pour la
même raison, qui m'était lors cachée, dont j'ai parlé de même sur autre
chose, il n'y a pas longtemps. Par conséquent M. le duc d'Orléans, qui
n'y entendait pas finesse, désirait aussi ce pardon, et l'un et l'autre
me l'avaient très particulièrement recommandé, et m'en avaient écrit en
Espagne depuis le plus fortement du monde. Le duc de Liria, qui le
souhaitait ardemment avec grande raison, me pressait aussi là-dessus,
tellement que j'en parlai à Grimaldo. Ce ministre me dit que je ne
pouvais parler de cette affaire à personne qui l'eût plus à coeur que
lui, par son ancien et véritable attachement pour le duc de Berwick, et
pour la fidèle amitié qui était entre le duc de Liria et lui, mais que
je ne devais point me tromper sur cet article\,; que le roi et la reine
n'avaient encore rien rabattu de leur première indignation\,; qu'il leur
en échappait de temps en temps des marques fort vives et telles que lui,
qui les connaissait, se garderait bien de toucher cette corde auprès
d'eux\,; qu'à mon égard, après cet avis, il n'avait rien à me dire, mais
que je pouvais me régler là-dessus. Ce début me parut fâcheux. J'avais
espéré de l'amitié de Grimaldo pour le père et le fils qu'il me
frayerait un chemin que je n'aurais qu'à suivre. Son refus me le fit
voir bien plus difficile que je ne m'y étais attendu. Je nie tournai
vers le P. Daubenton sans lui parler de ma tentative. Mais j'eus beau
lui parler conscience et son caractère de confesseur, il me fit toutes
les protestations possibles pour le duc de Berwick et même pour le duc
de Liria, me dit que c'était une affaire en quelque sorte d'État dans
laquelle il ne devait point entrer de lui-même\,; m'en laissa entendre
toute la difficulté, et me renvoya à Grimaldo, à qui aussi je me gardai
bien de dire que j'en eusse parlé au confesseur, et que j'en avais été
éconduit. Je lui dis seulement que réflexion faite je ne pouvais manquer
à des ordres si précis\,; que je ne pouvais m'imaginer que Leurs
Majestés Catholiques me pussent savoir mauvais gré de les exécuter\,;
que je m'en acquitterais avec tout le respect, les mesures et
l'attention à ne les point blesser que j'y pourrais mettre, qu'au pis
aller, si je ne réussissais pas, j'aurais fait ce que je devais, et
évité de me faire une affaire de l'inexécution d'ordres si précis et
réitérés. Dans cet esprit, je demandai une audience. Je dis à Leurs
Majestés Catholiques que j'avais à m'acquitter auprès d'elles d'un ordre
de bouche avant mon départ, et réitéré très fortement depuis\,; que ce
dont il s'agissait était une grâce que le roi et M. le duc d'Orléans
avaient extrêmement à coeur d'obtenir de Leurs Majestés\,; qu'ils la
leur demandaient avec toute la confiance qu'ils devaient prendre non
seulement en leur générosité, mais encore en leur piété\,; que néanmoins
Sa Majesté et Son Altesse Royale en prenaient encore une nouvelle de ce
moment de réunion aussi parfaite et aussi intime de Leurs Majestés avec
elles\,; et que Leurs Majestés se pouvaient assurer d'une reconnaissance
parfaite si elles en obtenaient ce dont Sa Majesté et Son Altesse Royale
étaient si véritablement touchées et qu'elles désiraient avec tant de
passion. Ils me laissèrent tout dire, puis le roi me demanda ce que
c'était donc que le roi et M. le duc d'Orléans lui demandaient. Je
répondis\,: le retour de l'honneur de leurs bonnes grâces pour le duc de
Berwick, qui ne se consolait point d'avoir eu le malheur de les perdre.
À ce nom le roi rougit, m'interrompit, et me dit d'un air allumé et d'un
ton ferme\,: «\,Monsieur, Dieu veut qu'on pardonne, mais il ne faut pas
m'en demander davantage.\,» Je baissai la tête, puis regardant la reine
comme pour lui demander assistance, je dis en rebaissant la tête\,: «\,
Votre Majesté me ferme la bouche\,; et le respect m'empêchera de la
rouvrir là-dessus, sans néanmoins éteindre les espérances que je mettrai
toujours en la générosité et la piété de Votre Majesté.\,» Je me tus
ensuite, comprenant bien à leur contenance qu'insister davantage serait
sans autre fruit que les opiniâtrer et les aigrir. Après quelque
silence, la reine parla d'autre chose, mais de simple conversation qui
dura quelque peu, et l'audience finit de la sorte. Grimaldo, à qui je
rendis ce qui s'était passé, n'en fut pas surpris\,: il me l'avait bien
prédit. Le duc de Liria en fut très affligé, quoique toujours
personnellement bien traités. L'un et l'autre, qui furent les deux seuls
qui surent cet office, né jugèrent pas à propos que j'en reparlasse
davantage. J'en pensais comme eux, et les choses en demeurèrent là.

La seconde affaire, la cour n'y avait nulle part et n'en avait pas même
de connaissance. La duchesse de Beauvilliers qui par le mariage de sa
fille au duc de Mortemart, dont elle était dans le repentir depuis
longtemps, avait fait passer presque toute la fortune du duc de
Beauvilliers sur ce gendre, était touchée après coup de voir Sa
Grandesse sortie de sa maison. Elle m'en témoigna sa peine avant mon
départ, et me pria de voir si je ne pourrais point obtenir une grandesse
pour le duc de Saint-Aignan qui avait peu de biens et beaucoup
d'enfants. J'aimais et je respectais extrêmement la duchesse de
Beauvilliers, et M. de Beauvilliers était vivant et agissant dans mon
coeur dans la dernière vivacité du sentiment le plus tendre et le plus
rempli de vénération. Quoique le duc de Saint-Aignan ne m'eût jamais
cultivé que suivant la mesure de son besoin, et que sa futilité me fût
désagréable, il m'était cher, parce qu'il était frère du duc de
Beauvilliers, et par cette raison, lui et tout ce qui porta son nom, me
l'a été toute ma vie, sans nul égard à rien de tout ce qui aurait dû
émousser les pointes de ce vif attachement. Je partis donc bien résolu
de ne rien oublier pour le succès d'une chose que je désirais assez
passionnément pour ne savoir de bonne foi ce que j'aurais choisi, si on
m'eût donné en Espagne l'option de cette grandesse ou de la mienne. Les
services et la reconnaissance pour de tels morts, et desquels ni des
leurs on ne peut rien attendre, sont d'une suavité si douce, et jettent
dans l'âme quelque chose de si vif, de si délicieux, de si exquis que
nulle sorte de plaisir n'y est comparable et dure toujours, et je
l'éprouve encore sur la charge de premier gentilhomme de la chambre que
le duc de Mortemart avait eue du duc de Beauvilliers, sur laquelle j'ai
raconté en son temps ce qui se passa. Plein de ce désir, j'en fis la
confidence à Grimaldo, à qui, en peu de mots, j'en expliquai la cause
pour qu'il ne crût pas cet office que je voulais rendre du nombre de
ceux dont on se soucie peu, pourvu qu'on s'en soit acquitté, et qu'il
sentit au contraire à quel point le succès m'en tenait au coeur. Sa
réponse m'affligea. Après la préface de politesse et d'amitié, il
m'avertit que je trouverais dans Leurs Majestés Catholiques un grand
éloignement, parce que, outre que le duc de Saint-Aignan y avait donné
lieu lui-même par force futilités, et petites choses pendant son
ambassade à Madrid, où le soin tardif de sa parure avait souvent
impatienté Leurs Majestés Catholiques, en attendant souvent fort
longtemps qu'il fût arrivé pour ses audiences, le cardinal Albéroni, qui
ne l'aimait pas, avait jeté dans leur esprit des impressions fâcheuses
qui y étaient toujours restées, qui paraissaient toutes les fois que le
hasard leur rappelait le nom de duc de Saint-Aignan, et qui formeraient
un obstacle que j'aurais bien de la peine à surmonter, ce qu'il ne
pouvait me cacher qu'il n'espérait pas. Je le pressai vainement d'en
jeter quelques propos à Leurs Majestés Catholiques. Il m'assura que,
bien loin de me préparer la voie, cela nuirait et les arrêterait au
refus\,; au lieu que, s'il y avait un moyen de réussir, c'était la
surprise et l'embarras de me refuser en face\,; que s'ils ne me
refusaient ni n'accordaient, alors il m'offrait de venir de son côté à
l'appui, et de m'y rendre tout le service qu'il lui serait possible.
C'était parler raison\,; il fallut bien s'en contenter. Je cherchai à
prendre un temps de satisfaction et de bonne humeur de Leurs Majestés
Catholiques\,; un temps où la conduite de la princesse des Asturies,
dont je parlerai bientôt, m'attirait leur confidence et de fréquents
particuliers\,; un temps enfin où j'avais lieu de nie flatter que je
leur étais personnellement fort agréable. L'extrême désir me faisait
espérer sur ce que la duchesse de Beauvilliers avait été l'unique
personne, en femmes et en hommes, dont le roi d'Espagne, la maison
royale à part, m'eût demandé des nouvelles. Je pris donc des moments de
pure conversation en tiers avec eux pour la jeter sur la jeunesse du roi
d'Espagne, et par là sur le duc et la duchesse de Beauvilliers.
J'excitai, tant que je pus, les souvenirs d'estime et d'amitié\,; puis
me mettant sur la morale du renversement des fortunes les plus sagement
et les mieux établies, je parlai de la perte des deux fils du duc de
Beauvilliers, qui avait jeté toute sa fortune sur son gendre, dont les
enfants privaient le duc de Saint-Aignan de la décoration que Sa Majesté
avait donnée à sa maison. Je me tus quelques moments pour voir si le roi
prendrait à ce discours\,; mais, son silence continuant, j'ajoutai que
ce serait une grâce de sa générosité, et digne de son ancienne amitié
pour le duc et la duchesse de Beauvilliers, de remettre la grandesse à
sa destination première, et de l'accorder au duc de Saint-Aignan\,; et
je dirais, si je l'osais, qu'un tel souvenir si dignement placé ferait
un honneur infini à la gloire de Sa Majesté\,; que comblé comme je
l'étais de ses bienfaits, j'oserais encore moins hasarder ma très humble
et très instante intercession, mais que l'extrême désir que j'en avais
me forçait d'avouer que ce serait pour moi la plus grande satisfaction
de ma vie, égale, pour le moins, à celle que je ressentais des grâces
qu'elle avait daigné de répandre sur moi. Pendant cette reprise
j'aperçus le roi piétiner, comme il faisait toujours quand il voulait
finir l'audience\,; et quand j'eus achevé, au lieu de me répondre, il se
mit à tirer la robe de la reine, qui était le signal de me congédier, ce
qu'elle fit fort poliment quelques moments après. Je sortis pénétré de
douleur d'un silence et d'une fin d'audience de si mauvais augure. Je
descendis tout de suite dans la cavachuela du marquis de Grimaldo, à qui
je fis le récit de ce qui venait de se passer. Il n'en fut point
surpris, et me répéta les mêmes choses qu'il m'avait dites du peu de
disposition qu'il avait prévu que je trouverais. Au lieu de me plaindre
du peu de digne souvenir que j'avais trouvé dans le roi d'Espagne de son
gouverneur et de sa famille, au lieu de prier Grimaldo de faire quelque
effort, je crus plus efficace et moins embarrassant pour lui de me
contenter de lui exposer amèrement les motifs de mon désir, et de
l'affliction où me jetait le mauvais succès qu'il avait eu, parce que je
ne pouvais interpréter un silence si opiniâtre, suivi incontinent de
l'impatience de finir l'audience, que comme un refus tacite. Je me
répandis là-dessus si pathétiquement avec Grimaldo, sans lui faire même
aucune sorte d'insinuation, qu'il me dit enfin de la meilleure grâce du
monde qu'il ne manquerait pas de prendre son temps de parler à Leurs
Majestés de la douleur où il m'avait vu au sortir de cette audience, et
de faire tout ce qui lui serait possible pour le duc de Saint-Aignan. Je
lui répondis que je n'aurais osé lui demander rien là-dessus\,; mais que
cette offre si obligeante me comblait, et je l'embrassai de tout mon
coeur. Mais ce ministre ne réussit pas plus que moi. Il en parla deux
fois, il fut refusé, et à la dernière, le roi d'Espagne lui dit qu'après
tout ce qu'il avait fait pour moi je devais être content. De sorte que
Grimaldo me conseilla et me pria même par l'amitié qu'il avait pour moi
de ne pas tenter l'impossible, et de ne me pas rendre désagréable à
Leurs Majestés Catholiques en les pressant de nouveau de ce que très
certainement elles ne feraient pas. Je le sentis bien moi-même, et je
n'osai plus rien dire ni rien faire sur une chose que j'avais si
ardemment désirée. Revenons maintenant à la princesse des Asturies.

Sa convalescence avançait, et son humeur se manifestait en même temps.
Je sus par l'intérieur qu'elle résistait avec opiniâtreté à aller chez
la reine, après tous les soins et les marques extraordinaires de bonté,
les visites continuelles, qu'elle en avait reçues pendant sa maladie et
qu'elle en recevait encore tous les jours. Elle ne voulait point sortir
de sa chambre\,; elle s'amusait à sa fenêtre où elle se montrait en
bonne santé.

Son appartement de plain-pied à celui de la reine n'en était séparé que
par cette petite galerie intérieure dont j'ai souvent parlé, car elle
était dans l'appartement qu'avait l'infante. Elle ne voulait plus
écouter sur rien les médecins sur sa santé, ni ses dames sur sa
conduite, et répondait même à la reine fort sèchement lorsqu'elle
essayait à la ramener par les insinuations les plus douces. La reine
même m'en parla et m'ordonna de la voir et de lui aider à la rendre plus
traitable. Je répondis que je n'étais que trop informé de ce que j'étais
très peiné qui fût\,; que je ne devais pas me flatter de pouvoir plus
que Sa Majesté sur l'esprit de la princesse\,; et après un peu de
conversation sur ce qu'elle croyait m'en apprendre, et que j'y eus
ajouté ce que je sa vois de plus, qu'elle ne me nia pas, je pris la
liberté de lui dire qu'il y avait aussi trop de bonté et de
ménagement\,; que Sa Majesté gâtait la princesse\,; qu'il fallait la
ployer sans retardement à ses devoirs, et que si dans l'excès de la
patience de la reine, la considération de M. le duc d'Orléans y entrait
pour quelque chose, non seulement je me chargeais de tout auprès de lui,
mais que je répondais à Sa Majesté que non seulement il trouverait bon
tout ce qu'il plairait à Sa Majesté de dire à la princesse, et de faire,
mais que lui en serait aussi extrêmement obligé, parce que personne ne
connaissait mieux que moi ses sentiments pour Leurs Majestés, combien il
se sentait aise du retour de leurs bonnes grâces et désireux de les
conserver, combien aussi il se sentait honoré du mariage de sa fille,
combien, par conséquent, il désirait qu'elle sentît son bonheur et sa
grandeur, et qu'elle s'en rendît digne par sa reconnaissance, son
obéissance, ses respects pour Leurs Majestés et par une application
continuelle non seulement à leur plaire et à répondre à leurs bontés,
mais à deviner même tout ce qui pourrait la leur rendre plus agréable et
à s'y porter continuellement\,; qu'outre que M. le duc d'Orléans
regardait cette conduite comme le devoir de M\textsuperscript{me} sa
fille le plus juste et le plus pressant, il le considérait aussi comme
le seul fondement solide du bonheur de la princesse et comme ce qui
pouvait le plus contribuer au sien, par savoir que sa fille ne fît rien
qu'à leur gré, et par se pouvoir flatter de leur avoir fait un présent
dont l'agrément pouvait contribuer à la continuation de leurs bontés
pour lui-même et au resserrement de plus en plus de cette heureuse union
qu'il avait toujours si passionnément désirée.

Ce discours fut fort bien reçu. La conversation s'étendit sur de pareils
détails à ceux qui l'avaient commencée, et finit par des ordres fort
exprès du roi et de la reine de voir souvent la princesse et de lui
parler. La duchesse de Monteillano et les autres dames m'en pressaient
continuellement. J'avais déjà vu la princesse bien des fois, même au
lit\,; il n'y avait donc rien de nouveau à m'y voir retourner.
D'ailleurs cette opiniâtreté à demeurer dans sa chambre perçait au
dehors, parce qu'elle suspendait les fêtes qui étaient préparées, et que
chacun attendait avec impatience. J'allai donc chez la princesse deux ou
trois fois sans en avoir eu aucune parole que \emph{oui} et \emph{non}
sur ce que je lui demandais de sa santé, et encore pas toujours. Je pris
le tour de dire à ses dames devant elle ce que je lui aurais dit à
elle-même\,; ses dames y applaudissaient, et y ajoutaient leur mot. La
conversation se faisait ainsi devant la princesse, en sorte qu'elle lui
était une véritable leçon, mais elle n'y entrait en aucune façon.
Néanmoins elle alla pourtant une fois ou deux chez la reine, mais en
déshabillé et d'assez mauvaise grâce.

Le grand bal demeurait toujours préparé et tout rangé dans le salon des
grands et n'attendait que la princesse qui n'y voulait point aller. Le
roi et la reine aimaient le bal, comme je l'ai dit ailleurs. Ils se
faisaient un plaisir de celui-là, le prince des Asturies aussi, et la
cour l'attendait avec impatience. La conduite de la princesse
transpirait au dehors, et faisait le plus fâcheux effet du monde. Je fus
averti du dedans que le roi et la reine en étaient très impatientés, et
pressé par les dames de la princesse de lui en parler, j'allai chez elle
et fis avec ses dames la conversation sur la santé de la princesse, qui
apparemment ne retarderait plus les plaisirs qui l'attendaient. Je mis
le bal sur le tapis\,; j'en vantai l'ordre, le spectacle, la
magnificence, je dis que ce plaisir était particulièrement celui de
l'âge de la princesse\,; que le roi et la reine l'aimaient fort, et
qu'ils attendaient avec impatience qu'elle pût y aller. Tout à coup elle
prit la parole que je ne lui adressais point, et s'écria comme ces
enfants qui se chêment\footnote{Mot ancien et familier\,; il se disait
  des enfants qui éprouvaient un dégoût ou un mal dont la cause était
  inconnue.}\,: «\,Moi, y aller\,! je n'irai point. --- Bon, madame,
répondis-je, vous n'irez point, vous en seriez bien fâchée, vous vous
priveriez d'un plaisir où toute la cour s'attend à vous voir, et vous
avez trop de raisons et de désir de plaire au roi et à la reine pour en
manquer aucune occasion.\,»

Elle était assise et ne me regardait pas. Mais aussitôt après ces
paroles, elle tourna la tête sur moi, et d'un ton le plus décidé que je
n'en ouïs jamais\,: «\,Non, monsieur, me dit-elle, je le répète, je
n'irai point au bal\,; le roi et la reine y iront s'ils veulent. Ils
aiment le bal, je ne l'aime point\,; ils aiment à se lever et à se
coucher tard, moi à me coucher de bonne heure. Ils feront ce qui est de
leur goût, et je suivrai le mien.\,» Je me mis à rire, et lui dis
qu'elle voulait se divertir à m'inquiéter, mais que je n'étais pas si
facile à prendre sérieusement ce badinage\,; qu'à son âge on ne se
privait pas si volontiers d'un bal, et qu'elle avait trop d'esprit pour
priver toute la cour et le public de cette attente, encore moins à
montrer un goût si peu conforme à celui du roi et de la reine, et qui
paraîtrait si étrange à son âge et à son arrivée\,; mais qu'après cette
plaisanterie le mieux était de ne prolonger pas plus longtemps une
attente, dont le délai d'un bal, tout rangé et tout prêt depuis si
longtemps, devenait indécent. Les dames m'appuyèrent, et la conversation
entre elles et moi continua de la sorte sans que la princesse fît
seulement contenance de nous entendre.

En sortant, la duchesse de Monteillano me suivit avec la duchesse de
Liria et M\textsuperscript{me} de Riscaldalgre. Elles m'entourèrent hors
de la porte de la chambre, et me témoignèrent leur effroi d'une volonté
si arrêtée dans une personne de cet âge contre devoir et plaisir, et
dans un pays où elle ne faisait que d'arriver, et toute seule parmi tous
gens inconnus. J'en étais plus épouvanté qu'elles\,; j'en voyais des
conséquences capables d'apporter de grandes suites. Mais j'essayai de
les rassurer sur un reste de maladie et d'humeurs en mouvement qui
pouvaient causer ce méchant effet, mais qui cesserait avec le retour de
la pleine santé. Toutefois j'étais bien éloigné de m'en flatter. Je me
gardai bien néanmoins de faire ce récit au roi et à la reine\,; mais
comme ils me parlèrent du bal, et le roi surtout avec amertume sur la
fantaisie de la princesse, je pris la liberté de lui dire que je
n'imaginais pas qu'il se voulût gêner pour le caprice d'une enfant qui
venait sûrement de sa maladie, ni priver sa cour et tout le public d'une
fête aussi agréable et aussi superbe qu'était le premier bal que j'avais
vu au palais, et que j'avouais qu'en mon particulier j'en serais
affligé, parce que je m'en étais fait un fort grand plaisir. «\,Oh\,!
cela ne se peut pas, reprit le roi, sans la princesse. --- Et pourquoi
donc, sire\,? lui répliquai-je. C'est une fête que Votre Majesté donne à
sa joie et à la joie publique. Ce n'est pas à la princesse, quoique à
son occasion, à régler les plaisirs de Votre Majesté, et ceux qu'elle
veut bien donner à sa cour qui s'y attend et les désire. Si la princesse
croit que sa santé lui permette, elle y viendra, sinon la fête se
passera sans elle.\,»

Tandis que je parlais, la reine me faisait signe des yeux et de la tête
de presser le roi, tellement que j'ajoutai que tout ce qui se faisait et
se passait n'était et ne pouvait être que pour Leurs Majestés\,;
qu'elles en étaient le seul objet et la décoration unique\,; que quelque
grands princes que fussent les infants, ils n'y étaient que comme les
premiers courtisans et pour illustrer l'assemblée, mais jamais
l'objet\,; que la confiance dont Sa Majesté daignait m'honorer sur ce
qui regardait la princesse m'engageait par devoir à supplier Leurs
Majestés de considérer qu'il ne fallait pas accoutumer la princesse à
croire que tout se fit pour elle, et que rien ne se pouvait faire sans
elle\,; que plus la fête était digne de la présence de Leurs Majestés,
plus cette leçon de la faire sans elle lui ferait impression\,; que je
ne pouvais m'empêcher de regarder cela comme appartenant très
essentiellement à une éducation si importante, et dont le bonheur de la
princesse dépendait, en lui faisant sentir dès la première
{[}occasion{]} qu'elle n'était rien, et qu'on se passait très aisément
d'elle. La reine appuya fort ce discours\,; mais le roi ne répondant
rien, elle tourna doucement la conversation ailleurs. En finissant
l'audience, elle prit l'instant que le roi se retournait après ma
révérence pour me faire signe de la tête et des yeux que j'avais bien
parlé, et me montrant le roi du doigt et comme le poussant sur lui, elle
me fit entendre de ne me pas rebuter. Cela fit que je me hâtai de dîner
pour me trouver à leur sortie pour la chasse, et je demandai tout haut à
la reine pour quel jour enfin serait le bal, dont j'avouais que je
mourais d'envie. Elle me répondit avec action qu'il fallait le demander
au roi, et lui demanda s'il m'avait entendu. Il lui répondit\,:
\emph{Mais nous verrons}. Ce court dialogue les conduisit au haut du
petit degré qui était tout proche par où ils descendaient et remontaient
toujours, et je demeurai au haut, parce qu'à peine y pouvait-on passer
deux de front.

Le lendemain je trouvai moyen de leur parler en particulier sur quelque
bagatelle, puis je remis le bal sur le tapis. La reine me dit en riant
qu'il était vrai que j'en avais bien envie, et elle aussi, et se mit
doucement à presser le roi. Comme il souriait sans répondre, je pris la
liberté de leur dire que je les suppliais de se souvenir que j'avais
pris celle de leur représenter que Leurs Majestés gâtaient la
princesse\,; qu'aujourd'hui j'osais ajouter qu'elles s'en
repentiraient\,; qu'elles y vaudraient remédier quand il n'en serait
plus temps\,; que M. le duc d'Orléans en serait au désespoir, et que
s'il pouvait avoir le même bonheur que j'avais d'être en leur présence,
il leur parlerait là-dessus en même sens que moi, mais bien plus
fortement, comme il lui convenait. Ce propos tourna par eux-mêmes la
conversation sur de nouvelles bagatelles fort maussades d'opiniâtreté,
de fantaisie, d'inconsidération pour ses dames, qui échappaient à la
princesse, de la brèveté\footnote{On a déjà vu que Saint-Simon écrivait
  \emph{brèveté} pour \emph{brièveté}.} de ses visites chez Leurs
Majestés, de la sécheresse de ses manières avec elles, sur quoi je les
suppliai de me pardonner si je leur disais que c'était la faute de Leurs
Majestés plus que d'une enfant qui ne savait ce qu'elle faisait, et
qu'au lieu de l'accoutumer par leur trop de bonté à ne se refuser aucun
caprice, rien n'était plus pressé ni plus important que de les réprimer,
de lui imposer, de lui faire sentir tout ce qu'elle montrait ignorer à
leur égard, et même à l'égard de ses dames\,; enfin l'accoutumer au
respect et à la crainte qu'elle leur devait, à lire dans leurs yeux et
jusque dans leur maintien leurs volontés, pour s'y conformer à l'instant
et avec un air comme si c'était la sienne par l'empressement à leur
obéir et à leur plaire. Tout cela fut encore poussé de ma part et
raisonné de la leur assez longtemps, après quoi je me retirai. Je
n'allais plus chez la princesse, et je le dis à Leurs Majestés, parce
que j'en voyais l'inutilité. Je ne reparlai plus de bal à leur retour de
la chasse, au passage de leur appartement dans la crainte de rebuter le
roi. Le surlendemain je me trouvai à leur passage pour la chasse. Au
sortir de l'appartement, la reine me dit qu'il n'y aurait point de
bal\,; que l'ordre était donné d'ôter le préparatif qui était rangé
depuis si longtemps, en me faisant des signes d'en parler encore au roi.
Je lui dis donc que j'en serais désolé par le plaisir que je m'en étais
fait, et que si j'osais je lui demanderais ce bal comme une grâce.

Ce dialogue conduisit à ce petit degré qui était tout contre. À
l'entrée, la reine me fit signe de suivre. Je me fourrai donc à côté de
celui qui lui portait la queue, lui parlant haut de ce bal pour que le
roi, qui marchait devant elle, pût entendre. Un moment après elle se
tourna à moi avec un air que je dirais \emph{penaud} si on pouvait
hasarder ce terme, et me fit signe de ne plus rien dire. Apparemment que
le roi lui en avait fait quelqu'un là-dessus, car cette rampe était
obscure, et je ne pus l'apercevoir. Au repos du degré, qui était assez
long, la reine s'approcha du roi. Je demeurai où j'étais sans m'avancer.
Ils se parlèrent bas, puis la reine m'appela, et quand je fus près
d'elle\,: «\,Voilà qui est fait, me dit-elle, il n'y aura point de
bal\,; mais pour s'en dépiquer, ce fut son terme, le roi en aura un
petit ce soir, après souper, dans notre particulier, où il n'y aura que
du palais, et le roi veut que vous y veniez.\,» Je leur fis une profonde
révérence et mon remerciement, tout cela, arrêtés sur ce repos du degré.
La reine me répéta\,: «\,Mais vous y viendrez donc\,?» Je répondis à cet
honneur comme je devais. Le roi me dit\,: «\, Au moins, il n'y aura que
nous.\,» Et la reine continua\,: «\,Et nous danserons tout à notre aise
et en liberté\,;» et tout de suite {[}ils{]} achevèrent de descendre, et
je les vis monter en carrosse.

Le bal fut dans la petite galerie intérieure. Il n'y eut que les
seigneurs en charge, le premier écuyer, les majordomes de semaine, la
camarera-mayor, les dames du palais, les jeunes señoras de honor et
caméristes. Le roi, la reine, le prince des Asturies s'y divertirent
fort\,; tout le monde y dansa force menuets, encore plus de
contredanses, jusque sur les trois heures après minuit que Leurs
Majestés se retirèrent et le prince des Asturies. Ce fut là où je vis et
touchai à mon aise la fameuse \emph{Pérégrine}, que le roi avait ce
soir-là au retroussis de son chapeau, pendant d'une belle agrafe de
diamants. Cette perle, de la plus belle eau qu'on ait jamais vue, est
précisément faite et évasée comme ces petites poires qui sont musquées,
et qu'on appelle de \emph{sept-engueule} et qui paraissent dans leur
maturité vers la fin des fraises. Leur nom marque leur grosseur,
quoiqu'il n'y ait point de bouche qui en pût contenir quatre à la fois
sans péril de s'étouffer. La perle est grosse et longue comme les moins
grosses de cette espèce, et sans comparaison plus qu'aucune autre perle
que ce soit. Aussi est-elle unique. On la dit la pareille et l'autre
pendant d'oreilles de celle qu'on prétend que la folie de magnificence
et d'amour fit dissoudre par Marc-Antoine dans du vinaigre, qu'il fit
avaler à Cléopâtre. Quoique l'appartement de la princesse des Asturies
fit à l'un des bouts de cette galerie intérieure, elle ne parut pas un
instant. Je ne prédis que trop vrai à Leurs Majestés Catholiques. La
princesse en fit de toutes les façons les plus étranges, excepté la
galanterie\,; et à son retour ici on eut le temps de voir quelle elle
était, dans le peu d'années qu'elle a vécu veuve et sans enfants. J'ai
rapporté ce bal tout de suite de ce qui regarde la princesse\,; il faut
parler maintenant des autres fêtes qui furent données à l'occasion des
doubles mariages.

Elles commencèrent le 15 février par une illumination et un feu
d'artifice dans la place qui est devant le palais. J'ai déjà parlé ici
de la surprenante beauté des illuminations d'Espagne. Les feux
d'artifice ne leur y cèdent point. Ils durent plus d'une heure et
ordinairement davantage dans joute leur plénitude, et dans une variation
perpétuelle de paysages, de chasses, de morceaux d'architecture
admirables, de places et de châteaux. Les fusées merveilleuses,
innombrables à la fois, continuelles, les fleuves et les cascades de
feu, en un mot, tout ce qui peut remplir et orner le spectacle et le
rendre toujours surprenant ne cesse, ne diminue, ne s'affaiblit pas un
moment, en sorte qu'on n'a pas assez d'yeux pour voir le tout ensemble.
Nos plus beaux feux d'artifice ne sont rien en comparaison.

Le lendemain, Leurs Majestés Catholiques allèrent en cérémonie à
Notre-Dame d'Atocha, telle qu'{[}elle{]} a été ici décrite ailleurs.
Mais en celle-ci elles étaient dans un carrosse tout de bronze doré et
de glaces, avec le prince et la princesse des Asturies sur le devant, et
suivies de trente carrosses remplis de grands et de toute la cour. Je
n'y fus point ni Maulevrier, comme nous n'y avions point été la première
fois, sur l'avis du marquis de Montalègre, sommelier du corps, à qui je
le demandai, mais qui ne m'en dit point la raison. J'appris, à
l'occasion de celle-ci, que c'était parce que les grands étaient avertis
de se trouver à ces cérémonies, et y avaient leurs places et non les
ambassadeurs. J'aurais pu m'y trouver comme grand, ainsi que je faisais
en d'autres fonctions où les ambassadeurs ne se trouvent pas\,; mais
celle-ci était si solennelle et si marquée sur le double mariage que,
n'y pouvant assister comme ambassadeur, je crus m'en devoir abstenir
quoique grand. Au retour de l'Atoche, le roi passa par la place Major,
tout illuminée, et s'y arrêta quelque temps. J'y étais à une fenêtre. Il
trouva, en arrivant au palais, la place qui est devant, illuminée.
J'avais eu l'honneur d'être admis sur le balcon de Leurs Majestés
Catholiques et près d'elles au feu d'artifice dont j'ai parlé\,; mais je
me retirai peu à peu à une autre fenêtre gardée pour mes enfants et ma
compagnie, et je ne retournai au balcon du roi que pour en voir sortir
Leurs Majestés et les accompagner à leur appartement.

On eut un autre jour, dans la place Major illuminée, un divertissement
fort galant. La maison où j'étais était vis-à-vis de celle du roi, et de
l'une à l'autre une lice entre deux barrières. Rien ne pouvait être plus
brillant, plus rempli ni avec un plus grand ordre. Le duc de
Medina-Coeli, le duc del Arco et le corrégidor de Madrid avaient chacun
leur quadrille de deux cent cinquante bourgeois ou artisans de Madrid,
toutes trois diversement masquées, c'est-à-dire magnifiquement parées en
mascarades diverses, mais à visage découvert, tous montés sur les plus
beaux chevaux d'Espagne avec de superbes harnais. Les deux ducs,
couverts des plus belles pierreries, ainsi que les harnais de leurs
admirables chevaux, étaient, ainsi que le corrégidor, en habits
ordinaires, mais extrêmement magnifiques. Les trois quadrilles, leur
chef à la tête, suivies de force gentilshommes, pages et laquais,
entrèrent l'une après l'autre dans la place, dont elles firent le tour,
et toutes leurs comparses, dans un très bel ordre et sans la moindre
confusion, au bruit de leurs fanfares, celle de Medina-Coeli la
première, celle del Arco après, puis celle de la ville. Les chefs, l'un
après l'autre, se rendirent après les comparses sous le balcon de Leurs
Majestés Catholiques, où étaient le prince et la princesse, les infants
et leurs plus grands officiers, tandis que la brigade arrivait
vis-à-vis, sous le balcon où j'étais. De cet endroit ils partirent deux
à la fois, prenant chacun à l'entrée de la lice un grand et long
flambeau de cire blanche, bien allumé, qui leur était présenté de chaque
côté en même temps, d'où prenant d'abord le petit galop quelques pas,
ils poussaient leurs chevaux à toute bride tout du long de la lice, et
les arrêtaient tout à coup sur cul sous le balcon du roi. L'adresse de
cet exercice, où pas un ne manqua, est de courir de front sans se
dépasser d'une ligne ni rester d'une autre plus en arrière, tête contre
tête et croupe contre croupe, tenant d'une main le flambeau droit et
ferme, sans pencher d'aucun côté et parfaitement vis-à-vis l'un de
l'autre, et le corps ferme et droit. La quadrille del Arco suivit dans
le même ordre\,; puis celle de la ville. Chaque couple de cavaliers
n'entrait en lice qu'après que l'autre était arrivée, mais partait au
même instant, et à mesure qu'ils arrivaient ils prenaient leur rang en
commençant sous le balcon du roi, et quand chacune avait achevé de
courir, force fanfares en attendant que l'autre commençât. Les courses
de toutes trois finies, leurs chefs en reprirent chacun la tête de la
sienne, et dans le même ordre, mais alors se suivant toutes trois,
firent leurs comparses et le tour de la place au bruit de leurs
fanfares, sortirent après de la place et se retirèrent comme elles
étaient venues. L'exécution en fut également magnifique, galante et
parfaite, et dans un ordre et un silence qui en releva beaucoup la
grâce, l'adresse et l'éclat.

On eut une autre fête dans la même place, avec la même illumination, que
la cour vit de la même maison dans la place, et moi vis-à-vis dans celle
d'où j'avais vu la course des flambeaux avec le nonce, Maulevrier et
tout ce qui était de chez moi. J'ai expliqué ailleurs les places des
grands, et comment les balcons des cinq étages de la place tout autour
sont remplis et les toits chargés de peuple, ainsi que le fond de la
place en foule, mais sans faire au spectacle le plus petit embarras. Ce
fut un combat sur mer d'un vaisseau turc contre une galère de Malte, qui
eut la victoire après deux heures de combat, le désempara et le brûla.
L'eau était si parfaitement représentée, et les mouvements des deux
bâtiments si aisés, leur manoeuvre si vive et si multipliée, les
événements des approches et du combat si vifs, si justes, si variés, si
souvent douteux pour la victoire, qu'on ne se doutait plus que ce fût un
jeu qui se passait à terre. Le spectacle dura plus de deux heures et fut
toujours également intéressant. Les agrès, les habillements, les armes,
rien d'oublié, et tout représentait si naïvement un vaisseau turc et une
galère maltaise, les services et les mouvements des combattants et des
manoeuvres des gens de mer, qu'on ne pouvait se rappeler que tout cela
fût factice. Jusqu'au vent favorisa la fête en dissipant la fumée de la
mousqueterie et des bordées de canon. La mêlée de l'abordage fut surtout
merveilleusement exécutée, repoussée et reprise à diverses fois. Enfin
ce combat parut tellement effectif et sérieux que l'événement seul
déclara la victoire.

Enfin il y eut encore un autre feu d'artifice, dans la place du palais,
tout différent, mais tout aussi beau que le premier, où Leurs Majestés
Catholiques me firent l'honneur de me retenir fort longtemps près
d'elles sur leurs balcons.

\hypertarget{chapitre-viii.}{%
\chapter{CHAPITRE VIII.}\label{chapitre-viii.}}

1722

~

{\textsc{Buen-Retiro.}} {\textsc{- Morale et pratique commode des
jésuites sur le jeûne en Espagne.}} {\textsc{- Je veux voir la prison de
François Ier.}} {\textsc{- Délicate politesse de don Gaspard Giron.}}
{\textsc{- Expédient de Philippe III contre l'orgueil des cardinaux.}}
{\textsc{- Prison de François Ier.}} {\textsc{- Je vais voir Tolède.}}
{\textsc{- Causes particulières de ma curiosité.}} {\textsc{- Contes et
sorte de forfait des cordeliers de Tolède.}} {\textsc{- Différence de
notre prononciation latine d'avec celle de toutes les autres nations.}}
{\textsc{- Le carême fort fâcheux dans les Castilles.}} {\textsc{-
Vesugo, excellent poisson de mer.}} {\textsc{- Église métropolitaine de
Tolède.}} {\textsc{- Humble sépulture du cardinal Portocarrero.}}
{\textsc{- Beauté admirable des stalles du choeur.}} {\textsc{- Chapelle
et messe mosarabiques.}} {\textsc{- Évêques mêlés avec les chanoines
sans aucune distinction.}} {\textsc{- Drapeau blanc au clocher de
l'église de Tolède pour chaque archevêque ou chanoine devenu cardinal,
qui n'en est ôté qu'à sa mort.}} {\textsc{- Députation du chapitre de
Tolède pour me complimenter.}} {\textsc{- Ville et palais de Tolède.}}
{\textsc{- Aranjuez.}} {\textsc{- Amusement de sangliers.}} {\textsc{-
Haras de buffles et de chameaux.}} {\textsc{- Lait de buffle exquis.}}

~

Le carême mit fin aux fêtes, et Leurs Majestés Catholiques quittèrent le
palais et allèrent habiter celui de Buen-Retiro. Ce fut aussi le temps
de l'anniversaire de la feue reine dite la \emph{Savoyana}, dans
l'église de l'Incarnation, qui est grande et belle, quoique ce soit un
couvent de religieuses. Les grands y furent invités à l'ordinaire, par
conséquent mon second fils et moi, et non les ambassadeurs. Le banc des
grands et le siège ployant du majordome-major du roi y étaient disposés
comme en chapelle, mais sans prie-Dieu du roi, sans siège de cardinaux
et sans banc d'ambassadeurs. Mais les majordomes du roi s'y trouvèrent
debout à leurs places comme en chapelle, et le clergé, comme en
chapelle, assis vis-à-vis des grands, et tous autres debout. Le duc
d'Abrantès, évêque de Cuença, y fit pontificalement l'office dans une
chaire à l'antique, dont j'ai fait la description, et donné la figure
ici avec le plan de la séance du roi tenant chapelle. Il y eut la veille
des premières vêpres\,; j'y allai avec le duc de Liria. Il n'y avait
encore personne en place. Nous entrâmes dans la sacristie, où nous
trouvâmes deux ou trois grands. Il s'y en amassa bientôt davantage, et
quand nous fûmes une quinzaine, quelqu'un proposa d'aller prendre place
et d'envoyer prier le prélat de commencer. Quand ce fut pour sortir de
la sacristie, aucun ne voulut passer devant moi, et par conséquent
{[}ils{]} me voulaient céder la première place sur le banc. Après
quelques compliments, je leur dis que je leur parlerais comme me faisant
un grand honneur d'être leur confrère\,; que j'avais en même temps ceux
d'être ambassadeur et grand d'Espagne\,; que si j'acceptais ce qu'ils
avaient la bonté de m'offrir, cela ferait un exemple et fort aisément
une règle pour d'autres cérémonies et pour d'autres ambassadeurs\,; que
quelque estime que je fisse d'un si grand caractère, il n'était que
passager\,; que je faisais bien plus de cas de la dignité solide,
permanente, héréditaire de grand d'Espagne\,; et que par ces raisons je
leur conseillais et les suppliais de passer cinq ou six devant moi pour
entrer dans l'église et se placer sur le banc\,; que de cette façon il
n'y aurait rien à dire, et qu'ils éviteraient un exemple qui pourrait
leur devenir désagréable. Ils me remercièrent avec beaucoup de
reconnaissance, et me crurent. Le duc de Medina-Coeli passa le premier,
quatre ou cinq autres le suivirent, moi ensuite, puis les autres, et
nous nous rangeâmes de même sur le banc. Aussitôt la musique du roi
commença les vêpres, le prélat étant arrivé tout revêtu à son siège
comme nous nous placions. Une vingtaine de grands arrivèrent ensuite les
uns après les autres.

Le lendemain nous nous trouvâmes en bien plus grand nombre à la messe
chantée par la musique du roi et célébrée par le même prélat. Ma
politesse fit un grand effet à la cour\,; tous les grands m'en surent un
gré infini, et beaucoup d'entre eux me le témoignèrent. Je n'étais point
là comme ambassadeur, et je me crus en liberté et en raison d'en user de
la sorte.

Le Retiro, dont je ne ferai point la description, parce que celles
d'Espagne en sont remplies, est, à mon gré, un palais aussi magnifique
que le palais de Madrid, plus grand et beaucoup plus agréable. Il a des
cours, dont une est réservée, comme ici pour ce qui s'y appelle les
honneurs du Louvre, où entrent les carrosses des cardinaux, des
ambassadeurs et des grands seulement, et un parc admirable si les arbres
y venaient mieux, et que l'eau des fontaines et des magnifiques pièces
d'eau fût plus abondante. Rien ne ressemble tant, de tout point, à son
parterre en face du palais, que celui de Luxembourg, à Paris\,: mêmes
formes, mêmes terrasses, même contour et même tour de fontaines et de
jets d'eau. Le mail y est admirable et d'une prodigieuse grandeur. J'ai
observé qu'en cette saison, qui est toujours belle en Espagne, le mail
succède tous les jours à la chasse, où le roi n'allait plus qu'un peu
après Pâques\,; et j'ai aussi expliqué comment se passait ce jeu de mail
et cette promenade, où j'allais presque tous les jours faire ma cour. Un
jour que je vis la reine y prendre plusieurs fois du tabac, je dis que
c'était une chose assez extraordinaire de voir un roi d'Espagne qui ne
prenait ni tabac ni chocolat. Le roi me répondit qu'il était vrai qu'il
ne prenait point de tabac\,; sur quoi la reine fit comme des excuses
d'en prendre, et dit qu'elle avait fait tout ce qu'elle avait pu, à
cause du roi, pour s'en défaire, mais qu'elle n'en avait pu venir à
bout, dont elle était bien fâchée. Le roi ajouta que pour du chocolat il
en prenait avec la reine les matins, mais que ce n'était que les jours
de jeûne. «\,Comment, sire, repris-je de vivacité, du chocolat les jours
de jeûne\,! --- Mais fort bien, ajouta le roi gravement, le chocolat ne
le rompt pas. --- Mais, sire, lui dis-je, c'est prendre quelque chose,
et quelque chose qui est fort bon, qui soutient, et même qui nourrit.
--- Et moi je vous assure, répliqua le roi avec émotion et rougissant un
peu, qu'il ne rompt pas le jeûne, car les jésuites, qui me l'ont dit, en
prennent tous les jours de jeûne, à la vérité sans pain ces jours-là,
qu'ils y trempent les autres jours. » Je me tus tout court, car je
n'étais pas là pour instruire sur le jeûne\,; mais j'admirai en moi-même
la morale des bons pères et les bonnes instructions qu'ils donnent,
l'aveuglement avec lequel ils sont écoutés et crus privativement à qui
que ce soit, du petit des observances au grand des maximes de l'Évangile
et des connaissances de la religion. Dans quelles ténèbres épaisses et
tranquilles vivent les rois qu'ils conduisent\,!

Pendant le séjour de la cour au Retiro, le palais de Madrid était vide
et je le voulus voir en détail. Je m'adressai pour cela à don Gaspard
Giron, qui voulut bien se donner la peine de me promener partout. C'est
encore une description que je laisse aux voyageurs et à ceux qui ont
traité localement de l'Espagne\,; mais j'en donnerai un morceau que je
n'ai rencontré nulle part.

En nous promenant, je dis à don Gaspard que je craignais sa politesse et
qu'elle ne me privât de ce que je désirais voir principalement. Le bon
homme m'entendit bien, car il était spirituel et fin\,; mais la
galanterie espagnole lui fit faire le sourd. Il m'assura toujours qu'il
ne me cacherait rien. «\,Je parie que si, señor don Gaspard, lui
dis-je\,: la prison de François Ier\,? --- Eh\,! fi\,! fi\,! \emph{señor
duque}, de quoi parlez-vous là\,?» Et {[}il{]} changea tout de suite de
propos en me montrant des choses. Je l'y ramenai, et à force de
compliments et de propos, je le forçai de m'accorder ma demande\,; mais
ce fut avec des façons si polies, si honteuses, si ménagées qu'il ne se
pouvait marquer plus d'esprit et de délicatesse. Il voulut que je me
défisse de ce qui était avec moi, excepté M. de Céreste et ma famille\,;
puis me mena dans une salle, très vaste par où nous avions passé, qui
est entre la salle des gardes et l'entrée du grand appartement du roi.
En attendant que les clefs fussent venues, qu'il avait envoyé chercher,
il me montra deux enfoncements faits après coup, vis-à-vis l'un de
l'autre, dans l'épaisseur de la muraille, qui avaient chacun un siège de
pierre, tous deux égaux, dans l'enfoncement d'une fenêtre. Cette pièce
avait quatre fenêtres de chaque côté sur la cour et sur le Mançanarez,
et la muraille du côté du Mançanarez est si épaisse qu'elle fait de
chaque fenêtre de ce côté-là comme un vrai cabinet enfoncé, tout ouvert.
Après m'avoir fait remarquer et bien considérer ces deux sièges de
pierre, il me demanda ce qu'il m'en semblait. Je lui dis que cette
curiosité me paraissait fort médiocre et ne pas mériter la peine de la
remarquer. «\,Vous allez voir que si, me répliqua-t-il, et vous en
conviendrez tout à l'heure.\,» Il me conta alors que Philippe III,
fatigué de l'orgueil de cardinaux qui prenaient un fauteuil devant lui
dans leurs audiences, se mit à ne leur en plus donner que debout dans
cette salle, en s'y promenant, et que, lassé ensuite d'être debout ou de
se promener quand les audiences s'allongeaient, il fit creuser ces deux
enfoncements avec ces sièges de pierre pour s'y asseoir d'un côté, le
cardinal de l'autre, et de cette façon éviter le fauteuil. Et voilà où
conduisent l'usurpation, d'une part, et la faiblesse, de l'autre. Il me
dit ensuite, toujours en attendant les clefs, que François Ier avait
d'abord été logé dans la maison, alors bien plus petite, où le duc del
Arco demeurait actuellement, qu'on avait accommodée en prison, et qui
est au centre de Madrid\,; mais qu'au bout de quelques mois, on ne l'y
avait pas cru assez en sûreté\,; et que, le trouvant trop ferme sur les
propositions qu'on lui faisait, on avait voulu le resserrer pour tâcher
de l'ébranler, et qu'on l'avait mis dans le lieu qu'il m'allait montrer,
puisque je m'obstinais si opiniâtrement à le voir.

Les clefs à la fin arrivées, et tout étant prêt à entrer, don Gaspard
nous mena, tout au bas bout de cette salle, dans l'enfoncement de la
dernière fenêtre sur le Mançanarez. Arrivé là, je regardai de côté et
d'autre, et n'y aperçus point d'issue. Don Gaspard riait cependant et me
laissait chercher ce que je ne trouvais point\,; puis il poussa une
porte dans l'épaisseur du mur, du côté d'en bas de l'espèce de cabinet,
dans l'épaisseur de la longue muraille, où était cette fenêtre, si
artistement prise, et sa serrure tellement cachée qu'il n'était pas
possible de s'en apercevoir. La porte était basse et étroite, et me
présenta un escalier entre deux murs, qui ne l'était pas moins. C'était
une espèce d'échelle de pierre, d'une soixantaine de marches fort
hautes, ayant pourtant assez de giron, au haut desquelles, sans tournant
ni repos, on trouvait un petit palier qui, du côté du Mançanarez, avait
une fort petite fenêtre bien grillée et vitrée, de l'autre côté une
petite porte à hauteur d'homme et une pièce assez petite avec une
cheminée, qui pouvait contenir quelque peu de coffres et de chaises, une
table et un lit, qui ne tirait de jour que, la porte ouverte, par la
petite fenêtre vis-à-vis du palier. Continuant tout droit, on trouvait
au bout de ce palier, c'est-à-dire quatre ou cinq pieds après la
dernière marche, quatre ou cinq autres marches aussi de pierre -et une
double porte très forte avec un passage étroit entre deux, long de
l'épaisseur du mur d'une fort grosse tour. La seconde porte donnait dans
la chambre de François Ier, qui n'avait point d'autre entrée ni sortie.
Cette chambre n'était pas grande, mais accrue par un enfoncement sur la
droite en entrant, vis-à-vis de la fenêtre, assez grande pour donner du
jour suffisamment, vitrée, qui pouvait s'ouvrir pour avoir de l'air,
mais à double grille de fer, bien forte et bien ferme, scellée dans la
muraille des quatre côtés. Elle était fort haute du côté de la chambre,
donnait sur le Mançanarez et sur la campagne au delà. Il y avait de quoi
mettre des sièges, des coffres, quelque table et un lit. À côté de la
cheminée, qui était en face de la porte, il y avait un recoin profond,
médiocrement large, sans jour que de la chambre, qui pouvait servir de
garde-robe. De la fenêtre de cette chambre au pied de la tour, au bord
du Mançanarez, il y a plus de cent pieds, et tant que François Ier y
fut, deux bataillons furent jour et nuit en garde sous les armes, au
pied de cette tour, au bord du Mançanarez, qui coule tout le long et
fort proche. Telle est la demeure où François Ier fut si longtemps
enfermé, où il tomba si malade, où la reine sa soeur l'alla consoler, et
contribua tant et si généreusement à sa guérison et à disposer sa
sortie, et où Charles-Quint, craignant enfin de le perdre, et avec lui
tous les avantages qu'il se promettait de tenir un tel prisonnier,
l'alla enfin visiter, et commença à le traiter d'une manière plus
humaine.

Je considérai cette horrible cage de tous mes yeux et de toute ma plus
vive attention, malgré les soins de don Gaspard Giron à m'en distraire
et à me presser d'en sortir. Souvent je ne l'entendais pas, tant j'étais
appliqué à ce que j'examinais\,; souvent aussi en l'entendant je ne
répondais point. Ils n'avouèrent ni ne désavouèrent que l'escalier ne
fût gardé en dedans, et que cette chambre obscure sur le palier fût un
corps de garde d'officiers. Enfin il ne manquait rien aux précautions
les plus recherchées pour que François Ier ne pût se sauver.

Je pris ensuite cinq ou six jours pour un voyage que, dès en allant en
Espagne, j'avais bien résolu de faire. Je voulus voir Tolède où
plusieurs raisons de curiosité m'attiraient. Je voulais voir cette
superbe église si renommée par son étendue et sa magnificence, tout ce
qu'elle renferme de richesses, et ce clocher superbe, dont le revenu est
de cinq millions. Je voulais voir le lieu où s'étaient tenus ces
célèbres conciles de Tolède, d'où toute l'Église a adopté plusieurs
canons, et si augustes par la science et la sainteté de presque tous les
Pères qui les composèrent. Enfin je voulais voir et entendre le rit et
la messe connus sous le nom de Mosarabiques qui ne sont plus conservés
qu'à Tolède, où le grand cardinal Ximénès les a fondés pour toujours
dans une chapelle de la cathédrale et dans les sept paroisses de la
ville où on n'en célèbre point d'autres.

Cette liturgie, qui est latine, et qui, pour l'offertoire et le canon de
la messe est, pour tout l'essentiel, {[}en{]} tout semblable à la messe
d'aujourd'hui, c'est-à-dire à l'oblation, aux espèces, au \emph{memento}
des vivants et des morts, aux paroles et à la forme de la consécration,
à l'ostension et à l'adoration de l'eucharistie et du calice consacré, à
la communion et au même sens des différentes prières qui précèdent et
qui suivent, même à la lecture de l'épître et de l'évangile, est un
grand et précieux monument. C'est la messe qui se disait avant le
sixième\footnote{Il y a dans le manuscrit \emph{sixième siècle} et non
  \emph{huitième siècle}, comme on l'a imprimé dans les précédentes
  éditions pour rectifier une erreur de date. La conquête de l'Espagne
  par les Arabes n'eut lieu, en effet, qu'après la bataille de Xérès
  livrée en 711.} siècle, puisqu'elle est antérieure à la conquête d'une
partie de l'Espagne par les Arabes, ou, comme on dit communément, par
les Maures, dans les premières années du sixième siècle, excités et
introduits par le comte Julien, outré de ce que Roderic, ou comme on le
nomme plus communément, Rodrigue, roi d'Espagne, avait violé sa fille.
Je pris donc mes mesures avec l'archevêque de Tolède, avec qui on a vu
ici que j'étais en commerce fort particulier, et je fis ce petit voyage.

Quoiqu'il y ait près de vingt lieues, des environs de Paris, de Madrid à
Tolède, des relais bien disposés m'y firent arriver en un jour, et de
fort bonne heure. Le chemin est beau, ouvert, uni\,; mais Tolède est au
pied et dans la montagne. En arrivant dans le faubourg qui est en bas,
au pied d'un haut rocher, sur lequel est le reste de l'ancien château,
on me fit tourner le dos à l'entrée de la ville, et aller aux
Cordeliers, dont le couvent fut le lieu de l'assemblée de ces fameux
conciles de Tolède. À peine eus-je mis pied à terre que les notables du
couvent s'empressèrent autour de moi, et me firent d'abord remarquer une
vieille fenêtre grillée du château, d'où ils me dirent que le roi
Rodrigue avait vu la fille du comte Julien, qui demeurait dans
l'emplacement d'un côté de leur maison, et que c'était là que ce prince
s'était embrasé d'un amour qui avait été si funeste à lui et à toutes
les Espagnes. Cette tradition sur cette fenêtre ne me fit pas grande
impression, d'autant que la fenêtre et ses appartenances me parurent
fort éloignées de plus de mille ans d'antiquité.

Ces moines me conduisirent dans leur église, qui, non plus que son
portail, assez neuf, ne me semblèrent que fort communs. À peine y fus-je
entré qu'ils m'arrêtèrent et me demandèrent si je n'apercevoir pas
quelque chose de fort extraordinaire. Je vis un crucifix de grandeur
naturelle, de relief, au lieu de tableau du grand autel, en caleçon et
en perruque, comme ils sont presque tous en Espagne, qui ne me surprit
point, parce que j'en avais vu beaucoup d'autres pareils. Comme je ne
répondais point, cherchant des yeux ce qu'ils voulaient me faire
remarquer\,: «\,Eh\,! les bras\,!» me dirent-ils. En effet, j'en vis un
attaché à l'ordinaire, et l'autre pendant le long du corps. À mon, tour,
je leur demandai ce que cela signifiait. Un grand miracle toujours
existant, à ce qu'ils m'assurèrent d'un ton grave et dévot. Et aussitôt
me contèrent, en supprimant toute date, ce qu'était alors cette
église\,; qu'un riche bourgeois, ayant fait un enfant à une fille, sous
promesse verbale de l'épouser, il l'avait nié et s'était moqué d'elle\,;
mais qu'elle et ses parents, qui n'avaient point de preuve, l'engagèrent
à s'en rapporter à ce crucifix, tellement qu'étant tous venus à
l'église, suivis d'une foule de peuple, la fille et le garçon ne
s'étaient pas plutôt présentés devant le crucifix que son bras gauche
s'était détaché de la croix de soi-même, et doucement baissé et placé
tel qu'il était demeuré depuis et que nous le voyons, sur quoi on
s'était écrié au miracle, et le garçon avait épousé la fille.

Quoique à l'abri de l'inquisition par mon caractère d'ambassadeur, il
fallait éviter de donner du scandale dans un pays aussi dominé par la
superstition\,: j'avalai donc le plus doucement que je pus ce pieux
conte que ces moines exaltaient et me pressaient d'admirer. Ils me
menèrent faire un moment d'adoration au pied du grand autel, puis me
firent faire le tour des chapelles de l'église, dont chacune avait ses
miracles particuliers qu'il me fallut essuyer. D'une chapelle à l'autre
je les priai de me mener à la salle des conciles, ou à ce qui en
restait, qui était uniquement ce qui m'amenait chez eux. Ils me
répondirent\,: «\,Tout à l'heure, mais encore cette chapelle-ci, car
elle est bien remarquable.\,» Et il fallait y aller et entendre les
miracles auxquels je me refroidissais beaucoup. Enfin, quand tout fut
épuisé et qu'il fut question d'aller à la salle des conciles, ils me
dirent qu'il n'en restait rien, et que depuis cinq ou six mois, ils en
avaient abattu les restes pour y bâtir leur cuisine. Je fus saisi d'un
si violent dépit que j'eus besoin de me faire la dernière violence pour
ne les pas frapper de toute ma force. Je leur tournai le dos en leur
reprochant cette espèce de sacrilège en termes fort amers. Je gagnai mon
carrosse sans vouloir mettre le pied dans leur maison, et y montai sans
leur faire la moindre civilité. Voilà ce que deviennent les monuments
les plus précieux de l'antiquité, par l'ignorance, l'avarice ou la
convenance, sans que la police ni que personne se mette en peine de les
revendiquer et de les faire conserver. J'eus à celui-ci un regret
extrême.

L'archevêque de Tolède m'avait engagé à loger chez lui, où j'allai
descendre. Céreste, le comte de Lorges, mes enfants, l'abbé de
Saint-Simon et son frère, l'abbé de Mathan, et deux officiers principaux
de nos régiments étaient avec moi, et furent logés dans l'archevêché ou
dans les maisons joignantes. J'y fus reçu par les deux neveux de
l'archevêque, et servi par ses officiers qu'il y avait envoyés exprès.
Les neveux étaient chanoines, et le cadet montrait de l'esprit et de la
politesse\,; nous nous parlions latin. L'aîné, quoique inquisiteur,
croyant que je lui parlais une autre langue qu'il n'entendait pas, me
pria de me servir avec lui de la latine. C'est que nous autres,
Français, prononçons le latin tout autrement que les Espagnols, les
Italiens et les Allemands. À la fin pourtant il m'entendit. Ils ne
manquèrent à rien de la plus grande civilité, sans se rendre le moins du
monde incommodes. Le palais archiépiscopal n'est pas grand\,; toutes
petites pièces assez obscures et vilaines, fort simplement meublées. Il
est sur une petite place, latéralement au portail de la métropole. On
nous servit un grand nombre de plats et trois services, rien du tout de
gras\,; et nous fûmes servis de la sorte toujours soir et matin, mais le
soir de toutes choses de collation.

Le carême est fort fâcheux dans les Castilles. La paresse et
l'éloignement de la mer font que la marée est inconnue. Les plus grosses
rivières n'ont point de poisson, les petites encore moins, parce
qu'elles ne sont que des torrents. Peu ou point de légumes, si ce n'est
de l'ail, des oignons, des cardons, quelques herbes. Ni lait ni beurre.
Du poisson mariné, qui serait bon si l'huile en était bonne\,; mais elle
est si généralement mauvaise qu'on en est infecté jusque dans les rues
de Madrid, en carême, car presque tout le monde le fait, jeunes et
vieux, hommes et femmes, seigneurs, bourgeois et peuple. Ainsi on est
réduit aux oeufs de toutes les façons et au chocolat, qui est leur
grande ressource. Le \emph{vesugo} est l'unique poisson de mer qui se
mange à Madrid. Il vient de Bilbao vers Noël, et tout le monde se
félicite lorsqu'il commence à paraître. De figure et de goût il tient du
maquereau et de l'alose, et a la délicatesse et la fermeté des deux. Il
est excellent. On en mange les jours gras comme les maigres sans s'en
lasser. Mais il commence à piquer dès le commencement du carême, et
bientôt après on n'en peut plus manger. La chère que nous fîmes à Tolède
n'était donc pas friande, à l'espagnole et fort grande, mais il était
impossible de mieux.

Dès le matin, j'allai voir l'église ou plutôt les églises, car il s'en
détache deux chapelles à angle égal, grandes comme des églises, qui
s'appellent, l'une des anciens rois, l'autre des nouveaux rois, qui ont
de magnifiques tombeaux, et chacune un grand et beau choeur de
plain-pied devant le grand autel, et chacune un riche et nombreux
chapitre, où l'office se fait comme dans la grande église, sans
s'interrompre ni s'entendre réciproquement, toutes trois. La sacristie,
pleine de richesses immenses, est vaste et pourrait passer pour une
quatrième église. J'y vis la chape impériale de Charles-Quint, de toile
d'or fort ample et à queue d'un pied, semée près à près d'aigles noires
éployées, à double tête, le chaperon et les orfrois d'une étoffe qui
paraît avoir été magnifique et surbrodée, avec une large attache de même
étoffe et des agrafes d'or. On m'y ouvrit une armoire, entre bien
d'autres, remplie des raretés les plus précieuses, au fond matelassé de
laquelle était attachée la belle croix du Saint-Esprit de diamants, que
le feu roi avait envoyée au cardinal Portocarrero, environnée d'un grand
tour d'admirables diamants, d'où pendait la Toison d'or que portait
Charles II d'ordinaire et qu'il donna peu avant sa mort à cette
église\,: deux présents fort inutiles, comme ils sont.

Je ne m'arrêterai point ici à une description de structure ni de
richesses, qui est un des plus curieux et des plus satisfaisants
morceaux des relations et des voyages d'Espagne, et qui, seule et
exacte, ferait plus d'un volume\,; je me bornerai à de simples remarques
et en fort petit nombre. La tombe plate du cardinal Portocarrero est
sans nul ornement dans le passage entre le choeur métropolitain et la
chapelle des nouveaux rois, en sorte qu'elle est foulée aux pieds de
tout le monde, avec cette seule inscription et sans armes\,: \emph{Hic
jacet chais, pulvis, et nihil}, suivant qu'il l'ordonna expressément\,;
mais on a mis vis-à-vis sur la muraille une magnifique épitaphe en son
honneur. L'église métropolitaine n'a point le défaut de presque toutes
les églises d'Espagne. Le choeur y est de plain-pied, c'est-à-dire
relevé de trois ou quatre marches plus que la nef, entre la nef et le
grand autel, et fermé à peu près comme est celui de Notre-Dame, à Paris,
mais le choeur et la nef presque le double plus longue et plus large, et
haute à proportion. Le choeur a tout autour trois rangs de stalles, tous
trois plus élevés l'un que l'autre, ce qui en fait un nombre prodigieux.
Elles sont commodes, et tant les stalles que la boiserie entière, qui
est fort élevée et richement travaillée, sont de bois précieux. Pas une
stalle de trois rangs ne ressemble à une autre pour le travail. Le
dossier, les côtés, les dessus des séparations, le devant de chaque
stalle relevée, est d'une ciselure en bois plus finement travaillée et
plus exactement recherchée que les plus belles tabatières d'or. Les
sujets en sont pris de la vie de Ferdinand le Catholique et d'Isabelle,
sa première femme, qui, par leur mariage, réunirent les couronnes
d'Aragon et de Castille et leurs dépendances, et dont les conquêtes
éteignirent la domination des Maures en Espagne\,; et comme rien n'y est
oublié en aucun genre, jusques aux plus petites choses, les événements
depuis leur naissance jusqu'à leur mort ont pu fournir à toutes ces
stalles sans aucun vide. Il n'y en a aucune qui ne méritât plusieurs
heures d'application à la considérer, et dont la rare beauté ne fit
trouver ces heures courtes.

L'archevêque avait ordonné que, encore qu'on fût en carême, la messe
mozarabique fût chantée et célébrée devant moi aussi solennellement que
le jour de Pâques. Cette chapelle de la cathédrale, où cet office est
fondé, a son choeur particulier et est vers le bas de la nef. On mit un
prie-Dieu avec un tapis et quatre carreaux, deux en bas pour les genoux,
deux en haut pour les coudes, pour mon second fils et pour moi, qui est
le traitement des cardinaux, des ambassadeurs et des grands, dans toutes
les églises d'Espagne. Cela était préparé du côté de l'évangile, tout
près de l'autel, en sorte qu'étant à genoux je voyais pleinement dessus.
Mon second fils et moi fûmes conduits sur ce prie-Dieu, et on donna
seulement un carreau au comte de Larges, à Céreste, à mon fils aîné, à
l'abbé de Saint-Simon et à son frère.

Je vis et j'entendis cette messe avec une grande curiosité et un extrême
plaisir. Je ne la décrirai point ici, parce que je la vis telle que je
l'ai lue décrite et expliquée dans le cardinal Bona\footnote{Le cardinal
  Bona a laissé un grand nombre de traités. Il s'agit probablement ici
  de son traité \emph{De rebus liturgicis}, où l'on trouve des
  recherches sur les cérémonies et les prières de la messe.} et dans
d'autres livres liturgiques. Elle se dit en latin, avec les ornements
ordinaires, tant des célébrants que de l'autel. Il y a seulement
toujours deux livres aux deux côtés sur l'autel\,: l'un est pour tout ce
qui est de la messe, l'autre pour les collectes pour le peuple, qui sont
fort multipliées, ainsi que les \emph{amen} du choeur. Cela et la
séparation de l'Eucharistie en quinze parties en croix sur la patène, en
prononçant un nom de mystère sur chaque particule en la séparant et la
posant, et dans la suite en prenant pour se communier chaque particule
l'une après l'autre, en prononçant le même nom de mystère, rend la messe
un peu plus longue que les nôtres\,; mais cela est peu perceptible à une
grand'messe par le chant du choeur, qui allonge toujours.

De là je fus conduit au choeur, dont je voulus voir l'office, où je fus
placé au bout le plus près de l'autel, et sur le devant de ma stalle et
de celle de mon second fils, il y avait un tapis et des carreaux comme
dans la chapelle mozarabe\,; les autres eurent chacun leur stalle et un
carreau. Je remarquai avec surprise deux évêques en rochet et camail
violet, avec leur croix au cou, dans les stalles parmi les chanoines,
sans aucune distinction ni distance, et des chanoines également
au-dessous et au-dessus d'eux. Il y avait des bancs disposés en travers
dans le milieu, dans le large espace entre les stalles de chaque côté,
où les chanoines se vinrent asseoir pour entendre le sermon d'un jacobin
après l'évangile. Ces deux évêques s'y placèrent parmi les chanoines en
leur rang d'ancienneté, comme ils étaient dans les stalles, sans
distance, sans distinction, joignant les chanoines au-dessus et
au-dessous d'eux. C'étaient deux évêques \emph{in partibus} suffragants
pour soulager l'archevêque dans ses fonctions épiscopales, comme
confirmations, ordinations, consécrations des saintes huiles, etc. Ce
qui me parut singulier fut une espèce de drapeau blanc arboré et
flottant au plus haut du superbe clocher de cette église, qui est
prodigieusement élevé, et d'une riche et admirable structure. Je crus
qu'on était dans l'octave de la dédicace de l'église, mais on me
détrompa bientôt en m'apprenant que ce drapeau était là pour le cardinal
Borgia. C'est qu'aussitôt qu'un chanoine de Tolède, ou l'archevêque,
devient cardinal, on met ce drapeau au clocher\,; et s'il arrive qu'il
se trouve plusieurs chanoines cardinaux, on met un drapeau pour chacun
d'eux, et le drapeau de chacun n'est ôté qu'à sa mort.

Au retour de l'église, et avant le dîner, on m'annonça deux chanoines
qui venaient me complimenter au nom du chapitre. En même temps, je fus
averti que l'un était un Pimentel, archidiacre de l'église de Tolède,
par conséquent d'une des plus grandes maisons d'Espagne, et de la même
que le comte de Benavente\,; que ce chanoine avait quatre-vingt mille
livres de rentes de sa prébende, et qu'il avait refusé les archevêchés
de Séville et de Saragosse\,; qu'il était aussi chef de l'inquisition du
diocèse, et qu'il était accompagné d'un autre chanoine de qualité dont
la prébende lui valait soixante mille livres de rente. C'étaient là des
chanoines tant soit peu renforcés en comparaison des nôtres. Tout ce qui
était avec moi, et beaucoup d'autres gens de la ville, dont le corps
m'était venu saluer, les neveux et les principaux officiers de
l'archevêque remplissaient la pièce où j'étais, où nous étions tous
debout. Je fis quelques pas au-devant des deux chanoines\,; je leur fis
donner deux sièges à côté l'un de l'autre, et j'en pris un vis-à-vis
d'eux. Je les priai par signes de se couvrir, et nous nous couvrîmes
tous trois, tout le reste debout, faute de sièges et de place. Les
chanoines étoient en habit long avec un chapeau. Dès que je fus couvert,
je me découvris et ouvris la bouche pour les remercier\,; à l'instant,
le Pimentel, le chapeau à la main, se leva, s'inclina, me dit
\emph{domine} sans m'avoir donné l'instant d'articuler un seul mot, se
rassit, se couvrit, et me fit une très belle harangue en fort beau
latin, qui dura plus d'un gros quart d'heure. Je ne puis exprimer ma
surprise ni quel fut mon embarras de répondre en français à un homme qui
ne l'entendait pas. Quel moyen\,! en latin, comment faire\,? Toutefois,
je pris mon parti\,; j'écoutai de toutes mes oreilles, et tandis qu'il
parla, je bâtis ma réponse pour dire quelque chose sur chaque point, et
finir par ce que j'imaginai de plus convenable pour le chapitre et pour
les députés, en particulier pour celui qui parlait. Il finit par la même
révérence qui avait commencé son discours, et je voyais en même temps
toute cette jeunesse qui me regardait et riochait de l'embarras où elle
n'avait pas tort de me croire.

Le Pimentel rassis, j'ôtai mon chapeau, je me levai, je dis
\emph{domine}. En me rasseyant et me couvrant, je jetai un coup d'oeil à
cette jeunesse, qui me parut stupéfaite de mon effronterie, à laquelle
elle ne s'attendait pas. Je dérouillai mon latin comme je pus, où il y
eut sans doute bien de la cuisine et maints solécismes, mais j'allai
toujours, répondant point par point\,; puis, appuyant sur mes
remerciements, avec merveilles pour le chapitre, pour les députés et
pour le Pimentel, à qui j'en glissai sur sa naissance, son humilité, son
mépris des grandeurs, et son refus de deux si grands et si riches
archevêchés. Cette fin leur fit passer mon mauvais latin, et les
contenta extrêmement, à ce que j'appris. Je ne parlai pas moins
longtemps que le Pimentel avait fait. En finissant par la même
révérence, je jetai un autre coup d'œil sur la jeunesse, qui me parut
tout éplapourdie\footnote{Vieux mot qui ne se trouve pas dans les
  lexiques ordinaires. Il a le même sens que \emph{abasourdi}.} de ce
que je m'en étais tiré si bien. Il est vrai qu'elle n'admira pas mon
latin, mais ma hardiesse et ma suite, parce que j'avais répondu à tout,
et que je les avais après largement complimentés. Après quelques moments
de silence, ils se levèrent pour s'en aller, et je les conduisis jusque
vers le bout de la pièce suivante. Les neveux et l'assistance me
félicitèrent sur mon bien-dire en latin. Ce n'était pas, je pense,
qu'ils le crussent, ni moi non plus, mais enfin j'en étais sorti et
quitte.

Nous dînâmes bientôt après. Le maître d'hôtel, les porteurs de plats,
ceux qui nous donnaient à boire et des assiettes, ceux qui étaient au
buffet, tous me semblaient des jésuites, à qui je n'osais demander mes
besoins. J'ai déjà remarqué que tous les domestiques de l'archevêque de
Tolède, même tous ses laquais, cochers et postillons, étaient tous vêtus
en ecclésiastiques, sans aucune différence des prêtres, et que l'habit
ecclésiastique est demeuré en Espagne précisément le même que celui que
portent les jésuites, qui était l'habit de tous les ecclésiastiques du
temps de saint Ignace, leur instituteur. L'après-dînée, j'allai visiter
les deux chanoines qui m'étaient venus complimenter, qui, par politesse,
firent dire qu'ils étaient sortis. De là je fus voir le palais de Tolède
que Charles-Quint avait comme bâti de nouveau. Les troupes de l'archiduc
y mirent le feu la dernière fois qu'elles abandonnèrent cette ville et
les Castilles, et par le peu qui en est resté, on voit que ç'a été le
plus grand dommage du monde, et la plus insigne brutalité. Je retournai
ensuite à l'église que j'eus loisir de voir bien plus à mon aise que je
n'avais pu faire le matin. On m'y arracha de chaque endroit pour m'en
faire admirer d'autres. On y passerait bien du temps à satisfaire sa
curiosité. On ne m'indiqua rien d'ailleurs à voir à Tolède\,: la ville
est collée à une haute chaîne de montagnes\,; elle est toute bâtie sur
un penchant fort roide, les rues étroites et obscures, en sorte que les
voitures n'y peuvent presque aller. Elle est assez grande, impose par un
air d'antiquité, et, quoique vilaine et sans aucune maison d'une
certaine apparence, paraît beaucoup par la roideur de l'amphithéâtre
qu'elle occupe, et qui la montre tout entière. Je n'y séjournai qu'un
jour entier.

De Tolède, j'allai à Aranjuez, environ comme de Paris à Meaux. On me fit
descendre et loger chez le gouverneur qui était absent, dans un grand et
beau corps de logis, tout près du château, à droite en arrivant. C'est
le seul endroit des Castilles où il y ait de beaux arbres, et ils y sont
en quantité. De quelque côté qu'on y arrive, c'est par une avenue d'une
lieue ou de trois quarts de lieue, dont plusieurs ont double rang
d'arbres, c'est-à-dire une contre-allée de chaque côté de l'avenue. Il y
en a douze ou treize qui arrivent de toutes parts à Aranjuez, où leur
jonction forme une place immense, et la plupart percent au delà à perte
de vue. Ces avenues sont souvent coupées par d'autres transversales,
avec des places dans leurs coupures, et par leur grand nombre forment de
vastes cloîtres de verdure ou de champs semés, et se vont perdre à une
lieue de tous côtés dans les campagnes.

Le château est grand\,; les appartements en sont vastes et beaux,
au-dessus desquels les principaux de la cour sont logés. Le Tage
environne le jardin, qui a une petite terrasse tout autour, sur la
rivière, qui est là étroite et ne porte point bateau. Le jardin est
grand, avec un beau parterre et quelques belles allées. Le reste est
coupé de bosquets, de berceaux bas et étroits, et plein de fontaines de
belle eau, d'oiseaux et d'animaux, de quelques statues qui inondent les
curieux qui s'amusent à les considérer. Il sort de l'eau de dessous
leurs pieds\,: il leur en tombe de ces oiseaux factices, perchés sur les
arbres, une pluie abondante, et une autre qui se croise en sortant de la
gueule des animaux et des statues, en sorte qu'on est noyé en un
instant, sans savoir où se sauver. Tout ce jardin est dans l'ancien goût
flamand, fait par des Flamands que Charles-Quint fit venir exprès. Il
ordonna que ce jardin serait toujours entretenu par des jardiniers
flamands sous un directeur de la même nation, qui aurait seul le droit
d'en ordonner, et cela s'est toujours observé fidèlement depuis.
Accoutumés depuis au bon goût de nos jardins amenés par Le Nôtre, qui en
a eu tout l'honneur, par les jardins qu'il a faits et qui sont devenus
des modèles, on ne peut s'empêcher de trouver bien du petit et du
colifichet à Aranjuez. Mais le tout fait quelque chose de charmant et de
surprenant en Castille par l'épaisseur de l'ombre et la fraîcheur des
eaux. J'y fus fort choqué d'un moulin sur le Tage, à moins de cent pas
du château, qui coupe la rivière et dont le bruit retentit partout.
Derrière le logement du gouverneur sont de vastes basses-cours, et
joignant un village fort bien bâti. Derrière tout cela est un parc fort
rempli de cerfs, de daims et de sangliers, où on est conduit par ces
belles avenues\,; et ce pare est un massif de bois étendu, pressé,
touffu pour ces animaux. Une avenue fort courte nous conduisit à pied
sous une manière de porte fermée d'un fort grillage de bois qui donnait
sur une petite place de pelouse environnée du bois. Un valet monta assez
haut à côté de cette porte, et se mit à siffler avec je ne sais quel
instrument. Aussitôt cette petite place se remplit de sangliers et de
marcassins de toutes grandeurs, dont il y en avait plusieurs de grandeur
et de grosseur extraordinaires. Ce valet leur jeta beaucoup de grain à
diverses reprises, que ces animaux mangèrent avec grande voracité,
venant jusque tout près de la grille, et souvent se grondant, et les
plus forts se faisant céder la place par les autres, et les marcassins
et les plus jeunes sangliers, retirés sur les bords, n'osant s'approcher
ni manger que les plus gros ne fussent rassasiés. Ce petit spectacle
nous amusa fort, près d'une heure.

On nous mena de là en calèche découverte, par les mêmes belles avenues,
à ce qu'ils appellent la Montagne et la hier. C'est une très petite
hauteur isolée, peu étendue, qui découvre toute la campagne et cette
immense quantité d'avenues et de cloîtres formés par leurs croisières,
ce qui fait une vue très agréable. Presque tout le
\emph{planitre}\footnote{Terrain plat, plateau (\emph{planities}).} de
cette hauteur est occupé par une grande et magnifique pièce d'eau, qui
est là une merveille et qui n'aurait rien d'extraordinaire dans tout
autre pays. Elle est revêtue de pierre, et porte quelques petits
bâtiments en forme de galères et de gondoles sur lesquelles Leurs
Majestés Catholiques se promènent quelquefois et prennent aussi le
plaisir de la péché, cette pièce étant assez fournie pour cela du
poisson qu'on a soin d'y entretenir. D'un autre côté, il y a une vaste
ménagerie, mais rustique, où on entretient un haras de chameaux et un
autre de buffles.

Des officiers du roi d'Espagne m'amenèrent le matin, comme je sortais,
un grand et beau chameau, bien ajusté et bien chargé, qui se mit à
genoux devant moi, pour y être déchargé d'une grande quantité de
légumes, d'herbages, d'oeufs, et de plusieurs barbeaux, dont
quelques-uns avaient trois pieds de long, et tous les autres fort grands
et gros, mais que je n'en trouvai pas meilleurs que ceux d'ici,
c'est-à-dire mous, fades et pleins d'une infinité de petites arêtes. Je
fus traité aux dépens du roi, et je séjournai un jour entier. Ce lieu me
parut charmant pour le printemps et délicieux pour l'été\,; mais l'été
personne n'y demeure, pas même le peuple du village, qui se retire
ailleurs et ferme ses maisons sitôt que les chaleurs se font sentir dans
cette vallée, qui causent des fièvres très dangereuses et qui tiennent
ceux qui en réchappent sept ou huit mois dans une langueur qui est une
vraie maladie. Ainsi la cour n'y passe guère que six semaines ou deux
mois du printemps, et rarement y retourne en automne. D'Aranjuez à
Madrid le chemin est assez beau, à peu près de la distance de Madrid à
l'Escurial. Mais, pour aller de l'une de ces maisons à l'autre, il faut
passer par Madrid.

À mon retour, le roi et la reine me demandèrent comment j'avais trouvé
Aranjuez. Je le louai fort, autant qu'il le méritait, et dans le récit
de tout ce que j'y avais vu, je parlai du moulin, et que je m'étonnais
comment il était souffert si proche du château, où sa vue, qui
interrompait celle du Tage, et plus encore son bruit, étaient si
désagréables, qu'un particulier ne le souffrirait pas chez lui. Cette
franchise déplut au roi, qui répondit qu'il avait toujours été là, et
qu'il n'y faisait point de mal. Je me jetai promptement sur d'autres
choses agréables d'Aranjuez, et cette conversation dura assez longtemps.
J'y mangeai du lait de buffle, qui est le plus excellent de tous et de
bien loin. Il est doux, sucré, et avec cela relevé, plus épais que la
meilleure crème, et sans aucun goût de bête, de fromage ni de beurre. Je
me suis étonné souvent qu'ils n'en aient {[}pas{]} quelques-uns à la
\emph{Casa del Campo}, pour faire usage à Madrid d'un si délicieux
laitage.

\hypertarget{chapitre-ix.}{%
\chapter{CHAPITRE IX.}\label{chapitre-ix.}}

1722

~

{\textsc{Réception de mon fils aîné dans l'ordre de la Toison d'or.}}
{\textsc{- Indécence du défaut des habits de la Toison, et de la manière
confuse des chevaliers d'accompagner le roi les jours de collier, qui
sont fréquents.}} {\textsc{- Manière dont le roi prend toujours son
collier.}} {\textsc{- Sa Majesté et tous ceux qui ont la Toison et le
Saint-Esprit ne portent jamais un collier sans l'autre.}} {\textsc{-
Nulle marque de l'ordre dans ses grands officiers, quoique d'ailleurs
pareils en tout à ceux du Saint-Esprit.}} {\textsc{- Rang dans
l'ordre\,; d'où se prend.}} {\textsc{- Le prince des Asturies est le
premier infant qui ait obtenu la préséance.}} {\textsc{- Les chevaliers,
grands ou non, couverts au chapitre.}} {\textsc{- Les grands officiers
découverts.}} {\textsc{- Différence très marquée de leur séance d'avec
celle des chevaliers.}} {\textsc{- Préliminaires immédiats à la
réception.}} {\textsc{- Réception.}} {\textsc{- Épée du grand capitaine
devenue celle de l'État.}} {\textsc{- Son usage aux réceptions des
chevaliers de la Toison.}} {\textsc{- Singuliers respects rendus à cette
épée.}} {\textsc{- Courte digression sur le grand capitaine.}}
{\textsc{- Accolade.}} {\textsc{- Imposition du collier.}} {\textsc{-
Révérences et embrassades.}} {\textsc{- Visites et repas.}} {\textsc{-
Cause du si petit nombre de chevaliers espagnols.}} {\textsc{- Expédient
qui rend enfin les ordres anciens et lucratifs d'Espagne compatibles
avec ceux de la Toison, du Saint-Esprit, etc.}} {\textsc{- Fâcheux
dégoût donné sur la Toison à Maulevrier, qui rejaillit sans dessein sur
La Fare.}} {\textsc{- Mon fils aîné s'en retourne à Paris\,; voit
l'Escurial.}} {\textsc{- Sottise des moines.}}

~

La santé de mon fils aîné qui ne se rétablissait point, et son
impatience de quitter un pays où il avait toujours été malade, me
pressait de le renvoyer. Sa santé et celle de la princesse des Asturies,
qui voulut voir la cérémonie de la réception d'un chevalier de l'ordre
de la Toison d'or, avait retardé la sienne. Rien ne s'y opposant plus,
je pris ce temps de la faire faire, et voici quelle elle fut.

SÉANCE DU CHAPITRE DE L'ORDRE DE LA TOISON D'OR POUR LE CONFÉRER À UN
NOUVEAU CHEVALIER.

\begin{enumerate}
\def\labelenumi{\arabic{enumi}.}
\item
  Fauteuil du roi.
\item
  Carreaux à ses pieds.
\item
  Table ornée.
\item
  Son tapis.
\item
  Carreau aux pieds du prince des Asturies.
\item
  Banc des chevaliers.
\item
  Tapis dont ces bancs sont couverts.
\item
  \begin{enumerate}
  \def\labelenumii{\arabic{enumii}.}
  \setcounter{enumii}{8}
  \tightlist
  \item
    Lieu ou le grand écuyer et le premier écuyer viennent se mettre à
    genoux.
  \end{enumerate}
\item
  Tapis dont le parterre est couvert.
\item
  Lieu d'où la reine et la princesse des Asturies, etc., firent la
  cérémonie debout.
\item
  Lieu d'où je la vis avec beaucoup de seigneurs.
\item
  Banc nu et sans tapis pour le chancelier et les autres grands
  officiers de l'ordre.
\item
  Par où le parrain sort, rentre et amène le chevalier à recevoir.
\end{enumerate}

Il faut remarquer que le fauteuil du roi n'est pas au milieu, mais un
peu retiré sur la gauche à cause de la table, par le respect de ce qui
est dessus.

Les habits de l'ordre de la Toison d'or appartiennent à l'ordre, qui les
fournit en entier aux nouveaux chevaliers, à la mort desquels ils sont
rendus à l'ordre, au lieu qu'en l'ordre du Saint-Esprit, dont l'habit
est fait aux dépens de chaque chevalier et demeure à ses héritiers, le
collier seul appartient à l'ordre, qui le lui prête sa vie durant, et
est après sa mort rendu à l'ordre, ou mille écus d'or s'il se trouvait
perdu. Quoique depuis le retour de Philippe II en Espagne, après la mort
de Charles-Quint, ni lui ni aucun roi d'Espagne ne soit jamais retourné
aux Pays-Bas, les habits de la Toison y étaient toujours demeurés, et
furent perdus pour l'Espagne avec les Pays-Bas lorsque ces provinces
tombèrent entre les mains des Impériaux après la bataille de Ramillies.
On s'en soucia peu, mal à propos, en Espagne, parce qu'on y était
accoutumé, dès Philippe II, à y faire des promotions de la Toison sans
habits. D'ailleurs, la prétention de l'empereur, quelque mal fondée
qu'elle fût, ayant toujours persisté sur la grande maîtrise de cet
ordre, la restitution des habits aurait été nécessairement une matière
inséparable de celle du droit à la grande maîtrise.

Ce défaut d'habits, qui eût pu être réparé si aisément en Espagne, en en
faisant faire comme on y a fait des colliers, ne l'a point été, et on ne
peut nier qu'il ne gâte extrêmement les cérémonies. Au moins ici, où,
depuis 1662, qui est la dernière promotion faite aux Grands-Augustins
suivant les statuts, au moins pour les habits, les chevaliers du
Saint-Esprit ne paraissent en aucune cérémonie qu'en rabat et en manteau
court, avec le collier par-dessus, ce qui fait au moins une cérémonie
uniforme et dans un habit qui ne se porte qu'en ces occasions, si on
s'est affranchi du grand habit de cérémonie qui, excepté des occasions
fort rares depuis cette époque, n'est plus porté que par les chevaliers
novices le jour de leur réception.

En Espagne, rien de plus indécent, où les chevaliers de la Toison d'or
portent le collier de l'ordre toutes les fêtes d'apôtres, quelques
autres grandes fêtes encore, aux chapitres de l'ordre, aux grandes
occasions de cérémonies de la cour, par exemple à mon audience de la
demande de l'infante. Chaque chevalier a son habit ordinaire, qui est
l'habit entièrement français. L'un a un justaucorps brun, un autre noir,
un autre rouge, un autre bleu. Celui-ci a de l'or, celui-là de l'argent.
On est en velours ou en drap, en un mot à son gré et à sa manière, avec
une perruque nouée, et une cravate, et le collier autour des épaules
par-dessus le justaucorps. Ils se rendent ainsi chez le roi les uns
après les autres, et l'attendent. Quand le roi sort de l'appartement
intérieur, il s'arrête sur le pas de la porte. Les deux plus anciens
chevaliers de la Toison se mettent à ses côtés, y reçoivent d'un valet
intérieur, qui est derrière le roi, ses colliers de la Toison et du
Saint-Esprit, qui se tiennent par de courtes chaînettes d'espace en
espace, les lui mettent autour des épaules et les lui attachent. Et soit
que le roi aille à la messe, à une audience de cérémonie d'ambassadeur,
ou au chapitre, ils marchent en confusion comme tout autre jour qu'ils
ne sont point en collier, et le remènent de même après en son
appartement. S'il y a chapelle, les chevaliers qui ne sont point grands
vont jusqu'à la porte et n'y entrent guère, parce qu'ils n'y sont point
assis, et qu'ils n'y ont point de place. Ceux qui sont grands se mettent
sur le banc des grands parmi les autres grands, tout à l'ordinaire,
comme ils se trouvent et comme s'ils n'avaient point de collier. Tous
les jours de collier, les chevaliers de la Toison, qui le sont aussi du
Saint-Esprit, portent les deux colliers.

Le chancelier de l'ordre de la Toison, qui était lors le marquis de
Grimaldo, et qui dans la suite fut chevalier de la Toison, et les autres
grands officiers de l'ordre, dont pourtant je n'ai vu aucun, et qui sont
aussi considérables et aussi respectés par leurs places de secrétaire
d'État et de ministres que le sont les nôtres, ne portent aucune marque
de l'ordre, ni sur eux, pas même aux chapitres, ni aux réceptions de
chevaliers, ni à leurs armes. Nulle naissance, nulle dignité ne donne de
préséance dans l'ordre de la Toison. Elle n'est affectée qu'à
l'ancienneté dans l'ordre, et entre nouveaux chevaliers reçus en même
promotion, par leur âge. Le prince des Asturies est le premier de sa
naissance qui ait précédé les chevaliers plus anciens que lui. Le roi
son père demanda même au chapitre de {[}le{]} lui accorder comme une
grâce, et le chapitre opina et l'accorda\,; mais il ne fut que le
premier à droite sur le même banc des chevaliers, coude à coude avec le
chevalier son voisin, sans tapis autre que le tapis du banc sur lequel
tous les chevaliers sont assis comme lui. La seule distinction que je
lui vis est un carreau à ses pieds, plus petit et avec moins de dorure
que celui qui était aux pieds du roi, mais vis-à-vis précisément du
premier chevalier assis sur le banc de la gauche, car ils se rangent à
droite et à gauche par ancienneté, en sorte que les plus anciens sont le
plus près du roi, et ainsi de suite jusqu'aux deux derniers qui ferment
le banc, où, dés qu'ils sont tous assis, le roi se couvre et tous les
chevaliers en même temps, grands ou non, et demeurent couverts pendant
toute la cérémonie. Le chancelier et les autres grands officiers de
l'ordre s'asseyent aussi en même temps sur un banc de bois nu et sans
tapis, placé vis-à-vis du roi, au bas bout à la fin des bancs des
chevaliers, et ne se couvrent point pendant toute la cérémonie. C'est
ainsi que j'y vis toujours le marquis de Grimaldo. La reine, la
princesse des Asturies, leurs dames et leurs grands officiers, excepté
le prince Pio, chevalier de la Toison, virent la cérémonie debout, en
voyeuses, et arrivèrent en même temps que le roi. Je la vis de même avec
beaucoup de seigneurs vis-à-vis d'elle, fort proches, et la vîmes très
bien. Elle est assez longue, je vais tâcher de l'expliquer. L'heure fut
donnée pour le\ldots.\footnote{L'heure est laissée en blanc dans le
  manuscrit.}.

Le duc de Liria, accompagné du prince de Masseran, aussi chevalier de la
Toison, vint me prendre avec mon fils aîné dans son carrosse, attelé de
quatre parfaitement beaux chevaux de Naples, et se mirent tous deux sur
le devant, quoi que mon fils et moi pûmes faire. Mais ces beaux
napolitains, qui sont extrêmement fantasques, ne voulurent point
démarrer. Coups de fouets redoublés, cabrioles, ruades, fureurs, prêts à
tous moments à se renverser. Cependant l'heure se passait, et je priai
le duc de Liria que nous nous missions dans mon carrosse pour ne pas
faire attendre le roi et tout le monde. J'eus beau lui dire que cela ne
pouvait nuire à sa fonction de parrain, puisque nous étions dans son
carrosse, et que ce n'était que par la force de la nécessité que nous en
prendrions un des miens, il ne voulut jamais y entendre. Ce manège dura
une demi-heure entière, au bout de laquelle les chevaux consentirent
enfin à partir.

Tout mon cortège nous accompagnait et suivait, comme à ma première
audience et comme à la couverture de mon second fils. Je voulais
toujours faire voir aux Espagnols le cas que je faisais des grâces du
roi d'Espagne et des honneurs de leur cour. Au milieu du chemin la
fantaisie reprit aux chevaux de s'arrêter et de recommencer leur
manège\,; moi à insister de nouveau à changer de carrosse, et le duc de
Liria à n'en point vouloir ouïr parler. Cette pause néanmoins fut bien
moins longue\,; mais comme nous partions vint un message du roi dire
qu'il nous attendait. Enfin, nous arrivâmes, et dès que le roi en fut
averti, il sortit, prit ses colliers de la manière que j'ai expliquée,
traversa une pièce, entra dans une autre fort grande, où le chapitre
était disposé. Il alla droit se mettre dans son fauteuil, et en même
temps les chevaliers sur leurs bancs, en leur rang, comme je l'ai
expliqué, et Grimaldo sur le sien, seul des grands officiers et pas un
des petits, ainsi je n'ai point vu où ni comment ils se placent.

Pendant qu'ils se plaçaient, la reine, la princesse des Asturies, les
infants et leur suite s'allèrent mettre debout où le chiffre le marque,
et moi avec tout ce qui m'avait suivi, où le chiffre le marque, avec une
vingtaine de seigneurs, et quelque peu de voyeurs se tinrent éloignés
dans le bas de la salle par où nous étions entrés.

Tout ce que je viens de dire arrivé et rangé, la porte vis-à-vis du roi,
par laquelle nous étions tous entrés, fut fermée, et mon fils aîné
demeuré dehors avec beaucoup de gens de la cour. Alors le roi se couvrit
et tous les chevaliers en même temps, sans qu'il le leur dit ni leur en
fit signe, et en cet état le silence dura un peu plus d'un \emph{pater}.
Ensuite le roi proposa le vidame de Chartres pour être reçu dans
l'ordre, mais en deux mots. Tous les chevaliers se découvrirent,
s'inclinèrent sans se lever, et se couvrirent. Tout ce qui était
spectateur, et la reine même, qui n'avait point de siège près d'elle,
n'étaient là que comme n'y étant pas, parce que le chapitre doit être
secret, et n'y avoir personne que les chevaliers. Ainsi je ne fis aucune
révérence qu'à la reine, qui eut la bonté de me faire des signes de
compliments et de satisfaction. Après ce silence, le roi appela le duc
de Liria, qui se découvrit et s'approcha du roi avec une révérence, qui
lui dit sans se découvrir\,: \emph{Allez voir si le vidame de Chartres
ne serait point ici quelque part}. Le duc de Liria fit une révérence au
roi sans en faire aux chevaliers, quoique découverts en même temps que
lui, sortit et la porte fut refermée, et les chevaliers couverts. Il
sera souvent parlé de révérences\,; mais il faut entendre toutes
celles-ci, ainsi que les deux que le duc de Liria venait de faire, des
mêmes révérences qui se font ici aux réceptions des chevaliers du
Saint-Esprit et en toutes les grandes cérémonies.

Le duc de Liria demeura près d'un demi-quart d'heure dehors, parce qu'il
est censé que le nouveau chevalier ignore la proposition qui se fait de
lui, et que ce n'est que par un pur hasard qu'on le trouve quelque part
dans le palais, ce qui ne se peut faire si promptement. Si on avait des
habits de la Toison en Espagne, ce chapitre ne serait que préliminaire,
et il y en aurait un second, au bout de quelque temps, à la porte duquel
le chevalier admis se trouverait et serait introduit par son parrain
aussitôt que le chapitre serait assis en place. Le duc de Liria rentra
et aussitôt la porte fut refermée, et de la même façon qu'il s'était
approché du roi, il lui dit que le vidame de Chartres était dans l'autre
pièce.

Le roi lui ordonna d'aller demander au vidame s'il voulait accepter
l'ordre de la Toison d'or et y être reçu, et pour cela s'engager à en
observer les statuts, les devoirs, les cérémonies, en prêter les
serments et se soumettre à tous les engagements que promettent tous ceux
qui y sont reçus, et les promettre\,; enfin de se comporter en tout
comme un bon, loyal, brave et vertueux chevalier. Le duc de Liria se
retira et sortit comme il avait fait la première fois. La porte se
ferma. Il fut un peu moins dehors, puis rentra. La porte se referma, et
il se rapprocha du roi comme les autres fois, et lui apporta le
consentement et le remerciement du vidame. \emph{Hé bien\,!} répondit le
roi, \emph{allez le chercher et l'amenez}. Le duc de Liria se retira
comme les autres fois, sortit et aussitôt rentra, ayant mon fils à sa
gauche. La porte ouverte, le demeura et entra qui voulut, et se jeta où
il put pour voir la cérémonie.

Le duc de Liria entra au chapitre, suivi de mon fils, par l'endroit du
chiffre marqué, et le conduisit aux pieds du roi, puis alla s'asseoir à
sa place. Mon fils s'était doucement incliné à droite et à gauche,
entrant dans le parterre, aux chevaliers\,; et après avoir fait au
milieu du parterre une inclination profonde, s'alla mettre à genoux
devant le roi, sans quitter son épée, ayant son chapeau sous le bras et
sans gants. Les chevaliers, qui s'étaient tous découverts à l'entrée du
duc de Liria, se couvrirent lorsqu'il s'assit, et le prince des Asturies
aussi, qui se découvrit et se couvrit toujours comme eux. Le roi répéta
à mon fils les mêmes choses un peu plus étendues qu'il lui avait fait
dire par le duc de Liria, et reçut sa promesse sur chacune, l'une après
l'autre. Ensuite un sommelier de courtine, qui était debout, en rochet,
derrière la table, présenta au roi, par derrière, entre la table et sa
chaise, un grand livre ouvert, où était un long serment que mon fils
prêta au roi, qui avait le livre ouvert sur ses genoux, et le serment
sur d'autres papiers en français, sur le livre. Cela fut assez long.
Ensuite mon fils baisa la main du roi, qui le fit lever et passer devant
la table directement sans révérence, au milieu de laquelle il se mit à
genoux, le dos au prince des Asturies, vis-à-vis le sommelier de
courtine, qui lui montra la table entre deux, ce que et comment il
fallait faire. Il se mit à genoux. Il y avait sur cette table un grand
crucifix de vermeil sur un pied, un missel ouvert à l'endroit du canon,
un évangile de saint Jean, et des papiers de promesses et d'autres de
serments à faire et à lire en français, mettant la main tantôt sur le
canon, tantôt sur l'évangile. Cela fut encore long\,; puis, sans détour
ni révérence, il revint se mettre à genoux devant le roi.

Alors le duc del Arco, grand écuyer, et Valouse, premier écuyer, qui
n'eurent la Toison que depuis, et qui étaient auprès de moi, partirent,
le duc le premier, Valouse derrière lui, portant sur ses deux mains,
avec un grand air d'attention et de respect, l'épée du grand capitaine,
qui est don Gonzalve de Cordoue, qu'on n'appelle point autrement. Ils
firent à pas comptés le tour par derrière le banc des chevaliers de la
droite, tournèrent par derrière celui du marquis de Grimaldo, entrèrent
dans le chapitre par où le duc de Liria était entré avec mon fils,
coulèrent en dedans le long du banc des chevaliers à gauche, sans
révérence, mais le duc s'inclinant, et Valouse sans aucune inclination,
à cause du respect de l'épée\,; mais les grands ne s'inclinèrent point.
Le duc, en arrivant entre le prince des Asturies et le roi, se mit à
genoux, et Valouse derrière lui. Quelques moments après, le roi leur fit
signe, Valouse tira l'épée du fourreau, le mit sous son bras, prit
l'épée nue par la lame vers le milieu, en baisa la garde et la présenta
au duc del Arco, toujours tous deux à genoux. Le duc la prit un peu
au-dessus de ses mains, baisa la garde, la présenta au roi, qui, sans se
découvrir, en baisa le pommeau, prit la garde des deux mains, la tint
quelques moments droite\,; puis d'une main, mais presque aussitôt des
deux, en frappa trois fois alternativement chaque épaule de mon fils, en
lui disant\,: \emph{Par saint Georges et saint André, je vous fais
chevalier}. Et les coups tombaient assez pesamment par le grand poids de
l'épée. Pendant que le roi en frappait, le grand et le premier écuyer
étaient toujours à genoux en la môme place. Elle fut rendue comme elle
avait été présentée et baisée de même. Valouse la remit dans le
fourreau, après quoi le grand écuyer et lui se levèrent, et s'en
allèrent comme ils étaient venus.

Cette épée, avec sa poignée, avait plus de quatre pieds, la lame large
en haut de quatre gros doigts, épaisse à proportion, diminuant de
largeur et d'épaisseur insensiblement jusqu'à la pointe, qui était fort
fine. La poignée me parut d'un vieux vermeil travaillé, longue et fort
grosse, ainsi que le pommeau\,; la croisière longue et les deux bouts
larges, plats, travaillés, point de branche. Je l'examinai fort, et je
ne la pus lever en l'air d'une main, encore moins la manier avec les
deux que fort difficilement. On prétend que c'est l'épée dont se servait
le grand capitaine, avec laquelle il avait tant remporté de victoires.

J'admirai la force des hommes de ces temps, à quoi l'habitude de
jeunesse faisait, je crois, beaucoup. Je fus touché d'un si grand
honneur fait à sa mémoire, que son épée fût devenue l'épée de l'État, et
que, jusque par le roi même, il lui fût porté un si grand respect. Je
répétai plus d'une fois que si j'étais le duc de Sesse, qui en descend
directement par femme, car il n'y en a plus de mâles, il n'y a rien que
je ne fisse pour obtenir la Toison, afin d'avoir l'honneur et le plaisir
sensible d'être frappé de cette épée, et avec un si grand respect pour
mon ancêtre. Tout grand capitaine qu'il fût, il ne chassa les Français
du royaume de Naples que par la perfidie la plus insigne et la plus
sacrilège\,; et quand son maître, plus perfide que lui encore, n'en eut
plus besoin, il le retira en Espagne, où, en arrivant, jaloux et
soupçonneux de l'honneur si singulier, on peut dire si étrange, après ce
qu'il avait fait aux François, que Louis XII lui fit de le faire manger
à sa table au dîner qu'il donna à Ferdinand le Catholique et à Germaine
de Foix, que Ferdinand venait d'épouser en secondes noces, à l'entrevue
de Savone, ce prince ingrat, en arrivant en Espagne, l'accabla de tant
de dégoûts qu'il le força de se retirer loin de sa cour, où il mourut
bientôt après de chagrin. Mais revenons à la cérémonie après cette
petite digression qui m'a si naturellement échappé.

L'accolade donnée par le roi après les coups d'épée, nouveaux serments
prêtés à ses pieds, puis devant la table, comme la première fois, et ce
dernier encore plus long, après quoi mon fils revint se mettre à genoux
devant le roi, mais sans plus rien dire. Alors Grimaldo se leva, et sans
révérence sortit du chapitre par sa gauche, coula par derrière le banc
droit des chevaliers, prit le collier de la Toison, qui était étendu au
bout de la table. En ce moment le roi dit à mon fils de se lever et de
demeurer debout en la même place. En même temps le prince des Asturies
et le marquis de Villena se levèrent aussi et s'approchèrent de mon
fils, tous deux couverts, et tous les autres chevaliers demeurant assis
et couverts. Alors Grimaldo, passant entre la table et le siège vide du
prince des Asturies, présenta debout le collier au roi, qui le prit à
deux mains, et cependant Grimaldo, passant par derrière le prince des
Asturies, s'alla mettre derrière mon fils. Dès qu'il y fut, le roi dit à
mon fils de s'incliner fort bas sans se mettre à genoux, et dans ce
moment le roi s'allongeant sans se lever, lui passa le collier et le fit
se redresser aussitôt, et prit le collier par devant, tenant seulement
le mouton. En même temps le collier lui fut attaché sur l'épaule gauche
par le prince des Asturies, sur l'épaule droite par le marquis de
Villena, par derrière par Grimaldo, le roi tenant toujours le mouton.

Quand le collier fut attaché, le prince des Asturies, le marquis de
Villena et Grimaldo, sans faire de révérence, ni qu'aucun chevalier se
découvrît, allèrent se rasseoir en leurs places, et dans le même moment
mon fils se mit à genoux devant le roi et lui baisa la main. Alors le
duc de Liria, sans révérence, découvert, sans qu'aucun chevalier se
découvrît, vint se mettre devant le roi, à la gauche, à côté de mon
fils, et tous deux firent la révérence au roi\,; se tournèrent devant le
prince des Asturies, lui firent la révérence, qui se leva en pied, et
fit l'honneur à mon fils de l'embrasser, et dès qu'il fut rassis lui
firent la révérence, puis se tournèrent devant le roi, lui firent la
révérence\,; après devant le marquis de Villena, lui firent la
révérence, qui se leva et embrassa mon fils, et se rassit, et ils lui
firent la révérence, de là se tournèrent devant le roi, à qui ils firent
la révérence\,; puis devant le chevalier à côté du prince des Asturies,
lui firent la révérence, qui se leva et embrassa mon fils et se rassit,
lui firent la révérence, puis se tournèrent devant le roi, lui firent la
révérence\,; allèrent devant le chevalier à côté du marquis de Villena,
lui firent la révérence, qui se leva et embrassa mon fils et se rassit,
lui firent la révérence, et ainsi à droite et à gauche alternativement,
les mêmes cérémonies jusqu'au dernier chevalier, après quoi mon fils
s'assit à côté, joignant et après le dernier chevalier, et se couvrit,
et le duc de Liria retourna à sa place.

Pendant cette cérémonie des révérences si étourdissante pour ceux qui la
font, le chevalier qui la reçoit et qui embrasse se découvre dès qu'ils
sont devant lui, ne se lève que leur révérence faite, n'en fait point et
reçoit assis la seconde révérence, après quoi il se couvre\,; tous les
autres chevaliers ne se découvrent point. Le prince des Asturies observe
ce qui vient d'être remarqué, tout comme les autres chevaliers. Mon
fils, assis, couvert et en place dans le chapitre, le roi demeura plus
d'un bon \emph{credo} dans son fauteuil, puis se leva, se découvrit, et
se retira dans son appartement comme il était venu. J'avais averti mon
fils de se presser d'arriver devant le roi à la porte de son appartement
intérieur. Il s'y trouva à temps et moi aussi, pour lui baiser la main
et lui faire nos remerciements, qui furent fort bien reçus. La reine y
arriva, qui nous combla de bontés. Il faut remarquer que la cérémonie de
l'épée et de l'accolade ne se fait point à ceux qui, ayant déjà un autre
ordre, l'ont ou sont censés l'avoir reçue, comme sont les chevaliers du
Saint-Esprit et de Saint-Michel, et les chevaliers de Saint-Louis.

Leurs Majestés Catholiques retirées, nous nous retirâmes aussi chez moi,
où il y eut un fort grand dîner. L'usage est, avant la réception, de
visiter tous les chevaliers de la Toison, et lorsque le jour en est
pris, de retourner chez tous les convier à dîner pour le jour de la
cérémonie, où le parrain se trouve avec l'autre chevalier dont il s'est
accompagné, les invite encore au palais avant d'entrer au chapitre, et
aide au nouveau reçu à faire les honneurs du repas. J'avais mené mon
fils faire toutes ces visites. Presque tous les chevaliers vinrent dîner
chez moi, et beaucoup d'autres seigneurs. Le duc d'Albuquerque, que je
voyais assez souvent, et qui s'était excusé du repas de la couverture de
mon fils, sur ce qu'il s'était ruiné l'estomac aux Indes, me dit qu'il
ne pouvait me refuser deux fois, à condition que je lui permettrais de
ne manger que du potage, parce que les viandes étaient trop solides pour
lui. Il vint donc et en mangea de six et assez raisonnablement de
presque tous. Il se fit après des apprêtes de son pain, qu'il trempait
légèrement dans tout ce qu'on servit de ragoûts à sa portée, desquelles
il ne mangeait que l'extrémité, et trouvait tout cela fort bon. Il ne
buvait que peu de vin avec de l'eau. Le dîner fut gai malgré le grand
nombre. J'ai déjà remarqué que les Espagnols, si sobres, mangeaient
autant et plus que nous chez moi, et avec goût, choix et plaisir\,; mais
sur la boisson, fort modestes. Voici les noms de ceux qui, en tous pays,
étaient alors chevaliers de la Toison d'or d'Espagne.

CHEVALIERS DE L'ORDRE DE LA TOISON D'OR D'ESPAGNE EXISTANTS EN 1722.

DE CHARLES II.

Le prince Jacques Sobieski.

*Le comte de Lemos.

*Le duc de Bejar\footnote{Le signe * mis avant le nom marque ceux qui
  étaient grands d'Espagne avant d'avoir la Toison\,; après le nom, les
  chevaliers de la Toison qui, depuis, ont été faits grands d'Espagne.
  (\emph{Note de Saint-Simon}.)}.

Le prince de Chimay.*

Le duc de Lorraine.

Le marquis de Conflans. Ce dernier était du comté de Bourgogne\,; son
nom est Vatteville.

*Le marquis de Villena.

L'électeur de Bavière.

8

DE PHILIPPE V.

Le prince des Asturies.

M. d'Asfeld, depuis maréchal de France.

M. le duc d'Orléans.

*Le prince Pio.

Le duc de Noailles.*

Le prince de Robecque.

Le comte de Toulouse.

Le marquis de Beauffremont.

Le maréchal duc de Berwick.*

Le marquis d'Arpajon.

Le comte Toring,

Le prince Fr.~de Nassau.

*Le duc d'Albuquerque.

Le maréchal de Villars.*

*Le duc de Popoli.

Le duc de Bournonville.*

Le marquis de Lede.*

*Le comte de Montijo.

Le prince Ragotzi.

M. de Caylus.

Le marquis, depuis maréchal de Brancas.*

\begin{itemize}
\tightlist
\item
  Le duc de Liria.
\end{itemize}

D. Lelio Caraffa.

Le prince de Masseran.*

Le marquis Mari.

Le marquis de Béthune, depuis duc de Sully.

Le duc de Ruffec, lors vidame de Chartres.

*Le duc d'Atri.

28

\emph{Nommés et non reçus}.

MM\hspace{0pt}. de Maulevrier et de La Fare, tous deux depuis maréchaux
de France.

Trente-six chevaliers, et les deux nommés trente-huit, et {[}douze{]}
colliers vacants.

Sur lesquels quatre Espagnols, outre le prince des Asturies\,;

Quatre Flamands et un Franc-Comtois, et six Italiens des pays autrefois
possédés par l'Espagne\,;

Treize Français ou comptés pour tels, dont deux au service d'Espagne\,;
et six Allemands ou réputés tels, dont deux souverains\footnote{On ne
  retrouve pas ici exactement le nombre de chevaliers indiqués plus haut
  par Saint-Simon.}.

Il y a lieu de s'étonner que, l'ordre de la Toison étant de cinquante
chevaliers, le grand maître non compris, ni les grands officiers de
l'ordre, et n'y pouvant y avoir aucun prélat, il y eût tant de colliers
vacants. Mais ce qui l'est bien plus, est le si petit nombre d'Espagnols
naturels, et le si grand nombre d'étrangers, surtout de Français.

Revenons à la raison de ces choses. Les ordres anciens d'Espagne,
Saint-Jacques, etc., sont fort riches. Les plus grands seigneurs
d'Espagne les ont toujours pris pour en obtenir les meilleures
commanderies. La moindre noblesse et les domestiques principaux des
grands seigneurs y sont admis comme eux, la plupart pour s'honorer, et
dans l'espérance aussi des petites commanderies. Ces ordres étaient
incompatibles avec la Toison et avec tous les autres ordres. Les grands
seigneurs Espagnols préféraient presque tous l'utilité des commanderies
à l'honneur de porter la Toison, et les rois d'Espagne en étaient bien
aises et les entretenaient dans cet esprit pour avoir presque toutes les
Toisons à répandre dans leurs États d'Italie et des Pays-Bas, et en
donner aux empereurs de leur maison, tant qu'ils en voulaient, pour leur
cour et pour les princes d'Allemagne. Ces deux raisons cessèrent avec la
vie de Charles II, et par la guerre qui la suivit, qui fit perdre à
Philippe V\footnote{Il y a dans le manuscrit Philippe II\,; mais c'est
  une erreur évidente.} l'Italie et les Pays-Bas, qui étaient demeurés à
l'Espagne.

Le premier engouement de l'avènement de Philippe V à la couronne
d'Espagne donna aux plus grands seigneurs de l'émulation pour l'ordre du
Saint-Esprit, pour signaler leur attachement à la maison nouvellement
régnante, et porter une distinction qui montrait la considération et la
faveur qu'ils en avaient acquises. Bientôt la difficulté de parvenir à
l'ordre du Saint-Esprit, par la rareté des colliers accordés à
l'Espagne, donna du goût aux grands seigneurs, qui, de toute nation,
étaient attachés à la cour ou au service de Philippe V, pour la Toison,
dont ce prince disposait par lui-même, et dont le retranchement des
États de Flandre et d'Italie le rendait moins avare pour sa cour. Mais
l'intérêt des commanderies des ordres anciens d'Espagne les gênait par
la nécessité d'opter entre le profit et l'honneur. Ce fâcheux détroit
les engagea à chercher des moyens de réunir l'un à l'autre\,; et comme
les papes se sont peu à peu emparés en Espagne de ce qui est le moins de
leur dépendance, entre autres de l'ordre de la Toison, par la
confirmation qu'ils se sont arrogés d'en faire, et que les rois
d'Espagne ont bien voulu souffrir, cette union de l'honneur et du profit
d'ordres incompatibles parut enfin possible à ceux qui la désiraient, en
s'adressant à une cour qui avait su jeter le grappin sur les uns et sur
les autres, et où rien n'était impossible pour de l'argent. La
négociation en fut donc entreprise à Rome, qui, par ses politiques
lenteurs, en fit acheter le succès au prix qu'il lui plut d'y mettre. Il
y fut donc réglé qu'elle ne refuserait aucune dispense, à ceux qui
avaient les anciens ordres d'Espagne et qui en possédaient des
commanderies, d'accepter tous les autres grands ordres auxquels ils
pourraient être nommés, en payant une annate\footnote{Impôt qui
  consistait dans le revenu d'une année.} à Rome lorsqu'ils recevraient
ces autres ordres et tous les cinq ans une autre annate\,; moyennant
quoi les anciens ordres d'Espagne ni leurs commanderies n'étant plus un
obstacle pour la Toison et pour le Saint-Esprit, ces deux ordres
devinrent l'objet du désir et de l'espérance de tout ce qui, à la cour
ou dans le service d'Espagne, se flatta d'y pouvoir parvenir. Et comme
cette grande affaire ne venait que d'être consommée à Rome lorsque
j'arrivai en Espagne, je ne trouvai aussi que ce peu de chevaliers
espagnols et ce grand nombre de colliers vacants, qui peu à peu furent
presque tous bientôt remplis.

Cette autorité qu'on avait laissé prendre aux papes sur l'ordre de la
Toison fournit aux Espagnols une occasion de mortifier Maulevrier,
qu'ils haïssaient avec raison, et qu'ils ne ménageaient pas plus qu'ils
n'en étaient ménagés, d'autant plus désagréable que ce fut contre tout
exemple. Il fut nommé chevalier de la Toison dès que les mariages furent
déclarés et avant que je partisse de Paris. Il était commandeur de
l'ordre de Saint-Louis. Ce fut là-dessus que les Espagnols l'arrêtèrent
tout court. Ils prétendirent cet ordre incompatible avec celui de la
Toison, et qu'il ne la pouvait recevoir que par une dispense du pape.
Maulevrier, avare, qui vit que cette dispense lui coûterait de l'argent
et du temps, se récria contre cette chicane. Il allégua le grand nombre
de chevaliers du Saint-Esprit, et qui étaient aussi chevaliers de
Saint-Louis, à quoi on n'avait point objecté cette difficulté pour
recevoir la Toison. Il leur présenta même, dans la propre espèce dans
laquelle il se trouvait, l'exemple de MM. de Brancas et d'Asfeld,
commandeurs de l'ordre de Saint-Louis, comme il l'était, à qui on
n'avait point proposé cette chicane. L'exemple était existant et
péremptoire. Les Espagnols dirent que, si on s'était trompé à leur
égard, ce n'était pas une raison de continuer cette erreur, et ne se
cachèrent pas en même temps que ce n'était qu'une invention pour lui
faire de la peine. Il se plaignit, il cria, il s'adressa au roi
d'Espagne, il n'en fut autre chose malgré ses raisons sans réplique. Il
lui fallut recourir à Rome, y payer, en essuyer les lenteurs, qui depuis
six mois duraient encore, et que les Espagnols prenaient plaisir à
allonger. Cette niche et quelque chose de plus ne le raccommoda pas avec
eux ni eux avec lui, mais le contre-coup en tomba sur La Fare, qui n'y
avait rien de commun, et à qui les Espagnols ne se seraient pas avisés
de faire cette malice. Mais il était chevalier de Saint-Louis, et la
difficulté qui accrochait la réception de Maulevrier dans l'ordre de la
Toison d'or ne permit pas que La Fare, dans le même cas que lui, y fût
reçu sans dispense, tellement qu'il s'en retourna près d'un mois avant
moi à Paris, où il ne put recevoir la Toison que quelques mois après,
des mains de M. le duc d'Orléans, par commission du roi d'Espagne.

Deux jours après que mon fils aîné eut reçu la Toison, il prit congé de
Leurs Majestés Catholiques, etc., et partit pour Paris avec l'abbé de
Mathan, qui voulut bien nous faire l'amitié de s'en aller avec lui. Ils
passèrent par l'Escurial, qu'ils n'avaient point vu, chargés des lettres
du roi d'Espagne, du nonce, de Grimaldo, pour le prieur du monastère,
afin qu'ils fussent bien reçus et qu'on leur fît tout voir. Cela fut en
effet très bien exécuté\,; mais l'appartement où Philippe II mourut leur
demeura, comme à moi, inaccessible\,; et pour le pourrissoir, ils ne
purent jamais obtenir qu'il leur fût ouvert. Les moines étaient encore
fâchés des remarques que j'y avais faites sur le malheureux don Carlos,
et crurent s'en venger par là.

\hypertarget{chapitre-x.}{%
\chapter{CHAPITRE X.}\label{chapitre-x.}}

1722

~

{\textsc{Honneurs prodigués à l'infante, et fêtes à son arrivée à
Paris.}} {\textsc{- J'obtiens une expédition en forme de la célébration
du mariage du prince et de la princesse des Asturies, dont il n'y avait
rien par écrit.}} {\textsc{- Baptême de l'infant don Philippe.}}
{\textsc{- L'infant don Philippe reçoit le sacrement de confirmation et
l'ordre de Saint-Jacques.}} {\textsc{- Voyage très solitaire de quatre
jours, à Balsaïm, de Leurs Majestés Catholiques.}} {\textsc{- Je reçois
un courrier sur l'entrée des cardinaux de Rohan et Dubois au conseil de
régence, et sur la sortie des ducs, du chancelier et des maréchaux de
France du conseil de régence.}} {\textsc{- Manège du cardinal Dubois.}}
{\textsc{- Il présente au régent un périlleux fantôme de cabale.}}
{\textsc{- Lettre curieuse du cardinal Dubois à moi sur l'affaire du
conseil de régence.}} {\textsc{- Néant évident de la prétendue cabale.}}
{\textsc{- Dubois, par une lettre à part, veut que sur-le-champ j'en
fasse part à Leurs Majestés Catholiques, en quelque lieu qu'elles
fussent.}} {\textsc{- Second usage du fantôme de cabale pour isoler
totalement M. le duc d'Orléans.}} {\textsc{- Artifices de la lettre du
cardinal Dubois à moi.}} {\textsc{- Sa crainte de mon retour.}}
{\textsc{- Moyens qu'il tente de me retenir en Espagne.}} {\textsc{-
Autres pareils artifices du cardinal Dubois, qui me fait écrire avec
plus d'étendue et de force par Belle-Ile.}} {\textsc{- Remarques sur la
lettre de Belle-Ile à moi.}} {\textsc{- Je prends le parti de taire la
prétendue cabale, de ne dire que le fait existant, et d'aller à
Balsaïm.}} {\textsc{- Conversation avec Grimaldo.}}

~

Je ne m'étendrai point sur les honneurs prodigués à l'infante pendant
son voyage et là son arrivée à Paris, encore moins aux fêtes dont elle
fut suivie. J'étais trop loin pour les voir et pour m'en occuper. Je dis
prodigués, parce qu'elle fut en tout et partout traitée comme reine,
qu'elle fut même nommée et appelée l'infante reine, et qu'il ne lui
manqua que le traitement de Majesté. Je ne compris rien à l'engouement
auquel on s'abandonna là-dessus. M. le duc d'Orléans, glorieux sans la
moindre dignité, refusait tout en ce genre, ou en faisait litière\,: les
mesures et les bornes n'étaient jamais des choses auxquelles il voulut
donner le plus court moment de penser et de régler. D'ailleurs, tout
était abandonné au cardinal Dubois, de naissance et d'expérience fort
éloigné d'avoir les plus légères notions du cérémonial, si ce n'était
pour ce qui regardait les cardinaux. Il eut donc plutôt fait de se
laisser aller à ces profusions d'honneur que d'y donner la moindre
réflexion. Il crut faire sa cour en Espagne, et s'y porta avec d'autant
plus d'impétuosité que ce fut en chose où l'Angleterre ne pouvait
prendre aucun intérêt.

Le roi et la reine d'Espagne furent en effet très satisfaits, ainsi que
toute leur cour, de tout ce qui se passa en France en cette occasion,
c'est-à-dire de toutes les fêtes dont je leur rendis compte, qui
marquait la joie et l'empressement, car, pour les honneurs, ils furent
regardés comme dus et comme des choses qui ne pouvaient ne se pas faire.
L'infante était fille de France comme fille du roi d'Espagne, et cousine
germaine du roi, enfants des deux frères, et destinée à l'épouser. Ces
titres emportaient assez d'honneur pour s'y tenir, sans y ajouter encore
presque tous ceux des reines, qu'elle ne devait pas avoir, et qui
étaient contre tout exemple et toute règle. Si on les avait outrepassés
en faveur de la dernière dauphine, avant son mariage, le cas était bien
différent. Qui, dans un temps où une faible ombre d'ordre se laissait
encore apercevoir, eût pu s'accommoder des prétentions d'une fille de
Savoie, dont le père n'était pas roi, et cédait aux électeurs\,? Qui,
des princesses du sang, aurait osé lui céder\,? Qu'eût-elle pu obtenir
chez Madame, et même chez M\textsuperscript{me} la duchesse d'Orléans,
toute petite-fille qu'elle était de Monsieur, et destinée à épouser Mgr
le duc de Bourgogne\,?

Ce fut pour trancher toutes ces difficultés que le rang entier de
duchesse de Bourgogne lui fut avancé avant son mariage. Mais l'infante
n'avait besoin de rien\,; elle était fille de France et fille d'un grand
roi\,: par son rang personnel, elle précédait Madame. Elle n'avait donc
besoin ni de supposition ni de secours, et elle était trop grande pour
qu'ils pussent être à son usage. Les plus légers principes formaient ce
raisonnement\,; mais les principes et leurs conséquences n'étaient pas
du ressort du cardinal Dubois, ni familiers à la dissipation et à la
paresse d'esprit de son maître sur ce qu'il lui plaisait de mépriser
comme de petites choses, parmi lesquelles il en enveloppait trop souvent
de grandes.

Par cette raison, je m'avisai d'une chose à laquelle ils n'avaient pas
pris la peine de penser. Nous n'avions point de preuves par écrit de la
célébration du mariage de la princesse des Asturies, parce qu'en Espagne
les partis ne signent point avec leurs parents et leurs témoins sur le
registre du curé, comme on fait en France, et le roi même et les
personnes royales. En partant pour Tolède, j'en parlai au marquis de
Grimaldo. Il m'expliqua là-dessus l'usage d'Espagne, et néanmoins il me
promit de m'en donner une expédition en forme\,; je la reçus de lui à
mon retour de Tolède, et je l'envoyai au cardinal Dubois. Je crus devoir
cette précaution pour consolider de plus en plus un mariage qui ne
devait pas être consommé sitôt, quoiqu'il parût l'être, puisque le soir
du mariage du prince et de la princesse des Asturies, tout le monde
avait été admis à les voir au lit ensemble, contre tous les usages
d'Espagne, comme je l'ai rapporté en son lieu.

Je trouvai, en arrivant de Tolède, la grandesse fort intriguée sur le
baptême de l'infant don Philippe. Premièrement il y eut beaucoup de
jalousie sur le choix des représentants, qui furent le marquis de
Santa-Cruz pour l'électeur de Bavière, parrain, et la duchesse de La
Mirandole pour la duchesse de Parme, marraine, et ensuite du dépit sur
la fonction de porter les honneurs. La reine, dont c'était le fils, et
le roi, par complaisance pour elle, voulut charger des grands de cette
fonction, et les grands prétendirent qu'elle devait être donnée aux
majordomes de semaine, parce que l'infant n'était pas l'aîné et
l'héritier présomptif de la couronne. Ils s'assemblèrent plusieurs fois
chez le marquis de Villena, majordome-major du roi, qui lui porta deux
fois leurs représentations. Il fut mal reçu\,: les grands s'obstinèrent,
le roi menaça, nomma les grands des honneurs, qui cédèrent enfin et les
portèrent, mais d'une façon qui marquait leur dépit\,; et les autres
grands sais fonction, qui se trouvèrent à la cérémonie, parce que les
grands et les ambassadeurs de chapelle y furent invités, n'y laissèrent
guère moins apercevoir leur chagrin.

Le matin, les fonts sur lesquels saint Dominique fut baptisé, furent
apportés de chez les Dominicains, qui me parurent d'un beau granit, avec
des ornements de bronze doré, et un très lourd fardeau à transporter.
C'est l'usage de s'en servir pour les infants par respect et par
dévotion pour saint Dominique, qui était Espagnol, et de la maison de
Guzman. Les ambassadeurs étaient fort près de Leurs Majestés Catholiques
du côté de l'épître, qui arrivèrent après tout le monde sur les quatre
heures après midi. Le cardinal Borgia répétait alors sa leçon avec ses
aumôniers, entre la place de Leurs Majestés Catholiques et celle des
ambassadeurs, vêtu pontificalement avec la mitre. Il n'y parut ni plus
expert ni plus endurant qu'au mariage et à la vélation du prince des
Asturies\,: il cherchait, ânonnait, grondait ses aumôniers. Néanmoins il
fallut commencer la cérémonie, et il alla se placer de l'autre côté des
fonts, vis-à-vis de nous, suivi de deux aumôniers et des quatre
majordomes du roi, de semaine, et assisté des deux évêques \emph{in
partibus} suffragants de Tolède, résidant à Madrid, en rochet et en
camail. La duchesse de La Mirandole était fort parée et beaucoup de
pierreries\,; le marquis de Santa-Cruz portait le petit prince. Les
marquis d'Astorga et de Laconit, les ducs de Lezera ou de Licera, del
Arco, de Giovenazzo et le prince Pio portèrent les honneurs. Le cardinal
Borgia perdit tellement la tramontane qu'il ne savait ce qu'il faisait
ni où il en était\,; il fallut à tous moments le redresser malgré ses
impatiences\,: il brusqua tout haut, non seulement ses aumôniers, mais
les deux évêques qui voulurent venir au secours, et les majordomes qui,
pour les cérémonies extérieures, s'en mêlèrent aussi, et qu'il prit tout
haut à partie. Cette scène devint si ridicule que personne n'y put
tenir\,: tout le monde riait, et bientôt tout haut, et les épaules en
allaient au roi et à la reine qui en était aux larmes. Cela acheva
d'outrer et de désorienter le cardinal, qui, à tout moment, passait des
yeux de fureur sur toute l'assistance, qui n'en riait que plus
scandaleusement. Je n'ai rien vu de si étrange ni de plus plaisant\,;
heureusement pour chacun que tous furent également coupables, Leurs
Majestés Catholiques pour le moins autant qu'aucun, et que la colère du
cardinal ne put s'en prendre à personne en particulier. Elle alla
jusqu'à \emph{gourfouler}\footnote{Ce mot, qui a le sens de
  \emph{maltraiter}, ne se trouve que dans les anciens lexiques. Voy. du
  Cange, V° Affolare.}\emph{les} majordomes avec son poing, qui eurent
grand'peine, en riant, d'en contenir les éclats. Pour le prince et la
princesse des Asturies, ils ne s'en contraignirent pas.

Le 7 mars, le même prince reçut le sacrement de confirmation du même
cardinal Borgia, ayant le prince des Asturies pour parrain. Cela se fit
sans cérémonie. Il s'en fallait huit jours qu'il eût deux ans accomplis.
Cette confirmation me sembla bien prématurée. Le lendemain 8 mars, il
fut fait chevalier de l'ordre de Saint-Jacques et commandeur de la riche
commanderie d'Aledo, de la manière suivante. Le marquis de Bedmar,
président du conseil des ordres, chevalier de Saint-Jacques et de
l'ordre du Saint-Esprit, se plaça dans un fauteuil de velours à frange
d'or, loin, mais vis-à-vis de l'autel, ayant une table à sa droite,
ornée et parée, sur laquelle étaient un crucifix, l'évangile, etc. Une
vingtaine des plus considérables chevaliers de Saint-Jacques, avertis,
grands et autres, étaient assis des deux côtés, vis-à-vis les uns des
autres sur deux bancs couverts de tapis, en rang d'ancienneté dans
l'ordre, les plus anciens étant des deux côtés les plus proches du
marquis de Bedmar, et tous, ainsi que ceux qu'on va voir en fonctions,
vêtus de leurs habits ordinaires, ayant par-dessus un grand manteau
jusqu'aux talons, de laine blanche, avec l'épée de Saint-Jacques, bordé
en rouge, sur le côté gauche. Ce manteau était ouvert par devant comme
une chape de moine, et attaché autour de leur cou par de gros cordons
ronds, de soie blanche, ajustés en sorte qu'ils faisaient quelques
godrons\footnote{Plis.} en tombant tous deux sur le côté gauche, plus
bas que la broderie de l'ordre, terminés par deux grosses houppes de
soie blanche, telles pour leur forme qu'on en voit en vert aux armes des
évêques, à leurs chapeaux. Tous les chevaliers étaient couverts, et
derrière eux force spectateurs debout. Le roi, la reine, le prince, la
princesse des Asturies et leur accompagnement étaient dans une tribune,
et moi dans une autre au-dessus de la leur, avec ce qui était de chez
moi.

Le marquis de Santa-Cruz, portant le petit prince, vint de la sacristie
par le côté de l'épître, longeant par derrière le banc des chevaliers,
du même côté, avec assez de suite, mais d'aucuns chevaliers, et se tint
quelques moments debout entre la tête du banc et la table, où le marquis
de Bedmar, sans se découvrir, me parut se tourner et dire quelque chose,
et Santa-Cruz répondre. Il vint après, toujours découvert, se mettre à
genoux devant Bedmar, qui demeura couvert, ainsi que les deux bancs.
Cela dura peu. De là Santa-Cruz, toujours portant le petit prince,
s'alla mettre devant la table, apparemment pour d'autres serments, où il
fut plus longtemps. Il revint après devant le marquis de Bedmar, où il
se tint debout. Comme j'étais par derrière, je ne vis pas, et ne pus
entendre si Bedmar parlait. Je le crus, parce que cela dura un peu\,;
mais Santa-Cruz, que je voyais en face, ne dit rien. Ensuite Santa-Cruz
tourna entre ses bras le petit prince, de façon qu'il présentait le dos
à Bedmar, à qui en même temps deux personnes de la suite de l'infant
présentèrent un petit manteau pareil au sien et à celui de tous les
autres chevaliers. Le marquis de Bedmar le prit à deux mains et le mit
sur le petit prince, et le reçut aussitôt après sur ses genoux, où le
marquis de Santa-Cruz le plaça, et se retira quelque peu. Alors le
marquis de Montalègre, sommelier du corps, et le duc del Arco, grand
écuyer, se levèrent de dessus leurs bancs, et vinrent gravement, à côté
l'un de l'autre, au marquis de Bedmar, suivis du marquis de Grimaldo,
aussi chevalier, qui portait les éperons dorés. Ils étaient tous trois
découverts. Grimaldo, arrivé devant le marquis de {[}Bedmar{]}, fit avec
les deux autres la révérence, que le marquis de Santa-Cruz n'avait point
du tout faite, à cause de l'embarras de porter le prince, et présenta à
Montalègre l'éperon pour le pied droit, et au duc del Arco l'éperon pour
le pied gauche, que ces deux seigneurs chaussèrent ou attachèrent comme
ils purent, et que peu de moments ensuite ils lui ôtèrent, après quoi le
marquis de Santa-Cruz se rapprocha et prit le petit prince entre ses
bras, et s'en retourna comme il était venu. Quand le marquis de Bedmar
l'eut à peu près vu près de rentrer dans la sacristie, il se découvrit,
se leva, s'inclina aux chevaliers, qui se découvrirent et se levèrent en
même temps que lui, et chacun s'en alla sans cérémonie.

Ce qui me surprit au dernier point fut la paix et la tranquillité d'un
enfant de ce petit âge, qui, accoutumé à ses femmes, se trouva là sans
pas une au milieu de tous visages à lui inconnus et bizarrement vêtus,
se laisser porter, mettre sur les genoux, se laisser affubler d'un
manteau, manier les pieds ou au moins leur voisinage, puis remporter
sans jeter un cri ni une larme, et regarder tout ce monde inconnu sans
frayeur et sans impatience.

Le lendemain 9, le roi et la reine seuls s'en allèrent pour quatre jours
en relais à Balsaïm, uniquement accompagnés du duc del Arco, du marquis
de Santa-Cruz, du comte de San-Estevan de Gormaz, capitaine des gardes
en quartier\,; de Valouse, de la princesse de Robecque, dame du
palais\,; de la nourrice de la reine, et d'une seule camériste. Je les
vis partir assez matin, et fort peu après dîner je me mis en marche par
la ville, pour commencer mes adieux, comptant -prendre congé de Leurs
Majestés Catholiques fort peu de jours après leur retour de Balsaïm.

Dans la première que je fis, par laquelle on savait chez moi que je
devais commencer, on vint m'avertir de l'arrivée d'un courrier qui
m'était annoncé depuis longtemps et toujours différé parce qu'il devait
m'apporter des réponses et des ordres sur plusieurs choses auxquelles le
cardinal Dubois n'avait jamais le temps de travailler. Je m'en revins
donc chez moi tout court. Je trouvai d'abord une lettre du cardinal
Dubois, qui m'envoyait une relation de tout ce qui s'était fait à Paris
à l'arrivée de l'infante, et des fêtes qui l'avaient suivie, pour la
présenter et la faire valoir au roi et à la reine d'Espagne\,; une boîte
de lettres de toute la maison d'Orléans sur le mariage de la princesse
des Asturies, qui étaient bien tardives\,; et ce que j'attendais avec
impatience, la lettre de remerciement de M. le duc d'Orléans au roi
d'Espagne, sur les grâces que j'en avais reçues, et celles du cardinal
Dubois sur le même sujet au P. Daubenton et au marquis de Grimaldo. Il y
en avait des mêmes aux mêmes à part sur la Toison de La Fare. Rien dans
ce paquet, ni dans un autre, dont je vais parler, de tout ce qui me
devait être envoyé sur les affaires que le cardinal m'annonçait, et du
délai de quoi il s'excusait tous les ordinaires.

L'autre paquet était celui qui avait fait dépêcher le courrier. Le
cardinal Dubois entretenait toujours le cardinal de Rohan de l'espérance
de le faire bientôt déclarer premier ministre, comme il lui en avait
donné parole, à laquelle, comme on l'a vu ici en son temps, le cardinal
de Rohan avait eu la sottise d'ajouter une telle foi qu'il en avait
donné part au pape et à plusieurs cardinaux en partant de Rome, où la
chose était devenue publique, et où on ne s'était pas trouvé si crédule
que lui. Dubois, quoique secrétaire d'État des affaires étrangères, et
déjà le maître de toutes, s'était modestement abstenu d'entrer dans le
conseil de régence depuis son cardinalat, quoiqu'il y entrât toujours
auparavant. Il ne se sentait pas assez fort tout seul pour hasarder le
combat de préséance. C'était un poulet trop nouvellement éclos, qui
traînait encore sa coque. Il fit donc entendre au cardinal de Rohan
qu'il fallait commencer par être ministre avant d'être premier ministre,
et qu'il était temps qu'il demandât au régent d'entrer au conseil de
régence, qui, en arrivant de Rome, où il avait, disait-il, si grandement
servi, n'oserait l'en refuser, en l'assurant, de plus, qu'il ferait
réussir la chose. Rohan était le pont dont Dubois se voulait servir pour
y entrer lui-même, peu en peine après de s'en défaire quand il le
voudrait. Ainsi, mettant Rohan en gabion devant lui, il n'avait plus à
craindre les mépris personnels, les comparaisons odieuses, les brocards
de ceux qui se trouveraient indignés de lui céder. La dispute
s'adresserait en commun, et le cardinal de Rohan étant son ancien, tout
le personnel disparaissait nécessairement, dont rien n'était applicable
au cardinal de Rohan, duquel il ferait le plastron de la querelle, et
lui, modestement derrière lui, n'aurait qu'à profiter du triomphe qu'il
procurerait au cardinalat. C'est en effet ce qui arriva.

Comme j'étais, Dieu merci, à trois cents lieues de cette scène, je ne
rapporterai point ce qui se passa. Les ducs furent tondus à leur
ordinaire\,; mais ceux qui étaient du conseil de régence cessèrent d'y
entrer ainsi que le chancelier. Ce qu'il y eut de plaisant, fut que les
maréchaux de France qui en étaient en sortirent aussi, dont pas un
jusqu'alors n'avait imaginé de disputer rien aux cardinaux. C'est ce
dont Dubois fut ravi. Il prit cette fausse démarche aux cheveux pour
persuader au régent que cette prétention commune contre les cardinaux
était uniquement prétexte, et réellement cabale contre lui et contre son
gouvernement. Ce courrier me fut donc dépêché pour m'instruire de cet
événement, et la lettre que le cardinal Dubois m'écrivit là-dessus ne
peut s'extraire et mérite d'être rapportée ici tout entière, pour y
remarquer tout l'art de ce venimeux serpent.

«\,Paris, 2 mars 1722.

«\,On vous aura rendu compte, sans doute, monsieur, des mouvements qu'il
y a eus dans le conseil de régence à l'occasion de la place que Mgr le
duc d'Orléans a permis à M. le cardinal de Rohan d'y prendre.\,» (Dubois
l'y prit en même temps, mais il n'en dit rien par modestie.) «\,S'il ne
s'était agi que de la préséance entre les cardinaux et les ducs et
pairs, je n'aurais pas été fâché que vous eussiez été absent pendant
cette contestation. Mais comme cette difficulté, dans cette occasion,
n'a été qu'un prétexte qu'on n'a pas même dissimulé longtemps, et que
c'est une cabale formée et ménagée par un homme (le duc de Noailles) qui
n'a pas su se conserver votre estime, et qui ne paraît pas avoir de
bonnes intentions pour Son Altesse Royale, et qu'elle tend à troubler
son gouvernement et à renverser ses ouvrages (lui Dubois), je n'ai
jamais regretté plus sincèrement votre absence, ni souhaité avec plus de
passion le secours de votre indignation et de votre courage. Je vous
conjure, monsieur, de vous en tenir à cette idée jusqu'à ce que vous
puissiez voir les choses par vous-même, et que vous soyez à portée de
signaler votre zèle pour ce que vous croirez le mériter davantage pour
le bien de l'État, l'union des deux couronnes, le soutien de la dernière
liaison qui a été faite, et le maintien de Mgr le duc d'Orléans.\,»
(C'est ce qu'il entendait ci-dessus par détruire son ouvrage, mais qu'il
sentait bien plus véritablement de lui-même.) «\,Je puis y ajouter et
pour votre propre défense\,; car je vous assure que, si on venait à bout
de ce que l'on trame, je suis persuadé que, si vous n'étiez pas la
première victime, vous seriez la seconde. Ces orages me conduisent bien
naturellement à penser à votre retour. Tout me persuade que votre
présence serait nécessaire encore pendant quelque temps à Madrid. Le
seul moyen de vous laisser sur cela la liberté que vous souhaiterez,
serait que vous pussiez y accréditer un peu M. de Chavigny, ce que l'on
me dit n'être pas facile par les mauvaises impressions qu'on a voulu
donner à Madrid contre lui. Cependant il ne les mérite pas, et jusqu'à
ce que Son Altesse Royale envoie en Espagne un ambassadeur, il n'y a que
lui qui puisse exécuter les ordres que vous laisserez en partant.
Tâchez, monsieur, de le mettre en état d'être écouté et d'avoir les
accès nécessaires, et disposez après cet arrangement du temps de votre
retour à votre gré. Je suis également combattu entre les grands services
que vous pouvez rendre à Madrid et les secours que vous pouvez donner
ici à Son Altesse Royale, et, si j'ose me mettre en ligne de compte,
j'ajouterai entre l'impatience que j'ai de cultiver les nouvelles bontés
que vous m'avez marquées, et vous donner, s'il m'est possible, de
nouvelles preuves, monsieur, de mon respect et de mon attachement.\,»

Les fausses lueurs de cette lettre y éclatent de toutes parts. Un groupe
de tant de seigneurs à abattre sous ses pieds fit peur au cardinal
Dubois, malgré le bouclier du cardinal de Rohan dont il avait su se
couvrir. Il connaissait la faiblesse de son maître, sa légèreté sur les
rangs, qu'il s'y moquait de la justice des raisons, qu'il ne se décidait
que par le besoin et le nombre qui lui faisait toujours peur\,; que
douze ou quinze des premiers seigneurs, par le caractère des uns et les
établissements des autres, pèseraient dans sa balance plus que deux
cardinaux, dont l'un ne pouvait rien, et l'autre n'était que ce qu'il
l'avait fait. Le poids du chancelier l'embarrassait encore. Il fallut
donc étouffer dans M. le duc d'Orléans la crainte d'offenser tant de
gens à la fois, presque tous si considérables, par une frayeur plus
grande d'une cabale formée contre lui pour renverser son gouvernement.

Il avait appris en Angleterre l'art de faire paraître une conjuration
prête à éclater, pour tirer du parlement plus de subsides, et
l'entretien de plus de troupes qu'il n'était disposé d'en accorder.
Dubois érigea de même en cabale pour renverser le gouvernement du régent
et le régent lui-même, la chose du monde la plus simple, la plus
naturelle, qui tenait le moins par aucun soin aux affaires et au
gouvernement, et qu'il n'avait tenu qu'à Dubois d'empêcher de naître, en
s'abstenant d'introduire dans le conseil de régence l'inutile et
dangereuse chimère du cardinalat. Mais cette exclusion entraînait
nécessairement celle de sa personne. Quoique le conseil de régence fût
devenu un néant, il y voulait primer et dominer, et il ne put avoir la
patience d'attendre le peu de mois qui restaient jusqu'à la majorité,
qui dissolvait à l'instant le conseil de régence par elle-même, pour en
composer de pareils à ceux du feu roi, où il n'aurait mis que des gens à
son choix et d'état à n'avoir rien à lui disputer, comme il fit en effet
dans la suite. Mais il fut si impatient qu'il fallut tout forcer, et
après si effrayé du nombre et de l'unanimité des résistances à lui céder
qu'il fallut inventer la cabale, le danger du prince, le péril de
l'État, les revêtir de toutes les couleurs qu'il imagina de leur
donner\,; ne laisser approcher du régent, pendant ce court mouvement de
simple préséance, que des gens bien instruits à augmenter sa frayeur.

Ce fut pour la porter au dernier degré qu'il y ajouta le dessein formé
de cette prétendue formidable cabale de renvoyer l'infante, de rompre la
nouvelle union formée avec l'Espagne, et, pour en persuader mieux le
régent, me dépêcher un courrier là-dessus pour m'en informer, et me
charger d'en rendre compte au roi et à la reine d'Espagne, et de
n'oublier rien pour les rassurer là-dessus. C'est ce qui me fut si
expressément ordonné de faire par une autre lettre en deux mots du
cardinal Dubois, qu'il m'écrivit à part de celle que je viens de copier,
et de faire sur-le-champ, dans l'instant que j'aurais lu les lettres que
ce courrier m'apportait, en quelque lieu que Leurs Majestés Catholiques
pussent être, sans différer d'un instant.

Un peu de réflexion dans M. le duc d'Orléans eût fait disparaître ce
fantôme aussitôt que présenté. Quel besoin avait cette cabale prétendue
d'une dispute de préséance pour éclater, et d'une dispute qu'elle ne
pouvait prévoir, puisque le cardinal de Rohan voyait depuis si longtemps
un conseil de régence, sans qu'il eût été question pour lui d'y entrer,
et que Dubois, qui en était et de nécessité par ses emplois, avait cessé
d'y entrer depuis le moment qu'il avait reçu des mains du roi la calotte
rouge\,? Quelle puissance avait acquise cette cabale depuis que celle du
duc et de la duchesse du Maine, où tant de gens étaient entrés et à
Paris et dans les provinces, appuyées de l'argent, du nom, de la
protection d'Espagne, des menées de son ambassadeur, homme de beaucoup
d'esprit et de sens, et de toute la passion du cardinal Albéroni, maître
alors de l'Espagne, depuis, dis-je, l'avortement de ces complots si
promptement et si facilement détruits\,? S'élevant de nouveau contre le
régent et en même temps contre l'Espagne, son plus fort appui l'autre
fois par les droits de la naissance de Philippe V, de quelle puissance
étrangère aurait-elle pu s'appuyer\,? Ce ne pouvait être de la seule à
portée de la secourir. L'Angleterre était trop intimement liée alors
avec la France, et trouvait trop son intérêt au gouvernement de M. le
duc d'Orléans et au crédit démesuré du cardinal Dubois, son pensionnaire
et son esclave, pour ne les pas soutenir de toute sa puissance, bien
loin d'aider à la troubler. Qui est l'homme ayant les moindres notions,
qui pût se flatter que la Hollande, et par elle-même et par sa
dépendance de l'Angleterre, y eût voulu contribuer d'un seul florin ni
d'un seul soldat\,? Enfin, la cabale aurait-elle mis son espérance dans
le roi de Sardaigne, si connu pour n'y pouvoir compter qu'en lui livrant
les provinces de sa bienséance, et encore avec plus que la juste crainte
d'en être abandonnée dès qu'il s'en serait saisi de manière à les
conserver\,? Et de plus, comment la cabale s'y serait-elle pu prendre
pour parvenir à les lui livrer\,? tous les autres princes {[}étant{]}
trop faibles ou trop éloignés pour y pouvoir penser. Enfin par les seuls
Français\,? Le temps était passé de la puissance des seigneurs et des
gouverneurs des provinces, des unions et des partis. La Bretagne en
était un exemple récent, et que tout ce qui s'était passé à la
découverte des complots du duc et de la duchesse du Maine, de Cellamare
et de leurs adhérents, dont les promptes et faciles suites étaient des
leçons qui ne pouvaient pas être si promptement effacées.

Cette chimère aurait donc pu à peine faire impression sur un enfant.
Mais tout était sûr à l'impétuosité d'un fourbe qui avait su infatuer
son maître au point de pouvoir tout entre prendre, d'être seul redouté,
de l'avoir enfermé sans accès à tout ce qui n'était pas vendu à ses
volontés et à son langage, et qui, appuyé sur la paresse de penser et de
réfléchir de son maître, qui avait plus tôt fait de l'en croire sur tout
que d'y songer un moment {[}dans{]} le tourbillon qui emportait ses
journées, le rendait aussi hardi et aussi heureux à entraîner un prince
de tant d'esprit et de lumières que s'il en eût été entièrement
dépourvu.

Ce fantôme d'une cabale si dangereuse, outre l'usage présent qu'en fit
le cardinal Dubois, en renfermait un autre plus éloigné. Je l'ai
tacitement annoncé ici en deux endroits, dont le dernier a été la
tentative de remettre le duc de Berwick dans les bonnes grâces de Leurs
Majestés Catholiques. Il est temps de le déclarer, simplement pour
l'intelligence, sans avancer le récit du succès, éloigné encore de
quelques mois. Dubois, toujours en défiance de la facilité de son
maître, qu'il ne voulait que pour soi, méditait de s'affranchir de toute
crainte, et d'éloigner de lui, comme que ce fût, quiconque avait eu part
à sa familiarité en affaires et à sa confiance, et qu'il craignait
qu'ils n'en reprissent avec lui, soit par ancien goût, amitié, habitude,
soit par poids ou par hardiesse. Plusieurs de ceux-là, il les faisait
entrer dans la prétendue cabale, et subsidiairement tous ceux qu'il lui
convenait d'écarter. Il craignait sur tous le duc de Noailles par son
esprit, sa souplesse, le goût et la familiarité que M. le duc d'Orléans
avait eus pour lui, et dont il avait encore des restes\,; le poids du
chancelier, sur qui Noailles avait tout pouvoir\,; celui du maréchal de
Villeroy, même du maréchal d'Huxelles, qui imposaient au régent, quoique
sans goût ni amitié, mais qui avait le même effet\,; les divers tenants
de ceux-là, tels que Canillac, Nocé et d'autres. C'était de ceux-là dont
il voulait s'affranchir en les ruinant dans l'esprit de M. le duc
d'Orléans, et préparer leur perte, pour y procéder au premier moment
qu'il y verrait jour. Le Blanc, tout son homme qu'il fût, était trop
avant dans la confiance et les choses les plus secrètes de M. le duc
d'Orléans, et Belle-Ile, son compersonnier\footnote{Mot employé souvent
  par Saint-Simon dans le sens d'\emph{associé}.}, tous deux ses favoris
en apparence et ses consulteurs de tous les soirs, étaient secrètement
sur la liste de ses proscriptions. Le duc de Berwick et moi n'y étions
pas moins. L'Anglais avait trop acquis sur le régent par le sacrifice si
plein et si prompt qu'il lui avait fait de tout ce qu'il devait au roi
d'Espagne\,; et, pour ce qui me regardait, mes anciennes, intimes et
continuelles liaisons d'affaire et d'amitié dans les temps les plus
critiques, du plus entier abandon, et les plus éloignées de toute
apparence d'utilité pour moi, même de plus qu'apparences les plus
contraires, me rendaient d'autant plus odieux à ce solipse \footnote{Le
  mot \emph{solipse} désignait des ambitieux et des égoïstes. L'esprit
  de parti a appliqué le nom de \emph{solipses} aux jésuites. Ant.
  Arnauld indique nettement le sens de ce mot dans sa \emph{Morale
  pratique} (t. III, p.~86)\,: «\,On sait que c'est votre caractère (il
  parle aux jésuites) de vous porter avec ardeur à faire le bien, pourvu
  que vous le fassiez seuls\,; et, si vous voulez être sincères, vous
  avouerez que l'un de vos pères, auteur du livre intitulé
  \emph{Monarchia solipsorum}, vous connaissait bien.\,» L'auteur de ce
  livre est, selon quelques critiques, le P. Gasp. Schopp (Scioppius),
  ou, selon d'autres, le P. Inchoffer. Bayle, à l'article
  \emph{Inchoffer} (remarque C), lui attribue positivement cet ouvrage.
  Il ajoute qu'il courut une prétendue lettre d'Innocent XII à
  l'empereur, l'an 1696, dans laquelle le pape nomme la société des
  jésuites \emph{monarchiam Monopantorum}. Sur quoi le P. Papebroch a
  fait cette réflexion\,: «\,Forsitan quasi µÒvoip£nta(soli omnia) a
  velint esse et aestimari jesuitae, scilicet alludendo ad scomma
  satirici cc cujusdam commenti, quo scripsit anonymus aliquis
  \emph{Monarchiam solipsorum}, veluti innuere volens quod societas soli
  sibi arrogare nitatur omnia.\,»}, que M. le duc d'Orléans ne pouvait
oublier que mes conseils ne lui avaient pas été inutiles dans toutes les
différentes situations de sa vie, et que Dubois avait souvent éprouvé ma
hardiesse et ma liberté. D'essayer de faire peur de Berwick et de moi à
M. le duc d'Orléans, il le sentait impraticable.

Pour se défaire de Berwick, il lui destinait l'ambassade d'Espagne.
C'était pour cela que j'avais reçu des ordres si précis et si réitérés
de ne rien oublier pour lui réconcilier Leurs Majestés Catholiques. On
verra que le mauvais succès que j'y eus ne le rebuta pas. Pour moi,
j'ignore comment il avait projeté de s'y prendre. On verra aussi comment
je le servis sur les deux toits, en voyant avec indignation le règne
absolu de la bête, et mon inutilité auprès de M. le duc d'Orléans. Tel
était le plan du cardinal Dubois, que nous lui verrons effectuer dans la
suite. Revenons maintenant à sa lettre à moi qu'on vient de voir, et aux
artifices dont il tâcha de me circonvenir par lui-même, et par une autre
lettre plus étendue que la sienne, qui m'arriva par le même courrier.

Le cardinal Dubois commence sa lettre par une vérité pour donner plus de
créance à ce qui la devait suivre\,; mais vérité à qui il donne une
étendue qu'elle n'avait pas. Il fut bien aise, en effet, de mon absence,
lors de l'exécution d'un dessein contre lequel il ne se dissimulait pas
que je ne me fusse roidi de toutes mes forces, qui l'eussent sûrement au
moins embarrassé. Mais quoi qu'il en puisse dire, mon absence le
soulagea encore plus dans la création et la présentation hardie de ce
fantôme de cabale si dangereuse dont il osa effrayer le régent. J'étais
le seul des intéressés qu'il n'aurait pu en rendre suspect, et à qui il
n'eût pu fermer l'oreille de son maître. Il ne pouvait douter de l'usage
que j'en aurais fait\,; et j'ose dire que j'ai lieu de douter qu'il eût
osé produire ce fantôme en ma présence. Après avoir légèrement glissé
là-dessus en commençant, il essaye de détourner mes yeux de son odieuse
préséance, sur laquelle il ne fait qu'un saut léger, sans y appuyer le
moins du monde, et compte m'infatuer de la prétendue cabale, à la faveur
de ma haine ouverte et sans aucun ménagement pour celui qu'il lui
convient d'en faire le chef.

Noailles s'était si indignement conduit dans l'affaire du bonnet, et
avec tant de perfidie, qu'il était tout naturel de penser qu'il n'était
touché de la préséance des cardinaux que par prétexte. C'en fut un en
effet, qui, dans lui et dans quelques autres peu touchés de leur
dignité, mais beaucoup de ce qu'ils jugeaient être leur fortune, et à
quelque prix que ce fût, ne regardait en rien ni le régent ni son
gouvernement, mais la personne unique du cardinal Dubois, puisque après
sa mort et l'élévation de Fréjus \footnote{Fleury, évêque de Fréjus.} à
l'autorité et à la pourpre, les mêmes ducs et maréchaux, si blessés en
apparence de la préséance des cardinaux, n'oublièrent rien pour être
admis dans le conseil du roi où le cardinal de Fleury avait la première
place. Dubois n'oublia donc rien pour surprendre ma haine, et par elle
me persuader de ce qu'il se proposait que je crusse de la cabale que
Noailles avait formée contre l'État et le régent, me persuader que de
son succès dépendait ma perte personnelle, me piquer par le dessein de
renvoyer l'infante, que je venais pour ainsi dire d'envoyer en France,
et rompre l'union que mon ambassade venait d'achever de consolider\,;
enfin {[}pour{]} m'éblouir, {[}pour{]} m'entraîner par le concours de
ces différentes passions qu'il tâchait d'exciter ou d'augmenter en
moi\,; {[}pour{]} me faire oublier la préséance, et {[}pour{]} me
précipiter à agir selon ce qu'il se proposait. Pour y mieux réussir, il
se contenta d'un récit dont l'artifice emprunta tant qu'il put l'air
simple et modeste, la brèveté de s'en tenir au nécessaire et de passer
tout de suite à autre matière, mais qui ne lui tenait pas moins au
coeur. Parmi les louanges et les désirs de ma présence qu'il sut mêler à
son récit pour me capter et m'aveugler par tous les endroits possibles,
il mourait de peur de mon retour. Que ne craignent pas les tyrans, et
plus encore ceux qui ne sont pas couronnés\,? Pour allier ses prétendus
souhaits de son retour et les raisons dont il tâchait de les rendre
vraisemblables avec son véritable désir de me tenir éloigné, il se jette
sur les services importants que je puis rendre en Espagne\,; il les
balance avec ceux que le régent devait attendre uniquement de moi auprès
de lui, se joue avec cet artifice, et met mon retour à un prix qu'il
était si persuadé que je ne pourrais atteindre, que la vérité perce
malgré lui, et le force de l'avouer en convenant de toute la difficulté
que je rencontrerais à établir Chavigny, déshonoré en Espagne comme
partout, dans la confiance nécessaire à y servir utilement pendant qu'il
n'y aurait point d'ambassadeur. Cet artifice était pitoyable, mais les
fripons se trompent eux-mêmes à force de vouloir tromper les autres.

Tout était fait en Espagne\,; réconciliations, traités, mariages, et
tout s'était fait indépendamment du ministère de Laullez et de
Maulevrier. Il n'y avait plus rien à faire qu'à suivre et entretenir les
traités et l'union\,; et pouvait-il me croire assez stupide pour ignorer
sur les lieux qu'il y eût d'autres négociations à ménager, et que ce qui
restait à faire, qui était uniquement cet entretien d'union et de
traités, était uniquement dans la main des deux seuls ministres des deux
couronnes, et tout à fait hors de la sphère de leurs ambassadeurs\,? Et
Dubois savait de plus combien Grimaldo y était porté et l'avait toujours
été d'inclination et de maxime\,; et quand bien même, ce qui n'était
pas, un ambassadeur y eût été nécessaire, l'homme à y envoyer existait,
sur quiconque le choix pût tomber, et devait se faire incontinent, si ma
présence auprès du régent était aussi nécessaire et aussi désirée par
Dubois qu'il voulait me le faire accroire. Ces panneaux se trouvèrent
aussi trop légers pour arrêter mes pieds\,; mais comme il n'avait osé
leur donner toute l'étendue qu'il voulait, pour les mieux cacher, voici
le supplément qu'il imagina.

On a vu ci-dessus, il n'y a pas longtemps, que Le Blanc, secrétaire
d'État de la guerre, était devenu l'homme à tout faire du cardinal
Dubois, et par lui Belle-Ile, son ami intime, et que tous les soirs le
cardinal Dubois finissait sa journée chez lui entre eux deux seuls. Ce
sont deux hommes que j'aurai lieu d'expliquer dans la suite, et qui
méritent bien de l'être. On a vu aussi que Belle-Ile était de mes amis,
et tout à fait à portée de tout avec moi. Je trouvai dans les paquets
que le courrier m'apporta une longue lettre de lui, qui était la
paraphrase de celle du cardinal Dubois dont je viens de parler. Mais
Belle-Ile, qui ne voulait pas apparemment que je m'y méprisse, la
commença par me dire qu'il m'avait écrit le matin même, dans le paquet
de M\textsuperscript{me} de Saint-Simon, sans détail, pour ne pas
confier des choses si importantes à la poste\,; mais que la conversation
qu'il avait eue le soir avec le cardinal Dubois et Le Blanc, où il avait
été résolu de m'envoyer un courrier exprès, l'engageait à m'écrire celle
qu'il m'envoyait par cette voie sûre\,; et de là entre dans le détail de
ce qui s'est passé sur la préséance des cardinaux et la sortie du
conseil de ceux qui s'en tinrent blessés\,; de là entre dans celui de la
cabale qui veut culbuter M. le duc d'Orléans et son gouvernement\,;
l'arrange, l'organise, nomme le duc de Noailles et Canillac comme les
vrais chefs, et le maréchal de Villeroy, qui se persuade l'être\,;
l'entraînement du chancelier par Noailles\,; distingue ceux qui, de
bonne foi, ne pensent pas plus loin que la préséance, d'avec ceux qui de
tout temps, effectivement plus que suspects, ont pris feu sur une
apparence de rang qui ne les touche guère, mais qui, ennemis de tout
temps du régent, ou dépités de se voir si reculés de toutes parts au
gouvernement, n'ont de vues, de desseins et de projets que de le
renverser. Il appelle leur absence du conseil lever le masque, et un
attentat authentique à l'autorité du roi\,; dit que le régent en est
extrêmement piqué, et résolu à une fermeté inébranlable. Il prête toutes
sortes de discours qui marquent les desseins pour la majorité. Il vient
après à me dire qu'il comprend l'embarras où je me serais trouvé, dans
cette cause commune, avec mon attachement pour M. le duc d'Orléans\,; à
la joie de mon absence dans cette conjoncture\,; et à me conjurer d'être
en garde sur tout ce qui me sera mandé\,; de ne pas douter de la réalité
et du danger de la cabale, et de ne pas prendre un périlleux change
là-dessus. Il se jette ensuite sur des arrangements pris avec le
parlement pour éloigner à la majorité M. le duc d'Orléans du
gouvernement et pour renvoyer l'infante, et sur des discours imprudents
qui ne le cachent pas\,; enfin, qu'on saura bien faire entendre au roi
d'Espagne combien la continuation de son union personnelle avec M. le
duc d'Orléans, brouillé sans retour avec tous les grands et tous les
personnages du royaume, lui serait nuisible, et combien il lui importe
de se détacher de l'un et de s'attacher les autres.

De là Belle-Ile vient à l'importance de prévenir incontinent le roi
d'Espagne là-dessus, à quoi je ne saurais marquer trop de zèle et
employer trop de dextérité\,; surtout lui bien peindre les chefs de la
cabale et ses acteurs principaux, les lui nommer en confiance, surtout
les plus opposés à tout ce qui s'est fait pour l'infante, et les plus
capables de faire jouer toutes sortes de ressorts pour rompre son
mariage et pour la renvoyer\,; enfin, lui vanter la fermeté de M. le duc
d'Orléans en cette occasion\,; lui persuader qu'il est plus en état que
jamais d'être utile à Leurs Majestés Catholiques et d'exécuter tout ce
qu'{[}elles{]} pourront désirer. Il m'exhorte avec louange d'employer
tout mon bien-dire et tout mon savoir-faire pour cimenter et affermir de
plus en plus l'union et le crédit de M. le duc d'Orléans avec le roi et
la reine d'Espagne, et me dit franchement que c'est après mûre
délibération que le cardinal me dépêche ce courrier. Belle-Ile ajoute
ensuite que le chef de cette cabale est le chef des jansénistes, duquel
l'objet est également la destruction de la religion, de M. le duc
d'Orléans, de ses serviteurs, dont je suis l'un des plus intimes\,;
qu'ainsi tout doit m'engager à concourir dans les vues du cardinal
Dubois pour faire avorter leurs desseins, et pour éloigner à jamais du
gouvernement gens qui me sont personnellement opposés. Il me dit ensuite
que son attachement pour moi, et la part qu'il a eue à me raccommoder
avec le cardinal Dubois en dernier lieu, l'engagent à me parler comme il
fait, lequel, malgré toute l'opposition qu'il sait que j'ai pour la
préséance des cardinaux, m'avait extrêmement désiré présent dans cette
occasion importante, parce qu'il s'y agit de toute l'autorité de M. le
duc d'Orléans, à laquelle j'ai, dit-il, plus de part que personne.
Belle-Ile me pique d'honneur sur le soin et le plaisir que je prendrai à
prévenir le roi d'Espagne sur ce venin qu'on voudrait répandre dans son
esprit contre M. le duc d'Orléans, et me dit qu'après un service si
important à Son Altesse Royale et à moi-même, et après que j'aurai
accrédité et mis au fait Chavigny, rien ne sera plus pressé que mon
retour. Il finit par m'assurer qu'il est convaincu que lorsque j'aurai
vu les choses de près je n'y prendrais jamais de part et serai ravi
d'avoir été absent\,; enfin des compliments.

On n'a qu'à jeter les yeux sur la lettre que j'ai transcrite ici du
cardinal Dubois et sur celle de Belle-Ile, pour ne pas douter que toutes
deux sont de la même main. Ils n'ont pas même l'art de le cacher, et
l'avouent de plus, comme la lettre de Belle-Ile étant le fruit de sa
conférence avec le cardinal Dubois et Le Blanc, où il fut résolu de me
dépêcher un courrier. La seconde ne fait qu'étendre la première, essayer
plus à découvert de piquer davantage ma haine et mon intérêt personnel
en si grand péril, selon eux, m'exciter à ne rien épargner auprès du roi
d'Espagne, selon leurs vues, c'est-à-dire de perdre à fond auprès de
Leurs Majestés Catholiques ces prétendus entrepreneurs de renvoyer
l'infante, pour leur ôter à jamais toute ressource de ce côté-là, et me
bien infatuer de cette cabale aussi dangereuse pour moi que pour M. le
duc d'Orléans\,; pour m'ôter par cette muraille toute impression et tout
sentiment sur la préséance, et me livrer en aveugle au cardinal Dubois.

La lettre de Belle-Ile est si grossièrement la même du cardinal Dubois,
mais plus expliquée, plus étendue, plus appuyant sur la cabale, et
appuyant plus librement le poinçon pour m'irriter, m'effrayer, et me
fournir de quoi piquer le roi et la reine d'Espagne, que ce n'est plus
la peine d'en faire l'analyse après avoir fait celle du cardinal Dubois.
Deux articles suppléés à celle de Dubois méritent seulement qu'on s'y
arrête. Tous deux passent, comme chat sur braise, sur la préséance et
sur l'entrée des cardinaux dans le conseil de régence. Ils sentaient
l'inutilité de cette entrée et celle de tenter de me la faire trouver
bonne et leur préséance supportable. Mais ce qui me parut admirable fut
la qualification de Belle-Ile, dictée par Dubois, à la sortie du conseil
de régence de ceux qui s'en trouvèrent blessés, qu'il traite de levée de
masque et d'attentat authentique à l'autorité du roi. Mais que peuvent
faire de plus respectueux les plus grands et les premiers d'un royaume
que de se retirer dans une pareille occasion, et d'accommoder par cette
modeste soumission ce qu'ils se doivent à eux-mêmes avec le respect
qu'ils rendent même à l'injustice qu'on leur fait\,?

Les maîtres des requêtes ne s'asseyent point au conseil des parties, où
le roi n'est jamais, où son fauteuil est vide, où le chancelier, les
conseillers d'État et les simples intendants des finances sont assis
dans des fauteuils\,; beaucoup moins le sont-ils au conseil des finances
ou au conseil de dépêches \footnote{On a indiqué la signification de ces
  mots \emph{conseil des parties, conseil des finances, conseil de
  dépêches}, dans une note ajoutée au t. Ier des \emph{Mémoires de
  Saint-Simon}, p.~445.}, quand quelque affaire extraordinaire en amène
quelqu'un rapporter devant le roi, où le maître des requêtes rapporteur
est seul debout. Ils furent pourtant un an sans que pas un d'eux voulût
venir rapporter au conseil de régence, où le fauteuil du roi était vide,
et où M. le duc d'Orléans présidait assis comme nous tous sur un siège
ployant, parce que ces messieurs y voulaient rapporter assis, ou bien
que ceux du conseil qui n'étaient pas officiers de la couronne ou
conseillers d'État se tinssent debout comme eux. L'impertinence était
évidente. Elle fut pourtant soufferte plus d'un an sans que personne se
soit avisé de la traiter d'attentat ni de complot contre l'autorité du
roi. C'est qu'ils n'étaient pas ducs, mais seulement maîtres des
requêtes. Et M. le duc d'Orléans leur fut bien obligé quand, à
l'instigation de M. d'Aguesseau, devenu chancelier, ils voulurent bien y
venir rapporter debout, sans plus prétendre y faire lever personne.

Un autre endroit que je trouvai risible est celui où Belle-Ile, après
avoir déployé son éloquence sur les mouvements, les discours, les moyens
et les desseins prétendus de la cabale, en produit le chef comme l'étant
aussi des jansénistes, qui voulaient également renverser la religion et
l'État. Mais à qui le cardinal et son secrétaire, car Belle-Ile l'était
en cette occasion, à qui contaient-ils ces fagots\,? Ce chef était,
selon eux, le duc de Noailles, et en apparence le maréchal de Villeroy,
lequel, en bas et ignorant courtisan qu'il fut toute sa vie, avait
épousé la haine du feu roi et de M\textsuperscript{me} de Maintenon
contre tout ce qu'il avait plu aux jésuites, etc., de faire passer pour
jansénistes, et pour tout ce qui n'adorait pas la constitution
\emph{Unigenitus}, et qui, depuis la mort du roi, se signalait sans
cesse contre tout ce qui était soupçonné, bien ou mal à propos, de
n'être pas moliniste ou constitutionnaire.

À l'égard du duc de Noailles, il y avait longtemps qu'il s'était fait un
mérite de sacrifier son oncle à ses ennemis. Les Rohan, les Bissy, les
autres chefs n'avaient point de client plus rampant et plus souple, ni
les jésuites de serviteur plus empressé et plus respectueux. Ce n'était
pas un homme qui pût être retenu par aucun sentiment autre que ses vues
de fortune, quoique la sienne fût assez complète. Mais l'ambitieux
cesse-t-il jamais d'y travailler\,? Je ne pouvais oublier qu'il avait
empêché les appels de tous les corps et de tous les tribunaux, tout
prêts à suivre les écoles, les chapitres et les congrégations qui
venaient d'appeler. Et on a vu en son lieu que je l'appris de M. le duc
d'Orléans, même que l'avis de ce neveu du cardinal de Noailles avait
arrêté le consentement qu'il était prêt d'y donner. Je ne pus donc voir
sur quoi pouvait porter cette imputation, ni ce que le jansénisme
pouvait avoir de commun avec la respectueuse et toute simple retraite de
gens qui ne pouvaient moins, dont aucun ne passait pour janséniste ni
pour opposé à la constitution, et dont quelques-uns avaient épousé le
molinisme et la constitution jusqu'au fanatisme. Cette sottise était
bonne tout au plus à mander au P. Daubenton, digne fabricateur de la
constitution, comme on l'a vu ici en son lieu, et jésuite prêt à
s'évanouir au nom de jansénisme, pour faire, par son canal, valoir cette
calomnie, destituée de toute sorte de plus légère apparence, auprès du
roi d'Espagne, qu'il avait si bien monté sur ces deux points. Enfin
Belle-Ile finissait, comme Dubois, par faire dépendre mon retour de
l'accréditement et de la confiance que je procurerais à Chavigny pour
gérer les affaires en attendant un ambassadeur, ce qu'ils sentaient bien
qui me serait impossible.

Je lus et relus bien mes lettres. J'y fis tout seul mes réflexions, et
je pris mon parti aussitôt. Ce fut de n'être pas la dupe du cardinal
Dubois, et de ne pas hasarder la réputation que j'ose dire que j'avais
acquise à la cour d'Espagne, en y donnant un fantôme de cabale pour une
réalité, dont le faux et le néant ne tarderait pas à me démentir, et qui
n'était fabriquée que pour coiffer le seul régent et le persuader du
sérieux de la chose par les ordres qu'on se hâtait de me donner
là-dessus. En même temps, je ne voulus pas m'exposer à manquer dans
cette conjoncture à l'extérieur des ordres si exprès du cardinal Dubois,
et je résolus de lui complaire en forçant les barricades de Balsaïm, où
il ne serait pas derrière moi pour écouter ce que je dirais.

J'allai donc d'abord trouver Grimaldo, qui travaillait dans sa
cavachuela. Je lui expliquai fort simplement ce qui s'était passé au
conseil de régence, sans lui dire un seul mot de cabale\,; mais
seulement qu'on craignait que cette désertion de tant de gens
considérables ne fît plus d'impression qu'elle ne devait sur l'esprit de
Leurs Majestés Catholiques\,; que c'était pour y obvier que j'avais reçu
cette nouvelle par un courrier qu'on m'avait dépêché aussitôt\,; qu'il
venait d'arriver, et qu'il m'apportait un ordre fort précis d'en aller
rendre compte à Leurs Majestés Catholiques, dans le moment que j'aurais
lu mes lettres, en quelque lieu qu'elles pussent être. Grimaldo se mit à
rire de cet empressement. Il me dit que cette affaire serait fort
indifférente à Leurs Majestés Catholiques, et qu'elles n'avaient aucune
intention de se mêler de l'intérieur de la cour de France, ni des
disputes qui pouvaient y arriver\,; qu'ainsi son avis était que je
remisse à en parler à Leurs Majestés Catholiques à leur retour, qui
serait dans quatre jours\,; qu'elles n'étaient parties que de ce matin
avec la très courte suite que je savais\,; que la défense d'aller à
Balsaïm était sans aucune exception, et que sûrement le roi serait fâché
et embarrassé de m'y voir. Je répondis à Grimaldo que je pensais tout
comme lui, mais qu'il connaissait l'homme à qui j'avais affaire, qui, de
plus, m'en ferait une de mon retardement, et l'imputerait au
mécontentement qu'il ne pouvait douter que je n'eusse de cette
préséance\,; et là-dessus je lui donnai à lire la lettre particulière
qui ne contenait que l'ordre exprès de rendre compte à Leurs Majestés
Catholiques, quelque part qu'elles fussent, dans le moment de l'arrivée
du courrier.

Grimaldo lut et relut cette courte lettre. Il me dit, en me la rendant,
qu'il sentait tout mon embarras et ne savait que me dire. Je m'espaçai
quelques moments sur le cardinal Dubois avec lui, et je le priai de
faire en sorte que le roi voulût bien entrer avec bonté pour moi dans la
situation où je me trouvais. Je le priai de lui écrire en ce sens pour
le disposer à me recevoir, parce que, comme que ce fût, j'étais résolu
d'aller le lendemain à Balsaïm\,; et je lui avouai que j'aimais mieux
risquer à déplaire au roi d'Espagne pour un moment, et sur chose sans
conséquence, que de me perdre dans ma cour, où le cardinal Dubois me
guettait sans cesse pour y parvenir, à qui il ne fallait pas fournir le
plus petit prétexte. Grimaldo, haussant les épaules, convint que j'avais
raison, et me dit qu'il avait heureusement à dépêcher tout présentement
un courrier au roi d'Espagne, par lequel il l'avertirait de mon voyage,
de sa cause, de mes raisons personnelles, et n'oublierait rien pour le
disposer à me recevoir sans chagrin. Je remerciai beaucoup Grimaldo et
revins chez moi disposer mon voyage, envoyer sur-le-champ des relais et
des mules de selle, à quoi Sartine, qui connaissait le chemin, m'aida
fort.

Je partis donc le lendemain avant six heures du matin, et je fus bien
étonné de trouver la porte de Madrid fermée, le côté de la clef en
dehors, et celui qui la gardait à cent pas hors cette porte, en sorte
qu'il fallut faire escalader la muraille, heureusement assez basse, par
un laquais, qui eut encore grand'peine à se faire ouvrir par le portier,
qui vint enfin nous faire sortir de la ville. Le comte de Lorges, mon
second fils, l'abbé de Saint-Simon, son frère, et le major de son
régiment vinrent avec moi. Cette corvée ne tenta point le comte de
Céreste.

\hypertarget{chapitre-xi.}{%
\chapter{CHAPITRE XI.}\label{chapitre-xi.}}

1722

~

{\textsc{Voyage à Balsaïm.}} {\textsc{- Fraîche réception tôt
réchauffée.}} {\textsc{- Audience à Balsaïm.}} {\textsc{- Je couche à
Ségovie.}} {\textsc{- Ségovie.}} {\textsc{- Cordelier de M. de
Chalais.}} {\textsc{- Je dîne à Balsaïm, et suis Leurs Majestés
Catholiques à la Granja.}} {\textsc{- Comment la Granja devenue
Saint-Ildephonse.}} {\textsc{- Saint-Ildephonse.}} {\textsc{- Superbe et
riche chartreuse.}} {\textsc{- Manufactures de Ségovie fort tombées.}}
{\textsc{- Je réponds aux lettres du cardinal Dubois et de Belle-Ile.}}
{\textsc{- Bruit ridicule que fait courir mon voyage de Balsaïm.}}
{\textsc{- Hardiesse étrange de Leurs Majestés Catholiques allant et
venant de Balsaïm.}} {\textsc{- Autres lettres curieuses du cardinal
Dubois à moi.}} {\textsc{- Vif sentiment du duc d'Arcos sur la préséance
des cardinaux au conseil de régence.}} {\textsc{- Cardinaux, chanoines
de Tolède, mêlés avec les autres chanoines en leur rang d'ancienneté
entre eux.}}

~

Nous arrivâmes sur le midi au vrai pied de la Guadarama, après avoir
déjà assez longtemps monté et fait à peu près comme de Paris à Senlis.
Nos voitures y demeurèrent, et nous montâmes nos mules. Je ne vis jamais
un si beau chemin ni si effrayant en voiture. On affronte un mur de roc
d'une effroyable hauteur par un chemin uni, mais étroit, qui va en
zigzag, assez droit, avec peu de roideur, en sorte qu'en parlant un peu
haut on peut causer un peu avec des gens au-dessous et au-dessus de soi,
qui sont à près d'une lieue les uns des autres. La montagne et le chemin
étaient couverts de neige fort épaisse\,; tout était rempli d'arbres
entre les rochers, dont les branches, toutes chargées de frimas,
n'étaient que les plus belles grappes et les plus brillantes. Toute
cette singularité faisait dans son affreux quelque chose de charmant. On
parvient ainsi à la cime, à force de contours. Le terre-plein n'en est
pas long, et la descente de l'autre côté est bien plus aisée et plus
courte, à la moitié de laquelle on découvre Balsaïm, dans une vallée
étroite, placé à une distance assez grande du pied de la montagne.
Balsaïm, bâti par les Maures, et brûlé par malice sous Charles II, qui y
allait trop souvent, et point réparé depuis, est le reste d'un grand et
beau château. Ce reste est fort petit, avec un jardin médiocre et rien
autour qui s'aperçoive. Nous fûmes mettre pied à terre à un reste de
bâtiment bas, qui était du château tout contre, mais sans communication
à couvert.

On nous fit entrer dans l'office du duc del Arco, où ses sommeliers
travaillaient, qui nous quittèrent civilement la place, après nous avoir
présenté des chaises de paille auprès du feu, dont nous avions grand
besoin, et nous avoir offert de nous rafraîchir, dont nous les
remerciâmes. Il n'était guère que quatre heures après midi, et nous y
attendîmes une heure et demie le retour de Leurs Majestés de la Granja,
qui est devenu Saint-Ildephonse. La cuisine du duc del Arco était à
côté. Au-dessus, il y avait quatre petites cellules pour les trois
seigneurs qui étaient du voyage, et pour Valouse, et, en bas, près de la
cuisine, une espèce de petite salle longue et étroite, où le duc del
Arco tenait sa table. Avertis de l'arrivée de Leurs Majestés, nous
allâmes les voir descendre de carrosse. Grimaldo les avait averties\,;
elles s'attendaient à me trouver. La réception du roi fut froide, pour
ne pas dire rechignée, sans dire une parole\,; celle de la reine,
embarrassée mais plus humaine. Elle me dit quelques mots, mais leur
suite me fit la meilleure réception du monde. Le roi et la reine
montèrent un degré de bois, entre deux bâtons pour garde-fous, où on ne
pouvait aller qu'un à un. Il était en dehors appuyé contre le pignon, et
en l'air comme la montée d'un paysan dans son village. Au haut il y
avait un petit carré à tenir cinq ou six personnes pressées, d'où on
entrait directement dans la chambre du roi et de la reine, sans rien de
plus qu'une garde-robe au delà, et vis-à-vis la porte de la chambre de
Leurs Majestés, en repassant le petit carré, une autre chambre toute
seule. C'est là tout le logement avec quelques trous au-dessus\,; et
dessous, au rez-de-chaussée, la cuisine et l'office de Leurs Majestés.

Arrivé dans ce carré où Leurs Majestés s'étaient arrêtées pour
m'attendre, je leur demandai la permission de les suivre et d'avoir
l'honneur de leur dire un mot. Toute la suite demeura dans ce carré et
dans la chambre joignante, et je me trouvai en tiers avec Leurs Majestés
Catholiques, qui me menèrent dans la fenêtre, parce que le jour baissait
fort. «\,Qu'y a-t-il, monsieur, donc de si pressé\,?» me dit le roi
sèchement. Je commençai par des excuses d'être venu sans permission sur
les ordres les plus exprès que j'en avais reçus pour leur rendre compte
de ce qui s'était passé au conseil de régence, que je leur expliquai
fort simplement, sans dire un mot de cabale et seulement pour les
informer de la raison qui avait fait sortir les ducs, le chancelier et
les maréchaux de France du conseil, qui n'était autre que la préséance
des cardinaux et qui était chose toute simple et sans nulle sorte de
conséquence pour la tranquillité et pour les affaires, mais dont
l'attention de M. le duc d'Orléans à les informer des moindres
événements avait voulu que je fusse le premier à le leur apprendre tout
tel qu'il était de la part du roi et de la sienne, par son respect et
son attachement pour Leurs Majestés. Le roi, toujours sec, me répondit
que cela ne valait pas la peine d'être venu, que cela eût été aussi bon
à Madrid. Je regardai la reine, et, m'adressant à elle, je lui dis qu'on
était bien empoché quand on avait affaire au cardinal Dubois, et sur un
fait encore où le moindre retardement m'eût fait une affaire, parce
qu'il était persuadé, sans doute avec raison, que je ne serais pas plus
content de ce qui s'était passé que ceux qui en avaient quitté le
conseil. La reine se mit à rire, me dit qu'elle le comprenait bien, et,
s'adressant au roi, ajouta qu'il n'y avait pas grand mal, sinon ma
peine, et tout de suite me fit quelques questions sur ce qui s'était
passé, mais courtes et simples. Le roi se radoucit et me dit qu'il ne se
souciait point de ces choses-là, qu'il ne voulait point se mêler de
l'intérieur de la cour de France, encore moins des disputes et des
querelles. Je finis ce propos par leur présenter la relation que j'avais
reçue de tout ce qui s'était passé à l'arrivée de l'infante, et des
fêtes qui l'avaient suivie, ce qui plut fort au roi et le remit de belle
humeur. Je leur en dis les principaux articles. Ils furent fort
sensibles à l'appareil de la réception, et surtout de ce que le roi
était sorti assez loin de Paris au-devant d'elle.

Après quelques propos là-dessus qui achevèrent de les égayer, la reine
proposa au roi de faire entrer ce qui était dehors pour leur donner part
de ces nouvelles, et me dit de les appeler. Tous entrèrent. La reine
leur répéta ce que je venais de lui en dire, et ajouta qu'il fallait
lire la relation. Puis, s'interrompant, elle eut la bonté de se mettre
en peine de mon gîte et de ce qui m'accompagnait. Le duc del Arco
m'offrit un lit et à souper, mais en peine de lits et de chambres pour
ce que j'avais amené. Tout cela causa force compliments et de la
meilleure grâce du monde de la part du duc del Arco, même du marquis de
Santa-Cruz, où le roi entra un peu et la reine avec vivacité. Je ne
voulais incommoder personne, et, pour ce qui était avec moi, il n'y
avait nul moyen de les gîter. Je proposai donc qu'il nous fût permis
d'aller coucher à Ségovie, et cela finit par là. Le duc del Arco voulait
nous donner à souper, mais je fis si bien que je m'en exemptai. Il me
fit donner une berline à quatre personnes pour nous y mener. La reine,
pendant cette conclusion, avait parlé bas au roi, et me dit après qu'ils
ne me laissaient aller qu'à condition de revenir tous le lendemain dîner
chez le duc del Arco, et de les suivre après dîner à la Granja, où le
roi me voulait montrer les bâtiments et les jardins qu'il y faisait
faire. Le roi ajouta quelque chose du sien avec un air content et
ouvert, et la reine les plus gracieuses bontés. Valouse nous voulait
donner son lit et sa chambre, et le comte de San-Estevan de Gormaz fit
aussi très bien. Mais il n'y eut rien d'égal à la politesse et à
l'empressement du duc del Arco. Nous prîmes congé et nous partîmes pour
Ségovie, distant de Balsaïm comme de la place de l'ancienne porte de la
Conférence\footnote{La porte de la Conférence se trouvait à l'extrémité
  occidentale de la terrasse du jardin des Tuileries qui longe les
  quais. Elle avait reçu ce nom parce qu'il y avait eu, dans ce lieu,
  des conférences entre les députés de Henri IV et ceux de la ville de
  Paris en 1592. Elle fut détruite en 1739.} à Sèvres, par une plaine
fort unie qu'on gagne après avoir un peu monté fort doucement. On nous
fournit aussi des gens à cheval avec des flambeaux.

Ceux qui, venus avec moi, y allèrent à cheval, précédèrent l'arrivée de
la berline. Nous les trouvâmes dans la rue, n'ayant pu se faire ouvrir
aucune maison. On les renvoyait par les fenêtres comme des bandits dont
on avait peur. Malgré l'équipage nous eûmes le même sort partout où nous
frappâmes, en sorte que pendant près d'une heure nous eûmes toute la
peur de coucher sur ce pavé sans souper. Enfin nous fîmes tant de bruit
à la porte d'une grande maison, qu'après avoir longtemps prié et menacé
par la fenêtre, bravé par notre nombre et par la livrée du roi qui nous
menait, ces gens comprirent enfin que nous disions vrai et que nous
n'étions pas des bandits. Ce fut un grand contentement que de voir
ouvrir cette porte. On nous fit monter et montrer des chambres et des
lits. C'était déjà beaucoup. Mais quand on parla de souper, point de
pain ni de viande, ni de tout l'accompagnement. Le repas en chemin avait
été fort léger, et nous n'avions pas compté d'avoir rien à porter pour
le soir. Il fallut bien du temps et de l'industrie pour surmonter la
mauvaise humeur de gens qui nous avaient reçus malgré eux, qui
trouvaient fort mauvais que nous troublassions leur repos, et pour
ramasser de quoi souper et l'apprêter à l'heure qu'il était, et dans un
pays où les cabarets et les hôtelleries sont inconnus. Néanmoins avec de
la patience nous soupâmes et nous couchâmes pas trop mal.

La curiosité m'éveilla le lendemain de bonne heure. Mes fenêtres me
présentèrent tout près ce superbe aqueduc construit par les Romains\,;
qui paraît d'une seule pierre, et qui, sans s'être encore démenti, porte
l'eau de la montagne voisine par toute la ville, qui est grande, bien
bâtie, avec des places, de belles églises, et des rues moins étroites,
moins obscures, moins tortues que je ne les ai vues dans les autres
villes d'Espagne, excepté Madrid et Valladolid. En approchant tout
contre l'aqueduc, qui est d'une grande hauteur, et plus que les plus
hauts qu'on voit autour de Versailles et de Marly, et sans arcades que
quelques portes pour la communication nécessaire, on est surpris de
l'énormité des pierres dont il est bâti et de la presque
\emph{imperceptibilité} de leurs séparations, où il ne paraît pas trace
d'aucune espèce de liaison. Je ne pouvais me lasser de considérer ce
merveilleux édifice que tant de siècles ont respecté.

La ville est au fond d'une plaine de quatre ou cinq lieues, belle, unie,
fertile et appuyée à la montagne, qui est là fort haute et fort
escarpée. À l'autre bout, du côté de la plaine, est le château de
Ségovie qui, comme Vincennes, est un palais, mais vaste et beau, embelli
et presque tout rebâti par Charles-Quint, et une prison de criminels
d'État. Il a, chose rare en Espagne, une belle et vaste cour, et les
appartements des rois sont admirables par leur plain-pied, leur étendue,
leur structure et les ornements sages, magnifiques et très bien
exécutés, dont ils sont enrichis. Leur dorure épaisse, foncée, brillante
comme si elle venait d'être faite, les plafonds avec leurs peintures
exquises, et l'ordonnance des ornements, tant dés murailles, des portes,
des fenêtres et des plafonds, me rappela tout à fait ceux de
Fontainebleau, ne balançant pas toutefois à préférer ceux de Ségovie. La
principale vue donne sur une petite rivière qui serpente tout proche, et
sur toute cette magnifique plaine bordée de montagnes inégales et de
quelques hauteurs.

Au plus haut du donjon, qui a sept étages, et qui est tout contre le
château, dans la même cour, était ce cordelier fameux que H. de Chalais
amena à Paris avec tant de précaution et de mystère, dont il a été ici
parlé en son temps, et qu'il ramena bien escorté à Ségovie, d'où il
n'était pas sorti depuis. J'appris de celui qui avait soin des
prisonniers, car il y en avait dans ce donjon plusieurs autres, que ce
cordelier était insatiable de romans, et guère moins de vin et de
viande\,; qu'il jurait et blasphémait sans cesse, et qu'il passait sa
vie à hurler de fureur ou à chanter pour se divertir. Il criait à
l'injustice contre la cour d'Espagne, mais sans jamais rien laisser
entendre de la cause de sa prison\,; qu'il avait tenté bien des fois de
se sauver, ce qui l'avait fait mettre au plus haut étage\,; qu'il ne
s'accoutumait point à sa prison, et qu'il était comme désespéré. Ce
concierge me parut excédé d'un tel hôte, dont l'impiété et le goût de la
débauche lui faisait horreur, et qu'il lui donnait plus de soin et de
peine que tous les autres prisonniers ensemble. Je fis ce que je pus
pour le lorgner à sa fenêtre, mais je ne pus l'y apercevoir. Il y avait
du moins une belle vue, et on lui donnait les livres qu'il demandait, et
tant de vin et de nourriture qu'il voulait, mais on ne lui laissait voir
personne ni rien de quoi il pût s'aider pour écrire. La matinée se passa
en ces curiosités, et nous partîmes pour Balsaïm par les mêmes voitures
qui nous avaient amenés la veille.

Nous descendîmes chez le duc del Arco vers une heure après midi, et
bientôt après on y servit un fort splendide dîner et fort bon, quoique
presque tout à l'espagnole. Le marquis de Santa-Cruz, le comte de
San-Estevan de Gormaz et Valouse dînèrent avec nous, et le duc del Arco
en fit les honneurs le plus noblement et le plus poliment du monde. On
fut longtemps à table, de fort bons vins, de très bon café, bon appétit,
bons propos. Ces seigneurs espagnols étaient ravis de me voir donner sur
leurs mets de bonne grâce. Peu de moments après dîner, ils nous menèrent
au bas de ce petit escalier de bois, sur lequel, tôt après, nous vîmes
paraître le roi et la reine et monter en carrosse, dont je fus fort
accueilli. Le roi me parut tout accoutumé à me voir à Balsaïm, et lui et
la reine se faire un plaisir de me faire voir leurs ouvrages à la
Granja. Ce mot espagnol veut dire une grange. C'en était une en effet,
et tout esseulée, qui appartenait aux moines de l'Escurial, à une lieue,
de celles d'autour de Paris, de Balsaïm. De cette maison, le roi y avait
été faire des chasses. La solitude lui en avait plu\,; la facilité d'y
avoir de l'eau en abondance et beaucoup de chasse l'avait déterminé à
acheter de ces moines ce qu'ils y avaient, et à y bâtir la retraite dans
laquelle il méditait de se jeter dès que le prince des Asturies
commencerait à pouvoir porter la couronne, qu'il lui voulait remettre,
comme il l'exécuta depuis. Mais ce dessein, alors ni de longtemps après,
ne fut connu que de la reine et du P. Daubenton, qui tous deux en
mouraient de peur, et n'oubliaient aucune adresse pour l'en détourner
doucement.

Le duc del Arco et le marquis de Santa-Cruz se partagèrent pour nous
mener. Le chemin coulait le long de la vallée, traversant souvent de
beaux ruisseaux et des ravins, et se rapprochant du pied de la chaîne de
ces hautes montagnes que nous avions traversées en venant de Madrid.
Plus on approche de la Granja, plus la vallée s'étrécit. Tout y était
ouvert comme en plein champ, et nous arrivâmes par le côté. La cage de
la maison était faite, distribuée, couverte\,; on en était aux dedans,
mais encore en maçons\,; et la plupart des jardins étaient faits, mais
grossièrement encore. La chapelle, qui est au flanc par où nous
arrivâmes, était à peine sortie de terre, comme une fort grande église,
qui devait être accompagnée de logements pour le chapitre et les gens de
la chapelle, qui n'étaient pas commencés. Cette chapelle était déjà
fondée pour une riche collégiale\footnote{Les collégiales étaient des
  chapitres de clercs réguliers ou séculiers, réunis dans une église
  sans siège épiscopal. Le chapitre de Sainte-Geneviève, rétabli en
  1863, est une véritable collégiale.}. Son titre était destiné de
Saint-Ildephonse, sous l'invocation duquel elle devait être consacrée\,;
et c'est ce qui a donné le nom à ce vaste palais. Avant d'aller plus
loin, il faut donner l'idée de ce lieu, que la retraite de Philippe V,
pendant sa courte abdication, a rendu célèbre.

Il serait difficile de trouver une situation plus ingrate, ni d'avoir
mieux réussi à la rendre triste, pour ne pas dire affreuse, par le choix
de l'emplacement du château. Ce château est un long et vaste bâtiment
qui est double, presque au bas d'une pente fort douce et fort unie
partout, qui, en s'élevant peu à peu, arrive jusqu'au bord de la plaine
de Ségovie, que cette hauteur presque insensible dérobe au château, qui
l'aurait vue en plein, avec la ville de Ségovie, son aqueduc et le
couronnement de ses montagnes, s'il avait été placé vingt ou vingt-cinq
toises plus haut, ce qui aurait formé à ses pieds une terrasse telle
qu'on aurait voulu, dominant sur les jardins, mais avec une douceur très
agréable, et qui n'aurait que plus invité à y descendre, au lieu que
l'emplacement où il est ne lui laisse que la vue et le plain-pied de la
vallée, et masque entièrement la vue de tous les étages du double par
cette hauteur qui s'élève si doucement jusqu'à la plaine, et qu'on
semble toucher des fenêtres avec la main. Le rez-de-chaussée me parut
destiné en salle des gardes, pièces à tenir des tables, et quelques
logements. Tout le premier étage pour les appartements de Leurs Majestés
Catholiques, distribués en belles pièces, de belles mais de diverses
grandeurs, dont le double aveuglé, comme je viens de l'expliquer, en
commodités et en garde-robes, logements de caméristes et de petits
domestiques du roi les plus nécessaires, avec, au bout du flanc, des
tribunes percées sur la chapelle, mais point encore faites. Nous ne
vîmes pas l'étage de dessus. La cage de l'escalier vaste et agréable
dans sa forme, au milieu du bâtiment, et à droite et à gauche de jolis
escaliers dérobés par lesquels nous passâmes.

À l'autre flanc opposé à la chapelle était un bâtiment double, qui ne
débordait pas le château en avant, placé en potence à l'égard du
château, qui s'étendait assez loin en le débordant par derrière, avec
des cours et de grands bâtiments intérieurs. Il était bâti pour servir
de commun pour les équipages, les cuisines et les offices, et pour loger
les seigneurs et toute la suite de la cour. Du flanc du château à ce
bâtiment, il n'y avait au plus que trois toises. J'en témoignai ma
surprise à la reine, qui me répondit qu'ils voulaient entendre du bruit
et voir aller et venir. L'intention secrète, que je ne pouvais
comprendre alors, était de désennuyer leur retraite par entendre et voir
du monde auprès d'eux. Les jardins allaient jusqu'au pied de la
montagne, dont l'espace était court, et sur la fin montaient un peu dans
la racine de la montagne\,; mais, à droite et à gauche, ils s'étendaient
déjà fort loin, et ils ont été depuis fort allongés de part et d'autre,
remplissant toujours toute la largeur de la vallée.

Ces jardins assez unis pour donner de vastes plains-pieds, et point
assez pour manquer des agréments qu'on tire des terrains inégaux.
Beaucoup d'allées d'arbres plantés tous grands, comme le feu roi faisait
à Marly, des terrasses peu élevées, revêtues et bordées de gazons, des
bosquets sortant encore peu de terre, des bassins, des canaux, des
pièces d'eau sans nombre, de toutes les formes, des cascades, des
nappes, des effets d'eau de toutes les sortes, de la plus belle eau et
de la meilleure à boire, et dans la plus prodigieuse abondance, et des
jets d'eau partout en gerbe et de toutes les formes, dont plusieurs, qui
étaient seuls, jetaient gros comme la cuisse, le double de la hauteur de
ce beau jet d'eau de Saint-Cloud, qui faisait la jalousie du feu roi, et
que tout le monde admire avec raison. Les plus fâcheux inconvénients ont
quelquefois leur utilité\,: cette longue chaîne de montagnes qui bornait
les jardins, qui s'élevait presque jusqu'aux nues, toute de rochers
parsemés d'arbres mal semés, couverte de neige presque toute l'année,
dont la cime ne fondait jamais, dont la hideuse beauté faisait tout
l'aspect du château, et dont un mulet rapportait de la glace et de la
neige en moins de deux heures, aller et venir, cette chaîne de montagnes
fourmillait des plus grosses sources, à toutes hauteurs, et fournissait
sans cesse toutes les eaux des jardins, en telle quantité qu'on voulait,
et pour telle hauteur où on désirait les faire jaillir. Ces jardins
avaient déjà quantité d'orangers, et ils étaient aussi ornés de vases de
métal et de tous les plus précieux marbres, et les plus ornés
d'excellents bas-reliefs et des plus belles statues de bronze et de
divers marbres que le sont les jardins de Versailles et de Marly, avec
des ateliers dans les jardins mêmes, où travaillaient sans cesse les
meilleurs maîtres de France et d'Italie qu'on avait pu attirer. Mais ces
jardins, véritablement charmants par la variété et le bon goût,
l'agrément, la fraîcheur, la facilité, l'étendue des promenades, avaient
un inconvénient bien fâcheux\,; c'est que tout le terrain de ces jardins
n'était que roche vive et dure, avec une légère croûte de terre par
dessus, de manière qu'il avait fallu employer le pic et très
ordinairement le secours de la poudre pour écaver tous les bassins et
pièces d'eau, les trous de tous les arbres, les tranchées des
palissades, et tous les terrains des massifs, en emporter les pièces à
dos de mulet et rapporter de même la bonne terre de loin pour en remplir
toutes les excavations ou on avait planté, et qu'il en fallait user de
même pour toutes les nouvelles plantations et pièces d'eau qu'on y
voudrait ajouter dans la suite en allongeant les jardins {[}aux{]} deux
extrémités. Voilà pour la cherté qui ne pouvait être que fort grande\,;
mais le pis est que quelque profondeur qu'on eût pu donner aux endroits
destinés à planter, les racines des arbres, dont leur vie et leur beauté
dépend, s'étendent toujours tout autour d'elles, et il y en a qui
percent à pic. Dès qu'elles se trouveront arrêtées par le roc, ce qui y
touchera séchera bientôt, la terre rapportée se consumera et ne pourra
plus fournir autant de sucs qu'il en faudra pour la nourriture des
racines et des arbres, qui dépériront et mourront en peu d'années.

Je ne vis aucun projet de cour ni d'entrée. Ils me dirent que les deux
extrémités du jardin et le bas de cette petite hauteur, qui monte à la
plaine de Ségovie, se fermeraient le long des jardins avec un pavillon
pour porte à chacun des deux bouts\,; qu'on entrerait toujours par où
nous étions venus, et qu'un pavé étroit en rue ferait toute la
séparation entre le château et les jardins. La plus proche maison
d'autour du château était une méchante maison de garde-chasse, qui en
était à une demi-lieue, et nulle autre que beaucoup plus loin, ce qui
charmait le roi d'Espagne en effet, dont la reine faisait aussi le
semblant. J'eus l'honneur, et ce qui était venu avec moi, de suivre
Leurs Majestés Catholiques partout, qui se promenèrent d'abord dans la
maison, et après dans les jardins toute la journée sans se reposer,
qu'elles prirent plaisir à me faire voir, et moi à leur faire ma cour en
admirant tant de beautés et tant de miracles d'eaux, qui en effet sont
uniques. La conversation se soutint pendant toute la promenade, où ces
seigneurs espagnols et Valouse entraient fort aussi, {[}et{]} où la
reine était toujours charmante, Le roi s'y mêla quelquefois. Ils firent
l'honneur de parler aussi à ceux qui étaient avec moi, et s'amusèrent
fort à donner leurs ordres et à se faire rendre compte par ceux qui
avaient le principal soin des bâtiments, jardins, etc., sous la
direction du duc del Arco, gouverneur du lieu, par lequel tout pas soit.

Dans cette promenade, le courrier qui m'était arrivé se présenta sur
leur passage. Je l'avais amené pour l'expédier de Ségovie, qui est
presque sur le chemin de Madrid à Bayonne. C'était Bannière, si fort en
réputation par le nombre et la promptitude de ses courses, et qui était
fort connu du roi et de la reine par toutes celles qu'il avait faites à
l'occasion des deux mariages, tellement que Leurs Majestés l'appelèrent
et lui parlèrent assez longtemps. J'appris qu'au revers de cette chaîne
de montagnes, et presque vis-à-vis des jardins qu'elle bornait, était
une superbe et vaste chartreuse, de plus de cent mille écus de rentes,
dont le principal revenu était des laines fines de leurs immenses
troupeaux. Leurs Majestés Catholiques y allaient quelquefois sans y
coucher que rarement. Elles et leur suite y étaient parfaitement
défrayées. Mais la chère ne pouvait être bonne dans un pays sans poisson
et presque sans légumes. À l'égard de leurs laines, j'en vis les
manufactures à Ségovie, qui me parurent peu de chose et fort tombées de
leur ancienne réputation. La fin du jour approchant termina le voyage.

En arrivant à Balsaïm, le roi m'ordonna de monter et de le suivre dans
sa chambre. Là, en tiers avec lui et la reine, ils me demandèrent si
j'étais pressé de renvoyer Bannière, et que, si je pouvais attendre des
paquets dont ils avaient en vie de le charger, je leur ferais plaisir.
Je répondis qu'il n'y avait rien de pressé, mais que, quand je le
serais, leur ordre me suffirait pour différer aussi longtemps qu'il leur
plairait. Je pris congé d'eux, et fis après mes remerciements à ces
seigneurs, surtout au duc del Arco, dont les soins, les prévenances, la
politesse n'avaient rien oublié. Il me fournit la même voiture et des
montures de la veille pour aller coucher à Ségovie, qui le lendemain
nous menèrent au pied de la montagne, où nous trouvâmes nos mules pour
la passer, et nos voitures où nous les avions laissées, dans lesquelles
nous arrivâmes le même soir à Madrid. Le lendemain j'allai conter à
Grimaldo ce qui s'était passé en mon voyage, et je n'oubliai pas de lui
dire combien j'étais charmé de toutes les merveilles que j'avais vues,
mais combien aussi j'étais étonné de la situation et de la position. Il
me répondit qu'il s'était bien douté que tout s'y passerait comme je
venais de lui raconter, et qu'il était fort aise que le roi, malgré le
froid de l'abord et l'indifférence sur ce qui m'amenait, dit voulu me
faire voir ses ouvrages, et que la promenade l'eût remis dans son état
ordinaire avec moi. Grimaldo ne me dissimula point ce qu'il pensait du
choix du lieu et de sa disposition, et nous causâmes longtemps ensemble.

Il fallut après rendre compte de mon voyage au cardinal Dubois, et
répondre à sa lettre. Je lui mandai nettement que j'étais d'autant plus
aise de mon éloignement de Paris, que, s'y j'y avais été, rien ne
m'aurait empêché de sortir du conseil\,; qu'à l'égard de la cabale et de
ses desseins, je me flattais qu'ils ne feraient ni peur ni mal à M. le
duc d'Orléans ni à son gouvernement\,; que, dans le compte que j'avais
rendu à Leurs Majestés Catholiques, elles m'avaient paru ne faire aucun
cas de cet événement et y être fort indifférentes\,; qu'il ne devait
avoir aucune inquiétude des impressions que Leurs Majestés Catholiques
en pourraient prendre, non plus que M. de Grimaldo. Pour allonger une
réponse si courte, je me jetai sur la hardiesse que j'avais prise de
forcer les barricades de Balsaïm, sur les beautés et les singularités de
Saint-Ildephonse et sur le retardement du renvoi de Bannière, que le roi
d'Espagne m'avait demandé, ce qui faisait que je ne lui écrivais que par
l'ordinaire. Enfin je finissais par des compliments sur ses lumières à
prévenir, et sa sagesse et son habileté à détruire tous les complots
dont il m'avait écrit. Je tâchai d'ajuster cette fin, en sorte qu'il ne
crût pas que je me moquais de lui, comme néanmoins je faisais en effet.
J'écrivis à Belle-Ile en même sens, parce que je prévis bien qu'il ne
serait pas le maître de cacher sa réponse. J'y ajoutai ce que je n'avais
pas voulu dire si directement au cardinal sur Chavigny, qu'il n'y avait
que lui-même qui pût, par une conduite suivie, faire revenir les esprits
en sa faveur, et que cette entreprise serait pour moi de trop longue
haleine, à laquelle Maulevrier, après mon départ, pourrait le servir.
C'était encore me moquer d'eux et leur faire comprendre que je ne serais
pas la dupe de leurs prétextes de me retenir en Espagne. Je crus bien
que ces réponses ne plairaient pas au cardinal Dubois\,; mais il n'était
pas en moi de ployer misérablement sous sa préséance, ni de me ruiner
sans ressource pour me stabilier\footnote{Établir d'une manière fixe et
  définitive.} en Espagne à son gré.

Le 13 mars, Leurs Majestés Catholiques revinrent de Balsaïm au Retiro.
Le voyage si brusque que j'y avais fait sur l'arrivée d'un courrier, et
malgré les défenses si précises à qui que ce fût, sans exception, d'y
aller, et la journée que j'avais passée tout entière auprès d'elles à
Saint-Ildephonse, joint à la façon pleine de grâces et de bontés
constantes et si distinguées, avec lesquelles j'étais toujours traité
depuis que j'étais en Espagne, firent courir le bruit le plus ridicule,
qui prit assez de créance subite pour me surprendre beaucoup. Il se
répandit donc que je quittais le caractère d'ambassadeur de France, et
que j'allais être déclaré premier ministre d'Espagne. Le peuple, à qui
ma dépense apparemment avait plu, et à qui personne de chez moi n'avait
donné aucun sujet de plainte, se mit à crier après moi dans les rues, à
me le dire, à témoigner sa joie et jusque du dedans des boutiques. Il
s'en assembla même autour de ma maison avec les mêmes témoignages que je
dissipai le plus civilement et le plus promptement que je pus, en les
assurant qu'il n'en était rien, et que je partais incessamment pour
retourner en France.

Je ne puis pas dire que je fusse insensible à ces marques d'estime et
d'affection\,; mais ce qui me toucha véritablement fut ce qui m'arriva
avec le marquis de Montalègre, sommelier du corps. Je le rencontrai à
l'entrée des appartements du Retiro. Il accourut à moi, m'embrassa et me
dit qu'il était transporté de joie de ce que je leur demeurais et de ce
que j'allais être premier ministre. Je le remerciai de cette marque si
grande de l'honneur de son estime et de son amitié, et je l'assurai en
même temps qu'il n'en était rien, et que je partirais dans fort peu de
jours pour retourner en France. J'eus à peine achevé, que Montalègre,
jetant sur moi des yeux de dépit et de colère, tourna tout court, et me
quitta sans révérence et sans me répondre un seul mot. Beaucoup de
seigneurs m'en firent des compliments, à qui je répondis de même.

Je réparerai ici, quoiqu'en lieu déplacé, l'oubli d'une bagatelle, mais
singulière, sur le chemin dans la montagne, pour aller à Balsaïm\,:
c'est que le roi et la reine d'Espagne faisaient toujours ces voyages
dans un grand carrosse de la reine à sept glaces, en sorte qu'en passant
la montagne par le même chemin que je fis, et qui était l'unique, il n'y
avait pas deux doigts de marge entre leurs roues et le précipice,
presque tout le long du chemin, et qu'en plusieurs endroits les roues
portaient à faux et en l'air, tantôt cent, tantôt deux cents pas,
quelquefois davantage. Des paysans en grand nombre étaient commandés
pour tenir le carrosse par de longues et fréquentes courroies, qui se
relayaient en marchant à travers les rochers avec toutes les peines et
les périls qui se peuvent imaginer pour la voiture et pour eux-mêmes. On
n'avait rien fait à ce chemin pour le rendre plus praticable, et le roi
et la reine n'en avaient pas la moindre peur. Les femmes qui la
suivaient en mouraient, quoique dans des voitures exprès fort étroites.
Pour les hommes de la suite, ils passaient sur des mules. Je n'ajouterai
point de réflexions à un usage si surprenant.

Les lettres que le courrier Bannière m'avait apportées étaient du 2
mars. Un courrier, dépêché par le duc d'Ossone, qui était encore à
Paris, m'en apporta une du cardinal Dubois, du 8 mars, dont le singulier
entortillement me divertit et me confirma dans le parti que j'avais
pris. J'avais reçu, il y avait déjà quelque temps, mes lettres de
récréance\footnote{Les lettres de récréance étaient celles qu'un
  souverain envoyait à son ambassadeur pour les remettre au prince dont
  il prenait congé.} et tout ce qu'il fallait pour prendre congé. Le
cardinal, qui mourait de peur que je ne m'en servisse, n'en avait pas
moins de me la laisser apercevoir. Sa lettre fut donc un tissu de
\emph{oui} et de \emph{non}, de l'importance des services à rendre en
Espagne pour consolider l'union, du désir de mon retour pour des raisons
non moins pressantes pour le service de l'État et de M. le duc
d'Orléans, toujours la condition de ne partir point sans avoir accrédité
Chavigny jusqu'à la confiance, toutefois ne vouloir point entreprendre
sur ma liberté, et de tout laisser à ma prudence. Je compris par le
tissu de cette lettre que, pour peu que j'en attendisse d'autres, elles
se trouveraient d'un style décisif, qui se trouveraient appuyées de
celles de M. le duc d'Orléans, que le cardinal Dubois faisait telles que
bon lui semblait. Je pris donc mon parti sur cette lettre de n'en point
attendre d'autres, et, dès le lendemain que je l'eus reçue, je pris jour
pour mon audience de congé.

Depuis que je parlais de partir, il n'y avait rien que la reine et même
le roi ne fissent pour me retenir, ni amitiés et regrets que toute leur
cour ne me fît la grâce de me témoigner. J'avouerai même que ce ne fut
pas sans peine que je quittai un pays où je n'avais trouvé que des
fleurs et des fruits, et auquel je tenais et je tiendrai toujours par
l'estime et la reconnaissance. Je pressai une infinité de visites pour
mes adieux, afin de ne manquer à personne. Dans celle que je fis au duc
et à la duchesse d'Arcos, desquels j'avais reçu les politesses les plus
marquées, et que je voyais assez souvent, le duc d'Arcos me conjura de
ne rentrer point au conseil de régence et de ne céder point aux
cardinaux. Je le suppliai de n'avoir pas assez mauvaise opinion de moi
pour en être en peine, et qu'il pouvait être sûr que je ne mollirais pas
là-dessus. Quelque rang que les cardinaux eussent peu à peu usurpé en
Espagne, on ne l'y supportait qu'avec dépit\,; et depuis que l'affaire
du conseil de régence fut devenue publique, je ne vis, ni grands
surtout, ni même gens de qualité qui n'en fussent indignés, et qui ne
s'en expliquassent très fortement, nonobstant le silence et l'entière
réserve que je m'étais imposée là-dessus.

Mais à propos de cardinaux et de tout leur grand rang en Espagne, que
j'y laissai plus supposé qu'usité, je ne dois pas oublier de rapporter
une curiosité que j'eus sur eux. Le cardinal Borgia était, comme je l'ai
dit, chanoine de Tolède. Il prit le temps du voyage de Balsaïm pour y
aller passer quelques jours. La singularité d'y avoir vu deux évêques
portant les marques de leur dignité, confondus avec les chanoines sans
la moindre distinction d'avec eux, m'inspira le désir d'être précisément
informé de ce qui s'y passerait avec un cardinal. Je priai donc Pecquet
d'aller à Tolède le même jour que je me rendis à Balsaïm, d'y demeurer
autant que le cardinal Borgia, et d'avoir la patience de le suivre pas à
pas. Il l'exécuta dans toute l'exactitude, et il me rapporta que le
cardinal Borgia s'était trouvé assidûment au choeur, en rochet et camail
violet, à cause du carême, en calotte et bonnet rouge, ayant des
chanoines au-dessous et au-dessus de lui, sans chaire vide entre eux et
lui, mais ayant devant lui un tapis de la largeur de sa stalle, jeté sur
l'appui régnant le long des stalles, faisant le dossier des stalles
d'au-dessous, et sur ce tapis un carreau pour s'appuyer dessus, à ses
pieds un carreau pour s'y mettre à genoux, le tout de velours rouge avec
un peu d'or, qui est le traitement qu'ont les grands d'Espagne dans les
églises, et qu'on a vu ci-dessus que mon second fils et moi eûmes aussi,
mais à la tête du choeur. Le cardinal Borgia se découvrit et se couvrit
toujours comme les autres chanoines, en même temps qu'eux. Pendant qu'il
y fut, il y eut une procession du chapitre, que Pecquet ne manqua pas de
voir et d'observer. Il y vit le cardinal Borgia marcher en son rang
d'ancienneté de chanoine, qui allaient, en file, deux à deux, comme dans
toutes les processions, un chanoine marchant à côté de lui, comme chacun
des autres, et des chanoines devant et derrière lui sans aucune distance
que la même gardée entre eux, sans que la queue du cardinal Borgia fût
portée par personne, qui n'était pas plus longue que celles des autres
chanoines, et sans avoir près de lui ni écuyer ni aumônier. Voilà de ces
choses qu'il faut avoir vues pour les croire, avec la superbe
cardinalesque et les immenses usurpations de ces prétendus égaux des
rois.

\hypertarget{chapitre-xii.}{%
\chapter{CHAPITRE XII.}\label{chapitre-xii.}}

1722

~

{\textsc{Mon audience de congé.}} {\textsc{- Singularité unique de celle
de la princesse des Asturies.}} {\textsc{- Maulevrier reçoit enfin le
collier de l'ordre de la Toison d'or, mais avec un dégoût insigne.}}
{\textsc{- Je pars de Madrid.}} {\textsc{- Alcala de Henarez.}}
{\textsc{- Guadalajara.}} {\textsc{- Agreda.}} {\textsc{- Pampelune.}}
{\textsc{- Roncevaux.}} {\textsc{- Bayonne.}} {\textsc{- Réponse
curieuse du cardinal Dubois et de Belle-Ile.}} {\textsc{- Trois
courriers me sont dépêchés.}} {\textsc{- Je me détourne pour passer à
Marmande, où le duc de Berwick était venu m'attendre de Montauban, où il
commandait en Guyenne.}} {\textsc{- Bordeaux.}} {\textsc{- Blaye.}}
{\textsc{- Loches.}} {\textsc{- Chastres.}} {\textsc{- Belle-Ile vient à
Chastres me proposer, de la part du cardinal Dubois, le dépouillement du
duc de Noailles, et me presse d'y entrer, auquel je m'oppose.}}
{\textsc{- Je vais au Palais-Royal.}} {\textsc{- Long entretien entre le
régent, le cardinal Dubois et moi.}} {\textsc{- Friponnerie sur la
restitution aux jésuites du confessionnal du roi.}} {\textsc{- Je me
démets de ma pairie à mon fils aîné, et lui fais présent des pierreries
du portrait du roi d'Espagne.}} {\textsc{- Je visite pendant la tenue du
premier conseil de régence tous ceux qui en étaient sortis, et vais à
Fresnes voir le chancelier exilé.}}

~

Je pris le 21 {[}mars{]} mon audience de congé, en cérémonie, du roi et
de la reine séparément. Je fus de nouveau surpris de la dignité, de la
justesse et du ménagement des expressions du roi, comme je l'avais été
en ma première audience, où je lui fis la demande de l'infante, et les
remerciements de M. le duc d'Orléans sur le mariage de madame sa fille.
Je reçus aussi beaucoup de marques de bonté personnelles et de regrets
de mon départ de Sa Majesté Catholique, et surtout de la reine\,;
beaucoup aussi du prince des Asturies. Mais voici, dans un genre bien
différent, quelque chose d'aussi surprenant que l'exacte parité qu'on
vient de voir des cardinaux-chanoines de Tolède avec les autres
chanoines de cette église, et que je ne puis m'empêcher d'écrire,
quelque ridicule que cela soit. Arrivé avec tout ce qui était avec moi,
à l'audience de la princesse des Asturies, qui était sous un dais,
debout, les dames d'un côté, les grands de l'autre, je fis mes trois
révérences puis mon compliment. Je me tus ensuite, mais vainement, car
elle ne me répondit pas un seul mot. Après quelques moments de silence,
je voulus lui fournir de quoi répondre, et je lui demandai ses ordres
pour le roi, pour l'infante et pour Madame, M. {[}le duc{]} et
M\textsuperscript{me} la duchesse d'Orléans. Elle me regarda et me lâcha
un rot à faire retentir la chambre. Ma surprise fut telle que je
demeurai confondu. Un second partit aussi bruyant que le premier. J'en
perdis contenance et tout moyen de m'empêcher de rire\,; et jetant les
yeux à droite et à gauche, je les vis tous, leurs mains sur leur bouche,
et leurs épaules qui allaient. Enfin un troisième, plus fort encore que
les deux premiers, mit tous les assistants en désarroi et moi en fuite
avec tout ce qui m'accompagnait, avec des éclats de rire d'autant plus
grands qu'ils forcèrent les barrières que chacun avait tâché d'y mettre.
Toute la gravité espagnole fut déconcertée, tout fut dérangé\,; nulle
révérence, chacun pâmant de rire se sauva comme il put, sans que la
princesse en perdît son sérieux, qui ne s'expliqua point avec moi
d'autre façon. On s'arrêta dans la pièce suivante pour rire tout à son
aise, et s'étonner après plus librement.

Le roi et la reine ne tardèrent pas à être informés du succès de cette
audience, et m'en parlèrent l'après-dînée au Mail. Ils en rirent les
premiers pour en laisser la liberté aux autres, qui la prirent fort
largement sans s'en faire prier. Je reçus et je rendis des visites sans
nombre\,; et comme on se flatte aisément, je crus pouvoir me flatter que
j'étais regretté. Je comptais partir le 23, mais les bulles de dispense
étant arrivées depuis quelques jours à Maulevrier pour l'ordre de la
Toison d'or, et la cérémonie de sa réception étant fixée à ce même jour,
je crus devoir déférer à ses instances et ne pas affecter de partir ce
même jour, après tout ce qui s'était passé. J'assistai donc en voyeux à
sa réception, comme j'avais fait à celle de mon fils aîné, et je fus
témoin de l'insigne dégoût qu'il y essuya.

Quand ce fut à le revêtir du collier, le marquis de Villena s'approcha
de lui pour le lui attacher sur l'épaule droite, mais le prince des
Asturies ne branla pas de sa place, en sorte que le marquis de Grimaldo,
après avoir attaché le collier par derrière, l'attacha aussi sur
l'épaule gauche. Je remarquai la surprise du chapitre et de tous les
assistants, mais elle augmenta bien davantage aux révérences. Lorsque
Maulevrier la fit au prince des Asturies, ce prince, au lieu de se
découvrir, se lever et l'embrasser, demeura assis sans se découvrir ni
en faire aucun semblant, et dans cette posture lui présenta sa main à
baiser comme avait fait le roi, et il la baisa. Il me parut à l'instant
que ce procédé fut extrêmement senti, qui ne pouvait être que de concert
avec le roi. Maulevrier n'en parut point du tout embarrassé. Il avait
choisi le marquis de Santa-Cruz pour son parrain, qui ne l'aimait point,
et qui se moquait souvent de lui et en face. Aussi fit-il sa fonction
avec un air de dédain qui n'échappa à personne. Il vint pourtant dîner
chez lui après la cérémonie, où nous nous trouvâmes douze ou quinze au
plus. Avant de se mettre à table, je vis le peu de chevaliers de la
Toison qui étaient là se pelotonner, dont quelques-uns ne me cachèrent
pas leur scandale, et leur crainte que le mépris public qui venait
d'être fait de Maulevrier par le prince des Asturies, conséquemment par
le roi son père, sans l'aveu duquel il n'eût pas osé contrevenir à ce
qui s'était toujours pratiqué en toutes les réceptions jusqu'alors, ne
devînt un exemple qui serait suivi désormais, et je les laissai dans le
mouvement de se concerter pour faire là-dessus leurs représentations au
roi. Comme je partis le lendemain 24, je n'ai point su ce qui en est
arrivé. J'eus l'honneur de faire encore ma cour à Leurs Majestés
Catholiques toute cette après-dînée, au Mail, qui me comblèrent de
bontés, et de prendre un dernier congé d'elles en rentrant dans leur
appartement.

J'avais donné la plupart de ces derniers jours à ce qu'un aussi court
séjour qu'un séjour de près de six mois avait pu me faire regarder comme
des amis particuliers, surtout à Grimaldo. Quelque sensible joie et
quelque empressement que je sentisse d'aller retrouver
M\textsuperscript{me} de Saint-Simon et mes amis, je ne pus quitter
l'Espagne sans avoir le coeur serré, {[}sans{]} regretter des personnes
dont j'avais reçu tant de marques personnelles de s'accommoder de moi,
et dont tout ce que j'avais vu dans le gros de la nation m'avait fait
concevoir de l'estime jusqu'au respect, et une si juste reconnaissance
pour tant de seigneurs et de dames en particulier. J'ai conservé
longtemps quelque commerce de lettres avec quelques-uns, mais avec
Grimaldo tant qu'il a vécu, et après sa disgrâce et sa chute, qui
n'arriva que longtemps après, avec plus de soin et d'attention
qu'auparavant. L'attachement plein de respect et de reconnaissance pour
le roi et la reine d'Espagne m'engagea à me donner l'honneur de leur
écrire en toutes occasions, surtout à répandre mon extrême douleur à
leurs pieds au renvoi de l'infante\footnote{Ce fut le 5 avril 1725 que
  le duc de Bourbon, alors premier ministre, fit renvoyer la jeune
  infante, que l'on élevait à Paris depuis 1722.}. Je consultai
là-dessus l'évêque de Fréjus, déjà plus maître que M. le Duc, qui me
manda que je pouvais écrire, résolu, s'il m'eût refusé, de le dire à
Laullez et de le prier de le mander à Leurs Majestés Catholiques. Elles
me firent souvent l'honneur de me répondre avec toutes sortes de bontés,
et de charger toujours leurs nouveaux ministres en France, et les
personnes considérables qui y venaient se promener avec leur permission,
de me renouveler expressément les mêmes bontés de leur part.

Je partis donc, enfin, de Madrid le 24 mars, prenant ma route par
Pampelune. Une de mes premières dînées fut à Alcala. C'est une petite
ville fort bien bâtie, dont douze ou quinze collèges font tout
l'honneur, tous bâtis très bien, et encore plus splendidement fondés par
le cardinal Ximénès, qui n'est connu en Espagne que sous le nom du
cardinal Gisneros, et respecté presque autant que l'a mérité ce grand
homme. J'allai voir quelques-uns de ces collèges. Il est enterré dans la
chapelle du principal, qui ferait ici une jolie église. Son tombeau de
marbre est beau, environné d'une grille à hauteur d'homme, dans le
choeur, devant le grand autel. Il était assez gâté faute de soin et de
réparation, ce qui excita tellement mon indignation que je n'épargnai
pas les principaux de ce collège en reproches de leur négligence et de
leur ingratitude. Je couchai une nuit à Guadalajara, où arriva la
catastrophe de la princesse des Ursins, et où je vis le panthéon du duc
del Infantado, dont j'ai parlé ailleurs.

Une autre dînée fut à Agreda, assez gros bourg oie est un monastère de
filles, où la fameuse Marie d'Agreda a vécu et est morte, que la gent
quiétiste a fait enfin canoniser depuis, à toute peine, à l'appui de la
constitution \emph{Unigenitus}. J'allai à ce couvent, dont on m'ouvrit
l'église, qui n'a rien que de très simple et commun. On me montra à côté
du portail, qui est aussi plus que médiocre, comme un grand soupirail de
cave ouvert sur la rue, où on me dit que reposait son corps. Je n'en
voulais pas davantage, et j'avais déjà fait quelques pas pour aller
trouver mon dîner, lorsque les religieuses, informées que j'étais là,
m'envoyèrent prier de les aller voir. Je ne pus honnêtement refuser
cette demande, plus curieuse sûrement encore que civile. Je fus conduit
dans une grande cour, à une grande porte, qui était assez loin à gauche,
ce qui ne me laissa pas douter que le dessein ne fût de me faire entrer
dans le monastère. Aussitôt que j'en fus tout proche, la porte s'ouvrit
tout entière, qui se trouva bordée de religieuses, touchant le seuil,
mais en dedans. La supérieure me fit un compliment en assez bon
français, et me pria de m'asseoir dans un fauteuil qu'on avait mis
derrière moi. Elles s'assirent toutes sur de petites chaises de paille.
Après quelques courts propos sur mon voyage, on peut juger qu'il ne fut
plus mention que de leur sainte, déjà béatifiée, mais depuis peu. Elles
m'en firent apporter des choses de dévotion, un petit Jésus de cire,
quelques livres, quelques chapelets, dont elles me donnèrent
quelques-uns. J'admirai tout ce qu'elles me voulurent conter, mais
j'abrégeai poliment la conversation plus qu'elles n'auraient voulu, et
je m'en allai trouver mon dîner, peu satisfait de ma curiosité.

J'avais pris ma route par Pampelune. Le gouverneur vint aussitôt où
j'étais logé, et voulut me mener chez lui et me donner à souper et à
ceux qui étaient avec moi. Après force longs compliments, j'obtins de
demeurer où j'étais, à condition que nous irions souper chez lui. La
chère ne se fit point attendre, fut grande, à l'espagnole, mauvaise\,;
des manières nobles, polies, aisées. Il nous fit fête d'un plat
merveilleux. C'était un grand bassin plein de tripes de morue fricassées
à l'huile. Cela ne valait rien, et l'huile méchante. J'en mangeai, par
civilité, tant que je pus. En me retirant je lui demandai la permission
de voir la citadelle, où on ne laisse entrer aucun étranger. J'y fus
avec ce qui était avec moi le lendemain matin. Je visitai tout à mon
aise, et je la trouvai fort belle, bien entretenue, ainsi que la
garnison, qui me reçut sous les armes, au bruit du canon, et tout en
fort bel et bon ordre. Nous allâmes de là voir et remercier le
gouverneur, qui peu après revint chez moi nous voir partir.

À peu de distance, nous primes des mules pour passer les Pyrénées. Le
chemin est par là plus court et un peu moins rude que par Vittoria. Mais
il était devenu fort mauvais, parce que les Espagnols, qui l'avaient
fort aplani pour y pouvoir mener aisément de l'artillerie depuis qu'ils
avaient un roi français, en avaient soigneusement rompu tous les chemins
lors de la guerre que l'abbé Dubois leur fit faire par M. le duc
d'Orléans pour complaire aux Anglais et pour son chapeau, où le maréchal
de Berwick commanda. Nous couchâmes à Roncevaux, lieu affreux, tout
délabré, le plus solitaire et le plus triste de ce passage, dont
l'église n'est rien, ni ce qui reste de l'ancien monastère, où nous
fûmes logés. L'abbé me vint voir, vêtu de long, avec un grand manteau
vert, ce qui me surprit beaucoup. La visite fut courte. On nous montra
l'épée de Roland et force pareilles reliques romanesques. Nous partîmes
de bon matin de ce désagréable gîte, et arrivâmes enfin le jeudi saint à
Bayonne, chez M. d'Adoncourt, par une pluie effroyable et continuelle
qui ne nous avait point quittés depuis la sortie des montagnes. Il
semblait qu'elle n'osait les passer. Je n'en avais presque point vu
tomber en Espagne. Le ciel y est sans cesse d'une sérénité admirable, et
les vents ne s'y font presque point sentir.

D'Adoncourt, quoi que nous pussions dire, nous logea et nous fit la plus
grande et la meilleure chère du monde. J'assistai les jours saints aux
offices de la cathédrale, dans la place et avec le même traitement usité
pour le gouverneur de la province, l'évêque y officiant. J'eus l'honneur
de faire ma cour plusieurs fois à la reine douairière d'Espagne, qui
m'ordonna de dîner dans sa maison de la ville, le jour de Pâques, dont
le sieur de Bruges, dont j'ai parlé lors de mon passage, fit très bien
les honneurs\,; et comme on savait que j'étais affamé de poisson, on y
en servit en quantité et d'admirables, que je préférai à la viande.
L'évêque, dont j'ai parlé aussi en même temps, et quelques principaux du
lieu s'y trouvèrent. J'allai de là remercier et prendre congé de la
reine, qui me fit présent elle-même d'une fort belle épée d'or sans
diamants, avec beaucoup d'excuses de me donner si peu de chose. L'évêque
voulut me donner à souper si absolument qu'il fallut s'y rendre. J'y
trouvai bonne compagnie, bonne chère et force poisson, qui ne laissa pas
de trouver encore place.

Un courrier m'arriva à Bayonne, qui avait été précédé de deux autres
qui, pour ne me pas manquer, avaient pris, l'un par Vittoria, l'autre
par Pampelune. Tous trois apparemment portaient des duplicata, car je
n'ai point vu les dépêches des deux autres. Je fus agréablement surpris
de celles qui me trouvèrent à Bayonne. C'était la réponse à celle que
j'avais faite et à ce que m'avait apporté Bannière. Le cardinal y avait
vu fort nettement mon sentiment sur la préséance et sur la sortie du
conseil de ceux qu'elle blessait. Il pouvait bien avoir aussi aperçu ce
que je pensais de sa prétendue cabale. Enfin il avait vu que son
éloquence entortillée, ses prétextes recherchés et appuyés, ni la
crainte de lui déplaire, ne pouvaient me retenir en Espagne. Peut-être
les courriers qui m'étaient allés chercher jusqu'à Madrid me
portaient-ils des ordres si positifs qu'ils m'eussent embarrassé, qu'il
n'était plus temps de me donner en deçà des Pyrénées, et que ce fut pour
cela que je reçus à Bayonne ce troisième courrier avec des lettres
ajustées pour le lieu, au cas qu'il m'y trouvât, comme il arriva, avec
ordre de m'y attendre, et peut-être de rebrousser chemin avec ses
dépêches au bout d'un certain temps que j'aurais reçu en Espagne celles
qui m'y étaient portées par les deux courriers qui avaient passé et qui
ne m'avaient point rencontré, car ces sortes de ruses étaient tout à
fait dans le caractère du cardinal Dubois. Quoi qu'il en soit, j'ouvris
sa lettre avec curiosité.

Je n'y trouvai plus mention de rester encore en Espagne, ni de Chavigny,
ni d'aucun autre prétexte, et pas un mot qui laissât sentir que je lui
eusse répondu franchement sur l'affaire du conseil. Je n'eus que des
louanges de la promptitude avec laquelle j'avais été à Balsaïm, et de la
manière dont je m'étais acquitté de ce qui m'y avait fait aller\,; des
impatiences nonpareilles d'amitié et de besoin de mon arrivée\,; une
prière, qui allait à la défense, de m'arrêter nulle part, même de faire
le très petit détour de passer à Blaye, parce que les choses du monde
les plus pressées et les plus importantes m'attendaient, qui ne
pouvaient se faire sans moi. Cette lettre si singulière était
accompagnée d'une autre de Belle-Ile, qui en faisait le commentaire. Il
me répétait les mêmes choses, me disait que le cardinal Dubois était
charmé de la réponse que j'avais faite aux dépêches que j'avais reçues
par Bannière\,; qu'il m'écrivait par son ordre exprès pour me conjurer
d'arriver avec toute la diligence possible, et que je ne pouvais me
rendre assez tôt pour l'importance des choses que le régent et le
cardinal avaient à me communiquer, et sur lesquelles, toutes pressées
qu'elles fussent, il ne se pouvait rien faire sans moi. Il ajoutait
qu'il était chargé de m'assurer qu'il ne me serait rien proposé qui pût
m'être désagréable ou m'embarrasser, rien surtout qui pût en aucune
sorte intéresser ma dignité de duc et pair, sur {[}ce{]} qu'ils étaient
bien persuadés qu'il n'y avait rien à espérer de moi là-dessus. Rien de
plus pressant enfin ni de plus flatteur. Il finissait enfin en me
conjurant de ne m'arrêter pas un instant et de ne passer point à Blaye.

Un si grand changement de style et tant de merveilles à l'instant de mon
départ, malgré tant de fortes insinuations, et quelque chose même de
plus, d'y demeurer encore sous les prétextes qu'on a vus, me parut fort
suspect d'une part si peu sûre, car il était visible que le cardinal
avait pour ainsi dire dicté, au moins vu et corrigé, la lettre de
Belle-Ile, comme il avait fait celle que Bannière m'avait apportée\,: on
verra bientôt que je ne me trompais pas. Je leur mandai par une réponse
courte à chacun le jour que j'avais supputé pouvoir arriver\,; que
j'étais fatigué du voyage à tour de roue jusqu'à Bayonne\,; que cette
raison de m'y reposer et celle des jours saints m'y retiendraient
jusqu'au lundi de Pâques\,; enfin que je n'avais pu refuser au duc de
Berwick de prendre les petites landes pour l'aller trouver, où il venait
exprès de Montauban pour me voir\,; et du reste force compliments.

Le duc de Berwick, qui commandait en Guyenne, et qui trouvait Montauban
plus commode que Bordeaux pour fixer son séjour, m'avait en effet
demandé ce rendez-vous avec instance\,; et l'amitié qui était entre
nous, et toutes celles que j'avais reçues du duc de Liria, ne me
permettait pas un refus. Il était bien naturel au maréchal de désirer de
m'entretenir sur la situation de son fils en Espagne, sur une cour qu'il
avait tant fréquentée, et sur les dispositions, pour lui-même, de Leurs
Majestés Catholiques après tout ce qui s'était passé. Je partis donc de
Bayonne seul avec l'abbé de Saint-Simon, le lendemain de Pâques, et m'y
séparai jusqu'à Paris de tout ce qui était avec moi. Je passai un jour
franc avec le maréchal de Berwick à Marmande, et avec le duc de Duras,
qui était venu avec lui, et qui commandait en Guyenne sous lui. J'appris
là que nous n'étions qu'à quatre lieues de Duras. Je voulus y faire une
course pour en dire des nouvelles à M\textsuperscript{me} de Saint-Simon
et des beautés que le maréchal son oncle y avait fait faire toute sa vie
avec attache, sans jamais les avoir été voir. J'en avais aussi
curiosité\,; mais quoi que je pusse faire, jamais ils ne voulurent y
consentir. Malheureusement ils savaient, comme tout le pays, les
courriers qui m'avaient été dépêchés\,; ils n'osèrent prendre part à mon
retardement, dont j'eus un véritable regret. Je m'embarquai de bon matin
sur la Garonne, et j'arrivai de bonne heure à Bordeaux, chez Boucher,
intendant de la province. Les jurats me firent aussitôt demander par
Ségur, leur sous-maire, l'heure de me venir saluer. Je les priai à
souper, et dis à Ségur que les compliments se feraient mieux le verre à
la main. Ils vinrent donc souper, et me parurent fort contents de cette
honnêteté. Le lendemain la marée me porta de fort bonne heure à Blaye
par le plus beau temps du monde. Je n'y couchai qu'une nuit et ne passai
point à Ruffec pour abréger.

J'arrivai le 13 avril à Loches sur les cinq heures du soir. J'y couchai
parce que j'y voulus écrire un volume de détails à la duchesse de
Beauvilliers, qui était à six lieues de là, dans une de ses terres, que
je lui envoyai par un exprès\,; et je pus de la sorte lui écrire à
découvert sans rien craindre de l'ouverture des lettres. J'arrivai
d'assez bonne heure le lendemain 14 à Étampes, où je couchai, et le 15,
à dix heures du matin, à Chastres\footnote{Chastres, ou Châtres, aux
  environs d'Étampes (Seine-et-Oise), porte aujourd'hui le nom
  d'Arpajon. On a changé ce nom en celui de Chartres dans les anciennes
  éditions. L'itinéraire de Saint-Simon se rendant d'Étampes à Paris
  suffirait pour prouver qu'il ne faut pas lire Chartres.}, où
M\textsuperscript{me} de Saint-Simon devait venir dîner et coucher,
au-devant de moi, pour jouir du plaisir de nous recevoir, de nous
retrouver ensemble, de nous mettre réciproquement au fait de tout, en
solitude et en liberté, ce qui ne se pouvait espérer à Paris dans ces
premiers jours de mon retour. Le duc d'Humières et Louville vinrent avec
elle. Elle arriva une heure après moi dans le petit château du marquis
d'Arpajon, qu'il lui avait prêté, où la journée nous parut bien courte
et la matinée du lendemain 16 avril.

Comme nous causions, sur les dix heures du matin, arriva Belle-Ile.
Après les amitiés et les compliments, il me pria qu'il pût m'entretenir
en particulier. Après de nouveaux compliments, des louanges de ma
conduite en Espagne et de mes lettres, et une courte peinture de la
situation de la cour, se taisant sur la préséance et glissant sur la
cabale, il me peignit le duc de Noailles comme l'homme le plus
dangereux, et le plus ennemi de M. le duc d'Orléans et de son
gouvernement, et n'oublia pas d'animer ma haine autant qu'il lui fut
possible, et de me présenter tout l'intérêt que j'avais de saisir
l'occasion de le perdre sans ressource, qui s'offrait d'elle-même à moi,
et pour laquelle j'étais attendu avec tant d'impatience.

Après ce vif préambule, il me dit merveilles du cardinal Dubois à mon
égard, et enfin qu'il l'avait chargé de venir me trouver à Chastres pour
me confier de quoi il s'agissait\,; en quoi il ne doutait pas que
l'amour de l'État, mon attachement personnel pour M. le duc d'Orléans,
la connaissance expérimentale que j'avais du caractère du duc de
Noailles, enfin que mon intérêt, si fort uni à celui de M. le duc
d'Orléans, ne me portât à me joindre à lui, cardinal Dubois, dans ce qui
était projeté, pour l'exécution de quoi il m'avait attendu avec une
extrême impatience\,; en un mot, qu'il fallait chasser le duc de
Noailles et lui ôter sa charge de premier capitaine des gardes du corps.
Je répondis à Belle-Ile par une autre préface, mais bien plus courte que
n'avait été la sienne, sur tous les points qu'il avait traités. Je
m'étendis un peu plus sur ma haine pour le duc de Noailles, sur ses
causes, sur ma soif ardente de vengeance, sur ce que je n'avais nul
ménagement à garder avec lui, et sur ce qu'en effet je n'en gardais
publiquement aucun. Ensuite je lui dis qu'en affaires de cette nature ce
n'était pas son intérêt ni sa passion qu'il fallait consulter\,; que, si
je n'écoutais que l'un ou l'autre, il n'y avait rien à quoi je ne me
portasse pour écraser le duc de Noailles\,; mais que l'intérêt et la
passion étaient des conseillers dont un homme d'honneur et de bien se
devait garder, sans toutefois exclure la satisfaction qu'ils pourraient
prendre dans les conseils sages, justes et prudents qui, sans égard à
eux, et pour des causes réelles et sans reproche, se trouveraient
d'ailleurs concourir avec eux\,; que c'était ce que je ne pouvais
apercevoir dans la proposition qu'il me faisait, où je ne voyais nulle
raison qui pût imposer à personne, mais beaucoup de danger à s'y
abandonner. Belle-Ile, fâché de ce qu'il entendait, m'interrompit de
vivacité et voulut pérorer. À mon tour je lui demandai audience. Je le
priai de considérer que ce n'était pas tout de frapper de grands coups,
mais qu'il en fallait considérer la conséquence et les suites\,; que je
n'ignorais pas le pouvoir du roi sur les charges qui ne sont pas offices
de la couronne, mais que je savais aussi qu'il n'est pas souvent à
propos de faire tout ce qu'on peut exécuter\,; que quelque haine que
j'eusse pour le duc de Noailles, et quelque juste mépris que j'eusse de
son âme, de sa conduite et de ses quarts de talents, je le voyais
revêtu, et point de crime qui autorisât à le dépouiller. S'il y en avait
quelqu'un, il le fallait montrer, le prouver et l'établir publiquement,
avec tant de solidité, sans même rien de forme juridique, que cela
fermât la bouche au monde. Mais s'il n'y avait que des sujets de simple
mécontentement, le dépouiller serait et paraîtrait une violence qui
irriterait tout le monde, et en particulier tous ceux qui avaient des
charges, et tous leurs entours, dont chacun se dirait avec raison\,:
«\,Aujourd'hui le duc de Noailles, et demain moi, si la fantaisie en
prend\,; et qui me garantira d'une fantaisie\,?» Dès lors, voilà tout ce
qu'il y a de gens les plus établis et les plus considérables, et tout ce
qui tient à eux, dans le plus grand éloignement de M. le duc d'Orléans
et d'un gouvernement sous lequel il n'y a de sûreté pour personne\,; et
c'est la semence la plus fertile et la plus dangereuse des associations,
des complots, et de tout ce qu'ils enfantent de plus sinistre. «\,Voyons
les choses, ajoutai-je, comme elles sont et comme elles se présentent.
Bien ou mal à propos, le duc de Noailles est le troisième capitaine des
gardes, et le troisième gouverneur du Roussillon, de père en fils. Il a,
depuis qu'il a commencé à paraître, été sans cesse dans des emplois
brillants. Les établissements de ses soeurs et de toute sa famille sont
immenses, tous gens qui, par intérêt et par honneur ne peuvent pas ne
point sentir vivement le coup dont il sera frappé\,; et plus il tombe
sur un homme si grandement établi, et lui et ses plus proches et
nombreux entours, plus M. le duc d'Orléans s'en fait des ennemis
irréconciliables, plus toutes les charges du royaume tremblent et
s'indignent d'autant plus que la plupart de leurs possesseurs, quant à
leurs personnes, aucun, quant à leurs entours, n'ont pas à beaucoup près
des considérations de ménagement telles que les a le duc de Noailles.\,»
Je priai ensuite Belle-Ile de considérer la proximité du moment de la
majorité, et tout ce que M. le duc d'Orléans aurait à craindre de tous
les gens en charge d'approcher à toutes heures un roi dont l'esprit ne
pouvait pas être formé, encore moins le jugement, et qui serait en proie
aux flatteries, aux calomnies, aux adresses de tous gens si intéressés à
perdre auprès de lui le régent et sa régence, et qui auraient tant de
choses spécieuses à se ballotter entre eux, pour les mettre sans
défiance dans la tête du roi, sur les finances, sur la marine, sur
l'Angleterre, sur la guerre faite à l'Espagne, sur la vie particulière
de M. le duc d'Orléans, et sur tant d'autres points qui se présentent si
aisément quand on veut nuire et qu'un grand intérêt y pousse, sans
compter les autres mécontents. Belle-Ile ne sut que répondre de précis à
des objections si fortes et si évidentes. Mais pour ne pas se rendre, il
battit la campagne, et chercha tant qu'il put des ressources dans ma
haine et dans son bien dire.

Cette conférence, où il ne fut question que de ce point, dura plus d'une
heure, et finit par me prier de faire encore des réflexions. Je lui dis
qu'elles s'étaient toutes présentées à la première mention de sa
proposition\,; qu'elles se fortifiaient toutes l'une par l'autre\,; que
je ne voyais pas qu'il eût répondu à aucune\,; qu'ainsi je demeurais
dans mon sentiment\,; que je le priais de les porter toutes et dans
toute leur force au cardinal Dubois pour lui faire sentir les suites
funestes de ce projet, auquel l'accablement d'affaires de toutes les
sortes ne lui avaient pas permis de penser avec l'attention qu'il
méritait. J'assaisonnai cela de tous les compliments capables d'adoucir
le dépit de ma résistance, qui fut d'autant plus vif que le cardinal
n'osa le montrer. Belle-Ile dîna avec nous en sortant de cette
conversation parce que nous voulions arriver à Paris de fort bonne
heure, et partit avant nous.

Je ne fis que changer de voiture au logis, et j'allai au Palais-Royal,
droit chez le cardinal Dubois. Il accourut au-devant de moi. Ce fut des
merveilles\,; et sans rentrer ni s'arrêter, il me conduisit chez M. le
duc d'Orléans, dont la réception fut aussi bonne et plus sincère. Il
était dans son petit cabinet au bout de sa petite galerie. Nous nous
assîmes, moi vis-à-vis de lui, son bureau entre deux, et le cardinal au
bout du bureau. Je leur rendis compte de bien des choses, et je répondis
à bien des questions. Ensuite je parlai à M. le duc d'Orléans de la
conduite de la princesse des Asturies avec Leurs Majestés Catholiques,
de leur patience et de leurs bontés pour elle\,; et après ce sérieux je
le divertis de mon audience de congé chez elle, dont il rit beaucoup.
Ensuite il me parla de la sortie du conseil, glissant avec des patins
sur la préséance\,; et le cardinal se mit sur la cabale, sans toutefois
enfoncer matière, et dit que Son Altesse Royale n'avait pu moins faire
que de chasser le chancelier. Je laissai tout conter\,; puis je leur dis
que je ne pouvais qu'apprendre, ne m'étant pas lors trouvé ici et
n'ayant encore vu personne, sinon que je trouvais tout cela bien
fâcheux. Et tout de suite, me tournant tout à fait à M. le duc d'Orléans
et m'adressant à lui, j'ajoutai que, puisque le chancelier n'était à
Fresnes que pour la même chose que j'aurais faite si j'avais été ici,
j'espérais bien que Son Altesse Royale trouverait bon que j'y allasse le
voir incessamment. Cette parole fit comme deux termes du régent, qui
baissa les yeux, et du cardinal, qui égara les siens, rougissant de
colère. Je crois bien qu'ils n'avaient pas espéré me persuader de
rentrer au conseil\,; mais l'étonnement et le dépit d'une adhésion si
nette et si peu attirée à la sortie du conseil, et la liberté avec
laquelle je causais\footnote{J'indiquais la cause.} mon empressement
pour le chancelier déconcerta le régent comme un particulier, et le
tout-puissant ministre comme un courtisan. Je me repus avec complaisance
de l'état où je les vis, et du silence qui dura plusieurs moments. Le
cardinal le rompit en se secouant comme un homme qui se réveille, et me
dit, d'un air le plus bénin qu'il put, qu'ils avaient fait ce que le roi
d'Espagne avait désiré. Je lui demandai ce que c'était. Il me
répondit\,: «\,Donner au roi un jésuite pour confesseur, et c'est le P.
Linières. --- Le roi d'Espagne\,! repris-je, jamais il ne m'en a parlé.
--- Comment\,? dit le cardinal\,; il me semble pourtant qu'il vous a
parlé de jésuite, et que vous nous en avez écrit. --- Vous confondez,
monsieur, repris-je\,; le roi d'Espagne m'en a parlé pour l'instruction
de l'infante, et pour sa confession pour la suite\,; je vous en ai écrit
et à M. le duc d'Orléans, et cela {[}a{]} été fait\,; mais jamais le roi
d'Espagne ne m'en a dit un seul mot pour le roi. Bien est vrai que le P.
Daubenton m'en parla, et me dit que le roi d'Espagne avait dessein de me
charger de prier M. le duc d'Orléans, de sa part, de rendre le
confessionnal du roi aux jésuites\,; que je répondis au P. Daubenton que
pour moi je serais ravi d'y pouvoir contribuer comme particulier, mais
que je n'oserais pas me charger de faire cet office, parce que, comme le
roi d'Espagne aurait raison de trouver mauvais que notre cour se voulût
ingérer d'entrer dans les choses intérieures de sa cour, surtout de se
mêler de son confesseur, aussi notre cour voulait être en pleine liberté
sur ces mêmes choses, et me blâmerait aigrement de me charger d'une
pareille commission\,; qu'ainsi je le suppliais de détourner le roi
d'Espagne de me la proposer, parce que j'aurais la douleur de ne la
pouvoir accepter. Le P. Daubenton se rendit tout court à ces raisons,
qu'il trouva ou qu'il fit semblant de trouver bonnes. Jamais le roi
d'Espagne ne m'en a ouvert la bouche ni parlé de rien d'approchant, ni
le P. Daubenton depuis.\,» Le cardinal balbutia entre ses dents je ne
sais quoi qu'il n'achevait pas de prononcer, et M. le duc d'Orléans, qui
jusque-là l'avait laissé parler là-dessus et moi lui répondre, se mit à
rire et à me dire\,: «\,Oh bien\,! donc, tout ce que nous vous demandons
(je remarquai bien ce \emph{nous} de communauté avec le cardinal), c'est
que vous ne nous démentiez pas\,; car nous avons dit à tout le monde que
c'était aux pressantes instances du roi d'Espagne que nous avions donné
au roi un confesseur jésuite.\,» Je me mis aussi à rire, et lui répondis
que tout ce que je pouvais pour son service, si on m'en parlait dans le
monde, serait de faire le plat important, et de payer de silence pour ne
les point démentir et pour ne point mentir. Puis m'adressant au
cardinal, je lui dis qu'il avait toutes mes dépêches\,; que, pour en
avoir le coeur net, il prît la peine de les visiter, et qu'il n'y
trouverait que le fait d'un jésuite pour l'infante, et pas un mot pour
le confesseur du roi. Le saint prélat le savait de reste\,; il se mit à
rire aussi, mais du bout des dents\,; me dit qu'il se rappelait la
chose, qu'elle était telle que je la leur disais, mais qu'il était
important de la tenir secrète, et que je ne me laissasse pas entamer
là-dessus.

Cette conversation, qui dura près de deux heures, finit le mieux du
monde, mais, jointe à celle que j'avais eue le matin à Chastres avec
Belle-Ile, ne me mit pas bien dans les bonnes grâces du cardinal Dubois,
qui toutefois n'osa en rien faire paraître. Elle finit par la permission
que je demandai au régent de me démettre de ma pairie à mon fils aîné.
Je ne trouvais pas convenable que, destiné par son aînesse à être duc et
pair, il n'en eût pas le rang, tandis que je l'avais acquis à son cadet
par la grandesse.

Du Palais-Royal j'allai aux Tuileries faire ma révérence au roi, à son
souper, à la fin duquel je lui demandai la même permission. Je m'en
retournai de là chez moi, où je le dis à mon fils aîné, qui prit le nom
de duc de Ruffec. Je lui fis en même temps présent des pierreries qui
environnaient le portrait du roi d'Espagne, que le marquis de Grimaldo
m'avait apporté de sa part l'après-dînée de mon audience de congé. Elles
furent estimées quatre-vingt mille livres pair les premiers joailliers
de Paris. C'était le plus riche présent qui en eût été fait en Espagne à
aucun ambassadeur. Je me plus à en faire faire une magnifique Toison à
mon fils.

Il fallut se livrer pendant plusieurs jours aux visites passives et
actives. Toutefois je me hâtai d'aller voir le cardinal de Noailles. Je
ne voulais pas qu'il fût la dupe de la demande prétendue du roi
d'Espagne d'un confesseur jésuite pour le roi. Je lui fis confidence,
sous le secret, de ce qui s'était passé là-dessus, au Palais-Royal,
entre le régent, le cardinal Dubois et moi. Je fis aussi la même
confidence, et sous le même secret, à l'évêque de Fréjus et au maréchal
de Villeroy, qui s'étaient opposés de toutes leurs forces à un
confesseur jésuite, malgré l'ensorcellement de la constitution. Ils
furent fort sensibles à cette confidence, que je crus nécessaire, et
m'en ont toujours gardé le secret. Du reste, je fus fidèle à ne me
laisser entendre là-dessus à personne, et à payer les questions de
silence. C'était la condition remplie par Dubois à l'égard des jésuites
du concours qu'il en avait obtenu pour son chapeau. Je pris le premier
jour du conseil de régence et le temps de sa tenue pour visiter tous
ceux qui en étaient sortis. Cette affectation fut fort remarquée, comme
c'était bien aussi mon dessein. Je sus que le maréchal de Villeroy, qui
s'était conservé d'y assister, mais derrière le roi, sans opiner ni y
prendre la moindre part, avait envoyé voir dans la cour des Tuileries si
mon carrosse y était. Il ne put s'empêcher de me témoigner sa joie de ce
que je n'étais pas rentré au conseil. Je lui répondis froidement qu'il
ne me connaissait guère s'il m'en avait pu soupçonner. Six jours après
mon arrivée j'en allai passer trois à Fresnes. Cette visite fit grand
bruit, et fit au chancelier un plaisir sensible. Tant qu'il y fut je l'y
allai voir au moins deux fois l'année. Faisons maintenant une pause, et
rétrogradons pour voir ce qui s'était passé hors de l'Espagne depuis le
commencement de cette année.

\hypertarget{chapitre-xiii.}{%
\chapter{CHAPITRE XIII.}\label{chapitre-xiii.}}

1722

~

{\textsc{Façon plus que singulière dont l'officier dépêché avec le
contrat de mariage du roi fut enfin expédié de tout ce que j'avais
demandé pour lui.}} {\textsc{- Mort de M\textsuperscript{me} de Broglio
(Voysin).}} {\textsc{- Mort du comte de Chamilly.}} {\textsc{- Mort de
M\textsuperscript{me} de Montchevreuil, abbesse de Saint-Antoine.}}
{\textsc{- Cette abbaye donnée à M\textsuperscript{me} de Bourbon.}}
{\textsc{- Mort de l'abbé et du marquis de Saint-Hérem.}} {\textsc{-
Mort du comte de Cheverny, de l'abbé de Verteuil\,; de l'évêque de
Carcassonne (Grignan)\,; de Saint-Fremont\,; sa fortune.}} {\textsc{-
Mort du marquis de Montalègre à Madrid, et sa dépouille.}} {\textsc{-
Mort de la princesse Ragotzi (Hesse-Rhinfels)\,; de la duchesse de Zell
(Desmiers-Olbreuse)\,; sa fortune.}} {\textsc{- Mort du comte d'Althan,
grand écuyer et favori de l'empereur.}} {\textsc{- Mariage du prince
palatin de Soultzbach avec l'héritière de Berg-op-Zoom\,; du prince de
Piémont avec la princesse palatine de Soultzbach\,; du marquis de
Castries avec la fille du duc de Lévy\,; de Puysieux avec la fille de
Souvré\,; du duc d'Épernon avec la seconde fille du duc de Luxembourg\,;
de M\textsuperscript{lle} d'Estrées, déclarée, avec d'Ampus.}}
{\textsc{- P. de Linières, jésuite, confesseur de Madame, fait
confesseur du roi, avec des pouvoirs du pape, au refus de ceux du
cardinal de Noailles.}} {\textsc{- Armenonville garde des sceaux.}}
{\textsc{- Morville secrétaire d'État.}} {\textsc{- Le chancelier, sur
le point immédiat de son exil, marie sa fille au marquis de Chastelux.}}
{\textsc{- Caractère de ce gendre.}} {\textsc{- Cruel bon mot de M. le
duc d'Orléans.}} {\textsc{- Broglio l'aîné et Nocé exilés.}} {\textsc{-
M\textsuperscript{me} de Soubise gouvernante des enfants de France en
survivance de la duchesse de Ventadour.}} {\textsc{- Dodun contrôleur
général des finances en la place de La Houssaye.}} {\textsc{- Pelletier
de Sousy se retire à Saint-Victor.}} {\textsc{- Duc d'Ossone retourné à
Madrid.}} {\textsc{- Translations d'archevêchés et d'évêchés.}}
{\textsc{- Reims donné à l'abbé de Guéméné.}} {\textsc{- Ruses inutiles
des Rohan pour lui procurer l'ordre avant l'âge.}} {\textsc{- Mariage de
ma fille avec le prince de Chimay.}} {\textsc{- Mariage du comte de
Laval avec la soeur de l'abbé de Saint-Simon\,: l'un depuis évêque-comte
de Noyon, puis de Metz, en conservant le rang et les honneurs de son
premier siège\,; l'autre depuis maréchal de France.}} {\textsc{- Mort de
Courtenvaux.}} {\textsc{- Sa charge de capitaine des Cent-Suisses donnée
à son fils, à peine hors du berceau, et l'exercice à son frère.}}
{\textsc{- La cour retourne pour toujours à Versailles.}} {\textsc{- Je
m'oppose à l'exil du duc de Noailles, enfin inutilement.}} {\textsc{-
Bassesses du cardinal Dubois pour se gagner le maréchal de Villeroy,
inutiles.}} {\textsc{- Fatuité singulière de ce maréchal.}} {\textsc{-
Comte de La Mothe fait grand d'Espagne.}} {\textsc{- Mort de Plancy.}}
{\textsc{- Le pape donne à l'empereur l'investiture des Deux-Siciles.}}
{\textsc{- Mort du duc de Marlborough\,; de Zondedari, grand maître de
Malte.}} {\textsc{- Manoel lui succède.}} {\textsc{- Mort de la duchesse
de Bouillon (Simiane)\,; de l'épouse du prince Jacques Sobieski.}}

~

La première chose que j'appris fut de quelle façon l'officier du
régiment d'infanterie de Saint-Simon, que j'avais dépêché, chargé du
contrat de mariage du roi, avait enfin obtenu et reçu tout ce que
j'avais demandé pour lui. Le cardinal le rabrouait et le remettait
toujours, et avait tellement rebuté M. Le Blanc là-dessus, qu'il n'osait
plus lui en parler. Cet officier désolé se contentait de se présenter
devant le cardinal, sans plus rien dire, et à peine était-il remarqué.
Un jour qu'une foule de seigneurs, de dames, d'ambassadeurs, d'évêques
et le nonce du pape remplissaient son grand cabinet à l'attendre,
quelqu'un prit le cardinal en entrant, et lui parla toujours jusqu'au
milieu de cette compagnie. Apparemment qu'il l'importuna, car le
cardinal se tournant à lui de furie, l'envoya promener avec tous les
b\ldots. et les f\ldots. les plus redoublés, jurant à faire trembler et
criant à pleine tête. L'infamie d'une telle sortie au milieu de tout ce
que je viens de nommer saisit cet officier d'un si grand ridicule, qui
avait côtoyé le maltraité pour se pousser et tâcher de se faire voir,
que malgré lui il éclata de rire. À ce bruit le cardinal tourna la tête,
et le vit riant tant qu'il pouvait. Dans l'instant il lui mit la main
sur l'épaule\,: «\,Vous n'êtes pas trop sot, lui dit-il\,; je dirai
tantôt à M. Le Blanc d'expédier vos affaires.\,» Et aussitôt se mêla
avec tout ce qui l'attendait. Ce pauvre officier, qui se crut perdu dès
qu'il sentit la main du cardinal sur son épaule dans l'état où il le
surprenait, pensa tomber par terre de ce contraste, et n'eut ni la force
ni le temps de le remercier. Il alla le lendemain matin chez M. Le
Blanc, où il trouva toute son affaire faite et expédiée sans que rien y
manquât de tout ce que j'avais demandé pour lui, et accourut de là chez
M\textsuperscript{me} de Saint-Simon lui conter son aventure sans
pouvoir cesser d'en rire et de s'en étonner.

Je trouvai qu'il était mort bien des gens de connaissance depuis le
commencement de cette année, et quelques personnes considérables des
pays étrangers. La femme de Broglio, le roué de M. le duc d'Orléans, qui
était fille du feu chancelier Voysin, à trente-deux ans.

Le comte de Boulainvilliers, à soixante ans, qui avait prédit tant de
choses vraies et fausses, mais qui ne se trompa point à l'année, au
mois, au jour et à l'heure de sa mort, comme il avait aussi rencontré
juste à celle de son fils. Il s'y prépara avec courage, vit souvent le
curé de Saint-Eustache, dans la paroisse duquel il demeurait, et reçut
les sacrements. Ce fut dommage qu'un aussi savant homme se fût infatué
de ces curiosités défendues, qui rendaient son commerce suspect, et qui
était le plus doux, le plus aisé et le plus agréable du monde, sûr avec
cela, et si modeste qu'il ne semblait pas rien savoir, avec les
connaissances les plus étendues et les plus recherchées sur toutes les
histoires, et beaucoup de profondeur, de lumières et de bonne et sage
critique sur celle de France et sur son gouvernement primitif, ancien et
nouveau\footnote{Les principaux ouvrages historiques du comte de
  Boulainvilliers sont \emph{l'Histoire de l'ancien gouvernement de la
  France} (3 vol.~in-8°, la Haye, 1727). --- \emph{L'État de la France},
  extrait des mémoires dressés par les intendants du royaume (Londres,
  1727, 3 vol.~in-fol.). Il y a eu de nombreuses éditions de cet
  ouvrage, et une, entre autres, en 8 vol.~in-12. --- \emph{Histoire de
  la pairie de France et du parlement de Paris} (Londres, 1753, 2
  vol.~in-12).}. Son grand défaut était de travailler à trop de choses
en même temps, et de quitter ou d'interrompre un ouvrage commencé,
souvent fort avancé, pour se mettre à un autre. Je l'aurais vu bien plus
souvent pour m'instruire. Sans jamais chercher à rien apprendre aux
autres, il avait le talent, quand on l'en recherchait, de le faire avec
une simplicité, une netteté et une grâce qui plaisait infiniment. Mais
la crainte de donner à penser qu'on le recherchait pour connaître
l'avenir, me retenait et beaucoup d'autres de le fréquenter comme je
l'aurais voulu. Il fut toujours fort pauvre, honnête homme, malheureux
en famille, et ne laissa point de postérité masculine. Il était homme de
qualité et se prétendait de la maison de Croï, par la conformité des
armes, sans toutefois en être plus glorieux.

Le comte de Chamilly. C'était un grand et gros homme de bonne mine, de
savoir et d'esprit, mais qui le faisait trop sentir aux autres. Il avait
été ambassadeur en Danemark, où sa hauteur n'avait pas réussi. Le
maréchal de Chamilly, son oncle, l'avait fait succéder à son
commandement de Poitou, Saintonge, Angoumois, pays d'Aunis, la Rochelle
et îles adjacentes, et il {[}était{]} lieutenant général et gouverneur
du château de Dijon. Il n'avait que cinquante-huit ans, point d'enfants
mâles.

M\textsuperscript{me} de Montchevreuil, abbesse de Saint-Antoine, à
Paris. Elle était fort âgée, et soeur du feu marquis de Montchevreuil,
chevalier de l'ordre, si bien avec le feu roi et si intimement avec
M\textsuperscript{me} de Maintenon, duquel il a été parlé ici plusieurs
fois. Cette belle abbaye fut donnée à la fille aînée de
M\textsuperscript{me} la Duchesse, bossue et fort contrefaite de corps
et d'esprit, religieuse de Fontevrault, où elle n'avait pu durer, et
depuis longtemps au Val-de-Grâce, dont elle était le fléau, et le devint
de son abbaye.

J'eus aussi à regretter des amis. L'abbé de Saint-Herem, fils et frère
de deux évêques d'Aire, qui était d'une sûre et agréable compagnie, qui
savait, qui se conduisait très sagement, et qui de la naissance dont il
était, et le mérite qu'il avait, était fait pour remplir utilement les
premiers postes de l'Église.

Le marquis de Saint-Herem, son cousin, gouverneur de Fontainebleau, un
des plus honnêtes hommes que j'aie connus, avec qui j'avais passé ma
vie. Il n'était encore que dans la force de l'âge. Il avait eu la
survivance de Fontainebleau pour le fils qu'il laissa.

Enfin le comte de Cheverny, dans un âge fort avancé, dont j'ai parlé
souvent, que j'avais fait mettre dans le conseil des affaires
étrangères, qui fut après conseiller d'État d'épée et gouverneur de M.
le duc de Chartres, plus de titre que d'effet. Il n'avait point
d'enfants. Sa femme était gouvernante des soeurs de ce prince.

L'abbé de Verteuil mourut presque aussitôt après mon arrivée. On
m'accusa de l'avoir tué d'une indigestion d'esturgeon, dont, en effet,
il s'était crevé chez moi. C'était un excellent convive, homme de bonne,
plaisante et libre compagnie\,; médiocre ecclésiastique, avec de bonnes
abbayes, et charmant dans ses colères où on le mettait souvent. Il était
frère du feu duc de La Rochefoucauld, mais avec grande différence d'âge.
C'était un homme fort du monde et du meilleur.

L'évêque de Carcassonne, le dernier des Grignan, à soixante-dix-huit
ans. Il était frère du feu comte de Grignan, chevalier de l'ordre,
lieutenant général et commandant en Provence, gendre de
M\textsuperscript{me} de Sévigné.

Saint-Frémont, lieutenant général, fort entendu à la guerre, et qui n'y
avait pas négligé ses intérêts. C'était un homme de fortune, qui
s'appelait Ravend. Il se trouva lieutenant-colonel d'un régiment de
dragons, qu'eut un fils aîné de Villette, cousin germain de
M\textsuperscript{me} de Maintenon, fort protégé d'elle, et qui y fut
tué. Saint-Frémont, en habile homme qu'il était, s'y était attaché, et
M\textsuperscript{me} de Maintenon prit soin de l'avancer. Il eut l'art
d'être toujours au mieux avec les généraux des armées et avec les
ministres de la guerre. Homme d'esprit, de sens, de conduite, gaillard,
de bonne compagnie et fort honorable. Il était fort dans la bonne
compagnie partout. Il était extrêmement vieux, très bon officier
général, et avait prétendu au bâton.

J'appris, peu après mon arrivée, la mort à Madrid du marquis de
Montalègre, dont je fus affligé. Sa charge de sommelier de corps fut
destinée au duc d'Arion, en chemin de revenir des Indes\,; et celle de
majordome-major de la princesse des Asturies, qui lui était réservée,
fut donnée au duc de Bejar, et les hallebardiers au prince de Masseran.

La princesse Ragotzi mourut aussi dans un couvent, à Paris, où elle
était venue chercher à vivre, depuis que le prince Ragotzi était passé
en Turquie. On a vu ici ses singulières aventures, à l'occasion de
l'arrivée du prince Ragotzi à la cour. Elle était Hesse-Rheinfeltz, et
pour avoir tant fait parler d'elle, et en tant de pays, elle n'avait que
quarante-trois ans. Elle laissa deux fils, qui n'étaient pas faits pour
faire autant de bruit que leur père.

La duchesse de Zell, sur la fortune de laquelle il faut s'arrêter un
moment. Elle était fille d'Alexandre Desmiers, seigneur d'Olbreuse,
gentilhomme de Poitou, protestant, qui sortit du royaume à la révocation
de l'édit de Nantes, passa en Allemagne, et s'établit en Brandebourg, où
sa fille, belle et sage, fut fille d'honneur de l'électrice, veuve de
Charles-Louis duc de Zell, sans enfants en premières noces, et fille du
duc d'Holstein-Glucksbourg. Georges-Guillaume, frère du premier mari de
cette électrice, duc de Zell par la mort de son frère aîné, devint
amoureux de cette fille d'honneur de l'électrice, et l'épousa. Dans la
suite il obtint de l'empereur de la faire princesse de l'Empire pour
couvrir l'inégalité de ce mariage, et que leurs enfants, s'ils en
avaient, pussent succéder. Il mourut en août 1703, à quatre-vingt-un
ans, elle en février 1722, ne laissant qu'une fille mariée, 1682, à son
cousin germain Georges-Louis duc d'Hanovre, électeur et successeur de la
reine Anne à la couronne d'Angleterre, dont le fils y règne aujourd'hui,
et que son mari, jaloux d'elle, longtemps avant d'être roi d'Angleterre,
tint enfermée le reste de ses jours, après avoir fait jeter dans un four
ardent le comte de Koenigsmarck. Frédéric, frère cadet de
Christophe-Louis ci-dessus, et de Georges-Guillaume, avait usurpé le
duché de Zell sur Georges-Guillaume, mari dans la suite d'Éléonore
Desmiers, absent à la mort de leur père, qui par son testament avait
ordonné qu'Hanovre et Zell seraient chacun pour les deux aînés à
toujours. Georges-Guillaume conquit et garda le duché de Zell, et
Christophe-Louis demeura duc d'Hanovre. Il se fit catholique en 1657 et
mourut en 1679. Il avait épousé en 1667 Bénédicte-Henriette-Philippine,
palatine, soeur de la princesse de Salm et de la dernière princesse de
Condé, filles du second fils de l'électeur palatin, roi de Bohème, mort
proscrit en Hollande, dépouillé de tous ses États par l'empereur, sur
qui il avait usurpé la Bohême. Ainsi cette Éléonore-Desmiers Olbreuse
était belle-soeur de la duchesse d'Hanovre ou de Brunswick, que nous
avons vu mourir à Paris, au Luxembourg, il n'y a pas longtemps, et
belle-mère du second électeur d'Hanovre, premier roi d'Angleterre de sa
maison, et grand'mère du roi d'Angleterre, électeur d'Hanovre
d'aujourd'hui. Malgré l'inégalité de son mariage qui se pardonne si peu
en Allemagne, malgré les malheurs de sa fille, sa vertu et sa conduite
la firent aimer et respecter de toute la maison de Brunswick et du roi
d'Angleterre, son gendre, et considérer dans toute l'Allemagne.

Le comte d'Althan, grand écuyer de l'empereur, et son favori à quarante
ans. L'empereur ne le quitta point pendant sa maladie, et il mourut
entre ses bras. Il lui fit faire des obsèques magnifiques, se déclara le
tuteur de ses enfants, et nomma deux de ses ministres pour régir leurs
affaires et lui en rendre compte. Il est bien rare de voir l'amitié sur
le trône.

Je trouvai aussi quelques mariages faits. Ceux du prince de Soultzbach,
de la maison palatine, et de sa soeur avec le prince de Piémont. Lui
épousa l'héritière de Berg-op-Zoom, fille du feu prince d'Auvergne et
d'une soeur du duc d'Aremberg, desquels il a été parlé ici ailleurs.

Le marquis de Castries, chevalier d'honneur de M\textsuperscript{me} la
duchesse d'Orléans, avait perdu sa femme, son fils et sa belle-fille,
desquels on a parlé ici. Il ne lui restait aucune postérité. Il était
assez vieux et encore plus infirme, et ne se souciait pas trop de se
remarier. Son frère l'y engagea. Il était riche, et ne voulait pas
déchoir de sa première alliance. M\textsuperscript{me} de Saint-Simon
ménagea son mariage avec la fille du duc de Lévi, qui n'avait rien, et
qui dans la suite eut tout l'héritage par la mort de tous ses frères,
jeunes, et dont aucun ne fut marié. Elle était laide, mais avec beaucoup
d'esprit, et l'esprit fort aimable. Elle fut mère du marquis de Castries
d'aujourd'hui. Castries eut, en faveur de son mariage, cent cinquante
mille livres de brevet de retenue sur son gouvernement de Montpellier.

Puysieux épousa une fille de Souvré, fils de M. de Louvois et maître de
la garde-robe du roi. Il était fils de Sillery, écuyer de M. le prince
de Conti, gendre de M. le Prince, et neveu de Puysieux, ambassadeur en
Suisse, qui se fit chevalier de l'ordre par l'adresse qu'on a vue ici en
son temps.

Le duc d'Épernon, par la démission du duc d'Antin son père, épousa la
seconde fille du duc de Luxembourg\,;

Et M\textsuperscript{lle} d'Estrées, vieille fille, soeur du dernier duc
d'Estrées, déclara son mariage avec d'Ampus, gentilhomme provençal peu
connu, dont le nom est Laurent. Voilà les morts et les mariages que je
trouvai à mon arrivée, et voici les autres changements\,:

Le P. de Linières, jésuite, confesseur de Madame, bon homme, vieux et
rien de plus, fait confesseur du roi. On négocia fort avec le cardinal
de Noailles pour en obtenir des pouvoirs pour le roi, comme il en avait
donné à ce jésuite pour continuer d'entendre Madame\,; mais il fut
inflexible. Aussi était-il bien d'une autre importance de rendre le
confessionnal du roi aux jésuites que de laisser continuer Madame avec
son ancien confesseur, lorsque le cardinal de Noailles interdit les
jésuites. Le cardinal Dubois, qui n'en voulut pas avoir le démenti, fit
la plaie si éclatante à l'épiscopat de s'adresser à Rome, et le pape
envoya au roi un pouvoir de l'entendre en confession et de l'absoudre, à
quiconque il voudrait choisir, sans aucune exception.

Le chancelier, exilé à Fresnes, et d'Armenonville, garde des sceaux, et
son fils Morville, secrétaire d'État en sa place\,; Nocé, si bien et si
libre avec M. le duc d'Orléans, et qui avait été si longtemps l'intime
de Dubois, et celui par qui, étant à Hanovre et à Londres, ses lettres
passaient au régent, exilé à Blois\,; et Broglio, ce roué de M. le duc
d'Orléans, si impudent et si impie, chassé plus loin. Il y avait bien
longtemps qu'il le méritait, et pis. Le cardinal Dubois commença par ces
deux hommes, dont il craignait l'esprit hardi du premier, entreprenant
et audacieux du second, et la liberté et la familiarité de tous les deux
avec M. le duc d'Orléans, qui avait du goût et de l'amitié de tout temps
pour Nocé, fils du vieux Fontenay, qu'il avait fort estimé, et qui avait
été son sous-gouverneur. Tous d'eux avaient beaucoup d'esprit.

Le chancelier venait de marier sa fille au marquis de Chastelux, homme
de qualité de Bourgogne, du nom de Beauvoir, fort honnête homme, et
estimé à la guerre. L'arrêt du chancelier était intérieurement prononcé,
et M. le duc d'Orléans voulut ne rien déclarer que le mariage qui
s'allait faire ne fût achevé. Il en riait tout bas, et disait à ceux du
secret que ce pauvre Chastelux donnait dans le pot au noir, et s'allait
faire poissonnier la veille de Pâques. Il soutint ce subit exil de son
beau-père d'une façon respectable, et n'en vécut qu'avec plus de soins,
d'attentions et d'amitié pour sa femme, pour son beau-père et pour toute
sa famille.

M\textsuperscript{me} de Soubise, en fonction de gouvernante des enfants
de France, en survivance de la duchesse de Ventadour, grand'mère de son
mari, et Dodun, contrôleur général des finances, à la place de La
Houssaye, que son incapacité n'avait pu soutenir plus longtemps dans
cette place. Dodun, de président aux enquêtes, était passé dans les
conseils des finances, où il avait eu plusieurs commissions. Il avait de
la morgue et de la fatuité à l'excès, mais de la capacité, et autant de
probité qu'une telle place en peut permettre.

Pelletier de Souzy, qui était à la fin entré, comme il a été dit ici, au
conseil de régence, le quitta et se retira à Saint-Victor. Il était
doyen du conseil des parties. Il logeait avec des Forts, son fils, dans
une belle et agréable maison, qu'il avait bâtie, et toute sa vie avait
eu des emplois distingués, et vécu avec la meilleure compagnie, à qui il
faisait une chère fort recherchée. On crut que quelque mécontentement
qu'il eut de son fils lui fit prendre un parti dont il sentit le poids
et le vide, et qu'il ne soutint que par la honte de la variation.

Le duc d'Ossone était parti de Paris, qu'il avait rempli de sa
magnificence et des plus belles fêtes, lorsque j'y arrivai. Je ne le
rencontrai point en chemin.

Je trouvai l'archevêque de Tours, que j'avais voulu faire archevêque de
Reims, déjà transféré à Alby, et l'abbé d'Auvergne, nommé à Tours, passé
à Vienne. Tours fut donné à l'évêque de Toul, cet abbé de Camilly qui
avait eu cet évêché en récompense, comme je l'ai dit ailleurs, de tous
les tours de souplesse dont il avait si heureusement servi le cardinal
de Rohan, longtemps avant sa pourpre pour le faire recevoir dans le
chapitre de Strasbourg, où lui-même était alors depuis longtemps
chanoine du bas-choeur, et Toul fut donné à l'abbé Bégon, qui fut un
excellent évêque. Reims ne tarda pas à être donné à l'abbé de Guéméné
qui, pour le dire tout de suite, tenta bientôt, après avoir eu l'honneur
de sacrer le roi, d'être fait commandeur du Saint-Esprit, n'ayant pas
l'âge, car il était de 1695. Mais le propre des usurpateurs est de faire
semblant de se méconnaître pour que les autres les méconnaissent, et des
buts et des combles les plus désirés et les plus grands, de s'en faire
des degrés pour arriver à davantage. C'est par où les princes étrangers
vrais ou faux, sont parvenus où on les voit. Ainsi la Ligue ayant
conduit les Guise à tout ce qu'ils voulurent, à la couronne près qui
leur manqua, par des merveilles multipliées, les autres usurpations sont
demeurées à leur postérité, entre autres cette distinction qu'ils
imaginèrent après coup de fixer l'âge d'être capable d'être admis dans
l'ordre du Saint-Esprit, pour le mettre à trente-cinq ans, excepté pour
les princes du sang et pour les maisons souveraines, qu'ils firent
régler à vingt-cinq ans, pour s'égaler par là aux princes du sang, et à
côté d'eux se distinguer de tous les seigneurs. MM\hspace{0pt}. de Rohan
alors n'étaient que seigneurs\,; il s'en fallait bien que Louis XIV fût
né, ni M\textsuperscript{me} de Soubise, dont la beauté eut le don de
lui plaire, et elle d'en savoir si bien profiter. De gentilshommes, et
reçus comme tels dans l'ordre, comme on l'a vu du marquis de Marigny
tout à la queue des gentilshommes de la nombreuse promotion de 1619, où
il n'y en eut que cinq ou six après lui, quoique frère du duc de
Montbazon, devenus princes, et en ayant emblé par pièces la plupart des
distinctions peu à peu, rien ne se présentait plus à propos pour obtenir
l'ordre avant l'âge, et le tourner après en droit, que l'honneur d'avoir
sacré le roi avant l'âge de vingt-huit ans. Aussi l'occasion en fut-elle
saisie, et le malheur fut que la promotion fut différée au delà de la
vie du cardinal Dubois, qui sûrement ne les en eût pas éconduits, et
celle de M. le duc d'Orléans, qui ne sut comment la faire, pour avoir
promis quatre fois plus de colliers qu'il n'y en avait de vacants,
quoique presque toutes les places de l'ordre le fussent. M. le Duc, qui
la fit, ne jugea pas à propos d'accorder ce nouvel avantage à des gens
qui n'en avaient que trop usurpé, et qui voulaient persuader que tout
leur était dû. Encore que les charmes de M\textsuperscript{me} de
Soubise et la ténébreuse complaisance de son mari n'eussent pu obtenir
de Louis XIV un autre rang que parmi les gentilshommes, à lui et au
comte d'Auvergne, à la promotion de 1688, comme on l'a vu ici en
traitant de ces choses, où on a vu quelle fut la colère du roi et de
leurs refus, et par quel artifice l'exécution de ces ordres fut
corrompue sur les registres. Dans les promotions qui suivirent celle de
1728, cet archevêque de Reims ayant lors plus de trente-cinq ans, il ne
lui aurait pas été difficile d'y être compris\,; mais la distinction que
les Rohan s'y étaient proposée s'était évanouie avec les années. Il en
fallut donc chercher une autre ou un prestige pour éblouir dans la
suite\,: ce fut de n'entrer pas dans l'ordre après trente-cinq ans,
n'ayant pu y être admis auparavant, pour éviter d'en marquer la chasse.
M. de Reims prévint la chose de bonne heure. Ses nerfs furent attaqués
aussitôt après le sacre, en sorte qu'il ne marchait qu'avec une
difficulté qui s'est toujours augmentée, et qui lui en a enfin ôté
l'usage. Il déclara donc qu'il ne prétendait point à l'ordre, que la
faiblesse de ses jambes le mettait hors d'état de recevoir, et il s'en
est tiré de la sorte. Telles sont les entreprises, les artifices, les
ruses qui ont formé et enfin établi ces rangs prétendus étrangers, tant
en ceux qui sont en effet étrangers, qu'en ceux qui, à force de partager
avec eux, sont devenus honteux de ne pas l'être. La France est l'unique
pays de l'Europe où de tels abus, si dangereux et si flétrissants,
soient soufferts, et dont la Ligue est l'odieuse date, et qui porte avec
elle toute instruction trop souvent depuis bien rafraîchie.

À peine fus-je arrivé qu'il fallut achever un mariage qui m'avait été
proposé pour ma fille, avant que j'allasse en Espagne. Il y a des
personnes faites de manière qu'elles sont plus heureuses de demeurer
filles avec le revenu de la dot qu'on leur donnerait.
M\textsuperscript{me} de Saint-Simon et moi avions raison de croire que
la nôtre était de celles-là, et nous voulions en user de la sorte avec
elle. Ma mère pensait autrement, et elle était accoutumée à décider. Le
prince de Chimay se persuada des chimères en épousant ma fille dans la
situation où il me voyait. Dès avant d'aller en Espagne je ne lui
déguisai rien de tout ce que je pensais, ni du peu de fondement de tout
ce qui le persuadait de faire ce mariage. Je ne le voulus achever qu'à
mon retour, pour lui laisser tout le temps aux réflexions et au
refroidissement en mon absence. Il ne cessa de presser
M\textsuperscript{me} de Saint-Simon, ni elle de l'en détourner. Dès que
je fus de retour ses instances redoublèrent à un point qu'il fallut
conclure, et le mariage se fit à Meudon avec le moins de cérémonie et de
compagnie qu'il nous fut possible. Son nom était Hennin-Liétard, et ses
père et mère connus sous le nom de comte de Bossut, par leurs alliances,
leurs grands biens dans les Pays-Bas, et leurs grands emplois sous
Charles-Quint et depuis. Leur chimère était d'être de l'ancienne maison
d'Alsace, quoique la leur fût d'une antiquité assez illustre et assez
reconnue pour ne la pas barbouiller de fables. Néanmoins son frère, qui
était archevêque de Malines, avec une grande abbaye et cardinal, portait
hardiment le nom de cardinal d'Alsace, quoique espèce de béat. Lui et
son autre frère, le marquis de La Vère, étaient lieutenants généraux, et
fort distingués par leur valeur et leur service en Espagne, et en
quittant ce pays-là pour celui-ci, y avaient conservé le même grade. On
a vu ici ailleurs que l'électeur de Bavière fit donner la Toison d'or au
prince de Chimay tout jeune par Charles II. Il se signala en Flandre,
dans la guerre qui suivit la mort de ce monarque, par des actions fort
distinguées. Il passa ensuite en Espagne, où il fit sa cour à la
princesse des Ursins, qui le fit grand d'Espagne, et il y servit avec la
même distinction. Il s'y ennuya ensuite et vint en France, où il épousa
la fille du duc de Nevers, qu'il perdit quelques années après, sans
enfants. C'était un homme très bien fait, d'un visage très agréable,
dont l'air et toutes les manières sentaient le grand seigneur\,: aussi
l'était-il par de grandes et de belles terres, mais la plupart, de
longue main, en direction, et ses affaires fort embarrassées, dont il ne
laissait pas de tirer gros. C'était, de plus, un homme sans règle, qui,
avec de l'esprit et les meilleurs discours, se gouvernait lui et ses
affaires de fort mauvaise façon, plein de chimères et de fantaisies. La
duchesse Sforze, de chez qui il ne bougeait tous les soirs, tant que son
premier mariage dura, me prédit bien tout ce que j'en vis dans la suite.
Son frère avait quitté l'Espagne par la disgrâce du duc d'Havré, dans
laquelle il fut enveloppé, comme elle a été racontée ici en son temps.

Il se fit peu de jours après un autre mariage chez moi, à Meudon, de la
soeur de l'abbé de Saint-Simon avec le comte de Laval, maréchal de camp
alors, et enfin devenu maréchal de France. Son nom et cette juste
récompense de ses longs services dispensent d'en dire davantage.
M\textsuperscript{me} de Saint-Simon avait pris grand soin de cette
jeune personne, et l'eut chez elle tant que je fus en Espagne. Elle
était fort jolie, et son air de douceur, de modestie et de retenue
plaisait extrêmement. Le dedans était fort au-dessus du dehors\,: de
l'esprit, de l'agrément, de la gaieté, une piété et une vertu qui ne se
sont jamais démenties et qui n'ont effarouché personne\,; fort propre au
monde, et une conduite qui a infiniment aidé la fortune de son mari. Il
voulait une alliance et des entours qui le pussent porter. Il eut, en se
mariant, un petit gouvernement, et sa femme une pension.

Courtenvaux mourut fort jeune. Il était fils aîné du fils aîné de M. de
Louvois. Sa mère était soeur du maréchal d'Estrées, et sa femme soeur du
duc de Noailles, et il laissait un fils qui sortait tout au plus du
maillot. Il avait eu la belle charge de son père de capitaine des
Cent-Suisses. L'âge de l'enfant était ridicule\,; les services ni la
naissance n'y suppléaient pas. Néanmoins la facilité et le mépris de
toutes choses de M. le duc d'Orléans enhardirent le duc de Villeroy et
le maréchal d'Estrées\,: M. le duc d'Orléans ne put leur résister, et
l'enfant eut la charge. Le frère cadet de son père l'exerça en plein en
attendant que l'enfant fût d'âge à la faire.

On résolut enfin que le roi abandonnerait Paris pour toujours, et que la
cour se tiendrait à Versailles. Le roi s'y rendit en pompe le 15 juin,
et l'infante le lendemain. Ils y occupèrent les appartements du feu roi
et de la feue reine, et le maréchal de Villeroy fut logé dans les
derrières des cabinets du roi. Le cardinal Dubois eut toute la
surintendance entière pour lui seul, comme M. Colbert l'avait eue, et
après lui M. de Louvois. Il suivait à grands pas son projet de se faire
déclarer premier ministre, et pour cela d'isoler tant qu'il pourrait M.
le duc d'Orléans. Paris rendait son accès facile à bien des gens qui ne
pouvaient s'établir à Versailles ni y aller, les uns point du tout, les
autres que rarement et des moments. Ce changement dérangeait les soupers
avec les roués et des femmes qui ne valaient pas mieux. Il comprenait
bien que M. le duc d'Orléans les irait trouver à Paris tant qu'il
pourrait, mais que les affaires qu'il saurait lui présenter à propos le
dérangeraient souvent\,; que cette contrainte le dégoûterait,
l'ennuierait, et plus que toute autre chose, le préparerait à se
décharger sur lui, et pour acheter sa liberté, le déclarer premier
ministre et le supplément en titre de ses absences, qui ne seraient
plus, ou que bien rarement contrariées par les affaires, dont lui,
cardinal, devenu publiquement le maître, saurait bien se faire redouter,
de manière qu'il n'aurait rien à craindre des voyages de son maître à
Paris, où il le laisserait se replonger, dans sa petite loge de l'Opéra,
dans ses indignes soupers, s'éloigner des affaires, et lui, en profiter
pour voler de ses ailes et régner de son chef. M. le duc d'Orléans prit
l'appartement de feu Monseigneur en bas, et M\textsuperscript{me} la
duchesse d'Orléans demeura dans celui qu'elle avait en haut auprès du
sien, qui resta vide.

Quoique mon retour d'Espagne et ma conduite à l'égard de ceux qui
étaient sortis du conseil lui eussent fort déplu, et que ma résistance
au dépouillement du duc de Noailles lui eût donné un dépit qu'il ne me
pardonna jamais, il n'était pas temps encore de me le montrer. Je ne
trouvai donc point d'obstacle à ma familiarité ordinaire avec M. le duc
d'Orléans\,; le cardinal même m'en témoignait avec un mélange de
déférence. Mais sur ce qui était affaires autres que menues ou de cour,
j'en étais peu instruit, et par-ci, par-là, par morceaux, que l'habitude
arrachait à M. le duc d'Orléans dans mes tête-à-tête avec lui, sans
néanmoins que je l'y excitasse. Ce n'était pas pourtant que le cardinal
ne m'offrît de me les communiquer toutes. Il voulait que la réserve que
j'y éprouvais depuis mon retour tombât sur M. le duc d'Orléans. Mais,
outre que ce prince m'en disait trop pour que je ne visse pas à
découvert qu'il ne tenait pas à lui qu'il ne me dît tout comme
auparavant, je connaissais trop le cardinal pour être la dupe de ses
offres et de ses compliments. Il ne savait par où s'y prendre pour
m'éloigner de M. le duc d'Orléans\,: il me craignait pour sa déclaration
de premier ministre\,; il voulait également m'écarter des affaires et me
ménager, tellement qu'il m'accablait de gentillesses toutes les fois que
je le rencontrais, et s'y surpassait quand le hasard me faisait trouver
en tiers avec M. le duc d'Orléans et lui, tant pour me cajoler que pour
persuader à ce prince qu'il n'oubliait rien pour être bien avec moi,
m'embarrasser par là à résister aux choses qu'il entreprenait, et
affaiblir ce que je pourrais dire contre lui à M. le duc d'Orléans, sur
les choses où nous ne nous trouverions pas de même avis.

Il s'en présenta bientôt une\,: ce fut l'exil du duc de Noailles, dont
il n'osa pas me parler\,; mais M. le duc d'Orléans me le dit comme une
chose dont on le pressait. Je lui demandai à propos de quoi cet exil, et
il ne put me rien alléguer que de vague et de ce fantôme de cabale. Je
lui répondis qu'à ce dernier égard, où, depuis mon retour, je n'avais pu
apercevoir rien de réel, je ne voyais pas pourquoi l'exiler plutôt que
les autres\,; qu'il s'était contenté jusqu'alors de l'exil du
chancelier, qui était bien assez éclatant, et dont, à maints égards, il
aurait bien pu se passer\,; qu'y revenir sur d'autres sans cause
nouvelle et connue, c'était montrer un bâton levé sur les maréchaux de
Villars, d'Estrées, Tallard, Huxelles, même sur le maréchal de Villeroy,
qui étaient en personnages les principaux de ceux qui étaient sortis du
conseil, et qu'on lui donnait avec le duc de Noailles pour les prétendus
chefs de la prétendue cabale\,; que d'effaroucher tant de gens
considérables, considérés et si grandement établis, je n'en voyais que
du mal à attendre, et aucun bien à espérer\,; et qu'à l'égard du duc de
Noailles, il était, à mon avis, de ceux qu'il ne fallait jamais
bistourner\footnote{Mutiler, châtrer.} pour quelque cause que ce pût
être, mais le laisser entier ou l'écraser à forfait\,; qu'écraser un
homme, lui, et tous les siens, si grandement établi, et qui avait eu si
longtemps sa confiance et toutes les finances entre les mains, je n'y
voyais ni justice ni possibilité, sans crime qui pût être clairement
démontré, et tel que ce fût grâce que de ne pas {[}le{]} faire traiter
juridiquement par les formes\,; qu'un exilé, surtout de sa sorte, ne
pourrissait pas exilé\,; qu'on touchait à la majorité\,; que, de retour,
sa charge de capitaine des gardes l'approcherait nécessairement du
roi\,; qu'il n'oublierait ni fadeurs, ni bassesses, ni fertilité
d'esprit pour se l'apprivoiser, se familiariser, se rendre agréable, se
donner un crédit immédiat, se rallier les mécontents de la régence, qui
approcheraient le roi par leurs emplois ou par leur industrie\,; et
qu'alors Son Altesse Royale aurait tout lieu de se repentir de s'être
fait inutilement un ennemi du duc de Noailles, et de l'avoir laissé en
état et en moyens de s'en ressentir. J'ajoutai que, quand je m'opposais
à l'exil du duc de Noailles, ma voix en valait bien une douzaine
d'autres dans la situation publique où j'étais avec lui.

Cette proposition d'exil balança huit jours, pendant lesquels le
cardinal me détachait sans cesse Belle-Ile pour m'exorciser par ma haine
et par mon intérêt, et me dire ce que le cardinal n'osait lui-même, pour
n'avoir pas à se fâcher de la persévérance de mon opposition. Elle
l'emporta toutefois et m'indisposa le cardinal de plus en plus. Mais je
ne pus me résoudre de servir ses projets ni ma haine aux dépens de M. le
duc d'Orléans. Cette suspension d'exil ne fut pas longue.

Cinq semaines ou environ après, que je pensais qu'il n'en fût plus
question du tout, j'allai au Palais-Royal (car de Meudon, que
j'habitais, je voyais M. le duc d'Orléans à Versailles et à Paris, quand
il y était, les jours destinés par moi à le voir), et je trouvai La
Vrillière seul dans la petite galerie avant son petit cabinet, laquelle
était toujours vide, et on attendait dans la pièce qui la précédait.
Surpris de le voir là et encore plus de l'heure qui n'était pas la
sienne, je lui demandai ce qu'il y faisait. Il me dit qu'il avait un mot
à dire à M. le duc d'Orléans. J'entrai tout de suite dans le cabinet où
il était seul, avec l'air assez embarrassé. Je lui demandai ce qu'il y
avait, que La Vrillière était dans la petite galerie. «\,C'est pour
fondre la cloche, me répondit-il. ---Comment\,? dis-je, quelle cloche\,?
--- L'exil du duc de Noailles, reprit-il. --- Comment, lui dis-je, après
{[}avoir{]} senti et goûté la force de tout ce que je vous ai représenté
là-dessus\,! En vérité, monsieur, vous n'y pensez pas.\,» Et tout de
suite je repris les principales raisons. Nous étions debout. Alors il se
mit à se promener, la tête basse, par ce cabinet, quoique fort petit,
comme il faisait toujours quand il se trouvait debout et embarrassé de
quelque chose. Cette promenade et mon discours, avec peu de répliques de
sa part et faibles, dura un bon quart d'heure. Le silence succéda,
pendant lequel il se mit le nez tout contre les vitres de la fenêtre,
puis, se tournant à moi, me dit tristement\,: «\,Le vin est tiré, il
faut le boire.\,» Je vis qu'il avait combattu, qu'il sentait que j'avais
raison, mais qu'il craignait le cardinal, qui lui avait arraché la
chose. Je haussai les épaules et baissai la tête, en lui disant qu'il
était le maître, que je souhaitais qu'il s'en trouvât bien. Là-dessus il
alla ouvrir la porte de son cabinet, appela La Vrillière, lui parla
quelques moments, presque dans la porte. L'affaire fut ainsi consommée,
et le duc de Noailles eut son ordre le soir même, partit le lendemain
matin, et s'en alla dans ses terres, près du vicomté de Turenne, où il
fit le béat, porta chape aux processions et aux lutrins de ses
paroisses, et se fit moquer de lui là et à Paris, où on le sut, et où,
pour mieux faire sa cour au régent, il entretenait une comédienne depuis
le commencement de la régence, après avoir dit son bréviaire, fait les
carêmes, et fréquenté les saluts de la chapelle assidûment depuis son
retour d'Espagne jusqu'à la mort du feu roi, pour se raccommoder avec
lui et avec sa tante de Maintenon, à quoi il ne put réussir.

Défait du duc de Noailles, le cardinal Dubois, qui ne pouvait avoir la
même prise sur le maréchal de Villeroy, n'oublia rien pour le gagner.
Quelque justement perdu qu'il fût dans l'esprit de M. le duc d'Orléans,
il lui imposait toujours par habitude de jeunesse\,; et, comme il était
fat jusqu'au point de se croire invulnérable et de s'en vanter, il se
piquait de ne rien craindre, et, pour s'en mieux parer, il tenait
souvent des propos fort hardis au régent, qu'il paraphrasait au double
au public, où le cardinal Dubois n'était pas ménagé. Je viens de parler
de sa fatuité\,; il en venait de donner un rare spectacle à Paris. La
fête du saint-sacrement arriva cette année le 4 juin, et le roi n'alla à
Versailles s'établir que le 15. Il reconduisit la procession du
saint-sacrement, venue à la chapelle des Tuileries, jusqu'à
Saint-Germain l'Auxerrois où il entendit la grand'messe. Le maréchal de
Villeroy, à qui la goutte ne permettait guère de marcher sur le pavé des
rues, ne crut pas devoir perdre le roi de vue, depuis les Tuileries
jusqu'à sa paroisse, quoique environné de sa cour et de ses gardes, et
adoré alors à Paris, ni perdre une si belle occasion de se donner en
spectacle\,; il monta le plus petit bidet qu'il put trouver, sur lequel
il suivit le roi pas à pas, et se fit admirer de la populace et moquer
par tout ce qui accompagna le roi. Ce maréchal jetait donc un véritable
fléau pour le cardinal Dubois, sur lequel ni crainte, ni prudence, ni
bienséance même n'avaient aucune prise. Il ne pouvait souffrir
l'autorité que le cardinal Dubois avait prise dans les affaires, ni
supporter le rang, l'état et la préséance d'un homme qu'il avait vu si
longtemps ramper dans l'antichambre du chevalier de Lorraine, et qu'il
croyait combler alors d'un léger signe de tête en passant. Il n'y eut
donc rien que le cardinal Dubois ne fit pour arrêter une langue si
accablante à force de soumissions. Il se mit presque à ses genoux, il le
supplia de trouver bon qu'il lui apportât son portefeuille tous les
jours, entrer dans tout ce qu'il y aurait de plus secret, le conduire et
le rectifier par ses lumières.

Tout vain et tout borné que fût le maréchal de Villeroy, le long usage
du grand monde et de la cour, et la connaissance qu'il avait de longue
main du cardinal Dubois lui en avait assez appris pour ne pas compter
beaucoup sur de si grandes offres, ni pour croire qu'un homme de ce
caractère, qui dominait le régent, pût s'accommoder sérieusement de se
mettre en brassière sous lui. D'ailleurs, les chimères du maréchal ne
pouvaient s'accommoder d'entrer en part du gouvernement de M. le duc
d'Orléans\,; elles étaient de fronder, de faire contre, d'être le chef
et le ralliement des mécontents et des frondeurs, l'idole du peuple,
l'amour du parlement, surtout l'homme unique à la vigilance duquel toute
la France était redevable de la vie du roi. Établi sur de si beaux
principes, certain d'ailleurs de ne pouvoir être ébranlé depuis que, par
deux fois, il se fut rassuré sur sa place, dont pourtant il ne m'avait
pu pardonner la frayeur, on peut juger du peu de succès des bassesses du
cardinal Dubois, et combien elles gonflèrent la superbe et la morgue de
l'un, et augmentèrent le dépit et la rage de l'autre. Il les cacha tant
qu'il put, et redoubla d'efforts auprès de M. le duc d'Orléans pour lui
faire chasser le maréchal de Villeroy. C'est où ils en étaient les
quinze derniers jours de juin, qui furent les premiers quinze jours de
l'établissement de la cour à Versailles.

La duchesse de Ventadour, en pleine et seule possession de l'infante,
avec ce nouveau degré de faveur de sa survivance à sa petite-fille, tira
habilement sur le temps, et on fut tout étonné qu'il arrivât d'Espagne
une grandesse au comte de La Mothe, fils du frère aîné du maréchal de La
Mothe, père de la duchesse de Ventadour, et des duchesses d'Aumont et de
La Ferté. Une si heureuse fortune le consola du bâton de maréchal de
France, qu'on mourait d'envie de lui donner, et, comme on l'a vu en son
lieu, qu'il n'eut pas l'esprit de mériter. Il mourut en 1728, à
quatre-vingt-cinq ans.

Plancy mourut au même âge, sans alliance\,; le dernier des enfants de
Guénégaud, secrétaire d'État, après avoir servi et fort ennuyé, le
monde. Les ministres n'avaient pu encore parvenir à laisser leurs
enfants revêtus de l'image et des charges des seigneurs.

Le pape, accablé enfin par les troupes impériales qui désolaient l'État
ecclésiastique, donna à l'empereur l'investiture du royaume de Naples et
de Sicile dont il était en possession. L'Espagne éclata, mais il n'en
fut autre chose, sinon de se raccommoder après.

Le fameux Marlborough mourut à Londres, le 27 juin, à près de
soixante-quatorze ans, le plus riche particulier de l'Europe, mais sans
postérité masculine. Sa soeur était mère du duc de Berwick et l'avait
fait comte de Marlborough et capitaine des gardes du roi Jacques II
d'Angleterre. Il était de petite noblesse et fort pauvre. Il se nommait
Jean Churchill, et il était devenu duc de Marlborough, pair
d'Angleterre, capitaine général des armées, grand maître de
l'artillerie, colonel du premier régiment des gardes, chevalier de
l'ordre de la Jarretière et le plus heureux capitaine de son siècle. Sa
vie, ses actions, ses fortunes sont si connues qu'on s'en taira ici. Sa
victoire d'Hochstedt le fit prince de l'Empire et de Mindelheim, terre
dont l'empereur lui fit présent en même temps. Pour en perpétuer la
mémoire, il avait fait bâtir en Angleterre un château superbe auquel il
donna le nom de Pleintheim\footnote{Le nom est ainsi écrit dans
  Saint-Simon. La forme ordinairement adoptée est Bleinheim ou
  Blindheim.} , village où trente-six bataillons retranchés se rendirent
à lui sans attendre d'attaque. Les honneurs de ses obsèques et leur
magnificence égalèrent, à peu de chose près, celles des rois
d'Angleterre. Il fut inhumé à Westminster, dans la chapelle de Henri
VII\,; mais cet honneur n'est pas rare en Angleterre. Il y avait plus de
trois ans qu'une apoplexie l'avait tellement affaibli qu'il pleurait
presque sans cesse et n'était plus capable de rien.

Le grand maître de Malte, frère du cardinal Zondedari, mourut en ce même
temps, fort estimé et regretté dans son ordre. Antoine Manoel lui
succéda, des anciens bâtards de Portugal.

La duchesse de Bouillon (Simiane) mourut aussi à trente-neuf ans à
Paris.

Et l'épouse du prince Jacques Sobieski, fils aîné du fameux roi de
Pologne. Elle était soeur des électeurs de Mayence et palatin, de
l'impératrice, mère des empereurs Joseph et Charles VI, des reines
douairières d'Espagne et de Portugal, et mère de la reine d'Angleterre,
épouse du roi Jacques III, résidant à Rome. Elle mourut sans postérité
masculine, à cinquante ans, dans les terres du prince Jacques Sobieski
en Silésie.

\hypertarget{chapitre-xiv.}{%
\chapter{CHAPITRE XIV.}\label{chapitre-xiv.}}

1722

~

{\textsc{Extrême embarras du cardinal Dubois, qui tente encore de se
ramener le maréchal de Villeroy, qu'il ne pouvait perdre, et y emploie
le cardinal de Bissy.}} {\textsc{- Le cardinal de Bissy persuade le
maréchal de Villeroy, qui veut prévenir le cardinal Dubois, et va chez
lui avec le cardinal de Bissy, où, passant des compliments aux injures,
il fait la plus terrible scène qui se puisse imaginer au cardinal
Dubois.}} {\textsc{- Le cardinal Dubois, hors de lui, arrive tout de
suite dans le cabinet de M. le duc d'Orléans, m'y trouve seul, lui conte
devant moi la scène qu'il venait d'essuyer du maréchal de Villeroy, et
déclare qu'il faut opter entre l'un ou l'autre.}} {\textsc{- M. le duc
d'Orléans me presse de dire mon avis.}} {\textsc{- J'opine à l'exil du
maréchal de Villeroy.}} {\textsc{- Conférence entre M. le duc d'Orléans,
M. le Duc et moi, où il est convenu d'arrêter et d'exiler le maréchal de
Villeroy.}} {\textsc{- M. le duc d'Orléans m'envoie chez le cardinal
Dubois, au sortir de notre conférence, examiner et convenir de la
mécanique pour arrêter le maréchal de Villeroy.}} {\textsc{- Compagnie
que je trouve chez le cardinal Dubois.}} {\textsc{- Le duc de Charost,
en mue, pour être déclaré gouverneur du roi.}}

~

Le cardinal Dubois ne fut pas longtemps à sentir qu'il ne persuaderait
pas M. le duc d'Orléans de chasser le maréchal de Villeroy. C'était un
tour de force dont il avait senti tous les inconvénients toutes les fois
qu'il avait été tenté de l'entreprendre, qui devenait tous les jours
plus difficile et plus dangereux, auquel il avait tout à fait renoncé.
Chaque jour que le cardinal différait à se faire déclarer premier
ministre lui semblait une année, et toutefois il n'osait presser ce
grand pas sans s'être mis à couvert des vacarmes qu'en ferait le
maréchal de Villeroy, qui donneraient le signal et l'encouragement à
tant d'autres, lesquels, sans cet appui, n'oseraient parler haut, et
dont le groupe et les assauts que le maréchal se piquerait de donner au
régent feraient courir grand risque au cardinal d'être aussitôt
précipité qu'élevé à cette immense place, et par cela même fort éreinté
et en situation de regretter celle où il était auparavant. L'agitation
de ces pensées et la difficulté de se dépêtrer de l'embarras qui
l'arrêtait, l'occupait tout entier, redoublait ses humeurs et ses
caprices, le rendait de plus en plus inabordable, et jetait les affaires
les plus importantes et les plus pressées dans un entier abandon. Enfin,
il se résolut de faire encore un effort vers le maréchal de Villeroy\,;
mais n'osant plus s'y hasarder lui-même, il imagina de s'y prendre par
le cardinal de Bissy, charmé de sa conduite sur la constitution et du
confessionnal du roi si récemment rendu à ses bons amis les jésuites,
et, ce qui ne le touchait guère moins, en bravant le cardinal de
Noailles et le refus de ses pouvoirs.

Dubois lui fit donc part de ses peines, de la dureté de la conduite du
maréchal de Villeroy à son égard, de tous les devoirs où il s'était mis,
de tout ce qu'il avait tenté auprès de lui pour en obtenir une paix
qu'il n'avait jamais déméritée, et si nécessaire au bien des affaires et
à la bienséance, qui ne l'était pas moins entre un homme à qui le roi
était confié, et celui à qui le régent remettait le détail et le
principal soin des affaires. Il lui représenta le grand bien qui
naîtrait infailliblement du frein que sa médiation pourrait seule mettre
aux saillies continuelles du maréchal de Villeroy, le disposer à vouloir
bien le regarder comme un homme qui ne lui avait jamais manqué, qui
n'avait cessé, dans tous les temps, de mériter l'honneur de ses bonnes
grâces, qu'il n'avait rien oublié pour qu'il lui voulût permettre de lui
porter son portefeuille, et de lui faire part de toutes les affaires
avec la déférence la plus entière\,; enfin, qu'il espérait cette bonne
oeuvre de son amour pour le bien et de l'amitié du maréchal de Villeroy
pour lui, qui ferait bien recevoir les réflexions qu'il lui ferait
faire. L'intime et commune liaison du maréchal et du cardinal avec filme
de Maintenon, les intrigues de la constitution, la haine du cardinal de
Noailles, que le maréchal avait adoptée en bas courtisan, et fortifiée,
depuis la régence, par celle du duc de Noailles, avait uni Villeroy et
Bissy d'une manière étroite.

L'ambitieux béat saisit avidement une occasion si honnête et si décente
de rendre à son confrère un service si désiré. Parvenu de si loin où en
était Bissy, son étonnante fortune ne lui semblait guère que des degrés
pour se porter plus haut. Il voulait faire une grande fortune à son
neveu, et depuis qu'il voyait l'entrée du conseil ouverte aux cardinaux,
il désirait beaucoup d'y faire le troisième. Outre l'éclat qui en
résulterait pour lui, il comptait que c'était la voie la plus certaine
d'avancer son neveu à tout, et que, venant à bout de tirer du pied de
Dubois une si fâcheuse épine, et de le mettre en bonne intelligence avec
Villeroy, par conséquent de le rapprocher du régent, il n'y avait rien
qu'il ne pût se promettre de Dubois, et par lui de son maître. Il
travailla donc à bon escient auprès du maréchal de Villeroy, et fit si
bien qu'il le persuada et qu'il le pria d'en porter de sa part parole au
cardinal Dubois. Voilà les deux cardinaux au comble de leur joie. Dubois
pria Bissy de dire à Villeroy tout ce que la sienne pouvait exprimer de
plus touchant, et qu'il brûlait d'impatience qu'il lui permît d'aller
chez lui l'en assurer lui-même. Bissy ne tarda pas à exécuter une si
agréable commission, et Villeroy, pour ne demeurer pas en reste, convint
avec Bissy d'aller ensemble chez le cardinal Dubois. Le hasard fit
qu'ils y allèrent un mardi matin, et que je ne me souviens plus quelle
affaire me fit aller en même temps, contre mon ordinaire, parler à M. le
duc d'Orléans à Versailles, de Meudon, où j'habitais.

Bissy et Villeroy trouvèrent tous les ministres étrangers, dont c'était
le jour d'audience du cardinal Dubois, qui attendaient chacun la leur
dans la pièce d'avant le cabinet du cardinal. De longue main, l'usage
établi de ces audiences est que les ministres étrangers n'y étaient
introduits {[}que{]} l'un après l'autre, suivant qu'ils étaient arrivés
dans la pièce d'attente, pour éviter toute compétence\footnote{Rivalité.}
de rang entre eux. Ainsi Bissy et Villeroy trouvèrent Dubois enfermé
avec le ministre de Russie. On voulut avertir le cardinal de quelque
chose d'aussi nouveau que le maréchal de Villeroy chez lui, mais il ne
le voulut pas permettre, et s'assit avec Bissy sur un canapé en
attendant.

L'audience finie, Dubois sortit de son cabinet pour conduire
l'ambassadeur, et aussitôt avisa ce canapé si bien garni. Il ne vit plus
que lui à l'instant\,; il y courut, rendit mille hommages publics au
maréchal, avec force plaintes d'être prévenu, lorsqu'il n'attendait que
sa permission pour aller chez lui, et pria Bissy et lui de passer dans
son cabinet. Tandis qu'ils y allèrent, il en fit excuse aux ambassadeurs
sur ce que les fonctions et l'assiduité du maréchal de Villeroy auprès
du roi ne lui permettaient pas de s'absenter pour longtemps d'auprès de
sa personne\,; et, avec ce compliment, les quitta et rentra dans son
cabinet. D'abord, force compliments réciproques et propos du cardinal de
Bissy convenables au sujet. De là protestations du cardinal Dubois et
réponses du maréchal\,; mais à force de réponses, il s'empêtra dans le
musical de ses phrases, bientôt se piqua de franchise et de dire des
vérités, puis, peu à peu, s'échauffant dans son harnais, des vérités
dures et qui sentaient l'injure. Dubois, bien étonné, ne fit pas
semblant de sentir la force de ces propos\,; mais comme elle
s'augmentait de moment à autre, Bissy, avec raison, voulut mettre le
holà, interrompre, expliquer en bien les choses, persuader le maréchal
quelle était son intention. Mais la marée qui montait toujours tourna
tout à fait la tête au maréchal, et le voilà aux injures et aux plus
sanglants reproches. En vain Bissy le voulut faire taire, lui
représenter de combien il s'écartait de ce qu'il lui avait promis et
chargé de rapporter à Dubois, l'indécence sans exemple d'aller
maltraiter un homme chez lui, où il ne venait que pour achever de
consommer une réconciliation conclue. Tout ce que put dire Bissy ne fit
qu'animer le maréchal, et lui faire vomir tout ce que l'insolence et le
mépris peuvent suggérer de plus extravagant. Dubois, confondu et hors de
lui-même, rentrait en terre sans proférer un seul mot, et Bissy,
justement outré de colère, tâchait inutilement d'interrompre. Dans le
feu subit qui avait saisi le maréchal, il s'était placé de façon qu'il
leur avait bouché le passage pour sortir, et en disait toujours de plus
belle. Las d'injures, il se mit sur les menaces et sur les dérisions, il
dit à Dubois que maintenant qu'il s'était montré à découvert, ils
n'étaient plus en termes de se pardonner l'un à l'autre\,; qu'il voulait
bien encore l'avertir que tôt ou tard il lui ferait du pis qu'il
pourrait, mais qu'il voulait bien aussi, avec la même candeur, lui
donner un bon conseil. «\,Vous êtes tout-puissant, ajouta-t-il\,; tout
plie devant vous, rien ne vous résiste\,; qu'est-ce que les plus grands
en comparaison de vous\,? Croyez-moi, vous n'avez qu'une seule chose à
faire, usez de tout votre pouvoir, mettez-vous en repos, et faites-moi
arrêter, si vous l'osez. Qui pourra vous en empêcher\,? Faites-moi
arrêter, vous dis-je, vous n'avez que ce parti à prendre.\,» Et
là-dessus, à paraphraser, à défier, à insulter en homme qui très
sincèrement était persuadé qu'entre escalader les cieux et l'arrêter, il
n'y avait point de différence. On peut bien s'imaginer que tant de si
étonnants propos ne furent pas tenus sans interruptions et sans vives
altercations du cardinal de Bissy, mais sans en pouvoir arrêter le
torrent. Enfin, outré de colère et de dépit contre le maréchal qui lui
manquait si essentiellement à lui-même, il saisit le maréchal par le
bras et par les épaules et l'entraîna à la porte qu'il ouvrit, et le fit
sortir et sortit lui-même. Dubois, plus mort que vif, les suivit comme
il put\,; il se fallait garder de cette assemblée de ministres étrangers
qui attendaient. Tous trois eurent beau tacher de se composer, il n'y
eut aucun de ces ministres qui ne s'aperçût qu'il fallait qu'il se fût
passé quelque scène violente dans le cabinet, et aussitôt Versailles fut
rempli de cette nouvelle, qui fut bientôt éclaircie par les vanteries,
les récits, les défis et les dérisions publiques du maréchal de
Villeroy.

J'avais travaillé et causé longtemps avec M. le duc d'Orléans. Il était
passé dans sa garde-robe, j'étais debout derrière son bureau, où
j'arrangeais des papiers, lorsque je vis entrer le cardinal Dubois comme
un tourbillon, les yeux hors de la tête, qui me voyant seul, s'écria
plutôt qu'il ne demanda, où était M. le duc d'Orléans. Je lui dis qu'il
était entré dans sa garde-robe, et lui demandai à qui il en avait,
éperdu comme je le voyais. «\,Je suis perdu, je suis perdu,\,» dit-il,
et courut à la garde-robe. Il répondit si haut et si bref que M. le duc
d'Orléans, qui l'entendit, accourut presque de son côté, et le
rencontrant dans la porte, {[}ils{]} revinrent vers moi, lui demandant
ce que c'était. Sa réponse, entrecoupée de son bégaiement ordinaire, que
la rage et la frayeur augmentait, fut en bien plus longs détails le
récit que je viens de faire, après lequel le cardinal déclara au régent
que c'était à son Altesse Royale à sentir où tendait le maréchal de
Villeroy par un guet-apens aussi inouï et aussi peu mérité, paraphrasa
tout ce qu'il avait employé auprès de lui uniquement pour le bien des
affaires et le service de M. le duc d'Orléans, et conclut qu'après une
insulte de cette nature, et si faussement et traîtreusement préméditée,
il fallait que M. le duc d'Orléans vit tout à l'heure ce qu'il pouvait
et ce qu'il voulait faire, et choisît entre le maréchal de Villeroy et
lui, parce qu'il ne pouvait plus se mêler d'aucune affaire, ni rester à
la cour en honneur et en sûreté si le maréchal de Villeroy y demeurait
après ce qui venait de se passer.

Je ne puis exprimer dans quel étonnement nous demeurâmes M. le duc
d'Orléans et moi. Nous ne croyions pas entendre ce que nous entendions,
nous pensions rêver. M. le duc d'Orléans fit plusieurs questions, je
pris aussi la liberté d'en faire pour éclaircir et constater les faits.
Point de variations ni d'ambages dans les réponses du cardinal, tout
furieux qu'il était. À tous moments il présentait l'option, à toute
question, il proposait d'envoyer chercher le cardinal de Bissy, comme
témoin de tout. On peut juger quelle fut cette seconde scène, du hasard
de laquelle je me serais bien passé. Le cardinal insistant toujours sur
l'option, M. le duc d'Orléans, fort embarrassé, me demanda ce que je
pensais, comme, à ce qu'il me sembla, à un homme qui s'était toujours
opposé au renvoi du maréchal de Villeroy. Je répondis que je me trouvais
si étourdi et si ému d'une chose si étonnante, qu'il me fallait
auparavant reprendre mes esprits. Le cardinal, sans s'adresser à moi,
mais toujours à M. le duc d'Orléans, qu'il voyait dans l'embarras et le
trouble, insista fortement qu'il fallait prendre un parti. M. le duc
d'Orléans me pressant de nouveau, je lui dis enfin que jusqu'alors
j'avais toujours regardé le renvoi du maréchal de Villeroy comme une
entreprise fort dangereuse par les raisons que j'en avais alléguées
plusieurs fois à Son Altesse Royale\,; que je la regardais encore de
même pour le moins maintenant que le roi était plus avancé en âge et
touchait à sa majorité\,; mais que, quelque péril qu'il y eût, la scène
affreuse qui venait d'arriver me persuadait qu'il y avait un bien plus
grand danger à le laisser auprès du roi\,; que désormais on ne pouvait
se dissimuler que ce qu'il venait de faire n'était rien moins que tirer
l'épée contre M. le duc d'Orléans, et ses propositions ironiques de
l'arrêter que comme le sentiment d'un homme qui sentait qu'il le
méritait\,; qui se persuadait et qu'on ne l'oserait, et que, l'osant
même, l'exécution en était impossible\,; qui, sur ce principe, ne se
contraignait plus, ne se connaissait plus\,; qui, après avoir tramé en
secret contre M. le duc d'Orléans dès le premier jour de la régence,
sans cesser un moment depuis ni avoir pu être gagné par toutes les
grâces, les marques de confiance, même de déférence, enfin par une
chaîne non interrompue des traitements les plus distingués, levait
maintenant le masque, et ne se proposait rien moins que faire
publiquement autel contre autel\,: que c'était là mon avis, puisque Son
Altesse Royale le voulait savoir sans me donner le temps d'y réfléchir
avec plus de sang-froid\,; mais que pour l'exécution, quelque pressée
qu'elle pût être, il fallait penser mûrement à s'y prendre de manière
qu'on n'en pût avoir le démenti ni dans le temps même, ni dans la suite.

Pendant que je parlais, le cardinal, les oreilles dressées et les yeux
en dessous tournés sur moi, suçait toutes mes paroles, et changeait de
couleur à mesure, comme un homme qui entendrait prononcer son arrêt. Mon
avis exposé entier l'épanouit autant que la rage dont il écumait le lui
put permettre. M. le duc d'Orléans approuva ce que je venais de dire\,;
le cardinal, me jetant un coup d'oeil comme de remerciement, dit à M. le
duc d'Orléans qu'enfin il était le maître de choisir\,; qu'il voyait
bien qu'il ne pouvait rester le maréchal de Villeroy demeurant, et que
Son Altesse Royale prenant même la résolution de l'ôter, il fallait se
hâter, parce que les choses ne pouvaient subsister en la situation où
elles étaient. Enfin il fut conclu qu'on prendrait le reste de la
journée, et il était environ midi, et la matinée suivante pour y penser,
et que je me trouverais le lendemain à trois heures après midi chez M.
le duc d'Orléans.

Arrivé le lendemain chez ce prince, je le trouvai avec le cardinal
Dubois. M. le Duc y entra un moment après, qui était instruit de
l'aventure. Le cardinal Dubois ne laissa pas de lui en faire un récit
abrégé qu'il chargea un peu de commentaires et de réflexions. Il était
plus à lui que la veille par le temps qu'il avait eu de se remettre et
l'espérance de se voir défait dans peu du maréchal de Villeroy. J'y
appris toutes les vanteries qu'il avait publiées de la prise, disait-il,
qu'il avait eue avec le cardinal Dubois, et des défis et des insultes
qu'il lui avait faits, avec une sécurité qui invitait à l'en démentir,
et qui en rendait l'exécution de plus en plus nécessaire. Après quelques
propos debout, le cardinal Dubois s'en alla. M. le duc d'Orléans se mit
à son bureau, et M. le Duc et moi nous assîmes vis-à-vis de lui. Là il
fut question de délibérer tout de bon sur ce qu'il y avait à faire.

M. le duc d'Orléans exposa fort nettement les raisons de part et
d'autre, sans paraître trop pencher d'un côté, mais se montrant
embarrassé, et par conséquent fort en balance. Il développa fort
clairement toute sa conduite avec le maréchal de Villeroy et celle du
maréchal de Villeroy avec lui depuis l'instant de la mort du roi
jusqu'alors, mais en peu de mots, parce qu'il parlait à deux hommes qui
en étaient parfaitement instruits, à M. le Duc qui\,; conjointement avec
lui, avait voulu l'ôter d'auprès du roi et m'y mettre en sa place\,; à
moi, qui l'avais refusé deux autres fois, et cette dernière un mois
durant que ces deux princes m'en avaient pressé à l'excès, comme on l'a
vu ici en son temps, et qui, par mes refus et mes raisons, avais fait
demeurer le maréchal de Villeroy dans sa place. Le point véritablement
agité fut donc de savoir quel était le moins périlleux de l'y laisser ou
de l'en ôter, ce qui ne se pouvait plus que par une sorte de violence
dans la situation où il s'était si bien affermi qu'il ne doutait pas
qu'il ne fût impossible de l'en arracher. Après cet exposé assez court,
M. le duc d'Orléans m'ordonna de dire ce que je pensais là-dessus. Je
répondis que je le lui avais déjà dit la veille\,; que plus j'avais
réfléchi depuis au parti qu'il y avait à prendre, plus je m'étais
affermi dans l'opinion que le danger de laisser auprès du roi le
maréchal de Villeroy, après ce qui venait de se passer, était sans
comparaison plus grand que celui de l'en ôter, quel qu'il pût être\,;
que tant qu'il n'y avait eu dans la conduite du maréchal qu'une mauvaise
volonté impuissante, des liaisons et des projets mal bâtis et aussitôt
déconcertés qu'aperçus, la misère de se vouloir faire le singe de M. de
Beaufort, l'union timide avec tous gens qui mouraient de peur, et lui
qui en laissait voir plus qu'aucun, qui tremblait au moindre sérieux du
récent, et qui, après des démarches échappées souvent après celles qui
étaient ignorées, ne se pouvait rassurer qu'il ne vînt aux
éclaircissements, aux aveux, aux excuses, aux protestations avec la
frayeur et les bassesses les plus pitoyables, j'avais cru qu'il n'y
avait qu'à mépriser un homme sans tête et sans courage d'esprit, surtout
depuis l'effet de la découverte des complots du duc du Maine et de
Cellamare, et laisser piaffer et se panader\footnote{\emph{Panader}
  signifie faire la roue comme un paon. La Fontaine s'en est servi dans
  ce sens (fable du \emph{geai paré des plumes du paon})\,:Puis parmi
  d'autres paons tout fier se \emph{panada}.} ce personnage de théâtre
et de carrousel, dont le génie n'allait pas au delà de la fatuité,
continuellement arrêté par la crainte\,; mais que je changeais
entièrement d'avis sur ce qui venait de se passer\,; que cette scène
montrait de deux choses l'une, mais qui revenaient au même point\,: ou
un homme persuadé par le cardinal de Bissy, qui trouve son orgueil
satisfait par les hommages qu'il consentait de recevoir du cardinal
Dubois, et sa dignité assurée avec son repos par la part entière qui lui
était offerte dans les affaires, et qui, charmé de l'avoir amené à ce
point par ses hauteurs et par ses incartades, avait eu impatience de
s'en mettre en possession en prévenant le cardinal Dubois et en allant
chez lui avec le cardinal de Bissy, leur médiateur, sceller leur
réconciliation et leur paix\,; que là, dans cette intention effective,
la vue du cardinal Dubois l'avait troublé\,; l'arrangement de ses grands
mots et son ton d'autorité l'avaient barbouillé, qu'avec l'intention de
bien dire, le jugement lui avait manqué, l'air de franchise et de
supériorité l'avaient emporté\,; de l'un à l'autre, s'échauffant dans
son harnais, il n'avait pu reculer, la tête lui avait tourné\,; qu'après
avoir commencé en homme sage, il avait poursuivi et fini comme un fou,
et montré tout le venin de son âme et toute la superbe de sa sécurité
avec toute la complaisance d'un homme ivre qui attaque les murailles et
braverait une armée\,; ou bien c'est un homme qui, gonflé de vent,
charmé de réduire à ses pieds le cardinal Dubois, se persuade être
l'homme dont on {[}ne{]} peut se passer, qu'on n'a osé ôter de sa place,
et qu'on l'osera d'autant moins aujourd'hui qu'il est plus ancré, plus
chéri du public par la conservation de la personne du roi, qu'il a su
persuader lui être uniquement due, par l'approche de la majorité, par
toutes les raisons dans lesquelles un sot se mire, surtout par la
persuasion que les démarches vers lui du cardinal Dubois, chargé de
toutes les affaires, lui confirme l'excès de son importance ; plein,
dis-je, de toutes ces idées, qu'il ne sait ni peser ni digérer, il a
amusé le cardinal de Bissy, a fait semblant de se rendre à ses raisons
et aux hommages dont il lui a porté parole, dans la résolution de faire
à tous les deux l'affront qu'il leur a fait, d'éclater sans plus de
mesure, de se déclarer le persécuteur public du ministre qui s'humilie
devant lui, par conséquent l'ennemi du gouvernement et du régent qui
gouverne, enivré de la beauté de cette action qui, dans son sens qu'il
compte bien qui sera aussi celui du public, lui fait mépriser les
hommages du dépositaire de toute la confiance de celui qui gouverne, le
partage du secret et de la conduite des affaires, l'autorité qui y est
attachée, les fruits personnels et pour tous ceux qu'il voudra protéger,
enfin son repos à son âge, et à tant de si grands et de si doux
avantages {[}lui fait{]} préférer le bien public, le sage rétablissement
des affaires, le service du roi, les vues et la dernière confiance en
lui du feu roi, et à un si grand et si honorable travail illustrer et
consacrer les restes de sa vie avec le plus parfait désintéressement.
Ainsi, de quelque façon que le maréchal de Villeroy ait été conduit à la
scène qu'il vient de donner, la chose est égale et la fin la même, c'est
l'épée tirée contre le récent, et le Rubicon passé avec le plus grand
éclat. Le souffrir et laisser le maréchal de Villeroy en place, c'est
montrer une faiblesse et une crainte capables de lui réunir tous les
mécontents et tous les gens d'espérance pour la majorité\,; c'est rendre
au parlement ses premières forces et ses premières usurpations\,; c'est
former soi-même contre soi-même un parti formidable\,; c'est perdre
toute autorité au dedans et toute considération au dehors\,; c'est
encourir le mépris et toutes ses suites, et de la France et des pays
étrangers\,; c'est se creuser des abîmes pour la majorité. Je me tus
après ce court discours, pendant lequel M. le duc d'Orléans était fort
attentif, mais avec la contenance d'un homme fort embarrassé.

Dès que j'eus fini, il demanda à M. le Duc ce qu'il pensait. M. le Duc
dit qu'il pensait comme moi, et que, si le maréchal de Villeroy
demeurait dans sa place, il n'y avait qu'à mettre la clef sous la porte,
ce fut son expression. Il reprit ensuite quelques-unes des principales
raisons que j'avais alléguées, et les appuya, puis conclut qu'il n'y
avait pas un moment à perdre. M. le duc d'Orléans résuma quelque chose
de ce qui avait été dit, et convint de la nécessité de se défaire du
maréchal de Villeroy. M. le Duc insista encore sur s'en défaire
incessamment. Alors on se mit à voir comment s'y prendre.

M. le duc d'Orléans me demanda mon avis là-dessus. Je dis qu'il y avait
deux choses à traiter\,: le prétexte et l'exécution. Qu'il fallait un
prétexte tel qu'il pût sauter aux yeux de tout ce qui était impartial,
et qui ne pût être défendu par les amis mêmes du maréchal de Villeroy\,;
surtout se bien garder de donner lieu de croire que la disgrâce du
maréchal fût le fruit et le salaire de l'insulte qu'il venait de faire
au cardinal Dubois\,; que, quelque énorme qu'elle fût en elle-même à un
cardinal, à un ministre en possession de toute la confiance et de toutes
les affaires, le public qui l'enviait et qui ne l'aimait pas se
souvenait trop d'où il était parti, trouverait la victime trop
illustre\,; que le châtiment ferait oublier l'injure, et qu'on verrait
s'élever un cri public\,; qu'aux partis violents, quoique nécessaires,
il fallait toujours mettre de son côté et la raison et les apparences
mêmes, que je n'étais donc pas d'avis d'exécuter si brusquement ni si
près de l'insulte le châtiment qu'elle méritait\,; mais que M. le duc
d'Orléans avait heureusement en main le plus beau prétexte du monde, un
prétexte qui était connu de tout le haut et le bas intérieur du roi, un
prétexte entièrement sans réplique. Je priai M. le duc d'Orléans de se
souvenir qu'il m'avait dit plusieurs fois, et depuis peu encore, qu'il
n'avait jamais pu parvenir jusqu'à présent, non seulement de parler au
roi tête à tête, mais de lui parler à l'oreille devant tout ce qui était
dans son cabinet\,; que le maréchal de Villeroy, lorsqu'il l'avait voulu
essayer, venait devant tout le monde fourrer sa tête entre celle du roi
et la sienne, et après, sous prétexte d'excuse, lui avait déclaré que la
place qu'il avait auprès du roi ne lui permettait pas de souffrir que
qui que ce pût être, non pas même Son Altesse Royale, dit rien au roi
tout bas, et qu'il devait entendre tout ce qu'on lui voulait dire,
encore moins souffrir personne ni Son Altesse Royale être seule dans un
cabinet avec le roi. Que c'était à l'égard d'un régent, petit-fils de
France et le plus proche parent que le roi eût, une insolence à révolter
tout le monde et qui sauterait aux yeux\,; que le roi approchant de sa
majorité, gagnait un âge où il était temps et où le bien de l'État et
celui du roi demandait que le régent l'instruisit de bien des choses qui
ne se pouvaient dire que sans témoins, sans en excepter le maréchal de
Villeroy ni personne\,; que se targuer de la place de gouverneur et de
chargé de la personne du roi pour empêcher le régent de parler seul au
roi dans un cabinet, c'était porter l'audace jusqu'à jeter des soupçons
les plus fous et les plus injurieux, et que la porter jusqu'à ne vouloir
pas souffrir que le régent parlât bas au roi, même au milieu de tout ce
qui était dans son cabinet, sans venir fourrer son oreille entre eux
deux, était la dernière et la plus inutile insolence que qui que ce soit
ne pouvait excuser\,; que je croyais donc que c'était là un prétexte si
naturel dont il fallait se servir, et le piège que, entre-ci et fort peu
de jours, il fallait tendre au maréchal de Villeroy, qui s'y prendrait
sans doute de ce pinacle de sûreté et d'importance où il croyait être,
puisqu'il avait soutenu ce procédé jusqu'à présent\,; que le piège tendu
et succédant, il fallait que M. le duc d'Orléans s'offensât sur-le-champ
du refus, et que, le respect du roi présent ménagé, il parlât au
maréchal un langage nouveau qui, sans rien de fort, lui fit sentir que,
sous l'autorité et le nom du roi, il était le maître du royaume\,; que
cela suffirait pour un juste préparatif au public, que l'ivresse du
maréchal ne comprendrait pas, ni bien d'autres, qu'après l'exécution,
accoutumé qu'on était aux tolérances de Son Altesse Royale\,; mais que
ce piège ne devait être tendu que lorsque tout serait résolu, rangé et
tout prêt {[}pour{]} l'exécution la plus prompte, sans laisser
entre-deux tout le moins d'intervalle qu'il serait possible. Quand j'eus
cessé de parler\,: «\,Vous me le volez, me dit M. le duc d'Orléans\,;
j'allais le proposer si vous ne l'eussiez pas dit. Que vous en semble,
monsieur\,?» regardant M. le Duc. Ce prince approuva fort la proposition
que je venais de faire, la loua dans toutes ses parties en peu de mots,
et ajouta qu'il ne voyait rien de mieux à faire que d'exécuter ce plan
très ponctuellement.

Il fut convenu ensuite qu'il n'y avait d'autre moyen que d'arrêter le
maréchal, de l'envoyer tout de suite et tout droit à Villeroy, d'où on
verrait, après l'y avoir laissé se reposer un jour ou deux à cause de
son âge, mais bien veillé, si de là on l'enverrait à Lyon ou ailleurs.
Je dis après qu'il ne fallait pas oublier d'avoir un gouverneur tout
prêt pour le mettre en sa place\,; par conséquent songer dès à présent
au choix, et se souvenir plus que jamais d'éviter également un sujet peu
sûr, et tout serviteur particulièrement attaché à M. le duc d'Orléans,
qui était la raison qu'ils savaient l'un et l'autre qui m'avait fait si
opiniâtrement refuser cette importante place plus d'une fois. Là-dessus
M. le duc d'Orléans me dit que toute l'affaire était bien discutée et
résolue\,; qu'il s'en fallait tenir là parce qu'il n'y avait point
d'autre parti à prendre\,; qu'à l'égard de la mécanique à résoudre pour
arrêter le maréchal de Villeroy, il me priait d'aller chez le cardinal
Dubois, où je trouverais qu'on m'attendait pour en raisonner et la
résoudre. Je me levai donc et laissai M. le duc d'Orléans seul avec M.
le Duc, et m'en allai chez le cardinal Dubois, duquel je n'avais pas ouï
parler, ni d'aucun de ses émissaires, depuis son aventure, excepté le
peu que je l'avais vu en présence de M. le duc d'Orléans. Mais ce que ce
prince me dit en m'envoyant chez lui me fit nettement sentir que l'arrêt
du maréchal de Villeroy était résolu entre le régent et son ministre
avant la conférence que je viens de raconter, et qu'elle n'avait été
tenue sans autres que les deux princes et moi, pour y laisser un air de
liberté par l'absence du cardinal Dubois, et comme je m'étais ouvert la
veille entre le régent et le cardinal, lorsqu'il arriva furieux de la
scène qu'il venait d'essuyer, pour me donner lieu de parler devant M. le
Duc, et de l'entraîner dans mon avis de se défaire du maréchal de
Villeroy.

J'allai donc tout de suite chez le cardinal Dubois, et ma surprise fut
extrême de la compagnie que je trouvai avec lui, devant laquelle il me
dit d'abordée qu'elle était du secret, et que je pouvais parler devant
elle. Cette compagnie était le maréchal de Berwick, arrivé depuis peu de
jours de Guyenne, qui, non plus que moi, ne rentra pas au conseil de
régence\,; le cardinal et le prince de Rohan, Le Blanc et Belle-Ile,
assis en rond tout près et devant le canapé adossé à la muraille, où
étoient assis les deux cardinaux, et sur lequel je me mis auprès du
cardinal de Rohan. Le Blanc me parut une partie nécessaire pour
l'arrangement et les ordres de cette mécanique. Il était plein
d'inventions et de ressources, dans tout l'intérieur des opérations
secrètes du régent depuis longtemps, et sur le pied de secrétaire
renforcé du cardinal Dubois, avec caractère, par sa charge de signer en
commandement. Pour Belle-Ile, encore qu'à l'appui de celui-ci il se fût
introduit en tiers tous les soirs avec lui chez le cardinal Dubois, où
il se rendait compte, se résumaient et se résolvaient bien des choses,
il approchait si peu le régent, qui même ne l'aimait pas, que je le
trouvai là fort déplacé. À l'égard du maréchal de Berwick, qui, du temps
du feu roi, avait toujours été sur le pied de protégé du maréchal de
Villeroy, lequel, en courtisan qui savait le goût de son naître pour
toutes sortes de grands bâtards par leur homogénéité avec les siens,
avait eu grande part à la rapide élévation de celui-ci à la guerre, je
fus extrêmement étonné de le voir admis en ce conciliabule, et de l'y
entendre opiner aussi librement et aussi fortement qu'il fit, ayant
toujours fait profession jusqu'alors de cultiver le maréchal de Villeroy
et d'amitié particulière avec lui. Pour les deux frères Rohan, que le
cardinal Dubois ménageait avec une distinction singulière, et qu'il
avait admis là pour la leur témoigner d'une façon si marquée, je ne vis
jamais une joie plus scandaleuse, ni une plus âcre amertume que celle
qu'ils ne se mirent pas en peine même de voiler. On vit en plein éclater
toute la haine conçue de la rupture du mariage de leur fille boiteuse
avec le duc de Retz, sur des conditions méprisantes qu'ils ne
proposèrent que quand ils crurent qu'il n'y avait plus à s'en dédire, et
dont le maréchal de Villeroy, justement indigné, ne voulut jamais passer
malgré les charmes et les larmes de la duchesse de Ventadour, comme je
l'ai raconté en son temps, et le dépit que conçurent les Rohan de voir
incontinent après le duc de Retz épouser la fille aînée du duc de
Luxembourg, à conditions convenables, tandis qu'ils se trouvèrent trop
heureux de donner leur fille au duc Mazarin, d'une naissance et d'un
personnel peu agréables, sans charge ni autres réparations.

Je ne ferai point ici un détail superflu de tout ce qui fut discuté dans
cette petite assemblée. On y résolut ce qu'on va voir, qui fut très bien
exécuté. Seulement dirai-je que, dès que je fus assis et que le cardinal
Dubois m'eut déclaré que tout ce qui se trouvait en ce petit
conventicule était du secret et que je pouvais y parler sans réserve, il
me dit qu'on m'y attendait avec impatience pour apprendre ce que M. le
duc d'Orléans avait résolu, comme s'il l'eût ignoré, et que cette
assemblée, pour délibérer de la mécanique de l'exécution, n'eût pas
décelé la connaissance certaine qu'il avait de la résolution prise par
M. le duc d'Orléans. Je l'exposai donc en peu de mots\,; après quoi on
vint à la manière, à la forme, aux expédients de l'exécution, aux
remèdes des obstacles et des inconvénients du moment et de ses suites.

Ces discussions furent assez longues, auxquelles je pris assez peu de
part. Le fort en roula sur le cardinal Dubois et sur Le Blanc.
Belle-Ile, extrêmement bien avec les Rohan, et d'autre part avec le
maréchal de Berwick, se comporta avec sagesse. Le bon maréchal ne se
montra pas si mesuré. Je pense qu'il se trouvait fatigué des grands airs
d'ancien maître et d'ancien protecteur que le maréchal de Villeroy
déployait sur lui, et des emphases d'autorité et de toute supériorité
dont il l'accablait, et dont il était bien aise de se voir délivré. Je
convins avec Le Blanc que, dans l'instant que l'exécution serait faite,
il m'en avertirait par envoyer simplement à Meudon savoir de mes
nouvelles, sans rien de plus, et qu'à ce compliment inutile je
reconnaîtrais le signal que le maréchal était paqueté.

Je m'en retournai donc à Meudon sur le soir, où plusieurs personnes des
amies de M\textsuperscript{me} de Saint-Simon et des miens couchaient
souvent, et où la mode s'était mise à Versailles et à Paris de venir
dîner ou souper, de manière que la compagnie y était toujours fort
nombreuse. On n'y parlait que de cette scène du maréchal de Villeroy,
qui était universellement blâmée, mais sans aller plus loin, et sans
que, pendant les dix jours qui s'écoulèrent jusqu'à l'enlèvement du
maréchal de Villeroy, il fût entré dans la tête de personne qu'il pût
lui en arriver pis que le blâme général d'un emportement si démesuré,
tant on était accoutumé à l'impunité de ses incartades et à la faiblesse
de M. le duc d'Orléans. J'étais ravi cependant de voir une sécurité si
générale, qui augmentait celle du maréchal de Villeroy, rendrait plus
facile l'exécution de ce qu'on lui préparait, et qui ne cessait de le
mériter de plus en plus par l'indécence et l'affectation de ses
discours, et l'audace de ses continuels défis. Trois ou quatre jours
après j'allai à Versailles voir M. le duc d'Orléans. Il me dit que faute
de mieux, et sur ce que je lui avais dit plus d'une fois du duc de
Charost, il s'était résolu à lui donner la place de gouverneur du roi\,;
qu'il l'avait vu secrètement\,; qu'il avait accepté de fort bonne grâce,
et qu'il l'allait tenir en mue, claquemuré dans son appartement de lui
Charost, à Versailles, sans en sortir ni se montrer à qui que ce fût,
pour l'avoir tout prêt sous sa main à le mener au roi, et l'installer
dans le moment qu'il en serait temps. Il repassa avec moi toute la
mécanique concertée, et je m'en revins à Meudon, résolu de n'en bouger
qu'après l'exécution qui s'approchait, et sur laquelle il n'y avait plus
de nouvelles mesures à prendre.

\hypertarget{chapitre-xv.}{%
\chapter{CHAPITRE XV.}\label{chapitre-xv.}}

1722

~

{\textsc{Piège tendu au maréchal de Villeroy, qui y donne en plein.}}
{\textsc{- Le maréchal de Villeroy arrêté et conduit tout de suite à
Villeroy.}} {\textsc{- Le roi fort affligé.}} {\textsc{- Fuite inconnue
de l'évêque de Fréjus, découvert à Bâville, mandé et de retour
aussitôt.}} {\textsc{- Fureurs du maréchal de Villeroy.}} {\textsc{- Le
roi un peu apaisé par le retour si prochain de l'évêque de Fréjus.}}
{\textsc{- Mesures à prendre avec cet évêque, et prises en effet.}}
{\textsc{- Le duc de Charost déclaré gouverneur.}} {\textsc{- Désespoir
du maréchal de Villeroy.}} {\textsc{- Il dévoile la cause de la fuite de
Fréjus, dont cet évêque se tire fort mal.}} {\textsc{- Sa joie et ses
espérances fondées sur l'éloignement du maréchal.}} {\textsc{- Maréchal
de Villeroy exilé à Lyon, mais avec ses fonctions de gouverneur de la
ville et de la province.}} {\textsc{- Crayon léger de ce maréchal.}}
{\textsc{- Le roi tout consolé du maréchal de Villeroy.}} {\textsc{- Art
et ambition de la conduite de Fréjus.}} {\textsc{- Confirmation et
première communion du roi.}} {\textsc{- Cardinal Dubois, sans plus
d'obstacle, tout occupé de se faire brusquement déclarer premier
ministre, emploie Belle-Ile pour m'en parler.}} {\textsc{- Conversation
singulière entre M. le duc d'Orléans et moi sur faire un premier
ministre, dont je ne suis point d'avis.}} {\textsc{- Ennui du régent le
porte à faire un premier ministre\,; à quoi je m'oppose.}} {\textsc{-
Comparaison du feu prince de Conti, gendre du dernier M. le Prince.}}
{\textsc{- Aveu sincère de M. le duc d'Orléans.}} {\textsc{-
Considérations futures.}} {\textsc{- Cardinal Dubois bien connu de son
maître.}} {\textsc{- Faiblesse incroyable du régent.}} {\textsc{-
Belle-Ile resté en embuscade.}} {\textsc{- Réponse que je lui fais.}}

~

Le dimanche 12 août, M. le duc d'Orléans alla sur la fin de
l'après-dînée travailler avec le roi, comme il avait accoutumé de faire
plusieurs jours marqués de chaque semaine, et, comme c'était l'été, au
retour de sa promenade, qui était toujours de bonne heure. Ce travail
était de montrer au roi la distribution d'emplois vacants, de bénéfices,
de certaines magistratures, d'intendances, de récompenses de toute
nature, et de lui expliquer en peu de mots les raisons des choix et des
préférences, quelquefois des distributions de finances\,; enfin les
premières nouvelles étrangères, quand il y en avait à sa portée, avant
qu'elles devinssent publiques. À la fin de ce travail, où le maréchal de
Villeroy assistait toujours, et où quelquefois M. de Fréjus se hasardait
de rester, M. le duc d'Orléans supplia le roi de vouloir bien passer
dans un petit arrière-cabinet, où il avait un mot à lui dire tête à
tête. Le maréchal de Villeroy s'y opposa à l'instant. NI. le duc
d'Orléans, qui lui tendait le piège, l'y vit donner en plein avec
satisfaction. Il lui représenta avec politesse que le roi entrait dans
un âge si voisin de celui où il gouvernerait par lui-même, qu'il était
temps que celui qui, en attendant, était le dépositaire de toute son
autorité, lui rendit compte des choses qu'il pouvait maintenant
entendre, et qui ne pouvaient être expliquées qu'à lui seul quelque
confiance que méritât quelque tiers que ce pût être, et qu'il le priait
de cesser de mettre obstacle à une chose si nécessaire et si importante,
que lui régent avait peut-être à se reprocher de n'avoir pas commencée
plus tôt, uniquement par complaisance pour lui. Le maréchal s'échauffant
et secouant sa perruque, répondit qu'il savait le respect qu'il lui
devait, et pour le moins autant ce qu'il devait au roi et à sa place,
qui le chargeait de sa personne et l'en rendait responsable, et protesta
qu'il ne souffrirait pas que Son Altesse Royale parlât au roi en
particulier, parce qu'il devait savoir tout ce qui lui était dit,
beaucoup moins tête à tête dans un cabinet, hors de sa vue, parce que
son devoir était de ne le perdre pas de vue un seul moment, et dans tous
de répondre de sa personne. Sur ce propos, M. le duc d'Orléans le
regarda fixement, et lui dit avec un ton de maître qu'il se méprenait et
s'oubliait\,; qu'il devait songer à qui il parlait et à la force de ses
paroles, qu'il voulait bien croire qu'il n'entendait pas\,; que le
respect de la présence du roi l'empêchait de lui répondre comme il le
méritait et de pousser plus loin cette conversation. Et tout de suite
fit au roi une profonde révérence et s'en alla.

Le maréchal, fort en colère, le conduisit quelques pas, marmottant et
gesticulant sans que M. le duc d'Orléans fît semblant de le voir et de
l'entendre, laissant le roi étonné et le Fréjus riant tout bas dans ses
barbes. Le hameçon si bien pris, on se douta que ce maréchal, tout
audacieux qu'il était, mais toutefois bas et timide courtisan, sentirait
toute la différence de braver et de bavarder, d'insulter le cardinal
Dubois, odieux à tout le monde et sentant encore la vile coque dont il
sortait, d'avec celle d'avoir une telle prise, et en présence du roi,
avec M. le duc d'Orléans, et de prétendre anéantir les droits et
l'autorité du régent du royaume par les prétendus droits et autorité de
sa place de gouverneur du roi, et par ses termes de répondre de sa
personne, les appuyer ouvertement sur ce qu'il y a de plus injurieux. On
n'y fut pas trompé. Moins de deux heures après, on sut que le maréchal,
se vantant de ce qu'il venait de faire, avait ajouté qu'il s'estimerait
bien malheureux que M. le duc d'Orléans pût croire qu'il eût voulu lui
manquer, quand il n'avait songé qu'à remplir son plus précieux devoir,
et qu'il irait chez lui dès le lendemain matin, pour en avoir un
éclaircissement avec lui, dont il se flattait bien que ce prince
demeurerait content.

À tout hasard on avait pris toutes les mesures nécessaires dès que le
jour fut arrêté pour tendre le piège au maréchal. On n'eut donc qu'à
leur donner leur dernière forme, dès qu'on sut, dès le soir même, que le
maréchal viendrait s'enferrer. Au delà de la chambre à coucher de M. le
duc d'Orléans était un grand et beau cabinet, à quatre grandes fenêtres
sur le jardin, et de plain-pied, à deux marches près, deux en face en
entrant, deux sur le côté, vis-à-vis de la cheminée, et toutes ces
fenêtres s'ouvraient en portes, depuis le haut jusqu'au parquet. Ce
cabinet faisait le coin, où les gens de la cour attendaient, et en
retour était un cabinet joignant, où M. le duc d'Orléans travaillait et
faisait entrer les gens les plus distingués ou favorisés qui avaient à
lui parler. Le mot était donné. Artagnan, capitaine des mousquetaires
gris, était dans cette pièce, qui savait ce qui s'allait exécuter, avec
force officiers sûrs de sa compagnie, qu'il avait fait venir, et
d'anciens mousquetaires pour s'en servir au besoin, qui voyaient bien à
ce préparatif qu'il s'agissait de quelque chose, mais sans se douter de
ce que ce serait. Il y avait aussi des chevau-légers répandus en dehors
le long des fenêtres, et dans la même ignorance, et beaucoup d'officiers
principaux et autres de M. le duc d'Orléans, tant dans sa chambre à
coucher que dans ce grand cabinet.

Tout cela bien ordonné, arriva sur le midi le maréchal de Villeroy avec
son fracas accoutumé, mais seul, sa chaise et ses gens restés au loin,
hors la salle des gardes. Il entre en comédien, s'arrête, regarde, fait
quelques pas. Sous prétexte de civilité, on s'attroupe auprès de lui, on
l'environne. Il demande d'un ton d'autorité ce que fait M. le duc
d'Orléans. On lui répond qu'il est enfermé et qu'il travaille. Le
maréchal élève le ton, dit qu'il faut pourtant qu'il le voie, qu'il va
entrer, et dans cet instant qu'il s'avance, La Fare, capitaine des
gardes de M. le duc d'Orléans, se présente vis-à-vis de lui, l'arrête et
lui demande son épée. Le maréchal entre en furie et toute l'assistance
en émoi. En ce même instant, Le Blanc se présente. Sa chaise à porteurs,
qu'on avait tenue cachée, se plante devant le maréchal. Il s'écrie, il
est mal sur ses jambes, il est jeté dans la chaise qu'on ferme sur lui,
et emporté dans le même clin d'oeil par une des fenêtres latérales dans
le jardin, La Fare et Artagnan chacun d'un côté de la chaise, les
chevau-légers et mousquetaires après, qui ne virent que par l'effet de
quoi il s'agissait. La marche se presse, descend l'escalier de
l'orangerie du côté des bosquets, trouve la grande grille ouverte et un
carrosse à six chevaux devant. On y pose la chaise le maréchal a beau
tempêter, on le jette dans le carrosse. Artagnan y monte à côté de lui,
un officier des mousquetaires sur le devant, et du Libois, un des
gentilshommes ordinaires dû roi, à côté de l'officier\,; vingt
mousquetaires, avec des officiers à cheval, autour du carrosse, et
touche, cocher.

Ce côté du jardin, qui est sous les fenêtres de l'appartement de la
reine, occupé par l'infante, ne fut vu de personne à ce soleil de midi,
et, quoique ce nombre de gens qui se trouvèrent dans l'appartement de M.
le duc d'Orléans se dispersassent bientôt, il est étonnant qu'une
affaire de cette nature demeurât ignorée plus de deux heures dans le
château de Versailles. Les domestiques du maréchal de Villeroy, à qui
personne n'avait osé rien dire en sortant, je ne sais par quel hasard,
attendirent toujours avec sa chaise près de la salle des gardes\,; et
ceux qui étaient chez lui, dans les derrières des cabinets du roi, ne
l'apprirent qu'après que M. le duc d'Orléans eut vu le roi, et qu'il
leur manda que le maréchal était allé à Villeroy, où ils pouvaient lui
aller porter ce qui lui était nécessaire. Je reçus à Meudon le message
convenu. J'allais me mettre à table, et ce ne fut que vers le souper
qu'il vint des gens de Versailles qui nous apprirent à tous la nouvelle
qui y faisait grand bruit, mais un bruit fort contenu que la qualité de
l'exécution rendait fort mesuré par la surprise et la frayeur qu'elle
avait répandues.

Ce ne fut pas, après, un petit embarras que celui de M. le duc d'Orléans
pour en porter la nouvelle au roi, dès qu'elle fut répandue. Il entra
dans le cabinet du roi, d'où il fit sortir tous les courtisans qui s'y
trouvèrent, et n'y laissa que les gens dont les charges leur donnaient
cette entrée, et il ne s'en trouva presque point. Au premier mot le roi
rougit\,; ses yeux se mouillèrent\,: il se mit le visage contre le dos
d'un fauteuil, sans dire une parole, ne voulut ni sortir ni jouer. À
peine mangea-t-il quelques bouchées à souper, pleura et ne dormit point
de toute la nuit. La matinée et le dîner du lendemain 14 ne se passèrent
guère mieux. Ce même jour 14, comme je sortais de dîner à Meudon avec
beaucoup de monde, le valet de chambre qui me servait me dit qu'il y
avait là un courrier du cardinal Dubois, avec une lettre, qu'il n'avait
pas cru me devoir amener à table devant toute cette compagnie. J'ouvris
la lettre. Le cardinal me conjurait de l'aller trouver à l'instant droit
à la surintendance à Versailles, d'amener avec moi un homme sûr en état
de courir la poste pour le dépêcher à la Trappe aussitôt qu'il m'aurait
parlé, et de ne me point casser la tête à deviner ce que ce pouvait
être, parce qu'il me serait impossible de le deviner, et qu'il
m'attendait avec la dernière impatience pour me le dire. Je demandai mon
carrosse aussitôt, que je trouvai bien lent à venir des écuries, qui
sont fort éloignées du château neuf que j'occupais.

Ce courrier à mener au cardinal pour le dépêcher à la Trappe me tournait
la tête\,: je ne pouvais imaginer ce qui pouvait y être arrivé, qui
occupât si vivement le cardinal dans des moments si voisins de celui de
l'enlèvement du maréchal de Villeroy. La constitution, ou quelque
fugitif important et inconnu découvert à la Trappe, et mille autres
pensées m'agitèrent jusqu'à Versailles. Arrivant à la surintendance, je
vis par-dessus la porte le cardinal Dubois à la fenêtre, qui
m'attendait, et qui me fit de grands signes, et que je trouvai au-devant
de moi au bas du degré, comme je l'allais monter. Sa première parole fut
de me demander si j'avais amené un homme qui pût aller en poste à la
Trappe. Je lui montrai ce même valet de chambre qui en connaissait tous
les êtres pour y avoir été fort souvent avec moi, et qui était connu de
lui de tout temps, parce que de tout temps il venait chez moi, et que,
petit abbé Dubois alors, il l'entretenait souvent en m'attendant. Il me
conta, en montant le degré, les pleurs du roi, qui venaient bien
d'augmenter par l'absence de M. de Fréjus, qui avait disparu, qui
n'avait point couché à Versailles, et qu'on ne savait ce qu'il était
devenu, sinon qu'il n'était ni à Villeroy ni sur le chemin, parce qu'ils
venaient d'en avoir des nouvelles\,; que cette disparition mettait le
roi au désespoir, et eux dans le plus cruel embarras du monde\,; qu'ils
ne savaient que penser de cette subite retraite, sinon peut-être qu'il
était allé se cacher à la Trappe, où il fallait envoyer voir s'il y
était, et tout de suite me conduisit chez M. le duc d'Orléans. Nous le
trouvâmes seul, fort en peine, se promenant dans son cabinet, qui me dit
aussitôt qu'il ne savait que devenir ni que faire du roi, qui criait
après M. de Fréjus, et ne voulait entendre à rien, et de là crier contre
une si étrange fuite.

Peu de moments après arrivèrent le prince et le cardinal de Rohan, à qui
l'arrêt du maréchal de Villeroy avait ouvert toutes les portes\,; ils
étaient suivis de Pezé. Son attachement et sa parenté de
M\textsuperscript{me} de Ventadour, qui l'avait fort familiarisé avec
les deux frères, n'empêchait pas qu'il ne fût fort aise de se voir
délivré du maréchal de Villeroy, mais qui étant lié à Fréjus, étant
outré de cette escapade. Après plus de jérémiades que de résolutions,
Dubois me pressa d'aller écrire à la Trappe. Tout était en désarroi chez
M. le duc d'Orléans\,; ils parlaient tous dans ce cabinet\,; impossible
à tout ce bruit d'écrire sur son bureau, comme il m'arrivait souvent
quand j'étais seul avec lui. Mon appartement était dans l'aile neuve, et
peut-être fermé, car on ne m'attendait pas ce jour-là. J'eus plus tôt
fait de monter chez Pezé, dont la chambre était proche\,; au-dessus de
l'appartement {[}de la{]} reine, et je m'y mis à écrire. Ma lettre était
achevée, et Pezé, qui m'y avait conduit et qui était redescendu
aussitôt, remonta et me cria\,: «\,Il est trouvé, il est trouvé\,; votre
lettre est inutile, revenez-vous-en chez M. le duc d'Orléans.\,» Puis me
conta que tout à l'heure un homme à M. le duc d'Orléans, qui savait que
Fréjus était ami des Lamoignon, avait rencontré Courson dans la grande
cour, qui sortait du conseil des parties, à qui il avait demandé s'il ne
savait point ce qu'était devenu Fréjus\,; que Courson lui avait dit
qu'il ne savait pas de quoi on était si en peine\,; que Fréjus était
allé la veille coucher à Bâville, où était le président Lamoignon\,; sur
quoi cet homme de M. le duc d'Orléans lui avait amené Courson pour le
lui dire lui-même.

Nous arrivâmes Pezé et moi chez M. le duc d'Orléans, d'où Courson venait
de sortir. La sérénité y était revenue\,; Fréjus fut bien brocardé, et
le cardinal et le prince de Rohan ne s'y ménagèrent pas. Après un peu
d'épanouissement, le cardinal Dubois avisa M. le duc d'Orléans d'aller
porter au roi cette bonne nouvelle, et de lui dire qu'il allait dépêcher
à Bâville pour faire revenir son précepteur. M. le duc d'Orléans monta
chez le roi et me dit qu'il allait redescendre\,; les deux frères s'en
allèrent de leur côté avec Pezé, et je demeurai à attendre M. le duc
d'Orléans avec le cardinal Dubois. Après avoir un peu raisonné sur cette
fugue de Fréjus, il me conta qu'ils avaient des nouvelles de Villeroy\,;
que le maréchal n'avait cessé de crier à l'attentat commis sur sa
personne, à l'audace du régent, à l'insolence de lui Dubois, ni de
chanter pouille tout le chemin à Artagnan de se prêter à une violence si
criminelle\,; puis à invoquer les mânes du feu roi, à exalter sa
confiance en lui, l'importance de la place pour laquelle il l'avait
préféré à tout le monde\,; le soulèvement qu'une entreprise si hardie,
et qui passait si fort le pouvoir du régent, allait causer dans Paris et
dans tout le royaume, et le bruit qu'elle allait faire dans tous les
pays étrangers\,; les choix du feu roi, pour ce qu'il laissait de plus
précieux à conserver et à former, chassés, d'abord le duc du Maine, lui
ensuite\,; déplorations du sort du roi, de celui de tout le royaume\,;
puis des élans, puis des invectives, puis des applaudissements de ses
services, de sa fidélité, de sa fermeté, de son invariable attachement à
son devoir\,; après, des railleries piquantes à du Libois, gardien né de
tous les personnages qu'on arrêtait, sur ce qu'il avait été mis auprès
de Cellamare, auparavant de l'ambassadeur de Savoie. Enfin ce fut un
homme si étonné, si troublé, si plein de dépit et de rage, qu'il était
hors de soi et ne se posséda pas un moment. Le duc de Villeroy, le
maréchal de Tallard, Biron, furent à peu près ceux qui eurent la
permission d'aller à Villeroy, presque aucun autre ne la demanda. Mais
ce ne fut que le lendemain.

M. le duc d'Orléans revint de chez le roi, qui nous dit que la nouvelle
qu'il lui avait portée l'avait fort apaisé\,: sur quoi nous conclûmes
qu'il fallait faire en sorte que Fréjus revînt dans la matinée du
lendemain\,; que M. le duc d'Orléans le reçût à merveilles, prît tout
pour bon\,; l'amadouât, lui fît entendre que ce n'était que pour le
ménager et lui ôter tout embarras s'il ne lui avait pas confié le secret
de l'arrêt du maréchal de Villeroy\,; lui en expliquer la nécessité avec
d'autant plus de liberté que Fréjus haïssait le maréchal, ses hauteurs,
ses jalousies, ses caprices, et dans son âme serait ravi de son
éloignement et de posséder le roi tout à son aise\,; le prier de faire
entendre au roi les raisons de cette nécessité\,; communiquer à Fréjus
le choix du duc de Charost\,; lui en promettre tout le concert et les
égards qu'il en pouvait désirer\,; lui demander de le conseiller et le
conduire\,; enfin prendre le temps de la joie du roi du retour de Fréjus
pour lui apprendre le choix du nouveau gouverneur, et le lui présenter.
Tout cela fut convenu et très bien exécuté le lendemain.

Quand le maréchal le sut à Villeroy, il s'emporta d'une étrange manière
contre Charost, dont il parla avec le dernier mépris d'avoir accepté sa
place, mais surtout contre Fréjus, qu'il n'appelait plus que traître et
scélérat. Après les premiers {[}moments{]}, qui ne lui permirent que des
transports et des fureurs d'autant plus violentes que la tranquillité
qu'il apercevait partout le détrompait malgré lui de la certitude où son
orgueil l'avait jeté que le parlement, que les halles, que Paris se
soulèverait si on osait toucher à un personnage aussi important et aussi
aimé qu'il se figurait l'être, après l'avoir été à ses dépens, qu'on
n'aurait jamais l'assurance ni les moyens de l'arrêter. Ces vérités,
qu'il ne pouvait plus se dissimuler, succédant si fort tout à coup aux
chimères qui faisaient toute sa nourriture et sa vie, le nettoient au
désespoir et hors de lui-même. Il s'en prenait au régent, à son
ministre, à ceux qu'ils avaient employés pour l'arrêter, à ceux qui
avaient manqué à le défendre, à tout ce qui ne se révoltait pas pour le
faire revenir et faire tête au régent\,; à Charost, qui avait osé lui
succéder\,; surtout à Fréjus, qui l'avait trompé, et qui le trahissait
d'une manière si indigne. Fréjus était celui contre lequel il était le
plus irrité. Ses reproches d'ingratitude et de trahison pleuvaient sur
lui sans cesse\,; tout ce qu'il avait tenté près du feu roi pour lui\,;
comme il l'avait protégé, assisté, loué, nourri\,; que sans lui il n'eût
jamais été précepteur du roi\,; et tout cela était exactement vrai. Mais
la trahison qu'il rebattait à tous moments, il l'expliqua enfin\,: il
dit que Fréjus et lui s'étaient promis l'un à l'autre, dès les premiers
jours de la régence, une indissoluble union, et que, si par des troubles
et des événements qui ne se pouvaient prévoir et qui n'étaient que trop
communs dans le cours des régences, on entreprenait d'ôter l'un d'eux
d'auprès du roi sans que l'autre le pût empêcher, et sans lui toucher,
cet autre se retirerait sur-le-champ et ne reprendrait jamais sa place
que celle de l'autre ne lui fût rendue, et en même temps. Et là-dessus,
nouveaux cris de la perfidie que ce misérable, car les termes les plus
odieux lui étaient les plus familiers, prétendait sottement couvrir d'un
voile de gaze en se dérobant pour aller à Bâville se faire chercher et
revenir aussitôt, dans la frayeur de perdre sa place par la moindre
résistance et le moindre délai, et prétendait s'acquitter ainsi de sa
parole et de l'engagement réciproque que tous deux avaient pris
ensemble\,; et de là retournait aux injures et aux fureurs contre ce
serpent, disait-il, qu'il avait réchauffé et nourri tant d'années dans
son sein.

Ce récit revint promptement de Villeroy à Versailles avec les
transports, les injures, les fureurs, non seulement par ceux que le
régent y tenait pour le garder honnêtement, et pour rendre un compte
exact de tout ce qu'il disait et faisait jour par jour, mais par tout le
domestique, tant des siens que de ceux qui furent à Villeroy, qui
allaient et venaient, et devant qui il affectait de se répandre de plus
belle, soit à table, soit passant par ses antichambres, ou faisant
quelques tours dans ses jardins. Le contre-coup en fut pesant pour
Fréjus, qui avec toute la tranquillité apparente de son visage en parut
confondu. Il n'y répondit que par un silence de respect et de
commisération, dans lequel il s'enveloppa. Toutefois, il ne put le
garder tout entier au duc de Villeroy, au maréchal de Tallard et à
quelque peu d'autres\,; il s'en tira avec eux par leur dire
tranquillement qu'il avait fait tout ce qu'il avait pu pour remplir un
engagement qu'il ne niait pas, mais qu'après y avoir satisfait autant
qu'il était en lui, il avait cru ne pouvoir se dispenser d'obéir aux
ordres si exprès du roi et du régent, ni devoir abandonner le roi pour
opérer le retour du maréchal de Villeroy, qui était l'objet de leur
engagement réciproque, et qu'il était sensible que l'opiniâtreté de son
absence n'opérerait pas. Mais parmi ces excuses si sobres, on sentait la
joie percer malgré lui de se trouver défait d'un supérieur si incommodé,
de n'avoir plus affaire qu'à un gouverneur dont il n'aurait qu'à se
jouer, et de pouvoir désormais se conduire en liberté vers le grand
objet où il avait toujours tendu, qui était de s'attacher le roi sans
réserve, et de faire de cet attachement obtenu par toutes sortes de
moyens, la base d'une grandeur qu'il ne pouvait encore se démêler à
lui-même, mais dont le temps et les {[}conjonctures\footnote{Il y a
  \emph{conjectures} dans le manuscrit, mais c'est une erreur
  évidente\,: il faut lire \emph{conjonctures}.} {]} lui apprendraient à
en tirer les plus grands partis, et marcher en attendant fort couvert.
On laissa le maréchal se reposer et s'exhaler cinq ou six jours à
Villeroy, et comme il n'avait aucun talent redoutable, éloigné de la
personne du roi, on l'envoya à Lyon, avec la liberté d'exercer ses
courtes fonctions de gouverneur de la ville et de la province, en
prenant les mesures nécessaires pour le faire veiller de près, et
laissant auprès de lui du Libois pour émousser son autorité par cet air
de précaution et de surveillance qui lui ôtait tout air de crédit. Il
n'y voulut point recevoir d'honneurs en y arrivant. Une grande partie de
son premier feu était jetée\,; ce grand éloignement de Paris et de la
cour, où tout était non seulement demeuré sans le plus léger mouvement,
mais dans l'effroi et la stupeur d'une exécution de cette importance,
lui ôta tout reste d'espérance, rabattit ses fougues, et le persuada
enfin de se comporter avec sagesse pour éviter un traitement plus
fâcheux.

Telle fut la catastrophe de cet homme si fort au-dessous de tous les
emplois qu'il avait remplis, qui y montra le tuf dans tous, qui mit
enfin la chimère et l'audace à la place de la prudence et de la sagesse,
qui ne parut partout que frivole et comédien, et dont l'ignorance
universelle et profonde, excepté de l'art de bas courtisan, laissa
toujours percer bien aisément la croûte légère de probité et de vertu
dont il couvrait son ingratitude, sa folle ambition, sa soif de tout
ébranler pour se faire le chef de tous au milieu de ses faiblesses et de
ses frayeurs, et pour tenir un gouvernail dont il était si radicalement
incapable. Je ne parle ici que depuis la régence. On a vu ailleurs en
tant d'endroits le peu ou même le rien qu'il valait en tout
genre\,;-comment son ignorance et sa jalousie perdit la Flandre et
presque l'État, puis sa fatuité poussée à l'extrême, lui-même, et les
déplorables ressorts de son retour, qu'il est inutile de s'y arrêter
davantage. C'est assez de dire qu'il ne put jamais se relever de l'état
où le jeta cette dernière folie, et que le reste de sa vie ne fut plus
qu'amertume, regrets et mépris. Il avait persuadé au roi, et on en verra
la preuve, si j'ai le temps de remplir jusqu'au bout ce que je me suis
proposé, il avait, dis-je, persuadé au roi que lui seul, par sa
vigilance et par ses précautions, conservait sa vie qu'on voulait lui
ôter par le poison\,; c'est ce qui fut la source des larmes du roi quand
il lui fut enlevé, et de son presque désespoir lorsque Fréjus disparut.
Il ne douta point qu'on ne les eût écartés tous deux que pour en venir
plus aisément à ce crime.

Le retour si prompt de Fréjus dissipa la moitié de sa crainte, la
persévérance de sa bonne santé le délivra peu à peu de l'autre. Le
précepteur, qui avait un si grand intérêt à le conserver, et qui se
sentait si soulagé du poids du maréchal de Villeroy, ne s'oublia pas à
tâcher d'éteindre de si funestes idées, conséquemment à en laisser
tomber le criminel venin sur celui qui les avait inspirées et
persuadées. Il en craignait le retour quand le roi se trouverait le
maître, dont la majorité approchait\,: délivré de son joug, il ne
voulait pas y retomber. Il savait bien que les grands airs, les ironies
et les manières d'autorité sur le roi en public lui étaient
insupportables, et que le maréchal ne tenait au roi que par ces
affreuses idées de poison. Les détruire, c'était laisser le maréchal à
nu, et pis que cela, montrer au roi, sans paraître le charger, le
criminel intérêt de lui donner ces alarmes, et la fausseté et l'atrocité
de l'invention d'une telle calomnie. Ces réflexions, que la santé du roi
confirmait chaque jour, sapaient toute estime, toute reconnaissance,
laissaient même la bienséance en liberté de ne rapprocher pas de lui,
quand il en serait le maître, un si noir imposteur et si intéressé.
Fréjus sut user de ces moyens pour se mettre pour toujours à l'abri de
tout retour du maréchal, et de s'attacher le roi sans réserve\,: on n'en
a que trop senti depuis le prodigieux succès.

Cette expédition fut aussitôt après suivie de la confirmation du roi par
le cardinal de Rohan, et de sa première communion, qui lui fut
administrée par le même cardinal, son grand aumônier.

Défait enfin du maréchal de Villeroy, le cardinal Dubois n'eut plus
d'obstacle pour se faire déclarer premier ministre. Il crut même avec
raison devoir profiter de l'étonnement et de la stupeur où cet événement
avait jeté toute la cour, la ville, et plus que tous le parlement, pour
achever brusquement cet ouvrage également audacieux et odieux. Son
pouvoir sur l'esprit de son maître était sans bornes, et il avait pris
soin de le faire connaître tel pour se rendre redoutable à tout le
monde. Ce n'était pas que les affaires en allassent mieux. Tout
languissait, celles du dehors comme celles du dedans\,; il n'y donnait
ni temps ni soins, qu'en très légère apparence, et seulement pour les
retenir toutes à soi, où elles se fondaient et périssaient toutes. Son
crâne étroit n'était pas capable d'en embrasser plus d'une à la fois, ni
aucune qui n'eût un rapport direct et nécessaire à son intérêt
personnel. Il n'avait été occupé que d'amener tout à soi, et de conduire
son maître au point de n'oser, sans lui, remuer la moindre paille,
encore moins décider rien que par son avis, et conformément à son avis,
en sorte qu'en grâces comme en affaires, en choses courantes comme en
choses extraordinaires, il ne s'agissait plus de M. le duc d'Orléans, à
qui personne, pas même aucun ministre n'osait aller, pour quoique ce
fût, sans l'aveu et la permission du cardinal, dont le bon plaisir,
c'est-à-dire l'intérêt et le caprice, était devenu l'unique mobile de
tout le gouvernement. M. le duc d'Orléans le voyait, le sentait\,;
c'était un paralytique qui ne pouvait être remué que par le cardinal, et
dans lequel, à cet égard, il n'y avait plus de ressource.

Cet état causait, mais sourdement, un gémissement général, par la
crainte qu'avait répandue de soi cet homme qui pouvait tout, qui ne
connaissait aucune mesure, et qui s'était rendu terrible. Je m'en
affligeais plus que personne par amour pour l'État, par attachement pour
M. le duc d'Orléans, par la vue des suites nécessaires, et plus que
personne je voyais évidemment qu'il n'y avait point de remède, par ce
que je connaissais et j'approchais de plus près que personne. Malgré un
empire si absolu et si peu contredit, l'usurpateur du pouvoir suprême me
craignait encore et me ménageait. Il n'avait pu que contraindre la
confiance de M. le duc d'Orléans en moi, sa familiarité, l'habitude, le
goût, je n'oserais dire le soulagement de me voir et de me parler jusque
dans sa contrainte, dont il s'échappait quelquefois, et ma liberté, ma
vérité, dirai-je encore le désintéressement qui me rendait hardi à
n'écouter que le bien de l'État et mon attachement pour le régent, pour
lui parler ou lui répondre, retenait le cardinal en des mesures qu'il ne
gardait que pour moi, et qui me forçaient d'en conserver avec lui.

Dans cette situation personnelle, parmi tout ce mouvement, le cardinal
me détacha Belle-Ile pour me tourner sur la déclaration de premier
ministre, et tâcher non seulement de ranger tout obstacle de mon côté,
mais de n'oublier rien pour me rendre capable de l'y servir. Cet
entremetteur s'y prit avec tous les tours et toute l'adresse possibles.
Il me représenta que, par tout ce que nous voyions, il ne s'agissait que
du plus tôt ou du plus tard\,; que ne m'y pas prêter de bonne grâce
n'empêcherait pas à la fin que le cardinal ne l'emportât, et
m'exposerait à toute sa haine, dont je voyais tous les jours la
violence, la suite, la durée, le pouvoir\,; au lieu qu'en le servant en
chose qui était le but de ses plus ardents désirs, et chose que tôt ou
tard il n'était en ma puissance ni en celle de qui que ce fût de pouvoir
empêcher, je devais être assuré d'une reconnaissance proportionnée, qui
me ferait partager et les affaires et l'autorité de ce maître du régent
et du royaume. Je répondis à Belle-Ile qu'il pouvait bien juger que je
ne pouvais penser qu'il me vint faire une telle proposition de lui-même,
et il m'avoua sans peine que le cardinal l'avait chargé de me la faire,
et qu'il ne lui avait pas même défendu de me le dire.

C'était pour m'embarrasser que le cardinal s'y prit de la sorte, en me
réduisant de la sorte à répondre comme si c'eût été à lui-même. Je dis
donc à Belle-Ile de remercier le cardinal de cette confiance, que
j'accompagnai de force compliments\,; que la chose était de telle
importance qu'elle valait bien la peine de se donner le temps d'y
penser\,; qu'en attendant, je lui dirais ce qui me venait dans
l'esprit\,: qu'il me paraissait que le cardinal possédait tous les
avantages d'un premier ministre, déclaré tel par les plus expresses
patentes\,; que de se les faire expédier ne lui acquerrait rien de plus
du côté du pouvoir, de l'autorité, des pleines et entières fonctions,
mais que le titre, joint à l'effet et à la substance qu'il possédait et
qu'il exerçait sans contredit dans la plus vaste étendue, lui
soulèverait ceux qui étaient tout accoutumés à le voir et le sentir le
maître\,; et que, si quelque chose pouvait être capable de jeter par la
suite des nuages entre M. le duc d'Orléans et lui, ce serait la jalousie
et les soupçons qui naîtraient de cette qualité de premier ministre\,;
que je suppliais le cardinal, comme son serviteur, de peser cette
première réflexion qui me frappait sur cette affaire, de sentir que le
nom public et déclaré n'ajouterait quoi que ce soit à ce qu'il possédait
et qu'il exerçait en toute plénitude, et à quoi tout était déjà ployé et
accoutumé\,; que ce nom de plus n'en rendait pas la consistance plus
stable, parce que, dans la supposition, pour tout prévoir, qu'il pût
arriver qu'on lui voulût ôter le maniement des affaires, le titre, les
patentes, l'enregistrement et toutes les formes dont il serait revêtu,
ne le rendraient pas plus difficile à congédier que s'il n'en avait
point obtenu\,; que ces choses, ne faisant donc ni accroissement pour
lui, ni obstacles, ni rempart quelconque à une chute, ne lui devenaient
plus qu'un fardeau inutilement ajouté, mais avec danger d'en pouvoir
être entraîné, au lieu qu'en s'en tenant à sa situation présente, il
jouissait également de tout le pouvoir qu'il pouvait se proposer, et qui
était tel que nul titre ne pouvait l'accroître, qu'il ne réveillait et
ne révoltait personne par aucune nouveauté\,; qu'il ne semait ni
soupçon, ni jalousie, ni nuages dans l'esprit de M. le duc d'Orléans,
dont le germe pouvait produire des repentirs avec le temps, et de là des
suites\,; que l'intérêt de tous les deux n'était que de bien envisager
la proximité de la majorité, et de se conduire de telle sorte l'un et
l'autre, que l'habitude et la volonté du roi majeur, maître accessible,
succédât en leur faveur à ce que la nécessité avait fait pour le duc
d'Orléans, avait fait pour lui par le droit de sa naissance, et à ce que
l'estime, la confiance et le goût avaient obtenu de M. le duc d'Orléans
pour lui.

Mon but dans ce raisonnement, qui au fond était vrai et solide, était
d'éloigner tout engagement sans me rendre suspect de mauvaise volonté,
et de tacher de détourner le cardinal d'entreprendre ce que je sentais
bien que je tenterais en vain d'empêcher, mais que toutefois il n'était
pas en moi de ne pas tenter par toutes sortes de considérations
d'honneur, de probité, de fidélité pour l'État et pour l'intérêt
personnel de M. le duc d'Orléans. Belle-Ile avait trop d'esprit et de
sens pour ne pas sentir la force de ce que je lui exposais\,; mais il
connaissait trop bien le cardinal Dubois et sa passion effrénée pour le
titre public de premier ministre, pour espérer la moindre impression sur
lui de mon raisonnement, autre que le dépit, la fougue et la violence
d'un torrent qui ne cherche qu'à renverser toutes les digues qui se
rencontrent sur son chemin, et qui à la fin les brise. Il m'en avertit,
se remit sur tout ce que je ne pouvais promettre en servant une passion
si véhémente, et n'oublia rien de tout ce qu'il crut avoir le plus de
prise sur moi pour me toucher et m'ébranler, convenant d'ailleurs avec
moi de la tristesse de l'état des choses et d'une pareille nécessité.
Toutefois je demeurai ferme sur le principe secret qui me conduisait. Je
tâchai de lui faire entendre que des raisonnements sages et qui
n'allaient à rien moins qu'à diminuer le cardinal en quoi que ce soit,
n'étaient pas un refus, mais que j'estimais préalable à tout de lui
présenter des réflexions qui n'allaient qu'à, ses avantages avant que
d'aller plus loin.

Belle-Ile n'en pouvant tirer plus, se résolut de rendre compte au
cardinal de tout ce que je lui avais dit, et comme le cardinal ne
pouvait penser à autre chose, ce fut dès le soir même qu'il le lui
rendit. Il en arriva ce qu'il en avait prévu. Dès le lendemain il me le
renvoya avec des promesses nonpareilles, non seulement de conduire
toutes les affaires par mon conseil et de partager toute l'autorité avec
moi, mais de faire tout ce que je voudrais, et ce qu'il savait qui me
touchait le plus sur le rétablissement de tout ordre, droits et justice
dans les points qu'on me savait sensibles, où le désordre était devenu
plus grand. Je ris en moi-même de tant de magnifiques appâts. Dubois me
croyait sans doute aussi dupe que le cardinal de Rohan, à qui il avait
si solennellement promis de le faire premier ministre, et qui avait été
assez simple et assez follement ambitieux pour s'en être laissé
pleinement persuader. Mais ce manége, tout faux qu'il fût, m'acculait de
façon à ne pouvoir plus reculer. Toute mon adresse ne butta qu'à
m'assurer le privilège des Normands, dont il n'est rien de plus rare que
de tirer un oui ou un non. J'eus recours à véritablement bavarder sur
l'incertitude et la volubilité de M. le duc d'Orléans, qui change en un
moment tout ce qu'on croit tenir de sa facilité, de son crédit sur lui,
des impressions qu'il a reçues des raisons qu'on lui a présentées, après
quoi très souvent on se trouve non seulement à recommencer, mais plus
éloigné que l'on n'était avant d'avoir proposé\,; que ce que je ferais,
ce serait de le sonder et de profiter de ce que je trouverais de
favorable à mon dessein, la première fois que je le verrais. J'ajoutai
que je disais la première fois que je le verrais, parce que, si j'allais
le trouver en jour qui n'était pas l'ordinaire, il serait dès là en
garde sur ce qui m'amènerait, et par là je gâterais toute la besogne. Ce
que j'alléguais en effet pour différer et gagner du temps était en effet
tellement dans le vrai du caractère toujours soupçonneux de M. le duc
d'Orléans, et si parfaitement connu du cardinal et même de Belle-Ile\,;
par ce qu'il en savait de ceux qui en avaient l'expérience, par
eux-mêmes, que Belle-Ile s'en contenta, et le cardinal aussi, qui me le
renvoya le lendemain pour me le dire, me faire des remerciements infinis
des promesses réitérées, surtout bien confirmer la bonne volonté que je
lui témoignais, et tout doucement m'insinuer et me recorder ma leçon.

Enfin mon jour ordinaire venu, il me fallut aller chez M. le duc
d'Orléans, à Versailles, pour y arriver à mon heure, qui était sur les
quatre heures après midi, temps où il n'y avait plus personne chez lui.
Entrant tout de suite, je trouvai Belle-Ile seul dans ce grand cabinet,
où le maréchal de Villeroy avait été arrêté, qui m'attendait au passage,
pour me recommander l'affaire, et tâcher de la bombarder, proposition
qu'il ne m'avait point faite jusqu'alors, et qui venait apparemment tout
fraîchement d'éclore du cerveau embrasé du cardinal. Belle-Ile me lâcha
ce saucisson dans l'oreille. Je passai sans m'arrêter, et j'entrai dans
le cabinet de M. le duc d'Orléans.

Après quelques moments de conversation, je mis sur son bureau les
papiers dont j'avais à lui rendre compte. Il se mit à son bureau, et je
m'assis vis-à-vis de lui, comme j'avais accoutumé. Je trouvai un homme
occupé, distrait, qui me faisait répéter, lui qui était au fait avant
qu'on eût achevé, et qui se plaisait assez souvent à mêler quelques
plaisanteries dans les affaires les plus sérieuses, surtout avec moi, à
placer quelques bourdes et quelques disparates pour m'impatienter et
s'éclater de rire de la colère où cela me mettait toujours, et à se
divertir de ce que je ne m'y accoutumais point. Cette distraction et ce
sérieux me donna lieu, au bout de quelque temps, de lui en demander la
cause. Il balbutia, il hésita et ne s'expliqua point. Je me mis à
sourire et à lui demander s'il était quelque chose de ce qu'on m'avait
dit tout bas, qu'il pensait à faire un premier ministre et à choisir le
cardinal Dubois. Il me parut que ma question le mit au large, et que je
le tiroir de l'embarras de s'en taire avec moi, ou de m'en parler le
premier. Il prit un air plus serein et plus libre, et me dit qu'il était
vrai que le cardinal Dubois en mourait d'envie\,; que, pour lui, il
était las des affaires et de la contrainte où il était à Versailles d'y
passer tous les soirs à ne savoir que devenir\,; que du moins il se
délassait à Paris par des soupers libres dont il trouvait la compagnie
sous sa main, quand il voulait quitter le travail ou au sortir de sa
petite loge de l'Opéra. Mais qu'avoir la tête rompue toutes les journées
d'affaires pour n'avoir les soirs qu'à s'ennuyer, cela passait ses
forces et l'inclinait à se décharger sur un premier ministre, qui lui
donnerait du repos dans les journées et la facilité de s'aller divertir
à Paris. Je me mis à rire, en l'assurant que je trouvais cette raison
tout à fait solide, et qu'il n'y avait pas à y répliquer. Il vit bien
que je me moquais, et me dit que je ne sentais ni la fatigue de ses
journées, ni le vide presque aussi accablant de ses soirées, qu'il n'y
avait qu'un ennui horrible chez M\textsuperscript{me} la duchesse
d'Orléans, et qu'il ne savait où donner de la tête.

Je répondis que de la façon dont j'étais avec M\textsuperscript{me} la
duchesse d'Orléans depuis le lit de justice des Tuileries, je n'avais
rien à dire sur ce qui la regardait, mais que je le trouvais bien à
plaindre si cette ressource d'amusement lui manquait, de ne savoir pas
s'en faire d'autres, lui régent du royaume, avec autant d'esprit,
d'ornements dans l'esprit de toutes les sortes, et d'aussi bonne
compagnie quand il lui plaisait\,; que je le priais de se souvenir de ce
qu'il avait vu du feu prince de Conti, à qui il n'était inférieur en
rien, sinon en délaissement de soi-même, et faire une comparaison de ce
prince avec lui\,; que le roi le haïssait et le témoignait d'une façon
si marquée et si constante que personne ne l'ignorait\,; qu'il était
donc non seulement sans crédit, mais qu'il n'était point de courtisan
qui ne sentît qu'on déplaisait au roi de le fréquenter, qu'il n'avait
pas oublié non plus dans quelle frayeur on était de lui déplaire, et que
le désir de lui être agréable était généralement poussé jusqu'à
l'esclavage et aux plus grandes bassesses\,; que nonobstant des raisons
si puissantes sur l'âme d'une cour aussi complètement asservie, il avait
vu que M. le prince de Conti n'y paraissait jamais, et il y était
assidu, que dans l'instant il ne fût environné de tout ce qu'il y avait
de plus grand, de meilleur, de plus distingué de tout âge\,; qu'on se
pelotonnait autour de lui\,; que tous les matins sa chambre était
remplie à Versailles du plus important et du plus brillant de la cour,
où on était assis en conversation toujours curieuse et agréable, et où
on se succédait les uns aux autres deux ou trois heures durant\,; qu'à
Marly, où tout était bien plus sous les yeux du roi qu'à Versailles, non
seulement le prince de Conti était environné dans le salon dès qu'il y
paraissait, mais que ce qui composait la plus illustre, la plus
distinguée, la plus importante compagnie, s'asseyait en cercle autour de
lui, et en oubliait souvent les moments de se montrer au roi, et les
heures des repas. Dans la journée, à la cour comme à Paris, ce prince
n'était jamais à vide ni embarrassé de passer d'agréables soirées, tout
cela sans le secours de la chasse ni du jeu, qui n'étaient pour lui que
des effets rares de complaisance et nullement de son goût. Jamais dans
l'obscur, dans le petit, dans la crapule, ses débauches avec gens de
bonne compagnie, et de si bon aloi qu'en leur genre ils faisaient
honneur partout\,; d'ailleurs bonnes lectures de toute espèce et
fréquentation chez lui de gens de toute robe et de diverses sciences,
outre les gens de guerre et de cour, à tous lesquels il parlait leur
propre langage, et les savait ravir en se mettant à leur unisson\,;
attentif à plaire au valet comme au maître par une coquetterie pleine de
grâces et de simplicité qui était née avec lui. La princesse sa femme,
pour qui il avait toutes sortes d'égards, mais qui ne savait que jouer,
ne lui était point un obstacle, quoiqu'il vécût comme point avec elle,
et qu'il n'y pût trouver la moindre ressource. Il rendait avec attention
et distinction ce qui était dû à chacun\,; il était attentif à flatter
chaque seigneur, chaque militaire par des faits anciens ou nouveaux
qu'il savait placer naturellement\,; il entendait merveilleusement à
faire des récits agréables, où eux ou les leurs se trouvaient avec
distinction. En un mot, c'était un Orphée qui savait amener autour de
soi les arbres et les rochers par les charmes de sa lyre, et triompher
de la haine du feu roi, si redouté jusqu'au milieu de sa cour, sans
paraître y prendre la moindre peine, et avoir toutes les dames à son
commandement par l'agrément de sa politesse et la discrétion de sa
galanterie. En un mot, le contraste le plus parfait de M. le Duc, devant
qui tout fuyait, tout se cachait comme devant un ouragan, et qui passait
sa vie dans la tristesse, dans l'ennui, dans l'embarras que faire, où
aller, que devenir, et dans la rage de toutes les espèces de jalousies,
ayant toutefois beaucoup d'esprit, de savoir, de valeur, et toute la
faveur de sa double alliance avec le bâtard favori et la bâtarde du feu
roi.

Je demandai ensuite à M. le duc d'Orléans qui l'empêchait d'imiter ce
prince de Conti, ayant autant ou plus d'esprit et de savoir que lui,
sachant autant de faits d'histoire, de guerre et de cour que lui,
n'ayant pas moins de valeur, et {[}ayant{]} de plus commandé les armées,
vu l'Espagne à revers, non moins de grâces et de mémoire pour des récits
et des conversations charmantes, et, outre ces avantages encore plus
grands que dans le prince de Conti, se trouvant, au lieu de la disgrâce
dont ce prince n'était jamais sorti, tenir les rênes du gouvernement et
la balance des grâces, qui seule mettait tout le monde à ses pieds, et
lui présentait à choisir, à son gré, parmi tout ce qu'il y avait de
meilleur en chaque genre. J'ajoutai que pour cela il n'y avait qu'un pas
à faire, qui était de préférer la bonne compagnie à la mauvaise, de la
savoir distinguer et attirer, de souper joyeusement, mais seulement avec
elle\,; de sentir que ces soupers devenaient honteux passé dix-huit ou
tout au plus vingt ans, où le grand bruit, les propos sans mesure, sans
honnêteté, sans pudeur, faisaient injure à l'homme\,; où une ivresse,
continuelle le déshonorait, qui bannissait tout ce qui n'avait même
qu'un reste d'honneur extérieur et de maintien, et d'où la crapule et
l'obscurité des convives si déshonorés repoussaient tout homme qui ne
voulait pas l'être, et dont le public lui faisait un mérite\,; que de
tout cela je concluais que l'ennui de ses soirées à Versailles n'était
que volontaire, que celles qu'il y regrettait et qu'il allait chercher à
Paris ne seraient pas souffertes à aucun particulier de la moitié de son
âge, sans être éconduit de toutes les compagnies où il voudrait se
présenter, et que ce qu'il n'avait pas voulu retrancher pour Dieu, il le
bannit du moins pour les hommes et pour lui-même\,; que rien ne
l'empêchait d'avoir à Versailles un souper pour les gens distingués de
la cour, de la meilleure compagnie, qui s'empresseraient tous d'y être
admis, quand elle serait sur le pied de n'être point mêlée, ni salie
d'ordures, d'impiétés et d'ivrognerie, dont à ne considérer que son âge,
son rang et son état, le temps en était de bien loin outrepassé pour
lui\,; que la proximité de la majorité l'y conviait encore pour ôter de
dessus lui des prises si funestes et si sensibles qui seules pouvaient
l'écarter bien loin, et dont il ne pouvait se dissimuler l'indignation
publique, le mépris dans lequel nageait, pour ainsi dire, les obscurs
compagnons de ses scandaleuses soirées, tout ce qui en rejaillissait
sans cesse sur lui, le crédit qu'elles donnaient à tout ce que ses
ennemis voulaient imaginer et les pernicieuses semences qui s'en
jetaient pour des temps même peu éloignés. Je conclus par le prier de se
souvenir qu'il y avait des années que je gardais un silence exact sur sa
conduite personnelle, et que je ne lui en parlais maintenant que parce
qu'il m'y avait forcé en me montrant l'abîme où l'abandon à cette
conduite l'allait précipiter, de se dégoûter des affaires par l'ennui de
ses soirées, et de chercher à s'en délivrer, par se décharger sur un
premier ministre.

M. le duc d'Orléans eut la patience d'écouter, les coudes, sur son
bureau et sa tête entre ses deux mains, comme il se mettait toujours
quand il était en peine et embarras et qu'il se trouvait assis,
d'écouter, dis-je, cette pressante ratelée\footnote{Vieux mot qui ne
  s'employait que dans le style familier. \emph{Dire sa ratelée}
  signifiait \emph{dire librement tout ce que l'on pensait}.}, bien plus
longue que je ne l'écris. Comme je l'eus finie, il me dit que tout cela
était vrai, et qu'il y avait pis encore\,; c'était, ajouta-t-il, qu'il
n'avait plus besoin de femmes, et que le vin ne lui était plus de rien,
même le dégoûtait. «\,Mais, monsieur, m'écriai-je, par cet aveu, c'est
donc le diable qui vous possède, de vous perdre pour l'autre monde et
pour celui-ci, par les deux attraits dont il séduit tout le monde, et
que vous convenez n'être plus de votre goût ni de votre ressort que vous
avez usé\,; mais à quoi sert tant d'esprit et d'expérience\,; à quoi
vous servent jusqu'à vos sens, qui las de vous perdre, vous font, malgré
eux sentir la raison\,? Mais avec ce dégoût du vin et cette mort à
Vénus, quel plaisir vous peut attacher à ces soirées et à ces soupers,
sinon du bruit et des gueulées qui feraient boucher toute autre oreille
que les vôtres, et qui, plaisir d'idées et de chimères, est un plaisir
que le vent emporte aussitôt, et qui n'est plus que le déplorable
partage d'un vieux débauché qui n'en peut plus, qui soutient son
anéantissement par les misérables souvenirs que réveillent les ordures
qu'il écoute\,?» Je me tus quelques moments, puis je le suppliai de
comparer des plaisirs honteux de tous points, des plaisirs même qui se
dérobaient à lui sans espérance de retour avec des amusements honnêtes,
décents, des délassements de son âge, de son rang, de la place qu'il
tenait dans l'État, et que, sous un autre nom, il devait tâcher de
conserver après la majorité\,; des amusements qui le montreraient tel
qu'il était, et qui lui concilieraient tout le monde, par l'honneur de
vivre quelquefois avec lui, et par les espérances qui s'y attacheraient
et qui lui attacheraient dès lors tous ceux qui les concevraient pour
eux ou pour les leurs, ceux même qui seraient au-dessus et au-dessous de
ces espérances, par la joie de voir enfin mener une vie raisonnable et
digne au maître de toutes les affaires et de toutes les fortunes, et
d'être délivrés de la frayeur de voir, avec le temps, le roi tomber dans
des égarements plus pardonnables à la jeunesse, dont il lui donnerait
l'exemple, mais si insupportables sur le trône, et si peu connus des
têtes couronnées, plus étroitement esclaves de toutes bienséances, et
plus nécessairement que pas un de leurs sujets. Je lui dis encore de
penser à ce que dirait la cour, la ville, toute la France et les pays
étrangers, de voir un régent de son âge, et qui s'était montré si
capable de l'être, l'abdiquer, pour ainsi dire, et en revêtir un autre,
pour vaquer à la débauche plus librement et avec plus de loisir\,; et
quelle prise ne donnerait-il pas sur lui à ses ennemis, aux mécontents,
aux brouillons, aux ambitieux, d'intriguer auprès du roi pour le faire
remercier des soins qu'il ne voulait plus rendre\,; puisqu'il s'en était
déchargé sur un autre, et de congédier cet autre qui n'aurait plus de
soutien, pour le remplacer d'un ou de plusieurs de son goût et de son
choix\,; et que devient alors un prince de sa naissance, après avoir si
longtemps régné, tombé tout à coup dans l'anéantissement de l'état
particulier, et qui n'en jouit même que parmi les craintes et les
soupçons qu'on a ou qu'on fait semblant d'avoir, pour les inspirer à un
roi encore sans expérience et sans réflexion, facile à être conduit où
on le veut mener. Je terminai cette reprise par l'exemple de Gaston
confiné à Blois, où il passa les dernières années de sa vie, et où il
mourut dans la situation la plus triste, la plus délaissée, on ose dire
d'un fils de France, la plus méprisée.

Je crus alors en avoir dit assez, peut-être même trop emporté par la
matière, et devoir attendre ce que cela produirait. Après un peu de
silence, M. le duc d'Orléans se redressa sur sa chaise\,: «\,Hé bien\,!
dit-il, j'irai planter mes choux à Villers-Cotterêts\,;» se leva et se
mit à se promener dans le cabinet, et moi avec lui. Je lui demandai qui
le pouvait assurer qu'on les lui laisserait planter en paix et en repos,
même en sûreté\,; qu'on ne lui chercherait pas mille noises sur son
administration\,; que sur le pied qu'on l'avait fait passer en France et
en Espagne, du temps du feu roi, qui est-ce qui pouvait lui répondre
qu'on ne lui ferait pas accroire qu'il tramerait des mouvements et de
dangereux complots, et qu'on ne parvint à effrayer trop fortement le
roi, encore sans dauphin, d'un prince d'autant d'esprit, de valeur, de
capacité, qui avait si longtemps régné sous un autre nom, qui ne pouvait
être destitué de gens de main et de créatures, mais justement piqué,
outré de son état présent, et qui se trouvait jusqu'alors héritier
présomptif de la couronne, avec la liaison la plus intime, si
soigneusement achetée et ménagée entre lui et les Anglais, qui
gouvernaient l'empereur et la Hollande. Il y eut encore là quelques
tours de cabinet en silence, après lesquels il m'avoua que cela méritait
réflexion, et continua une douzaine de tours en silence.

Se trouvant à la muraille, au coin de son bureau où il y avait par
hasard deux tabourets, j'en vois encore la place, il me tira par le bras
sur l'un en s'asseyant sur l'autre, et se tournant tout à fait vers moi,
me demanda vivement si je ne me souvenais pas d'avoir vu Dubois valet de
Saint-Laurent, et se tenant trop heureux de l'être\,; et de là, reprit
tous les degrés et tous les divers états de sa fortune, jusqu'au jour où
nous étions, puis s'écria\,: «\, Et il n'est pas content\,; il me
persécute pour être déclaré premier ministre, et je suis sûr, quand il
le sera, qu'il ne sera pas encore content\,; et que diable pourrait-il
être au delà\,?» Et tout de suite se répondant à lui-même\,: «\, Se
faire Dieu le Père, s'il pouvait. --- Oh\,! très assurément,
répondis-je, c'est sur quoi on peut bien compter\,; c'est à vous,
monsieur, qui le connaissez si bien, à voir si vous êtes d'avis de vous
faire son marchepied, pour qu'il vous monte sur la tête. --- Oh\,! je
l'en empêcherais bien,\,» reprit-il. Et le voilà de nouveau à se
promener par son cabinet, sans plus rien dire, ni moi non plus, tout
occupé que j'étais de ce «\,je l'en empêcherais bien,\,» à la suite
d'une conversation si forte et de ce vif récit et encore plus vivement
terminé qu'il venait de me faire de la vie du cardinal Dubois \emph{ab
incunabulis}\footnote{Depuis le berceau.} jusqu'alors, où je ne l'avais
point porté ni donné aucune occasion. Cette seconde promenade dura assez
de temps et toujours en silence, lui la tête basse comme quand il était
embarrassé et peiné, moi comme ayant tout dit et attendant ce qui
sortirait de ce silence après une telle conversation. Enfin il se remit
à son bureau à sa place ordinaire, et moi vis-à-vis de lui assis, lui,
comme d'abord, ses coudes sur le bureau, sa tête fort basse entre ses
deux mains.

Il demeura plus d'un demi-quart d'heure de la sorte, sans remuer, sans
ouvrir la bouche ni moi non plus qui n'ôtais pas les yeux de dessus lui.
Cela finit par soulever sa tête sans remuer d'ailleurs, l'avancer vers
moi et me dire d'une voix basse, faible, honteuse, avec un regard qui ne
l'était pas moins\,: «\, Mais pourquoi attendre et ne le pas déclarer
tout à l'heure\,?» Tel fut le fruit de cette conversation. Je
m'écriai\,: «\,Ah\,! monsieur, quelle parole\,! Qui est-ce qui vous
presse si fort\,? N'y serez-vous pas toujours à temps\,? donnez-vous au
moins le temps de la réflexion à tout ce que nous venons de dire, et à
moi de vous expliquer ce que c'est qu'un premier ministre et le prince
qui le fait.\,» Il remit doucement sa tête entre ses deux mains sans
répondre une seule parole. Quoique atterré d'une résolution si prompte
après ce que lui-même avait dit des degrés et de l'ambition du cardinal
Dubois, je sentis que le salut de la chose, si tant était qu'il se pût
espérer, n'était plus dans les raisons d'opposition, qui étaient toutes
épuisées, mais uniquement dans le délai. Il fut court, car après un peu
de silence, il se leva et me dit\,: «\,Ho\,! bien donc, revenez ici
demain à trois heures précises raisonner encore de cela, et nous en
aurons tout le temps.\,» Je pris les papiers que j'avais à reprendre et
je sortis. Il courut après moi et me rappela pour me dire\,: «\,Au
moins, demain à trois heures\,; je vous prie, n'y manquez pas,\,» et
referma la porte. Je fus surpris de retrouver Belle-Ile en embuscade où
je l'avais laissé en entrant, et qui avait eu la patience d'y persévérer
plus de deux grosses heures à m'attendre. Il me suivit pour me demander
si cela était fait. Je lui dis que la conversation s'était étendue sur
plusieurs matières dont quelques-unes m'avaient conduit à tâter le pavé,
que je l'avais trouvé assez bon\,; mais qu'il connaissait M. le duc
d'Orléans soupçonneux, et qui n'aimait pas à conclure ni à être
pressé\,; que je reviendrais le lendemain où je verrais ce qui se
pourrait faire, sans toutefois lui répondre de rien. Je répondis de la
sorte à Belle-Ile, parce qu'il avait vu M. le duc d'Orléans me rappeler,
qu'il avait pu entendre l'ordre qu'il me donnait de revenir le
lendemain\,; que ce retour enfin ne pourrait être ignoré de lui ni du
cardinal Dubois, trop alerte pour n'être pas informé avec précision de
tous les moments de M. le duc d'Orléans dans une telle crise, et que la
cachotterie eût été également inutile et préjudiciable à moi, qui
voulais aller au bien, mais garder avec eux des mesures. D'ailleurs ma
réponse fut en des termes qui ne pouvaient blesser le cardinal.

\hypertarget{chapitre-xvi.}{%
\chapter{CHAPITRE XVI.}\label{chapitre-xvi.}}

1722

~

{\textsc{Autre conversation singulière et curieuse entre M. le duc
d'Orléans et moi sur faire un premier ministre, dont je persiste à
n'être pas d'avis.}} {\textsc{- Malheur des princes indiscrets et peu
fidèles au secret.}} {\textsc{- Exemples des premiers ministres en tous
pays depuis Louis XI.}} {\textsc{- Quel est nécessairement un premier
ministre.}} {\textsc{- Quel est le prince qui fait un premier
ministre.}} {\textsc{- Embuscade de Belle-Ile.}} {\textsc{- Le cardinal
Dubois déclaré premier ministre.}} {\textsc{- Il me le mande et veut me
faire accroire qu'il m'en a l'obligation, et n'oublie rien pour en
persuader le public.}} {\textsc{- Conches\,; quel.}} {\textsc{- Je vais
le lendemain à Versailles, où je vois le cardinal Dubois chez M. le duc
d'Orléans.}} {\textsc{- Indignité des Rohan.}} {\textsc{- Épisode
nécessaire.}} {\textsc{- Plénoeuf, sa femme et sa fille, depuis marquise
de Prie, et maîtresse déclarée de M. le Duc.}} {\textsc{- Infamie du
marquis de Prie.}} {\textsc{- Liaison intime de Belle-Ile et de Le Blanc
entre eux et avec M\textsuperscript{me} de Plénoeuf.}} {\textsc{- Elle
leur attire la haine, puis la persécution de M\textsuperscript{me} de
Prie et de M. le Duc.}} {\textsc{- Le cardinal Dubois, fort avancé dans
son projet d'élaguer}}

~

\footnote{Isoler.} entièrement M. le duc d'Orléans, se propose de perdre
Le Blanc et peut-être Belle-Ile. Conduite qu'il y tient. Désordre des
affaires de La Jonchère, trésorier de l'extraordinaire des guerres,
dévoué à M. Le Blanc. Belle-Ile toujours mal avec M. le duc d'Orléans.
Mariage futur de M\textsuperscript{lle} de Beaujolais avec l'infant don
Carlos, déclaré. Mariage du prince électoral de Bavière avec une
archiduchesse, Joséphine. Fort pour amuser le roi. Mort de Ruffé.
Étrange licence en France. Mort de Dacier. Érudition profonde de sa
femme, et sa modestie. Mort, famille et caractère de la duchesse de
Luynes (Aligre). Mort de Reynold. Mariage de Pezé avec une fille du
premier écuyer.

Le lendemain, 22 août, je vins au rendez-vous, et je trouvai encore
Belle-Ile dans ce grand cabinet, qui m'attendait au passage, et qui me
pressa de finir l'affaire du cardinal\,; je payai de mine et
d'empressement d'entrer dans le cabinet de M. le duc d'Orléans, que j'y
trouvai seul, qui s'y promenait avec l'air plus dégagé que la veille.
«\,Eh bien\,! me dit-il d'abordée, qu'avons-nous encore à dire sur
l'affaire d'hier\,? Il me semble que tout est dit, et qu'il n'y a plus
qu'à déclarer dès tout à l'heure le premier ministre.\,» Je reculai deux
pas et je lui dis que pour chose de telle importance, c'était là un
conseil bientôt pris. Il répondit qu'il y avait bien pensé, que tout ce
que je lui avais dit là-dessus lui était fort présent\,; mais qu'au
bout, il était crevé d'affaires tout le jour, d'ennui tous les soirs, de
persécutions du cardinal Dubois à tous les moments.

Je repris que cette dernière raison était la plus puissante\,; que je ne
m'étonnais pas de l'empressement du cardinal, mais beaucoup de son
succès sur lui qui était si soupçonneux\,; que je le suppliais de se
bien représenter deux choses la première, que pour le soulagement des
affaires et la liberté d'aller, tant qu'il voudrait, chercher l'Opéra et
ses soupers à Paris, il pouvait en jouir tant que bon lui semblerait,
parce que, le cardinal jouissait si pleinement et si ouvertement de la
toute-puissance, et que tout le monde le voyait et le sentait si
pleinement, qu'il n'y avait plus qui que ce fût, Français ou ministres
étrangers, qui osât se jouer à aller directement à Son Altesse Royale,
et qui ne fût bien convaincu, qu'affaire, justice ou grâces ne dépendit
uniquement du cardinal, n'allât à lui, ne se tînt pour battu s'il le
trouvait contraire, sans oser tenter d'aller plus haut, demeurait sûr de
ce qu'il demandait s'il trouvait le cardinal favorable, et le plus
souvent s'en tenait là, sans que lui régent en entendît parler, ou que
les gens ne venaient à lui que pour la forme, et lors seulement que le
cardinal le leur prescrivait, ce qu'il leur ordonnait aussi quelquefois
dans des cas de refus, dans l'espérance de leur faire prendre le change
et de se décharger du refus sur lui\,; que je m'étonnais qu'il fût
encore à s'apercevoir d'une chose si évidente qu'elle n'était ignorée de
personne\,; et que moi-même, depuis mon retour d'Espagne, si j'avais à
demander la moindre chose, et la plus facile et la plus raisonnable,
pour moi ou pour quelque autre à Son Altesse Royale, je me garderais
bien de lui en parler sans m'être assuré du cardinal auparavant, et me
tiendrais très sûr du refus si j'allais droit à elle sans l'attache du
cardinal, et au contraire, avec certitude morale de sa volonté que
j'obtiendrais ce que je lui aurais présenté à demander. Que les choses
étant à ce point d'autorité, et d'autorité affichée, je ne voyais nul
accroissement possible à l'exercice actuel qu'il en faisait
publiquement, par la déclaration ni par les patentes de premier
ministre, ni plus de soulagement et de liberté que Son Altesse Royale en
pouvait prendre dès à présent sans cela\,; mais que j'y apercevais pour
le cardinal Dubois une différence à la vérité imperceptible à l'exercice
actuel de sa toute-puissance, mais qui n'en était pas moins essentielle,
et que c'était là la seconde chose sur laquelle je demandais à Son
Altesse Royale toutes ses réflexions. C'est que, quelle que fût
l'étendue et la plénitude actuelle du pouvoir qu'avait saisi et
qu'exerçait pleinement le cardinal Dubois, il ne laissait pas de se
trouver, comme l'oiseau, sur la branche, exposé à être congédié au
premier instant que la volonté en prendrait à Son Altesse Royale, sans
autre forme ni embarras que de le renvoyer, de faire dire aux ministres
étrangers de ne se plus adresser à lui, et aux ministres et secrétaires
d'État de cesser de recevoir et de lui plus demander d'ordres, et de lui
plus rendre compte de rien\,; et sans même ce très peu de si courtes et
si simples mesures, envoyer un secrétaire d'État lui porter l'ordre de
s'en aller en son diocèse, prendre ou sceller ses papiers, et le faire
partir sur-le-champ. Que quoique la patente enregistrée et la
déclaration de premier ministre ne pût le parer de la chute, autre chose
était de pouvoir être renvoyé en un instant comme je venais de montrer
que cela se pouvait toutefois et quantes, autre chose de ne le pouvoir
que par des formes qui donnent du temps et des ressources, et moyen de
se raccommoder et de faire jouer des ressorts dans l'intervalle, de
dresser et de causer\footnote{Motiver.} une déclaration révocataire,
dont il pouvait être averti, de l'envoyer au parlement, de l'y faire
enregistrer. Je suppliai donc M. le duc d'Orléans de faire l'attention
si nécessaire à cette différence d'un homme qui est maître de tout sans
autre titre que la volonté de son maître, exprimée par le seul usage
dans lequel il l'autorise simplement de fait, ou qui le devient par
titre exprès, par déclaration, par enregistrement.

J'aurais bien ajouté à un autre qu'à M. le duc d'Orléans, de quel danger
il était pour lui d'établir premier ministre en titre un homme aussi
capable que l'était le cardinal Dubois de saisir toutes les avenues du
roi à force d'argent, de grâces, de souplesses, de se rendre maître de
l'esprit d'un enfant devenu majeur, et sans expérience de rien, et lui
revêtu en tigre, tandis que son premier ministre s'en trouvait dépouillé
de droit par la majorité, se délivrer d'une subordination importune, et
le faire renvoyer comme le cardinal Mazarin avait fait Gaston. Mais
c'était chose que l'ensorcellement de M. le duc d'Orléans le rendait
incapable d'entendre, puisque tout ce que je lui en avais dit la veille
avait fait si peu d'impression\,; et d'ailleurs, quoique je n'eusse rien
dit qui tendit à aucune diminution de la pleine puissance du cardinal
Dubois, je me commettais assez avec lui par la faiblesse et
l'indiscrétion de M. le duc d'Orléans, de m'opposer à sa déclaration de
premier ministre, pour ne m'exposer pas inutilement à me hasarder de
produire cette dernière réflexion, quelque importante qu'elle pût
être\,; et voilà comme le défaut de sentiment et de secret dans les
princes ferme la bouche à leurs meilleurs serviteurs, et les prive des
plus essentielles connaissances. Je me tus après un discours si
péremptoire, pour voir ce qu'il opérerait. La promenade continua sept ou
huit tours en silence, mais l'air embarrassé et la tête basse, puis il
s'alla mettre à son bureau dans l'attitude de la veille, et je m'assis
vis-à-vis, le bureau seulement entre lui et moi.

Ce mouvement n'interrompit point le silence. J'avais bien résolu de ne
le pas rompre le premier. Enfin il leva un peu la tête, me regarda et me
fit souvenir, je n'en avais pourtant pas besoin, que je lui voulais dire
quelque chose, dès la veille, sur l'état d'un premier ministre. Je lui
répondis qu'il savait trop bien l'histoire de son pays et des voisins
pour ignorer les maux et les malheurs que la Hongrie, Vienne,
l'Angleterre et l'Espagne avaient soufferts du gouvernement de leurs
premiers ministres, à l'exception unique et dans tous les points, du
seul cardinal Ximénès, dont la capacité, le désintéressement et la
droiture avait fait un phénix, et n'avait pu toutefois le garantir du
poison des Flamands\,; que ce serait perdre le temps de lui retracer les
faits de tous ces premiers ministres, excepté Ximénès\,; les désordres
et les ruines que leur intérêt personnel avait causés\,; la haine et le
mépris dont leur conduite avait couvert leurs maîtres, sans en excepter
même Henri VIII, qui ne s'en releva que par la ruine du cardinal Wolsey.
Que, pour se renfermer en France, le plus habile, pour ne rien dire de
plus, le plus soupçonneux, le plus rusé et le plus précautionné de tous
nos rois avait été livré au duc de Bourgogne par le cardinal Balue,
réduit à en subir la loi, à tout instant en peine de sa vie, réduit à
passer par tout ce que son ennemi voulut, et notamment à combattre en
personne avec lui deux jours après contre les Liégeois qu'il lui avait
soulevés, et qu'il se vit forcé à l'aider à réduire, c'est peu dire, à
les mettre sous son joug. Aussi Louis XI, rendu à lui-même, enferma-t-il
Balue, tout cardinal qu'il fût et qu'il l'avait fait, dans une cage de
fer pendant tant d'années, et se garda bien de lui donner un successeur.

Louis XII fut deux fois réduit à deux doigts de sa ruine, et la dernière
précipité dans le schisme, toutes les deux par l'ambition de son premier
ministre de se faire élire pape, dont toutes les deux fois il se crut
assuré, et toutefois les historiens sont pleins des louanges du cardinal
d'Amboise, parce qu'il n'eut point d'autres bénéfices que l'archevêché
de Rouen. Mais quelle y fut sa magnificence qui fait encore l'admiration
d'aujourd'hui\,? Sept ou huit frères ou neveux comblés des plus grands
bénéfices, de la grande maîtrise de Malte, grand maître de France,
maréchaux de France, gouverneurs de Milan, un neveu cardinal\,: voilà
pourtant le meilleur premier ministre et le plus applaudi qu'aient eu
nos rois.

La Ligue fut conçue et préparée, et l'intelligence et l'union avec
l'Espagne pour la faire éclore, par le cardinal de Lorraine, premier
ministre, pour transférer la couronne dans sa maison, et qui n'eut
d'autre objet pour la guerre et pour cette paix funeste par laquelle il
fit rendre plus de quarante places et de vastes pays à l'Espagne,
qu'elle n'eût pas repris en un siècle, et qu'il se dévoua par un si
perfide service, dont la mort du duc de Guise son frère, tué par
Poltrot, l'empêcha de voir le succès et l'accabla de la plus profonde
douleur à Trente, où, à l'acclamation de la clôture du concile, il
acclama tous les rois en nom collectif pour éviter, contre la coutume
constante jusqu'alors, de nommer le roi de France le premier, puis tous
les autres après, et gratifier l'Espagne en un point si sensible, depuis
que Philippe II avait osé le premier entrer en compétence si
boiteusement fondée sur la préséance de l'empereur Charles-Quint, parce
qu'il était aussi roi d'Espagne\,: ce dont le cardinal de Lorraine,
premier ministre, jeta de si solides fondements, dont l'effet ne fut
suspendu que par la mort de son frère\,; le fils de ce frère si jeune
alors, et depuis tué à Blois au moment qu'il allait enlever la couronne
à Henri III, à force de troubles, de partis, de guerre et de désordres,
sut trop bien en profiter, et le duc de Mayenne, son oncle, après lui,
en sorte que ce ne fut pas sans des miracles redoublés, et sans des
merveilles, qui, en tout genre, ont illustré Henri IV et la noblesse
française, que ce prince, après tant de hasards, de détresses, de
victoires, rassura la couronne sur sa tête et dans sa postérité, mais
dont la fin ne le rendit pas moins la victime de l'esprit encore fumant
de la Ligue abattue, comme Henri III l'avait été de sa force et de sa
fureur.

Vint après le faible et funeste gouvernement de la reine sa veuve, ou
plutôt du maréchal d'Ancre, sous son nom, dont la catastrophe rendit la
paix au royaume. Mais Louis XIII était si jeune, et, par une détestable
politique, si enfermé, si étrangement élevé qu'il ne savait pas lire
encore, et qu'il ignorait tout, comme il s'en est souvent plaint à mon
père, à quoi suppléa un sublime naturel, une piété sincère, une justice
exquise, la valeur d'un héros et la science des capitaines\,; mais si
malheureux en mère, en frère unique, en épouse, vingt ans stérile, en
santé, qui attirait les yeux de tous sur Gaston et qui faisait sa force,
en partis encore fumants, dont les plus grands obligeaient à compter
avec eux, et les huguenots armés, organisés, maîtres de tant de places
et de pays, formant un État dans l'État, forcèrent Louis XIII à faire un
premier ministre, qui fut un génie puissant et transcendant en tout,
mais qui, avec tant et de si grandes qualités, ne fut pas exempt de la
passion de se maintenir, et qui fit voler bien des têtes, à la vérité
presque toutes justement.

La minorité du feu roi soumit la France à une régente pour le moins
aussi espagnole d'inclination que de naissance, qui se choisit un
premier ministre étranger, et le premier qui fut de la lie du peuple.
Aussi ne songea-t-il qu'à lui et à s'asservir tellement la reine qu'elle
lui sacrifia tout, jusqu'à se précipiter deux fois au dernier bord des
derniers abîmes et de la guerre civile pour son unique intérêt, et pour
le maintenir ou le rappeler de ses proscriptions hors du royaume, à
toutes risques et affrontant tous les périls de toute la nation,
uniquement révoltée contre le cardinal Mazarin. Depuis on a vu ses
fautes aux Pyrénées, que Saint-Évremond développa avec tant de justesse
et d'agrément dans cette ingénieuse lettre qui lui coûta un expatriement
qui a duré aussi longtemps que sa très longue vie. Les lettres
particulières, les mémoires, toute l'histoire du traité de Westphalie
conclu enfin à Munster et Osnabruck, font foi qu'il en arrêta la
conclusion, aux risques de tout perdre, jusqu'à ce que son intérêt
particulier n'eut plus besoin de la guerre pour se soutenir, et se
mettre hors d'état de plus rien craindre. Ce furent ses ordres secrets à
Servien, son esclave, collègue indigne du grand d'Avaux, qui mirent bien
des fois la négociation au point de la rupture, qui rendirent la sienne
avec d'Avaux si scandaleuse et si publique, qui mit tous les ministres
employés à la paix par toutes les puissances du côté de d'Avaux, qui
produisirent ces lettres si insultantes de Servien à d'Avaux, et les
réponses de d'Avaux si pleines de sens, de modération et de gravité. Ce
fut enfin la conduite de Mazarin, si absurdement confite en félonie,
dont Servien avait tout le secret, conséquemment toute l'autorité de la
négociation, qui fit tout abandonner à d'Avaux au sein du triomphe des
longs travaux de son génie et de sa politique, qui avait su venir à bout
de la paix du nord, où plus d'un siècle après il est encore admiré, et
amener par là les choses à traiter et la plus glorieuse paix en
Westphalie, pour venir traîner dans sa patrie, dont il avait si bien
mérité, y être sans crédit sous le vain nom de surintendant des
finances, où il n'eut jamais la moindre autorité, ni la moindre part au
ministère, dont il vit récompenser Servien à son retour.

C'est à Mazarin que les dignités et la noblesse du royaume doit les
prostitutions, le mélange, la confusion, sous lesquels elle gémit, le
règne des gens de rien, les pillages et l'insolence des financiers,
l'avilissement de tout ordre, l'aversion et la crainte de tout mérite,
le mépris public que font de la nation tous ces vils champignons
dominant dans les premières places, dont l'intérêt à tout décomposer à
la fin a tout détruit. Tel fut l'ouvrage du détestable Mazarin, dont la
ruse et la perfidie fut la vertu, et la frayeur la prudence. Qui ne sera
épouvanté des trésors qu'il amassa en moins de vingt ans de règne,
traversés par deux furieuses proscriptions\,? Il fut prouvé en pleine
grand'chambre, au procès du duc Mazarin contre son fils, pour la
restitution de la dot de sa mère, qu'elle avait eu vingt-huit millions
en mariage. Ajoutez à cela les dots de la duchesse de Mercoeur, de la
connétable Colonne, de la comtesse de Soissons, même celle que trouva
après la mort du cardinal Mazarin la duchesse de Bouillon, toutes filles
de la seconde de ses soeurs, et les biens immenses qui ont fait le
partage du duc de Nevers leur frère. Ajoutez-y les dots de la princesse
de Conti et de la duchesse de Modène, filles de la soeur aînée du
cardinal Mazarin. Tous ces trésors tirés uniquement de ceux qu'il avait
su amasser, non dans un long cours d'abondance et de prospérités, mais
du sein de la misère publique et des guerres civiles qu'il avait
allumées, et des étrangères qu'il trouva, qu'il renouvela, qu'il
entretint jusqu'à un an près de sa mort.

Le cardinal de Richelieu et lui ont eu la même maison militaire que nos
rois\,: des gardes, des gens d'armes, des chevau-légers, et le dernier
des mousquetaires de plus, tous commandés par des seigneurs et par des
gens de qualité sous eux. Personne n'ignore que le père du premier
maréchal de Noailles passa immédiatement de capitaine des gardes du
cardinal Mazarin à la charge de premier capitaine des gardes du corps,
et que le marquis de Chandenier, dont la valeur et la vertu ont été si
reconnues, et chef de la maison de Rochechouart, fut le seul des quatre
capitaines des gardes dépossédés pour la ridicule affaire arrivée aux
Feuillants de la rue Saint-Honoré\footnote{L'événement auquel
  Saint-Simon fait allusion eut lieu le 15 août 1648. Comme cette
  \emph{ridicule affaire} n'est pas toujours connue des lecteurs
  modernes, je citerai ici un passage du journal inédit d'Olivier
  d'Ormesson, où elle se trouve tout au long\,: «\,J'appris l'affaire du
  capitaine des gardes, qui était que le 15 août le roi étant à la
  procession dans les Feuillants, les archers du grand prévôt, qui n'ont
  droit que de tenir la porte de la rue, prirent la porte du cloître,
  d'où ayant refusé de sortir au commandement de M. de Gesvres,
  capitaine des gardes, il fit main basse sur eux et deux furent tués à
  coup de hallebarde. Cela fit bruit. M. le cardinal (Mazarin), qui
  était auprès du roi, envoya M. Le Tellier demander le bâton à M. de
  Gesvres, avec ordre de se retirer. M. de Gesvres refusa de lui donner
  le bâton, ayant fait serment de ne le rendre qu'au roi. La reine (Anne
  d'Autriche), étant de retour au Val-de-Grâce, traita M. de Gesvres
  d'étourdi, lui redemanda le bâton, lequel il rendit, et se retira. M.
  le comte de Charost (autre capitaine des gardes), étant commandé de
  prendre le bâton, refusa, disant qu'il était autant criminel que M. de
  Gesvres, qui n'avait rien fait que dans l'ordre et par son avis. M. de
  Chandenier fut ensuite mandé et refusa de même. M. de Tresmes vint se
  plaindre que, son fils ayant fait une faute, l'on eût voulu donner le
  bâton à un autre qu'à lui, à qui la charge appartenait\,; que l'on ne
  dépossédait point ainsi les officiers en France. Il eut ordre de se
  retirer chez lui. Aussitôt la reine pourvut à la charge de Charost et
  mit en sa place Jarzé, qui prêta le serment de capitaine des gardes,
  et en celle de M. de Chandenier, M. de Noailles.\,»C'est à l'occasion
  de ce dernier que Saint-Simon fait allusion à la disgrâce des
  capitaines des gardes. Les sentiments qu'il exprime étaient ceux des
  contemporains, comme on le voit par la suite du journal d'Olivier
  d'Ormesson, qui écrivait au moment même des événements\,: «\,Chacun
  était fort indigné de ce procédé. L'on disait que M. le cardinal avait
  pris cette occasion pour mettre de ses créatures (Saint-Simon dit son
  \emph{domestique}) auprès du roi et s'en rendre maître.\,»}, qui ne
put être rétabli, parce qu'il ne le pouvait être qu'aux dépens du
domestique du cardinal Mazarin, à qui sa charge avait été donnée.

«\,Voilà, monsieur, dis-je à M. le duc d'Orléans, quels ont été en tous
pays les premiers ministres depuis le temps de Louis XI, pour ne
remonter pas plus haut. Je ne fais ici que vous faire souvenir d'eux par
quelques traits généraux. Vous avez assez lu, et vu encore des gens du
temps des derniers, pour que ce peu que je vous en dis vous en rappelle
tout le reste, et vous démontre que la peste, la guerre et la famine,
qui de tout temps ont passé pour les plus grands fléaux dont la justice
de Dieu ait puni les rois et les États, ne sont pas plus à craindre que
celui d'un premier ministre, avec cette différence que celui-là seul se
peut éviter\,: et que diriez-vous d'un prince prêt à essuyer la peste et
la famine dans son royaume, à qui Dieu les montrerait prêtes à y fondre,
et promettrait en même temps de l'en garantir à la moindre prière qu'il
en ferait, qui non seulement ne daignerait pas demander la délivrance de
ces terribles fléaux, mais qui aurait la folie, ou si vous lui voulez
donner un nom plus propre, qui serait assez stupide pour les demander\,?
Tel est, monsieur, un prince qui fait un premier ministre quand il n'est
pas dans les termes où se trouvèrent la fameuse Isabelle et votre
incomparable aïeul, et dont le tact n'est pas juste ou assez heureux
pour choisir un Ximénès ou un Richelieu.

«\,En voilà beaucoup, monsieur, poursuivis je\,; mais ce n'est pas
encore tout\,: permettez-moi de vous dire avec ma vérité et ma fidélité
accoutumée quel est nécessairement un premier ministre et quel devient
la prince qui le fait.

«\,Un premier ministre, si on en excepte le seul Ximénès, est un
ambitieux du premier ordre, qui conserve l'écorce dont il a tant besoin
et selon la mesure que le besoin subsiste, mais qui, dans la vérité, n'a
d'honneur, de vertu, d'amour de l'État, ni de son maître qu'en simple
parure, et sacrifie tout à sa grandeur, et, quand il y est parvenu, à sa
toute-puissance, à sa sûreté et à son affermissement dans sa grande
place. Il ne connaît que cet unique intérêt, d'amis ni, d'ennemis que
par rapport à cela, et suivant les divers degrés qui s'y rapportent.
Conséquemment tout mérite lui est suspect en tout genre, excepté en ceci
le cardinal de Richelieu qui se laissait volontiers dompter par le
mérite et les talents\,; toute réputation lui est odieuse, toute
élévation par dignité ou par naissance lui est dure et pesante\,; tous
droits, privilèges, lois, coutumes de tout temps respectées, lui sont à
charge\,; l'esprit et la capacité de quiconque ne le laisse point dormir
en repos\,; sur toutes choses, la moindre familiarité avec le prince, la
plus légère marque de son goût pour quelqu'un, l'effraye. Ce sont tous
gens qu'il prend à tâche d'éloigner\,; heureux, mais rarement heureux
quand il ne va pas à les noircir et à les perdre. Sa principale
application est de se faire autant d'esclaves que de gens qui approchent
du prince, de se bien assurer qu'ils ne parleront et ne répondront au
prince que sur le ton qu'il leur aura prescrit, et qu'ils lui rendront
compte de tout ce qu'ils verront, entendront, sauront, soupçonneront
même, avec une parfaite fidélité et le plus scrupuleux détail, et à
ceux-là même il donnera des espions et des surveillants qu'ils ne
pourront connaître, et d'autres encore à ceux-ci. Son grand art est que
personne n'approche du prince que de sa main, et tant qu'il pourra, sans
que le prince s'en aperçoive\,; de perdre sans retour ceux qui s'en
approcheront sans lui ou par leur hardiesse ou par le goût du prince\,;
et, comme il s'en trouve toujours quelqu'un trop difficile à perdre, de
n'oublier rien pour les gagner. L'intérêt de l'État, toujours subordonné
au sien, rend tout conseil d'État, de finance, et tous autres inutiles,
et la fortune de ceux qui les composent toujours douteuse. Ils sont
réduits à chercher et à deviner la volonté du premier ministre, dont
l'ignorance leur devient dangereuse, et la moindre résistance fatale.

«\,Un roi n'a d'intérêt que celui de l'État\,: on n'a donc point ces
embarras avec lui. Il s'explique nettement et librement de ses
volontés\,: on sait donc à quoi s'en tenir. On obéit, ou, si on croit
lui devoir faire quelques représentations sages, ou lui faire apercevoir
ce qu'on soupçonne lui être échappé de réflexions à faire sur cette
volonté, on le fait avec respect et sans crainte, parce que le roi, dont
la place et l'autorité sont inamissibles, n'en peut concevoir aucun
soupçon\,; et, s'il persévère dans sa volonté, c'est sans mauvais gré à
qui l'a combattue. À l'égard du premier ministre, c'est précisément tout
le contraire. Quelque tout-puissant, quelque affermi qu'il soit, toute
représentation lui est odieuse. Plus elle est fondée, plus elle le
choque, plus il craint un esprit qu'il sent qui va au fait. Il redoute
d'être tâté, encore plus d'être feuilleté. Quiconque en a l'imprudence,
même sans mauvaise intention, sa perte est résolue et ne tarde pas.

«\,Le premier ministre a toujours un intérêt oblique qu'il cache sous
tous les voiles qu'il peut, et cela en toute espèce d'affaires. Malheur
à qui les perce, s'il s'en aperçait. Sa place et sa puissance, de
quelque façon qu'elles soient établies, ne tiennent qu'à la volonté du
prince. Le rien souvent, aussitôt que l'affaire la plus importante, peut
altérer cette volonté, et lui causer bien de cuisantes inquiétudes, et
bien du travail pour se rassurer dans sa place et dans son autorité. Le
moindre affaiblissement lui annonce sa ruine\,; un autre rien peut la
déterminer. Il n'y a donc point de riens pour un premier ministre, et
dès lors quelle multitude de soins pour lui, et quelle dangereuse glace
que celle sur laquelle marchent toujours les ministres à son égard\,? La
paix et la guerre, les liaisons bonnes ou mauvaises avec les puissances
étrangères, les traités et leurs diverses conditions, les conjonctures à
saisir ou à laisser tomber, tout est en la main du premier ministre, qui
combine, avise et ajuste tout à son intérêt personnel, qui, dans sa
bouche, n'est que celui de l'État. Les ministres qui travaillent sous
lui, à qui le vrai intérêt de l'État est clair et celui du premier
ministre dans les ténèbres, c'est à ces ministres à bien prendre garde à
eux, à examiner les yeux et la contenance du premier ministre, à se
garer même de ses discours tenus souvent pour les sonder, à ne parler
qu'avec incertitude, sans s'expliquer jamais nettement, parce que ce
n'est pas leur avis que le premier ministre cherche, mais leur aveu que
le sien, quand il jugera à propos de le dire, est la politique la plus
exquise et le plus solide intérêt de l'État. Il en est de même sur les
finances, et sur ce qui regarde les particuliers. La place de premier
ministre, qui décide de toutes les affaires et de toutes les fortunes,
est si enviée, si haïe, ne peut éviter de faire un si grand nombre de
mécontents de tout genre et de toute espèce, qu'il a continuellement à
redouter. Il doit donc multiplier et fortifier ses précautions. Rien de
tout ce qui peut le maintenir et le raffermir ne lui paraît injuste. En
ce genre il peut tout ce qu'il veut, et il veut tout ce qu'il peut. En
récompense de tant d'avisements, de soins, de précautions, de frayeurs,
de combinaisons, de mascarades de toutes les sortes, il accumule sur soi
et sur les siens les charges, les gouvernements, les bénéfices, les
chapeaux, les richesses, les alliances. Il s'accable de biens, de
grandeurs, d'établissements pour se rendre redoutable au prince même\,;
mais son grand art est de le persuader à fond, qu'il est l'homme unique
dont il ne peut se passer, à qui il est redevable de tout, sans qui tout
périrait, pour lequel il ne peut trop faire, et sans lequel il ne doit
rien faire, surtout être confus des soins, des peines, des travaux dont
il est accablé, uniquement pour son bien et pour sa gloire, et pour
lequel sa reconnaissance et son abandon ne sauraient aller trop loin, et
par une suite nécessaire, traiter ses ennemis comme ceux de sa personne,
de sa gloire et de son État, et n'avoir de rigueur et de bonté que pour
les personnes et suivant les degrés qu'il lui marque. Tel est, monsieur,
et très nécessairement tout premier ministre, dont pas un ne pourrait se
maintenir sans cela. Voyons maintenant quel est le prince qui fait un
premier ministre, et permettez-moi de ne vous en rien cacher. J'excepte
toujours Isabelle et Louis le Juste par les cas singuliers où ils se
sont trouvés, et par l'heureux discernement de leur choix.

«\,Ce crayon, quoique si raccourci, des exemples des fléaux que tous les
divers États ont éprouvés de l'élévation et du gouvernement de leurs
premiers ministres, la France en particulier, et celui de ce qu'est
nécessairement un premier ministre en lui-même, prépare au crayon du
prince qui en fait un. C'est la déclaration la plus authentique qu'il
puisse faire de sa faiblesse ou de son incapacité, peut-être de l'une et
de l'autre, sans rien persuader à personne du mérite de son choix,
quelques pompeux éloges qu'il lui donne dans ses patentes, sinon de la
misère du promoteur, et de l'adresse et de l'ambition du promu. Si Louis
XI punit la trahison du sien en l'enfermant dans une cage de fer durant
tant d'années, à Loches, la reconnaissance du premier ministre pour un
si énorme bienfait n'a que la même récompense pour son maître. Mais la
cage où il le met est d'or et de pierreries, elle est parfumée des plus
belles fleurs\,; elle est au milieu de sa cour\,; mais elle n'en est pas
moins cage, et le prince n'y est pas moins enfermé et bien exactement
scellé. Ses plus familiers courtisans sont ses plus sûrs geôliers. Il a
donné son nom, son pouvoir, son goût, son jugement, ses yeux, ses
oreilles à son premier ministre, bien jaloux de garder de si précieux
dépôts, et bien en garde qu'il n'en revienne au prince l'émanation la
plus légère. Son salut en dépend et il ne l'ignore pas. Ainsi tout est
transmis du prince au premier ministre\,; le premier ministre règne en
plein en son nom\,; plus de différence d'effet entre le premier,
ministre et nos anciens maires du palais\,; plus de différence effective
entre le prince et nos rois fainéants, sinon que la plupart étaient
opprimés par les puissantes factions de leurs maires, et que le prince
ne l'est que par sa \emph{fétardise}\footnote{Ce mot, que l'on ne trouve
  pas toujours dans les dictionnaires, exprime avec plus d'énergie, le
  même sens que \emph{fainéantise, paresse}. Il est employé par les
  anciens écrivains français, aussi bien que l'adjectif \emph{fétard}.
  Villon s'en est servi plusieurs fois\,:Car de lire je suis
  \emph{fétard}. Et encore\,:De bien boire oncques ne fut \emph{fétard}.
  Voy. le \emph{Dictionnaire étymologique} de Ménage\,; supplément.}**.
Je frémis, monsieur, de prononcer ce mot\,; mais où ne se précipite pas
le serviteur tendre et fidèle pour sauver son maître, qu'il voit emporté
dans le tournoyant d'un gouffre, et qui se trouve seul à oser le
hasarder\,? Le prince est longtemps et se trouve à son aise dans sa
cage. Il y dort, il s'y allonge, il y jouit de la plus douce oisiveté.
Tous les plaisirs, tous les amusements s'empressent autour de lui\,;
jamais leur succession n'est interrompue, tandis que tout lui crie\,:
Les travaux continuels du premier ministre, qui se tue pour le soulager,
et qui étonne à tous moments l'Europe par la justesse et la profondeur
de sa politique, qui n'oublie rien pour rendre ses peuples heureux, qui
fait d'ailleurs les délices de sa cour, et à qui il doit tant de solides
et de glorieux avantages, sans autre soin que de vouloir s'en servir et
l'autoriser en tout. Quel bonheur suprême pour un prince aveugle et
paralytique de tout voir, de tout faire par autrui, sans sortir du sein
du repos, des amusements, des plaisirs et de l'ignorance de tout la plus
consommée\,! C'est là le grand art de ne retenir que la grandeur et les
charmes de la royauté, et d'en bannir tous les soucis, les embarras, les
travaux, et n'est-ce pas la dernière folie à qui le peut de ne pas s'y
livrer\,? Le prince ne voit rien d'aucune des parties du gouvernement.
Les fautes, les choix indignes, et ce qui en résulte, la misère et les
cris des sujets, les injustices, les oppressions, les désespoirs de tous
les ordres de l'État, les imprécations, les désolations, la ruine, le
dépeuplement, les désordres, le profit et les partis immenses que les
étrangers savent en tirer, leurs dérisions, le mépris du premier
ministre qu'ils payent quelquefois en plus d'une sorte de monnaie,
qu'ils séduisent, qu'ils trompent, et qui retombe bien plus à plomb sur
le prince qui y perd tout et qui n'y gagne rien, comme son premier
ministre, ce sont toutes choses si soigneusement éloignées de la cage,
que le prisonnier ne s'en peut pas douter. Il lui est si doux de croire
régner, et de sentir qu'il n'a rien à faire qu'à s'abandonner à ses
goûts et à son oisiveté, qu'il n'imagine pas un plus heureux que lui sur
la terre, et l'amour-propre et l'ignorance lui font encore ajouter foi
aux plus folles louanges qu'on est sans cesse occupé de lui prodiguer
par l'ordre du premier ministre, en sorte que le prince est persuadé
qu'il est le plus glorieux et le plus révéré de l'Europe, qu'il en tient
le sort entre ses mains\,; que de tant d'heur et de gloire, il n'en est
redevable qu'à son premier ministre, à ce grand choix qu'il a fait\,;
que l'unique moyen de se conserver dans cet état radieux est de
continuer à laisser maître de tout un si grand premier ministre, et
qu'il y va de toute sa gloire, de tout son bonheur, de tout celui de son
État, de le maintenir et de l'augmenter même, s'il est possible, en
puissance, en autorité, en toute espèce de grandeur.

«\,Mais rien de stable sur la terre. Le premier ministre porté si haut,
et qui a eu temps et moyens à souhait de se faire de grands et de
solides établissements, et de grandes et de vastes alliances, dont la
fortune dépend du maintien de la sienne, vient quelquefois à s'enivrer.
Il se figure ne pouvoir plus être entamé, il se croit au-dessus des
revers\,; il ne voit plus le tonnerre et la foudre que bien loin sous
ses pieds, comme ces voyageurs qui passent sur la cime des plus hautes
montagnes. Il devient insolent\,: la souplesse, la complaisance auprès
du prince l'abandonnent, parce qu'il compte n'en avoir plus besoin. Il
devient fantasque, opiniâtre\,; il le contrarie pour des riens, et il
refuse d'autres riens aux gardiens de la cage. Le prince, dont
l'entêtement est dur à entamer, a plus tôt fait de se croire indiscret
que son premier ministre impertinent. Son humeur se fortifie par le
succès. Il trouve dangereux d'accoutumer par sa complaisance le prince à
être importun, et ceux qui l'approchent à en être cause. Il y faut
couper pied, et cette méthode enfin commence à donner au prince du
malaise, et du dépit à ses geôliers. Ils négocient\,; ils sont
rebutés\,: le prince les plaint, intérieurement se fiche. Il commence à
s'apercevoir qu'encore serait-il de raison qu'il pût disposer des
bagatelles. Le premier ministre s'alarme, croit que, s'il abandonne des
bagatelles, bientôt tout lui échappera. Il se roidit, il éloigne ces
gardiens suspects, il en substitue de plus fidèles. Le prince ne sait
plus avec qui se plaindre de la dureté qu'il éprouve. Son angoisse
devient extrême\,; mais comment se passer d'un homme si nécessaire\,?
et, quand il serait capable d'en prendre le parti, comment s'y prendre
pour renverser le colosse qu'il a fait\,? Et quel usage tirer de
l'impuissance où il s'est bien voulu réduire pour élever un autre roi
que lui\,? De là, les partis, les cabales, les troubles, une lutte et
des malheurs profonds, qui rie sont pas même réparés par la chute du
premier ministre. L'abondance de la matière fournirait sans fin. Ce
court précis peut suffire aux réflexions de Votre Altesse Royale. Elle
se souviendra seulement de ce que c'est qu'un cardinal\,; que Dubois ne
peut être en rien au-dessous du Mazarin pour la naissance, et qu'il a
par-dessus lui l'avantage d'être né Français, dont cet Italien a
toujours tout ignoré jusqu'à la langue.\,»

Un assez long silence succéda à ce fort énoncé. La tête de M. le duc
d'Orléans, toujours entre ses mains, était peu à peu tombée fort près de
son bureau. Il la leva enfin et me regarda d'un air languissant et
morne, puis baissa des yeux qui me parurent honteux, et demeura encore
quelque temps dans cette situation. Enfin il se leva et fit plusieurs
tours, toujours sans rien dire. Mais quel fut mon étonnement et ma
confusion au moment qu'il rompit le silence\,! Il s'arrêta, se tourna à
demi vers moi sans lever les yeux, et se prit tout à coup à dire d'un
ton triste et bas\,: «\,Il faut finir cela, il n'y a qu'à le déclarer
tout à l'heure. --- Monsieur, repris-je, vous êtes bon et sage, et
par-dessus le maître. N'avez-vous rien à m'ordonner pour Meudon\,?» Je
lui fis tout de suite la révérence et sortis, tandis qu'il me cria\,:
«\,Mais vous reverrai-je pas bientôt.\,» Je ne répondis rien, et je
fermai la porte. Le fidèle et patient Belle-Ile était encore depuis plus
de deux grosses heures au même endroit où je l'avais laissé en entrant,
sans le temps qu'il y avait attendu mon arrivée. Il me saisit aussitôt
en me disant avec empressement à l'oreille\,: «\,Hé bien\,! où en
sommes-nous\,? --- Au mieux, lui répondis-je en me contenant tant que je
pus\,; je tiens l'affaire faite et tout sur le petit bord d'être
déclarée. --- Cela est à merveille, reprit-il\,: je vais tout à l'heure
faire un homme bien aise.\,» Je ne le chargeai de rien, et je me hâtai
de le quitter pour me sauver à Meudon, et m'y exhaler seul à mon aise.

Je sentis dès le lendemain la raison des quatre embuscades de Belle-Ile,
que je n'avais attribuées qu'à curiosité, à l'envie de se mêler et de
faire sa cour au cardinal Dubois. Ni moi ni personne n'en aurions jamais
deviné la cause, qui fut toute de projet d'une hardiesse démesurée. Sur
les deux heures après midi du 23 août, lendemain de la conversation qui
vient d'être racontée, le cardinal Dubois fut déclaré premier ministre
par M. le duc d'Orléans, et par lui présenté au roi comme tel, à l'heure
de son travail. Sur les quatre heures après midi arriva Conches à
Meudon, qui vint m'apprendre cette nouvelle de la part du cardinal
Dubois, qui l'envoyait exprès m'en porter, me dit-il, son hommage, comme
à celui à qui il en avait toute l'obligation. Je répondis fort sec et
avec grande surprise que j'étais fort obligé à M. le cardinal de la part
qu'il voulait bien me donner d'une chose pour laquelle il savait mieux
que personne qu'il n'avait besoin que de lui-même, et Conches, sans
autre propos de moi ni guère plus de lui, s'en retourna aussitôt.
Conches était un homme de rien et de Dauphiné, dont la figure lui avait
tenu lieu d'esprit. M. de Vendôme lui avait fait avoir une compagnie de
dragons, puis commission de lieutenant-colonel. Il s'était attaché
depuis à Belle-Ile, mestre de camp général des dragons\footnote{Colonel
  général des dragons. On a eu tort, dans les anciennes éditions, de
  substituer \emph{maréchal de camp général} à \emph{mestre de camp
  général}.}, qui ramassait alors tout ce qu'il pouvait pour se faire
des créatures, et qui savait très bien se servir des gens quels qu'ils
fussent, et les servir lui-même utilement. Je vis donc par ce message
que le cardinal Dubois se voulait parer de mon suffrage pour son
élévation à la place de premier ministre, tandis qu'il était
radicalement impossible et hors de toute vraisemblance qu'il ne sût par
M. le duc d'Orléans ce qui s'était passé, du moins en gros, entre ce
prince et moi là-dessus. Je fus vraiment indigné de cette effronterie,
dont sa prétendue reconnaissance remplit la cour et la ville.
Heureusement on nous connaissait tous deux\,: mais ce n'était pas le
plus grand nombre, de ceux surtout qui n'approchaient pas de la cour. Je
ne laissai pas de dire à des amis, à quelques autres personnes
distinguées, que j'étais fort éloigné d'y avoir part, et je remis au
lendemain, quoiqu'il fût de si bonne heure, à aller à Versailles.

Comme j'entrais dans les premières pièces de l'appartement, où j'étais
apparemment guetté à tout hasard, un des officiers de sa chambre me vint
dire que M. le cardinal Dubois me priait de passer par la petite cour,
et que je le trouverais à la porte du caveau. Ce caveau était une pièce,
une espèce d'enfoncement moins réel que d'ajustement, qui faisait une
petite pièce assez obscure, où monseigneur couchait souvent l'hiver,
dans les derrières de sa chambre naturelle, par la ruelle de laquelle on
y entrait, qui avait un degré fort étroit et fort noir en dégagement,
qui rendait dans la seconde antichambre du roi, d'un côté, et dans les
derrières de l'appartement de la reine de l'autre, et qui avait un autre
dégagement de plain-pied dans la petite cour, à travers une manière de
très petite antichambre. Ce fut dans cette antichambre que je trouvai le
cardinal Dubois. Je n'ai point su ce qui l'y avait mis. Peut-être averti
de mon arrivée, puisque dès l'entrée de l'appartement j'y fus envoyé de
sa part, y était-il allé pour m'y faire en particulier toutes ses
protestations et ses caracoles, qu'il craignait apparemment qui ne
fussent démenties par le froid dont il craignait que je les pourrais
recevoir. Quoi qu'il en soit, je le trouvai avec Le Blanc et Belle-Ile
seuls. Dès qu'il m'aperçut, il courut à moi, n'oublia rien pour me
persuader que je l'avais fait premier ministre, et son éternelle
reconnaissance me protesta qu'il voulait ne se conduire que par mes
conseils, m'ouvrir tous ses portefeuilles, ne me cacher rien, concerter
tout avec moi. Je n'étais pas si crédule que le cardinal de Rohan, et je
sentais tout ce que valait ce langage d'un homme qui savait mieux qu'il
ne disait, et qui ne cherchait qu'à se cacher sous mon manteau, et à
jeter, s'il l'eût pu, tout l'odieux de sa promotion sur moi, comme
l'ayant conseillée, poursuivie et procurée. Je répondis par tous les
compliments que je pus tirer de moi, sans jamais convenir que j'eusse la
moindre part à sa promotion, ni que je prisse à l'hameçon de tant de
belles offres sur les affaires. Il ne tenait pas à terre de joie. Nous
entrâmes par les derrières, lui et moi, dans le cabinet de M. le duc
d'Orléans, qui, à travers l'embarras qui le saisit à ma vue, me fit
aussi merveilles, mais sans qu'il fût question de la déclaration du
premier ministre. J'abrégeai tant que je pus ma visite et m'en revins
respirer à Meudon. Cette déclaration, incontinent suivie de la plus
ample patente et de son enregistrement, fut extrêmement mal reçue de la
cour, de la ville et de toute la France. Le premier ministre s'y était
bien attendu, mais il y était parvenu et il se moquait de l'improbation
et des clameurs publiques, que nulle politique ni crainte ne put
retenir.

Les Rohan firent preuve de la leur en cette occasion qui les touchait de
si près\,; ils avalèrent la chose doux comme lait, affectèrent de
l'approuver, de la louer, de publier que cela ne se pouvait autrement,
sinon que cela avait été trop différé. Ils ont tous en préciput une
finesse de nez qui les porte sans faillir à l'insolence et à la
bassesse, qui les fait passer de l'une à l'autre avec une agilité
merveilleuse, et dont l'air simple et naturel surprendrait toujours, si
leur extrême fausseté était moins connue, jusqu'à douter, avec raison,
s'ils ont soif à table quand ils demandent à boire. En vérité, la
souplesse ni l'étude des plus surprenants danseurs de corde n'égala
jamais la leur. Leur coup était manqué\,; en user autrement eût blessé
le cardinal Dubois jusque dans le fond de l'âme par la conviction de sa
longue perfidie\,; l'avaler comme ils firent était se l'acquérir autant
qu'il en pouvait être capable, par la reconnaissance de cacher son
forfait autant qu'il était en eux, et par l'effort d'approbation et de
joie de ce qu'il leur enlevait après des engagements si forts et si
redoublés. Laissons-les s'ensevelir dans cette fange, et Dubois dans le
comble de sa satisfaction et de la toute-puissance, pour exposer un
épisode indispensable à placer ici pour les étranges suites qu'eurent de
si chétives sources.

Plénœuf était Berthelot, c'est-à-dire de ces gens du plus bas peuple qui
s'enrichissent en le dévorant, et qui, des plus abjectes commissions des
fermes, arrivent peu à peu, à force de travail et de talents, aux
premiers étages des maltôtiers et des financiers, par la suite. Tous ces
Berthelot, en s'aidant les uns les autres, étaient tous parvenus, les
uns moins, les autres plus\,; celui-ci s'était gorgé par bien des
métiers, et enfin dans les entreprises des vivres pour les armées. Ce
fut cette connaissance qui le fit prendre à Voysin, devenu secrétaire
d'État de la guerre, pour un de ses principaux commis. Il avait épousé
une femme de même espèce que lui, grande, faite au tour, avec un visage
extrêmement agréable, de l'esprit, de la grâce, de la politesse, du
savoir-vivre, de l'entregent et de l'intrigue, et qui aurait été faite
exprès pour fendre la nue à l'Opéra et y faire admirer la déesse. Le
mari était un magot, plein d'esprit, qui voulait en avoir la meilleure
part, mais qui du reste n'était pas incommode, et dont les gains
immenses fournissaient aisément à la délicatesse et à l'abondance de la
table, à toutes les fantaisies de parure d'une belle femme, et à la
splendeur d'une maison de riche financier.

La maison était fréquentée\,; tout y attirait\,; la femme adroite y
souffrait par complaisance les malotrus amis de son mari qui, de son
côté, recevait bien aussi des gens d'une autre sorte qui n'y venaient
pas pour lui. La femme était impérieuse, voulait des compagnies qui lui
fissent honneur\,; elle ne souffrait guère de mélange dans ce qui venait
pour elle. Éprise d'elle-même au dernier point, elle voulait que les
autres le fussent\,; mais il fallait en obtenir la permission. Parmi
ceux-là elle savait choisir\,; elle avait si bien su établir son empire,
que le bonheur complet ne sortait jamais à l'extérieur des bornes du
respect et de la bienséance, et que pas un de la troupe choisie n'osait
montrer de la jalousie ni du chagrin. Chacun espérait son tour, et en
attendant, le choix plus que soupçonné était révéré de tous dans un
parfait silence, sans la moindre altération entre eux. Il est étonnant
combien cette conduite lui acquit d'amis considérables, qui lui sont
toujours demeurés attachés, sans qu'il fût question de rien plus que
d'amitié, et qu'elle a trouvés, au besoin, les plus ardents à la servir
dans ses affaires. Elle fut donc dans le meilleur et le plus grand
monde, autant qu'alors une femme de Plénœuf y pouvait être, et s'y est
toujours conservée depuis parmi tous les changements qui lui sont
arrivés.

Entre plusieurs enfants, elle eut une fille, belle, bien faite, plus
charmante encore par ces je ne sais quoi qui enlèvent, et de beaucoup
d'esprit, extrêmement orné et cultivé par les meilleures lectures, avec
de la mémoire et le jugement de n'en rien montrer. Elle avait fait la
passion et l'occupation de sa mère à la bien élever. Mais devenue
grande, elle plut, et à mesure qu'elle plut elle déplut à sa mère. Elle
ne put souffrir de voeux chez elle qui pussent s'adresser à d'autres\,;
les avantages de la jeunesse l'irritèrent. La fille, à qui elle ne put
s'empêcher de le faire sentir, souffrit sa dépendance, essuya ses
humeurs, supporta les contraintes\,; mais le dépit s'y mit. Il lui
échappa des plaisanteries sur la jalousie de sa mère qui lui revinrent.
Elle en sentit le ridicule, elle s'emporta\,; la fille se rebecqua, et
Plénoeuf, plus sage qu'elles, craignit un éclat qui nuirait à
l'établissement de sa fille, leur imposa en sorte qu'il en étouffa les
suites, qui n'en devinrent que plus aigres dans l'intérieur domestique,
et qui pressèrent Plénoeuf de l'établir.

Entre plusieurs partis qui se présentèrent, le marquis de Prie fut
préféré. Il n'avait presque rien, il avait de l'esprit et du savoir\,;
il était dans le service, mais la paix l'arrêtait tout court. L'ambition
de cheminer le tourna vers les ambassades, mais point de bien pour les
soutenir\,; il le trouvait chez Plénœuf, et Plénoeuf fut ébloui du
parrain du roi, d'une naissance distinguée et parent si proche de la
duchesse de Ventadour du seul bon côté, et qui, avec raison, le tenait à
grand honneur. L'affaire fut bientôt conclue\,; elle fut présentée au
feu roi par la duchesse de Ventadour\,; sa beauté fit du bruit\,; son
esprit, qu'elle sut ménager, et son air de modestie la relevèrent.
Presque incontinent après, de Prie fut nommé à l'ambassade de Turin, et
tous deux ne tardèrent pas à s'y rendre. On y fut content du mari, la
femme y réussit fort, mais leur séjour n'y fut pas fort long. La mort du
roi et l'effroi des financiers pressèrent leur retour\,; l'ambassade ne
roulait que sur la bourse du beau-père. M\textsuperscript{me} de Prie
avait donc vu le grand monde français et étranger\,; elle en avait pris
le ton et les manières en ambassadrice et en femme de qualité distinguée
et comme\,; elle avait été applaudie partout. Elle ne dépendait plus de
sa mère\,; elle la méprisa et prit des airs avec elle qui lui firent
sentir toute la différence de la fleur d'une jeune beauté d'avec la
maturité des anciens charmes d'une mère, et toute la distance qui se
trouvait entre la marquise de Prie et M\textsuperscript{me} de Plénœuf.
On peut juger de la rage que la mère en conçut\,; la guerre fut
déclarée, les soupirants prirent parti, l'éclat n'eut plus de mesure\,;
la déroute et la fuite de Plénœuf suivirent de près. La misère, vraie ou
apparente, et les affaires les plus fâcheuses accablèrent
M\textsuperscript{me} de Plénoeuf. Sa fille rit de son désastre et
combla son désespoir. Voilà un long narré sur deux femmes de peu de
chose, et peu digne, ce semble, de tenir la moindre place dans des
Mémoires sérieux, où on a toujours été attentif de bannir les
bagatelles, les galanteries, surtout quand elles n'ont influé sur rien
d'important. Achevons tout de suite.

M\textsuperscript{me} de Prie devint maîtresse publique de M. le Duc, et
son mari, ébloui des succès prodigieux que M. de Soubise avait eus, prit
le parti de l'imiter, mais M. le Duc n'était pas Louis XIV, et ne menait
pas cette affaire sous l'apparent secret et sous la couverture de toutes
les bienséances les plus précautionnées. C'est où ces deux femmes en
étaient, lorsque je fus forcé par M. le {[}duc{]} et
M\textsuperscript{me} la duchesse d'Orléans, comme on l'a vu en son
lieu, d'entrer en commerce avec M\textsuperscript{me} de Plénoeuf sur le
mariage d'une de leurs filles, que Plénœuf, retiré à Turin, s'était mis
de lui-même à traiter avec le prince de Piémont. M\textsuperscript{me}
de Prie, parvenue à dominer M. le Duc entièrement, fit par lui la paix
de son père, et le fit revenir. Elle l'aimait assez, et il la ménageait
dans la situation brillante où il la trouvait\,; car ces gens-là, et
malheureusement bien d'autres, comptent l'utile pour tout, et l'honneur
pour rien. Lui et sa fille avaient grand intérêt à sauver tant de biens.
Cet intérêt commun et la situation de M. le Duc, duquel elle disposait
en souveraine, serra de plus en plus l'union du père et de la fille aux
dépens de la mère\,; mais la fille, non contente de se venger de la
sorte des jalousies et des hauteurs de sa mère, qui ne put ployer devant
l'amour de M. le Duc, se mit à prendre en aversion les adorateurs de sa
mère, et la crainte qu'elle leur donna en fit déserter plusieurs.

Les plus anciens tenants et les plus favorisés étaient Le Blanc et
Belle-Ile. C'était d'où était venue leur union. Tous deux étaient nés
pour la fortune\,; tous deux en avaient les talents\,; tous deux se
crurent utiles l'un à l'autre\,: cela forma entre eux la plus parfaite
intimité, dont M\textsuperscript{me} de Plénoeuf fut toujours le centre.
Le Blanc voyait dans son ami tout ce qui pouvait le porter au grand, et
Belle-Ile sentait dans la place qu'occupait Le Blanc de quoi l'y
conduire, tellement que, l'un pour s'étayer, l'autre pour se pousser,
marchèrent toujours dans le plus grand concert sous la direction de la
divinité qu'ils adoraient sans jalousie. Il n'en fallut pas davantage
pour les rendre l'objet de la haine de M\textsuperscript{me} de Prie.
Elle ne put les détacher de sa mère, elle résolut de les perdre. La
tentative paraissait bien hardie contre deux hommes aussi habiles, dont
l'un, secrétaire d'État depuis longtemps, était depuis longtemps à
toutes mains de M. le duc d'Orléans, et employé seul dans toutes les
choses les plus secrètes. Il était souple, ductile, plein de ressources
et d'expédients, le plus ingénieux homme pour la mécanique des diverses
sortes d'exécutions où il était employé sans cesse, enfin l'homme aussi
à tout faire du cardinal Dubois, tellement dans sa confiance qu'il
l'avait attirée à Belle-Ile, et que tous deux depuis longtemps passaient
tous les soirs les dernières heures du cardinal Dubois chez lui, en
tiers, à résumer, agiter, consulter et résoudre la plupart des affaires.
Tel en était l'extérieur, et très ordinairement même le réel. Mais, avec
toute cette confiance, Le Blanc était trop en possession de celle du
régent pour que le cardinal pût s'en accommoder longtemps.

On a déjà vu ici que son projet était d'ôter d'auprès de M. le duc
d'Orléans tous ceux pour qui leur familiarité avec lui pouvait donner le
moindre ombrage, et qu'il avait déjà commencé à les élaguer. Il était
venu à bout de chasser le duc de Noailles, Canillac et Nocé, ses trois
premiers et principaux amis, qui l'avaient remis en selle, Broglio
l'aîné, quoiqu'il n'en valût guère la peine\,; qu'il avait échoué au
maréchal de Villeroy, qui bientôt après s'était venu perdre lui-même\,;
enfin qu'il avait tâché de raccommoder le duc de Berwick avec l'Espagne
pour l'y envoyer en ambassade, ne pouvant s'en défaire autrement, et on
verra bientôt qu'il ne se tenait pas encore battu là-dessus. Par tous
ces élaguements il ne se trouvait plus embarrassé que du Blanc et de
moi. Il me ménageait, parce qu'il ne savait comment me séparer d'avec M.
le duc d'Orléans. Il me faisait la grâce du Cyclope\,; en attendant ce
que les conjonctures lui pourraient offrir, il me réservait à me manger
le dernier. D'ailleurs je m'étais toujours contenté d'entrer où on
m'appelait\,; et à moins de choses instantes et périlleuses, je ne
m'ingérais jamais, et il ne pouvait manquer de s'apercevoir que la
conduite du régent et le gouvernement de toutes choses me déplaisaient
et me faisaient tenir à l'écart. Cela lui donnait le temps d'attendre
les moyens de faire naître des occasions\,; et m'attaquer sans
occasions, c'eût été trop montrer la corde et se gâter auprès de M. le
duc d'Orléans, à la façon dont j'étais seul à tant de titres auprès de
lui. Le Blanc était bien plus incommode. Sa charge, et plus encore les
détails de la confiance des affaires secrètes, lui donnaient
continuellement des rapports et publics et intimes avec M. le duc
d'Orléans. La soumission, la souplesse, les hommages de Le Blanc, ne le
rassuraient point. C'était un homme agréable et nécessaire à M. le duc
d'Orléans, de longue main dans sa privance la plus intime. Il était de
son choix, de son goût, utile et commode à tout, il l'entendait à
demi-mot, il ne tenait qu'à lui\,: c'étaient autant de raisons de le
craindre, par conséquent de l'éloigner\,; et si, par les racines qui le
tenaient ferme, il ne pouvait l'éloigner qu'en le perdant et l'accablant
absolument, il n'y fallait pas balancer. Et pour le dire encore en
passant, voilà les premiers ministres\,!

Celui-ci, uniquement occupé que de son fait et des choses intérieures,
était instruit de l'ancienne et intime liaison de Le Blanc et de
Belle-Ile avec M\textsuperscript{me} de Plénoeuf, de la haine extrême
que se portaient la mère et la fille, que celle de M\textsuperscript{me}
de Prie rejaillissait en plein sur ces deux tenants de sa mère. Dubois
résolut d'en profiter. En attendant que les moyens s'en ouvrissent, il
se mit à cultiver M. le Duc. Fort tôt après il sut que le désordre était
dans les affaires de La Jonchère. C'était un trésorier de
l'extraordinaire des guerres\footnote{L'\emph{extraordinaire des
  guerres} était un fonds réservé pour payer les dépenses
  extraordinaires de la guerre.}, entièrement dans la confiance de Le
Blanc, qui l'avait poussé et protégé, et qui s'en était servi, lui et
Belle-Ile, en bien des choses. Je n'ai point démêlé au clair si le
cardinal en voulait aussi à Belle-Ile, ou si ce ne fut que par
concomitance avec Le Blanc, par l'implication dans les mêmes affaires et
dans la haine de M\textsuperscript{me} de Prie. Je pencherais à le
croire, parce que, ayant plusieurs fois voulu servir Belle-Ile auprès de
M. le duc d'Orléans, je lui ai toujours trouvé une opposition qui allait
à l'aversion. Je ne crois pas même m'être trompé d'avoir cru
m'apercevoir qu'il le craignait, qu'il était en garde continuelle contre
lui de s'en laisser approcher le moins du monde, et certainement il n'a
jamais voulu de lui pour quoi que ç'ait été, d'où il me semble que, lié
comme il était avec Le Blanc, qui ne cherchait qu'à l'avancer, et qui en
était si à portée avec M. le duc d'Orléans, quelque prévention qu'eût
eue ce prince, elle n'y aurait pas résisté, si elle n'eût été étayée des
mauvais offices du cardinal Dubois, qui, avec tous les dehors de
confiance pour Belle-Ile, avait assez bon nez pour le craindre
personnellement, et comme l'ami le plus intime du Blanc, qu'il avait
résolu de perdre. Quoi qu'il en soit, Belle-Ile passait pour avoir trop
utilement profité de l'amitié du Blanc, et pour avoir infiniment tiré
des manéges qui se pratiquent dans les choses financières de la guerre,
et en particulier de La Jonchère, dans les comptes, les affaires et le
crédit duquel cela avait causé le plus grand désordre sous les yeux et
par l'autorité du Blanc.

Au lieu d'étouffer la chose, et d'y remédier pour soutenir le crédit
public de cette partie importante au bien général des affaires, le
cardinal la saisit pour s'en servir contre Le Blanc, et en faire sa cour
à M. le Duc et à M\textsuperscript{me} de Prie, qui aussitôt lâcha M. le
Duc au cardinal. Il fit donc grand bruit, pressa Le Blanc d'éclaircir
cette affaire, et bientôt vint à déclarer ses soupçons de la part qu'il
avait en ce désordre. M. le Duc, poussé par sa maîtresse, se mit à
poursuivre vivement cette affaire, et à ne garder plus aucunes mesures
sur Le Blanc ni sur Belle-Ile. M. le duc d'Orléans, qui aimait Le Blanc,
se trouva dans le dernier embarras des vives instances de M. le Duc,
qu'il redoublait tous les jours sous prétexte du bon ordre à maintenir,
et du discrédit que causait aux affaires publiques la faillite énorme
qu'un trésorier de l'extraordinaire des guerres était prêt à faire pour
n'avoir pu ne se pas prêter à toutes les volontés du secrétaire d'État
de la guerre, son supérieur et son protecteur, et de Belle-Ile, ami de
Le Blanc jusqu'à n'être qu'un avec lui. Le régent n'était pas moins
embarrassé des semonces doctrinales de son premier ministre qui, sans
lui montrer tant de feu que M. le Duc, le pressait plus solidement, et
avec une autorité que le régent ne s'entendait pas à décliner. Cette
affaire en était {[}là{]} quand les préparatifs d'une nouvelle liaison
avec l'Espagne et ceux du sacre du roi la suspendirent pour quelque
temps.

Le mariage de M\textsuperscript{lle} de Beaujolais, cinquième fille de
M. le duc d'Orléans, avec l'infant don Carlos, troisième fils du roi
d'Espagne mais aîné du second lit, fut traité avec tant de promptitude
et de secret qu'il fut déclaré presque avant qu'on en eût rien
soupçonné. Ce prince n'avait pas encore sept ans, étant né à Madrid le
20 janvier 1716, et la princesse avait un an plus que lui, étant née à
Versailles le 18 décembre 1714. C'était cet infant que regardait la
succession de Parme et de Plaisance, aux droits de la reine sa mère, et
celle de Toscane aussi. Cet établissement en Italie n'était pas prêt
d'échoir, par l'âge des possesseurs actuels. Elle avait besoin d'un
grand appui pour n'être point troublée par la jalousie de l'empereur, si
attentif à l'Italie, par celle du roi de Sardaigne, qui se trouverait
par là enfermé par la maison royale de France, enfin par celle de toute
l'Europe, qui portait déjà si impatiemment la domination de cette maison
en Espagne, et qui avait fait tant d'efforts pour l'en arracher.
L'intérêt de cette auguste maison était donc également grand et sensible
de se conserver une si belle partie de l'Italie, dont le droit lui était
évident et reconnu, et en particulier celui de la reine d'Espagne, de
qui il dérivait, à qui il était si glorieux d'augmenter d'une si belle
et si importante succession la maison où elle avait eu l'honneur
d'entrer, à la surprise de toute l'Europe et au grand mécontentement du
feu foi, comme on l'a vu en son lieu. Un intérêt plus personnel à la
reine d'Espagne s'y joignait encore. Elle avait toujours regardé avec
horreur l'état des reines d'Espagne veuves. Elle était accoutumée à
régner pleinement par le roi son époux\,; la chute lui en paraissait
affreuse si elle venait à le perdre, comme la différence de leurs âges
le lui faisait envisager. Son but avait donc été toujours de n'oublier
rien pour faire un établissement souverain à son fils, où elle pût se
retirer auprès de lui, hors de l'Espagne, quand elle serait veuve, et
s'y consoler en petit de ce qu'elle perdrait en grand. Pour y réussir,
elle ne pouvait s'appuyer plus solidement que de la France\,; et le
régent, de son côté, ne pouvait établir sa fille plus grandement, ni
mieux s'assurer personnellement de plus en plus l'appui de l'Espagne. La
surprise de la déclaration de ce mariage fut grande en Europe, et non
moindre en France, où tout ce qui n'aimait pas le régent et son
gouvernement en laissa voir du chagrin. Malheureusement, on vit bientôt
après que ces mariages, simplement conclus et signés avec l'Espagne,
n'avaient pas été faits au ciel.

Un autre mariage, entièrement parachevé en même temps, acheva
l'apparente réconciliation de la maison de Bavière avec celle
d'Autriche. Ce fut celui du prince électoral de Bavière avec la soeur
cadette de la reine de Pologne, électrice de Saxe, toutes deux filles du
feu empereur Joseph, frère aîné de l'empereur régnant. Quoique accompli
dès lors avec toute la pompe et la joie la plus apparente, il ne fut pas
heureux, et ne réussit point à réunir les deux maisons.

En attendant le sacre qui s'allait faire, on amusa le roi de l'attaque
d'un petit fort dans le bout de l'avenue de Versailles, et à lui montrer
ces premiers éléments militaires.

Il perdit Ruffé, un de ses sous-gouverneurs, qui était homme fort sage,
lieutenant général, et qui ne jouit pas longtemps du gouvernement de
Maubeuge, qu'il avait eu à la mort de Saint-Frémont. Il était aussi
premier sous-lieutenant de la première compagnie des mousquetaires.
Ruffé était du pays de Dombes, fort attaché au duc du Maine, et se
prétendait de la maison de Damas, dont il n'était point, et n'en était
point reconnu de pas un de cette illustre et ancienne maison. Son frère
néanmoins, qui fut aussi lieutenant général, s'est toujours fait
hardiment appeler le chevalier de Damas. En France, il n'y a qu'à
vouloir prétendre entreprendre en tout genre, on y fait tout ce que l'on
veut.

Les lettres perdirent aussi Dacier, qui s'y était rendu recommandable
par ses ouvrages et par son érudition. Il avait soixante et onze ans, et
il était garde des livres du cabinet du roi, ce qui l'avait fait
connaître et estimer à la cour. Il avait une femme bien plus
foncièrement savante que lui, qui lui avait été fort utile, qui était
consultée de tous les doctes en toutes sortes de belles-lettres grecques
et latines, et qui a fait de beaux ouvrages. Avec tant de savoir, elle
n'en montrait aucun, et le temps qu'elle dérobait à l'étude pour la
société, on l'y eût prise pour une femme d'esprit, mais très ordinaire,
et qui parlait coiffures et modes avec les autres femmes, et de toutes
les autres bagatelles qui font les conversations communes, avec un
naturel et une simplicité comme si elle n'eût pas été capable de mieux.

Il mourut en même temps une femme d'un grand mérite\,: ce fut la
duchesse de Luynes, fille du dernier chancelier Aligre, veuve en
premières noces de Manneville, gouverneur de Dieppe, qui sont des
gentilshommes de bon lieu, et mère de Manneville, aussi gouverneur de
Dieppe, qui avait épousé une fille du marquis de Montchevreuil, qui fut
quelque temps dame d'honneur de la duchesse du Maine. Le duc de Luynes
voulant se remarier en troisièmes noces, le duc de Chevreuse, son fils
aîné, lui trouva ce parti plein de sens, de vertu et de raison, et eut
bien de la peine à la résoudre. Elle s'acquit l'amitié, l'estime et le
respect de toute la famille du duc de Luynes, qui l'ont vue
soigneusement jusqu'à sa mort. Lorsqu'elle perdit le duc de Luynes, ils
ne purent l'empêcher de se retirer aux Incurables. On voyait encore, à
plus de quatre-vingts ans, qu'elle avait été belle, grande, bien faite
et de grande mine. Le duc de Luynes n'en eut point d'enfants.

Reynold, lieutenant général et colonel du régiment des gardes suisses,
très galant homme, et fort vieux, la suivit de près. Il avait été mis
dans le conseil de guerre\,: il en est ici parlé ailleurs.

Pezé, dont il a été souvent parlé ici, qui avait le régiment
d'infanterie du roi et le gouvernement de la Muette, épousa une fille de
Beringhen, premier écuyer.

\hypertarget{chapitre-xvii.}{%
\chapter{CHAPITRE XVII.}\label{chapitre-xvii.}}

1722

~

{\textsc{Préparatifs du voyage de Reims, où pas un duc ne va, excepté
ceux de service actuel et indispensable, et de ceux-là mêmes aucun ne
s'y trouva en pas une cérémonie sans la même raison.}} {\textsc{-
Désordres des séances et des cérémonies du sacre.}} {\textsc{- Étranges
nouveautés partout.}} {\textsc{- Bâtards ne font point le voyage de
Reims.}} {\textsc{- Remarques de nouveautés principales.}} {\textsc{-
Cardinaux.}} {\textsc{- Conseillers d'État, maîtres des requêtes,
secrétaires du roi.}} {\textsc{- Maréchal d'Estrées non encore alors duc
et pair.}} {\textsc{- Secrétaires d'État.}} {\textsc{- Mépris outrageux
de toute la noblesse, seigneurs et autres.}} {\textsc{- Mensonge et
friponnerie avérée qui fait porter la première des quatre offrandes au
maréchal de Tallard, duc vérifié.}} {\textsc{- Barons, otages de la
sainte ampoule.}} {\textsc{- Peuple nécessaire dans la nef dès le
premier instant du sacre.}} {\textsc{- Deux couronnes\,; leur usage.}}
{\textsc{- \emph{Esjouissance des pairs} très essentiellement
estropiée.}} {\textsc{- Le couronnement achevé, c'est au roi à se mettre
sa petite couronne sur la tête et à se l'ôter quand il le faut, non à
autre.}} {\textsc{- Festin royal\,; le roi y doit être vêtu de tous les
mêmes vêtements du sacre.}} {\textsc{- Trois évêques, non pairs,
suffragants de Reims, assis en rochet et camail à la table des paris
ecclésiastiques vis-à-vis-les trois évêques-comtes pairs.}} {\textsc{-
Tables des ambassadeurs et du grand chambellan placées au-dessous de
celles des pairs laïques et ecclésiastiques.}} {\textsc{- Lourdise qui
les fait placer sous les yeux du roi.}} {\textsc{- Cardinal de Rohan
hasarde l'Altesse dans ses certificats de profession de foi à MM. les
duc de Chartres et comte de Charolais\,; est forcé sur-le-champ d'y
supprimer l'Altesse, qui l'est en même temps pour tous certificats et
tous chevaliers de l'ordre nommés, avec note de ce dans le registre de
l'ordre.}} {\textsc{- Ce qui est observé depuis toujours.}} {\textsc{-
Grands officiers de l'ordre couverts comme les chevaliers.}} {\textsc{-
Ridicule et confusion de la séance.}} {\textsc{- Princes du sang
s'arrogent un de leurs principaux domestiques près d'eux à la cavalcade,
où {[}il y a{]} plus de confusion que jamais.}} {\textsc{- Fêtes à
Villers-Cotterêts et à Chantilly.}} {\textsc{- La Fare et Belle-Ile à la
Ferté.}} {\textsc{- Leur inquiétude, et mon avis que Belle-Ile ne peut
se résoudre à suivre.}} {\textsc{- Survivance du gouvernement de Paris
du duc de Tresmes à son fils aîné.}} {\textsc{- Signature du contrat du
futur mariage de M\textsuperscript{lle} de Beaujolais avec l'infant don
Carlos.}} {\textsc{- Départ et accompagnement de cette princesse.}}
{\textsc{- Laullez complimenté par la ville de Paris, qui lui fait le
présent de la ville.}} {\textsc{- Mort à Rome de la fameuse princesse
des Ursins.}} {\textsc{- Mort de Madame\,; son caractère.}} {\textsc{-
Famille et caractère de la maréchale de Clerembault.}} {\textsc{- Sa
mort.}} {\textsc{- Mariage de M\textsuperscript{me} de Cani avec le
prince de Chalais, et du prince de Robecque avec M\textsuperscript{lle}
du Bellay.}} {\textsc{- Paix de Nystadt entre le czar et la Suède.}}

~

Le temps du sacre s'approchait fort. À la façon dont tout s'était passé
depuis la régence, je compris que le sacre, qui est le lieu où l'état et
le rang des pairs a toujours le plus paru, se tournerait pour eux en
ignominie. Le principal coup leur était porté par l'édit de 1711, qui
attribuait aux princes du sang, et, à leur défaut, aux bâtards du roi et
à leur postérité, la représentation des anciens pairs au sacre, de
préférence aux autres pairs. L'ignorance, la mauvaise foi, et la
malignité éprouvée du grand maître des cérémonies, l'orgueil du cardinal
Dubois de tout confondre et de tout abattre pour relever d'autant les
cardinaux, le même goût de confusion, par principe, de M. le duc
d'Orléans, me répondaient du reste. Je le sondai néanmoins\,; je
représentai, je prouvai inutilement\,; je ne trouvai que de l'embarras,
du balbutiement, et un parti pris. Le cardinal Dubois, qui sut
apparemment de M. le duc d'Orléans que je lui avais parlé, et que je
n'étais pas content, m'en jeta des propos, et tâcha de me faire accroire
des merveilles. Il craignit ce qui arriva. Il voulut m'amuser et laisser
les ducs dans la foule. Il me pressa sur ce que je croyais qu'il
convenait aux ducs. Je ne voulus point m'expliquer que je n'eusse parlé
à plusieurs, quelque résolution que j'eusse prise, comme on l'a vu
ailleurs, de ne me mêler plus de ce qui les regardait. Pressé de nouveau
par le cardinal, je lui dis enfin ce que je pensais. Il bégaya, dit oui
et non, se jeta sur des généralités et des louanges de la dignité, sur
la convenance, même la nécessité qu'ils se trouvassent au sacre, et
qu'ils y fussent dignement, s'expliquant peu en détail. Je lui déclarai
que ces propos n'assuraient rien\,: mais que d'aller au sacre pour y
éprouver des indécences, et pis encore, ce ne serait jamais mon avis\,;
que si M. le duc d'Orléans voulait que les ducs y allassent, il fallait
convenir de tout, l'écrire par articles, et que M. le duc d'Orléans le
signât double, et en présence de plusieurs ducs\,; qu'il en donnât un au
grand maître des cérémonies, avec injonction bien sérieuse de l'exacte
exécution, l'autre à celui des ducs qu'il en voudrait charger.

Dubois, qui n'avait garde de se laisser engager de la sorte, parce qu'il
voulait attirer les ducs et se moquer d'eux, se récria sur l'écriture,
et vanta les paroles. Je lui répondis nettement que l'affaire du bonnet
et d'autres encore avaient appris aux ducs la valeur des paroles les
plus solennelles, les plus fortes, les plus réitérées\,; qu'ainsi il
fallait écrire ou se passer de gens qu'il regardait comme aussi
inutiles, sinon à grossir la cour. Le cardinal se mit sur le ton le plus
doux, même le plus respectueux, car tous les tons différents ne lui
coûtaient rien, et n'oublia rien pour me gagner. Il me détacha après
Belle-Ile et Le Blanc pour me représenter que je ne pouvais m'absenter
du sacre sans quelque chose de trop marqué, le désir extrême du cardinal
que je m'y trouvasse et de m'y procurer toutes sortes de distinctions.
M. le duc d'Orléans me demanda si je n'y viendrais pas, et sans oser ou
vouloir m'en presser, fit ce qu'il put pour m'y engager. Comme ils
sentirent enfin qu'ils n'y réussiraient pas, le cardinal se mit à me
presser par lui-même et par ses deux envoyés de ne pas empêcher les
autres ducs d'y aller, et de considérer l'effet d'une telle désertion.
Je répondis que c'était à ceux qui pouvaient l'empêcher, en mettant
l'ordre nécessaire, à y faire leurs réflexions\,; que je ne gouvernais
pas les ducs, comme il n'y avait que trop paru, mais que je savais ce
qu'ils avaient à faire, et me tins fermé \footnote{Il y a \emph{fermé}
  dans le manuscrit\,; on a déjà vu ce mot employé par Saint-Simon dans
  le sens de \emph{fixe} et \emph{fermement attaché}.} à cette réponse.

Je m'étais assuré plus facilement que je ne l'avais espéré que pas un
d'eux n'irait, excepte ceux à qui leurs charges rendaient le voyage
indispensable, et que de ceux-là mêmes aucun ne se trouverait dans
l'église de Reims, ni à pas une seule des cérémonies, comme celle des
autres églises, et celle du festin royal et de la cavalcade, excepté
ceux que leurs charges y forceraient, et qu'ils sacrifieraient toute
curiosité à ce qu'ils se devaient à eux-mêmes, ce qui fut très
fidèlement et très ponctuellement exécuté. Quand je fus bien assuré de
la chose, j'allai, quatre ou cinq jours avant le départ du roi, prendre
congé de M. le duc d'Orléans et dire adieu au cardinal Dubois avec un
air sérieux, pour m'en aller à la Ferté, et je partis le lendemain. Tous
deux s'écrièrent fort\,; mais, ne pouvant me persuader le voyage de
Reims, ils firent l'un et l'autre ce qu'ils purent pour m'engager à me
trouver au retour à Villers-Cotterêts, où M. le duc d'Orléans préparait
de superbes fêtes. Je répondis modestement que, ne pouvant avoir de part
aux solennités de Reims, je me trouverais un courtisan fort déplacé à
Villers-cotterêts, et tins ferme à toutes les instances. J'étais convenu
avec les ducs que pas un n'irait de Paris ni de Reims, hors ceux qui ne
pouvaient s'en dispenser par le service actuel de leurs charges. Et cela
fut exécuté avec la même ponctualité et fidélité. J'allai donc à la
Ferté cinq ou six jours avant le départ du roi, et n'en revins que huit
ou dix après son retour.

Le désordre du sacre fut inexprimable, et son entière dissonance d'avec
tous les précédents. On y en vit dans le genre de ceux qui eurent ordre
de s'y trouver et de ceux qui n'en eurent point, et le projet de
l'exclusion possible de toutes dignités et de toute la noblesse y sauta
aux yeux. Il ne fut pas moins évident qu'on l'y voulut effacer par la
robe et jusque par ce qui est au-dessous de la robe, ces deux genres de
personnes y ayant été nommément mandées et conviées, et nul de la
noblesse, excepté le peu d'entre elles qui y eurent des fonctions qui ne
se pouvaient donner hors de leur ordre. Le même désordre par le même
projet régna dans les séances de l'église de Reims, la veille aux
premières vêpres du sacre, le jour du sacre, et le lendemain, pour
l'ordre du Saint-Esprit, que le roi reçut, puis conféra\,; au festin
royal\,; à la cavalcade, enfin partout. C'est ce qui va être expliqué
par quelques courtes remarques. Il y en aurait tant à faire qu'on ne
s'arrêtera qu'à ce qui regarde le sacre, le festin royal et l'ordre du
Saint-Esprit. Je n'ai point su quelles furent les prétentions des
bâtards\,; mais le duc du Maine, ni ses deux fils, ni le comte de
Toulouse ne firent point le voyage de Reims\,; et le comte de Toulouse,
qui en fut pressé, le refusa nettement et demeura à Rambouillet. Des six
cardinaux qu'il y avait à Paris, le seul cardinal de Noailles n'y fut
point invité. Ce fut un hommage que le cardinal Dubois voulut rendre au
cardinal de Rohan et à la constitution \emph{Unigenitus}, qui l'avaient
si bien servi à Rome pour son chapeau. Par cette exclusion, le cardinal
de Rohan se trouva à la tête des quatre autres cardinaux. La même
reconnaissance pour les deux frères d'avoir si onctueusement avalé la
déclaration de premier ministre, après en avoir été si cruellement
joués, fit aussi choisir le prince de Rohan pour faire la charge de
grand maître de France, au lieu de M. le Duc qui l'était, mais qui
représentait le duc d'Aquitaine.

Les pairs ecclésiastiques devaient à deux titres avoir la première place
de leur côté. Ils avaient sans difficulté, avec les pairs laïques, la
fonction principale dans toute la cérémonie, et l'archevêque de Reims
était le prélat officiant et dans son église\,: les cinq autres le
joignaient sur la même ligne, et y étaient les principaux officiers.
Voilà donc deux raisons sans réplique. L'usage des précédents sacres en
était une troisième. Le cardinal Dubois voulait signaler son cardinalat,
et primer à l'appui de ses confrères. Il ne voulut donc pas les placer
derrière les pairs ecclésiastiques, et il n'osa les mettre devant eux
pour troubler toute la cérémonie. Il fit donner aux cardinaux un banc un
peu en arrière de celui des pairs ecclésiastiques, mais poussé assez
haut pour qu'il n'y eût rien entre ce banc et l'autel, et que le dernier
cardinal, qui était Polignac, ne fût pas effacé par l'archevêque de
Reims, ni par l'accompagnement ecclésiastique qui était près de lui
debout. Ainsi les archevêques et évêques, et à leur suite le clergé du
second ordre, fut placé sur des bancs derrière celui des pairs
ecclésiastiques, et plus arriéré que celui des cardinaux. Sur même ligne
que les bancs des archevêques, évêques et second ordre, et au-dessous,
étaient trois bancs, sur lesquels furent placés dix conseillers d'État,
dix maîtres des requêtes, et, pour que rien ne manquât à la dignité de
cette séance, six secrétaires du roi, tous députés de leurs trois
compagnies ou corps, qui avaient été invités.

De l'autre côté, les pairs laïques vis-à-vis des pairs ecclésiastiques,
et rien vis-à-vis des cardinaux. Derrière les pairs laïques les trois
maréchaux de France nommés pour porter les trois honneurs. Il faut se
souvenir que le maréchal d'Estrées qui, comme l'ancien des deux autres,
était destiné pour la couronne, ne devint duc et pair que le 16 juillet
1723, par la mort sans enfants du duc d'Estrées, gendre de M. de Nevers.
Au-dessous du banc des honneurs, et un peu plus reculé, était le banc
des seuls secrétaires d'État, et rien devant eux qu'un bout de la fin du
banc des pairs laïques. Il est vrai qu'il y eut un moment court de la
cérémonie, où on mit devant les secrétaires d'État un tabouret placé
vis-à-vis l'intervalle entre le banc des pairs laïques et celui des
honneurs, où se mit le duc de Charost\,; mais outre que cela fut pour
très peu de temps, la séance accordée aux secrétaires d'État n'en fut
pas moins grande, puisque le duc de Charost ne prit cette place pendant
quelques moments qu'en qualité de gouverneur du roi, qui n'est pas une
charge qui existe ordinairement lors d'un sacre.

Derrière le banc des trois maréchaux de France destinés à porter les
honneurs, les maréchaux de Matignon et de Besons y furent placés\,; et
sur le reste de leur banc, qui s'étendait derrière celui des secrétaires
d'État, les seigneurs de la cour et d'autres que la curiosité avait
attirés, sans que pas un fût convié, y furent placés au hasard et sur
d'autres bancs derrière. Ainsi les conseillers d'État, maîtres des
requêtes et secrétaires du roi d'un côté, et les secrétaires d'État de
l'autre, tous conviés, eurent les belles séances, et les gens de qualité
furent placés en importuns curieux où ils purent, comme le hasard ou la
volonté du grand maître des cérémonies les rangea pour remplir les vides
d'un spectacle où ils n'étaient point conviés, et où leur curiosité fit
nombre inutile\,; tant, jusqu'aux secrétaires du roi, tout homme à
collet fut là supérieur à la plus haute noblesse de France.

Les quatre premières chaires du choeur, de chaque côté, les plus proches
de l'autel, furent occupées par les quatre chevaliers de l'ordre qui
devaient porter les quatre pièces de l'offrande, et par les quatre
barons chargés de la garde de la sainte ampoule. On a ici remarqué
ailleurs la friponnerie mise exprès dans un livre des cérémonies du
sacre du feu roi, que le grand maître des cérémonies fit imprimer et
publier quelques mois auparavant celui-ci, où mon père était nommé comme
portant une de ces offrandes. J'eus beau dire, publier et déclarer
alors, que c'était une faute absurde dans la prétendue relation de ce
livre du sacre du feu roi\,; que c'était mon oncle, frère aîné de mon
père, et chevalier de l'ordre en 1633, en même promotion que lui, qui
porta un des honneurs, et non mon père, qui était alors depuis longtemps
à Blaye, et qui y demeura longtemps depuis, fort occupé pour le service
du roi contre les mouvements, puis de la révolte de Bordeaux et de la
province. Ce même service occupait beaucoup de pairs dans leurs
gouvernements, et en fit manquer pour la représentation des anciens
pairs au sacre, en sotte que si mon père se fût trouvé à Paris, il eût
représenté un de ces anciens pairs, puisqu'à leur défaut il fallut avoir
recours à un duc non vérifié, ou, comme on parle, à brevet, qui fut M.
de Bournonville, père de la maréchale de Noailles.

Cette fausseté n'avait pas été mise pour rien dans ce livre répandu
exprès dans le public avec bien d'autres fautes. Le parti était pris. On
avait résolu de confondre les ducs avec des seigneurs ou autres qui ne
l'étaient pas, de la manière la plus solennelle, et on en choisit un qui
n'avait garde de se refuser à rien, et conduit par des gens dont les
chimères avaient le même intérêt. Ce fut le maréchal de Tallard, duc
vérifié, et non pas pair, qui fut mis à la tête du comte de Matignon, de
M, de Médavy, depuis maréchal de France, et de Goesbriant, tous
chevaliers de l'ordre, et Tallard fit ainsi la planche inouïe et
première de dette association, en même fonction d'un duc\,; même d'un
maréchal de France, avec trois autres qui ne l'étaient pas, et qui
n'avait jamais été faite par un maréchal de France, beaucoup moins par
un duc.

À l'égard des quatre barons de la sainte ampoule, placés vis-à-vis, ce
fut une indécence tout à fait nouvelle, accordée à leur curiosité de
voir le sacre, et c'en fut une autre bien plus marquée de placer dans
les quatre chaires basses, au-dessous d'eux, leurs quatre écuyers tenant
leurs pennons\footnote{Étendards à longue queue flottante.} flottants à
leurs armes au revers de celles de France, tandis que les princes du
sang, représentant les anciens pairs, ni pas un autre homme en fonction,
n'avaient ni écuyers ni pennons. La fonction de ces quatre barons en
était interceptée. Leur charge est d'être otages de la restitution de la
sainte ampoule à l'église abbatiale de Saint-Remi après le sacre. Pour
cet effet, ils doivent marcher ensemble, à cheval, avec leurs écuyers
portant chacun le pennon éployé aux armes de son maître, et point avec
les armes de France, à cheval aussi devant le sien, et les barons
environnés de leurs pages et de leur livrée, et aller ainsi depuis
l'archevêché, comme députés pour ce par le roi, à l'abbaye de
Saint-Remi, où arrivés, ils doivent être de fait, ou supposés enfermés
dans un appartement de l'abbaye, et sous clef, depuis l'instant que la
sainte ampoule en part jusqu'à celui où elle y est rapportée et
replacée, et alors être délivrés, comme dûment déchargés de leur
fonction d'otages et de répondants de la restitution et remise de la
sainte ampoule, et retourner, de l'abbaye de Saint-Remi à l'archevêché
avec le même cortège qu'ils en étaient venus. Ainsi leurs pennons
uniques ne préjudiciaient à personne, puisque, ni dans la marche à
l'aller et au retour, les quatre barons étaient seuls ainsi que dans
l'abbaye, et ces pennons de plus ne devaient servir en effet qu'à être
appendus dans l'église de l'abbaye, en mémoire et en honneur de la
fonction d'otage de la restitution de la sainte ampoule, faite et
remplie par ces quatre barons.

Voici bien une autre faute sans exemple en aucun des sacres précédents
et tout à fait essentielle, et telle que je ne puis croire qu'elle ait
été commise en effet dans la cérémonie, mais que le goût d'énerver tout,
et l'esprit régnant de confusion a fait mettre dans les relations de la
Gazette, et publiques et autorisées. Elle demande un court récit. Le
peuple, qui depuis assez longtemps fait le troisième ordre, mais
diversement composé, le peuple, dis-je, simple peuple ou petits
bourgeois, ou artisans et manants, a toujours rempli la nef de l'église
de Reims au moment que le roi y est amené. Il est là comme autrefois aux
champs de Mars, puis de Mai, applaudissant nécessairement, mais
simplement à ce qui est résolu et accordé par les deux ordres du clergé
et de la noblesse. Dès que le roi est arrivé et placé, l'archevêque de
Reims se tourne vers tout ce qui est placé dans le choeur, pour demander
le consentement de la nation. Ce n'est plus, depuis bien des siècles,
qu'une cérémonie, mais conservée en tous les sacres, et qui, suivant
même les relations des gazettes, et autres autorisées et publiées, l'a
été en celui-ci. Il faut donc que, comme aux anciennes assemblées de la
nation aux champs de Mars, puis de Mai, puisque cette partie de la
cérémonie en est une image, que la nef soit alors remplie de peuple pour
ajouter son consentement présumé à celui de ceux qui sont dans le
choeur, comme dans ces assemblées des champs de Mars, puis de Mai, la
multitude éparse en foule dans la campagne, acclamait, sans savoir à
quoi, à ce que le clergé et la noblesse, placés aux deux côtés du trône
du roi, consentait aux propositions du monarque, sur lesquelles ces deux
ordres avaient délibéré, puis consenti. C'est donc une faute énorme,
tant contre l'esprit que contre l'usage constamment observé en tous les
sacres jusqu'à celui-ci, de n'ouvrir la nef au peuple qu'après
l'intronisation au jubé.

On se sert au sacre de deux couronnes\,: la grande de Charlemagne, et
d'une autre qui est faite pour la tête du roi, et enrichie de
pierreries. La grande est exprès d'une largeur à ne pas pouvoir être
portée sur la tête, et c'est celle qui sert au couronnement. Elle est
faite ainsi pour donner lieu aux onze pairs servants d'y porter chacun
une main au moment que l'archevêque de Reims l'impose sur la tête du
roi, et de le conduire, en la soutenant toujours, jusqu'au trône du
jubé, où se fait l'intronisation. Il est impossible, par la forme de
cette ancienne couronne, que cela ait pu se pratiquer autrement\,; mais
les relations approuvées et publiées ont affecté de brouiller cet
endroit si essentiel de la cérémonie, ne parlant point exprès, pour
exténuer tout, du soutien de la couronne de Charlemagne sur la tête du
roi par les pairs, et laissent croire qu'il l'a portée immédiatement sur
sa tête. Ce n'est pas la seule réticence affectée de cet important
endroit de la cérémonie. Elles taisent la partie principale de
l'intronisation, qui s'appelle \emph{l'esjouissance des pairs}, et voici
ce qui a été soigneusement omis par ces relations tronquées. Chaque
pair, ayant baisé le roi à la joue assis sur son trône, fait de façon
que de la nef il est vu à découvert depuis les reins jusqu'à la tête\,:
le pair qui a baisé le roi se tourne à l'instant à côté du roi, le
visage vers la nef, s'appuie et se penche sur l'appui du jubé, et crie
au peuple\,: «\,Vive le roi Louis XV\,!» À l'instant le peuple crie
lui-même\,: «\,Vive le roi Louis XV\,!» À l'instant une douzième partie
des oiseaux tenus exprès en cage sont lâchés\,; à l'instant une douzième
partie de monnaie est jetée au peuple. Pendant ce bruit le premier pair
se retire à sa place sur le jubé même\,; le second va baiser le roi, se
pencher au peuple et lui crier le «\,Vive le roi Louis XV\,!» À
l'instant autres cris redoublés du peuple, autre partie d'oiseaux
lâchés, autre partie de monnaie jetée, et ainsi de suite jusqu'au
dernier des douze pairs servants.

Les relations disent tout hors cette proclamation des pairs au peuple,
et cette distribution d'oiseaux et de monnaie à chacune des douze
proclamations. La raison de ce silence est évidente\,; je me dispenserai
de la qualifier. Je ne parle point des fanfares et des décharges qui
accompagnent chaque proclamation, et dont le bruit, ainsi que celui de
la voix de tout ce qui est dans la nef ne cesse point, mais redouble à
chaque proclamation et ne commence qu'à la première. L'autre couronne se
trouve au jubé. Dès que le roi y est assis, la grande couronne est
déposée à celui qui est choisi pour la porter, et c'est le roi lui-même
qui prend la petite couronne et qui se la met sur la tête, qui se l'ôte
et se la remet toutes les fois que cela est à faire. Je ne sais si les
relations sont ici fautives, il serait bien plus étrange qu'elles ne le
fussent pas. La raison de cela est évidente\,; et quand il va à l'autel
pour l'offrande et pour la communion, et qu'il en revient au jubé, c'est
après avoir ôté sa petite couronne, qui demeure sur son prie-Dieu au
jubé, et les pairs lui tiennent la grande couronne sur sa tête, excepté,
pour ces deux occasions, l'archevêque de Reims qui demeure à l'autel.

Les relations ne disent pas un mot des fonctions de l'évêque-duc de
Langres, ni des évêques-comtes de Châlons et de Noyon\footnote{On a vu
  plus haut (t. IX, p. 445-446), quelles étaient les fonctions de ces
  évêques à la cérémonie du sacre.}.

Il y eut, au festin royal, ou une faute dans le fait, ou une méprise
dans les relations si la faute n'a pas été faite, et deux nouveautés qui
n'avaient jamais été à pas un autre festin du sacre avant celui-ci. La
faute ou la méprise est que les relations disent que le roi étant revenu
de l'église en son appartement, on lui ôta ses gants pour les brûler,
parce qu'ils avaient touché aux onctions, et sa chemise pour la brûler
aussi par la même raison\,; qu'il prit d'autres habits que ceux qu'il
avait à l'église, reprit par-dessus son manteau royal, et conserva sa
couronne sur sa tête. Les gants ôtés et brûlés, cela est vrai et s'est
toujours pratiqué, d'abord en rentrant dans son appartement, la chemise
aussi\,; mais, à l'égard de la chemise, ordinairement elle n'est ôtée
qu'après le festin, lorsque le roi, retiré dans son appartement, quitte
ses habits royaux pour ne les plus reprendre. Que si quelquefois il y a
eu des rois qui ont changé de chemise avant le festin royal, ils ont
repris tous les mêmes vêtements qu'ils avaient à l'église pour aller au
banquet royal. C'est donc une faute et une nouveauté s'il en a été usé
autrement, sinon une lourde méprise aux relations de l'avoir dit, et un
oubli d'avoir omis quel fut l'habit que ces relations prétendent que le
roi prit dessous son manteau royal pour aller au festin.

À l'égard des deux nouveautés, l'une fut faite pour tout confondre,
l'autre par une lourde imprudence qui vint d'embarras. La première fut
de faire manger à la table des pairs ecclésiastiques les évêques de
Soissons, Amiens et Senlis, comme suffragants de Reims, sans aucune
prétention ni exemple quelconque en aucun festin royal du sacre avant
celui-ci. La suffragance de Reims n'a jamais donné ni rang ni
distinction\,; c'est la seule pairie qui les donne. Cela est clair par
le siège de Soissons, qui n'en a point, quoique premier suffragant,
quoique cette primauté de suffragance lui donne le droit de sacrer les
rois en vacance du siège de Reims, ou empêchement de ses archevêques\,;
et le siège de Langres, dont l'évêque est duc et pair, et toutefois
suffragant de Lyon. Jamais qui que ce soit, avant ce sacre, n'avait été
admis à la table des pairs ecclésiastiques\,; aussi dans cette
entreprise n'osa-t-on pas y mettre d'égalité. Les pairs ecclésiastiques
étaient à leur table en chape et en mitre, comme ils y ont toujours été,
de suite et tous six du même côté, joignant l'un l'autre, l'archevêque
de Reims à un bout avec son cortége de chapes derrière lui debout, et sa
croix et sa crosse portées par des ecclésiastiques en surplis devant
lui, la table entre-deux, et l'évêque de Noyon à l'autre bout. Les trois
évêques, qu'on peut appeler parasites, furent en rochet et camail, et
apparemment découverts, puisque les relations taisent le bonnet carré,
et placés de l'autre côté de la table, et encore au plus bas bout qu'il
se put, vis-à-vis des trois évêques comtes-pairs. Outre le préjudice de
la dignité des pairs dans une cérémonie si auguste, et où ils figurent
si principalement, c'était manquer de respect au roi, en présence duquel
et à côté de lui dans la même pièce, c'est manger avec lui, quoiqu'à
différente table, et jamais évêque ni archevêque n'a mangé en aucun cas
avec nos rois s'il n'a été pair ou prince, comme il a été expliqué ici
ailleurs, jusqu'à ce que l'ancien évêque de Fréjus se fit admettre le
premier dans le carrosse du roi, puis à sa table, ce qui a été le
commencement de la débandade qui s'est vue depuis en l'un et en
l'autre\,; c'était faire une injure aux officiers de la couronne qui
sont bien au-dessus des évêques, qui en ce festin du sacre, tout grands
qu'ils sont, ne sont pas admis à la table des pairs laïques, et ne le
furent pas non plus en celui-ci. En un mot, il n'a jamais été vu en
aucun autre sacre que qui que ç'ait été ait mangé à la vue du roi au
festin royal, autres que les six pairs laïques et les six pairs
ecclésiastiques qui avaient servi au sacre.

L'autre nouveauté, qui fut une très lourde bévue, vint de l'embarras qui
était né de la facilité qu'on laisse à chacun de faire ce qui lui plaît,
sans penser aux conséquences. La pièce, de tout temps destinée au festin
royal du sacre, dans l'ancien palais archiépiscopal de Reims, était une
pièce vaste et fort extraordinaire, en ce qu'elle était en équerre, en
sorte que ce qui se passait dans la partie principale de cette pièce ne
se voyait point de ceux qui étaient dans la partie de la même pièce qui
était en équerre, et réciproquement n'était point vu de ceux qui étaient
dans la partie principale de la même pièce. L'équerre était aussi fort
spacieuse et profonde, et c'était dans cette équerre qu'étaient les
tables des ambassadeurs et du grand chambellan, tellement qu'elles
étaient également toutes deux dans la môme pièce où était la table du
roi, et celle des pairs laïques et ecclésiastiques, et toutefois
entièrement hors de leur vue. L'archevêque de Reims Le Tellier, qui
travailla beaucoup à ce palais archiépiscopal, trouvant cette pièce
immense baroque, la rompit sans penser aux suites, ou sans s'en mettre
en peine, et le feu roi l'ignora, ou ne s'en soucia pas plus que lui. De
là l'embarras où placer les tables des ambassadeurs et du grand
chambellan\,: on ne pouvait les placer dans la même pièce de celle du
roi, sans être sous sa vue, ni lui en dérober la vue qu'en les mettant
dans une autre pièce. On ne songea seulement pas qu'avant le changement
fait à cette pièce, elle était aussi capable qu'alors de contenir ces
deux tables, et qu'elles avaient néanmoins été toujours mises dans
l'équerre, que l'archevêque Le Tellier n'avait fait que couper, pour les
dérober à la vue du roi\,; ce qui devait déterminer à les mettre encore
dans cette même équerre, quoique coupée et faisant une autre pièce. On
sauta donc le bâton, on les mit dans la pièce où était la table du roi,
et on les plaça sur même ligne, mais au-dessous des deux tables des
pairs laïques et ecclésiastiques, d'où résulta nouvelle difformité, en
ce que ces évêques, non pairs, suffragants de Reims, qu'on fit manger
pour la première fois à la table des pairs ecclésiastiques, se
trouvèrent à une table supérieure à celle des ambassadeurs et à celle du
grand chambellan, avec qui ces évêques n'ont pas la moindre
compétence\,; et, pour rendre la chose plus ridicule, à une table
supérieure à celle où le chancelier mangeait, et placé comme eux au
bas-côté de la table inférieure à la leur, lui qui ne leur donne pas la
main chez lui, et dont le style de ses lettres à eux est si
prodigieusement supérieur. Ajoutons encore l'énormité de faire manger à
la vue du roi, en une telle cérémonie, les deux introducteurs des
ambassadeurs, tant par leur être personnel que par la médiocrité de leur
charge, parce qu'ils doivent manger à la table des ambassadeurs. Les
réflexions se présentent tellement d'elles-mêmes sur un si grand amas de
dissonances de toutes les espèces, nées de toutes ces nouveautés, que je
les supprimerai ici. Venons maintenant à ce qui se passa pour l'ordre du
Saint-Esprit, que le roi reçut le lendemain matin des mains de
l'archevêque de Reims, et qu'il conféra ensuite, comme grand maître de
l'ordre, au duc de Chartres et au comte de Charolais.

La règle est que ceux qui sont nommés chevaliers de l'ordre, entre
plusieurs formalités préparatoires, font à genoux, chez le grand
aumônier de France, qui l'est né de l'ordre, profession de la foi du
concile de Trente, et lecture à haute voix de sa formule latine, qui est
longue, et que le grand aumônier leur tient sur ses genoux, assis dans
un fauteuil, la signent, et prennent un certificat du grand aumônier
d'avoir rempli ce devoir. Les deux princes nommés au chapitre tenu à
Reims s'acquittèrent de ce devoir.

Le cardinal de Rohan, ne doutant de rien sur l'appui de la protection si
déclarée et si bien méritée du cardinal Dubois, saisit une si belle
occasion d'établir sa princerie, d'autant mieux que c'était la première
promotion de l'ordre qui se faisait depuis qu'il était grand aumônier.
Il donna ses ordres à son secrétaire qui, en signant les certificats de
ces princes au-dessous de la signature du cardinal de Rohan, mit
hardiment \emph{par Son Altesse Éminentissime}, au lieu de mettre
simplement \emph{par monseigneur}. Le secrétaire des commandements du
régent, qui retira le certificat de M. le duc de Chartres, y jeta les
yeux par hasard, et fut si étrangement surpris de \emph{l'Altesse
Éminentissime} qu'il alla sur-le-champ en avertir M. le duc d'Orléans.
La colère le transporta à l'instant malgré sa douceur naturelle et son
peu de dignité, mais au fond très glorieux. Il envoya sur-le-champ
chercher l'abbé de Pomponne, chancelier de l'ordre. C'était l'heure
qu'on sortait de dîner pour aller bientôt aux premières vêpres du sacre,
et le chapitre de l'ordre s'était tenu la veille. L'abbé de Pomponne m'a
conté qu'il fut effrayé de la colère où il trouva M. le duc d'Orléans,
au point qu'il ne sut ce qui allait arriver. Il lui commanda d'aller
dire de sa part au cardinal de Rohan d'expédier sur-le-champ deux autres
certificats à MM. les duc de Chartres et comte de Charolais, où il y eût
seulement \emph{par monseigneur}, d'y supprimer \emph{l'Altesse
Éminentissime} qu'il avait osé y hasarder, et de lui défendre de la part
du roi de jamais l'employer dans aucun certificat de chevalier de
l'ordre. Le régent ajouta l'ordre à l'abbé de Pomponne de faire écrire
le fait et l'ordre en conséquence, tant à l'égard du certificat expédié
à chacun de ces deux princes, que {[}pour{]} tous ceux à expédier à tous
chevaliers de l'ordre nommés à l'avenir, sur les registres de l'ordre.

Le cardinal de Rohan et son frère furent bien mortifiés de cet ordre,
dont ils ne s'étaient pas défiés par le caractère du régent et par la
protection du premier ministre. Ils obéirent sur-le-champ même et sans
réplique, et l'avalèrent sans oser en faire le plus léger semblant. De
pareilles tentatives, souvent avec succès, sont les fondements des
prétentions, et trop ordinairement de la possession de ces chimères de
rang de prince étranger je l'ai remarqué ici en plus d'une occasion.
Quand je fus chevalier de l'ordre, cinq ans après, j'avertis les
maréchaux de Roquelaure et d'Alègre et le comte de Grammont, qui furent
de la même promotion avec le prince de Dombes, le comte d'Eu et des
absents, de prendre bien garde à leurs certificats. M. le duc d'Orléans
n'était plus et les entreprises revivent. Je voulus voir le mien chez le
cardinal de Rohan môme, au sortir de ma profession de foi. Le
secrétaire, qui en sentit bien la cause, me dit un peu honteusement que
je n'y trouverais que ce qu'il y fallait, et me le présenta. En effet,
j'y vis \emph{par monseigneur} et point d'\emph{Altesse\,;} je souris en
regardant le secrétaire, et lui dis\,: «\,Bon, monsieur, comme cela,\,»
et je l'emportai. Je sus des trois autres que j'avais avertis, que les
leurs étaient de même. Cela me montra qu'ils avaient abandonné cette
prétention. Certainement le coup était bon à faire\,; si le premier
prince du sang, fils du régent, et un autre prince du sang avaient
souffert l'Altesse du cardinal de Rohan, qui eût pu après s'en
défendre\,?

Il n'y eut de séance à la cérémonie de l'ordre que pour le clergé et
pour la même robe, même les secrétaires du roi, qui y eurent les mêmes
qu'au sacre. Tout le reste n'y fut placé qu'à titre de curieux,
pêle-mêle, comme il plut au grand maître des cérémonies. Il n'y eut que
les chevaliers de l'ordre, qui étaient en petit nombre, qui formèrent
seuls la cérémonie. Ce qu'il y eut de nouveau, car il y eut du nouveau
partout, c'est que les officiers de l'ordre se couvrirent dans le
choeur, comme les chevaliers, eux qui dans les chapitres, excepté le
seul chancelier de l'ordre, sont au bout de la table, derrière lui,
debout et découverts, et les chevaliers et le chancelier assis et
couverts. Aussi, comme je l'ai remarqué ailleurs, ont-ils fait en sorte
qu'il n'y a plus de chapitre qu'en foule, en désordre, sans rang, où le
roi est debout et découvert, et qu'il n'y a plus de repas, parce que le
chancelier de l'ordre y mange seul avec le roi et les chevaliers en
réfectoire, et les autres grands officiers mangent en même temps avec
les petits officiers de l'ordre dans une salle séparée.

À l'égard de la cavalcade, il ne se put rien ajouter à l'excès de sa
confusion. Les princes du sang y prirent, pour la première fois, un
avantage que le régent souffrit pour l'intérêt de M. son fils contre le
sien. Chacun d'eux eut près de soi un de ses principaux domestiques.
Cela ne fut jamais permis qu'aux fils de France et aux petits-fils de
France, c'est-à-dire à M. le duc de Chartres, depuis duc d'Orléans,
enfin régent, seul petit-fils de France, qui ait existé depuis
l'établissement de ce rang pour Mademoiselle, fille de Gaston, et pour
ses soeurs, qui toutes n'avaient point de frères. Cette nouveauté en a
enfanté bien d'autres depuis que M. le Duc fut premier ministre.

Je ne parle point de beaucoup d'autres remarques, cela serait infini\,;
j'omets aussi les fêtes superbes que M. le duc d'Orléans et M. le Duc
donnèrent au roi, à Villers-Cotterêts et à Chantilly, en revenant de
Reims.

Tout en arrivant à Paris, La Fare et Belle-Ile me vinrent voir à la
Ferté. La Fare était aussi fort ami de M\textsuperscript{me} de
Plénoeuf, mais non son esclave comme ses deux amis Le Blanc et
Belle-Ile. Ils me parlèrent fort de leur inquiétude sur la vivacité avec
laquelle l'affaire de La Jonchère se poussait, lequel avait été conduit
à la Bastille, et qu'on ne parlait pas de moins que d'ôter à Le Blanc sa
charge de secrétaire d'État e et de l'envelopper avec Belle-Ile dans la
même affaire. Quoique La Fare n'y fût pour rien, ils venaient me
demander conseil et secours. Je leur dis franchement que je voyais
clairement la suite du projet d'écarter de M. le duc d'Orléans tous ceux
en qui il avait habitude de confiance, et ceux encore dont on pouvait
craindre la familiarité avec lui, dont les exemples des exils récents
faisaient foi\,; que Le Blanc étant celui de tous le plus à éloigner, en
suivant ce plan par l'accès de sa charge et par l'habitude de confiance
et de familiarité, le prétexte et le moyen en était tout trouvé par
l'affaire de La Jonchère\,; que le cardinal Dubois aurait encore à en
faire sa cour à M. le Duc et à M\textsuperscript{me} de Prie, et à tout
rejeter sur eux\,; qu'ils connaissaient tous deux l'esprit et la rage de
M\textsuperscript{me} de Prie contre les deux inséparables amis de sa
mère, et quel était son pouvoir sur M. le Duc\,; qu'ils ne connaissaient
pas moins l'impétuosité et la férocité de M. le Duc, la faiblesse
extrême de M. le duc d'Orléans, l'empire que le cardinal Dubois avait
pris sur lui\,; qu'il n'y avait point d'innocence ni d'amitié de M. le
duc d'Orléans qui pussent tenir contre le cardinal, M. le Duc et sa
maîtresse réunis par d'aussi puissants intérêts\,; que je ne voyais donc
nul autre moyen de conjurer l'orage que d'apaiser la fille en voyant
moins la mère, qui ne courait risque de rien, à qui cela ne faisait
aucun tort, et qui, si elle avait de la raison et une amitié véritable
pour eux, et qui méritât la leur, devait être la première à exiger de
ses deux amis à faire ce sacrifice à une fureur à laquelle ils ne
pouvaient résister, qu'en la désarmant par cette voie, même de ne voir
plus la mère, laquelle ne méritait pas qu'ils se perdissent pour elle,
si elle le souffrait.

La Fare trouvait que je disais bien, et que ce que je proposais était la
seule voie de salut, si déjà l'affaire n'était trop avancée. Belle-Ile
ne put combattre mes raisons ni se résoudre à suivre ce que je pensais,
et se mit, faute de mieux, à battre la campagne. J'avais beau le ramener
au point, il s'échappait toujours. À la fin, je lui prédis la prompte
perte de Le Blanc et la sienne, que le cardinal, M. le Duc et sa
maîtresse entreprenaient de concert, et dont ils ne se laisseraient pas
donner le démenti, si, en suivant mon opinion, ils ne désarmaient
promptement M. le Duc et sa maîtresse par le sacrifice que je
proposais\,; quoi fait, ils auraient encore bien de la peine à se tirer
des griffes seules du cardinal\,; mais que, quand ils n'auraient plus
affaire qu'à lui, encore y aurait-il espérance. Mais rien ne put
ébranler Belle-Ile. Question fut donc de voir quelle conduite il aurait,
si les choses se portaient à l'extrémité, comme je le croyais. Je
conclus à la fuite, et que Belle-Ile attendît hors du royaume les
changements que les temps amènent toujours.

La Fare fut aussi de cet avis, mais Belle-Ile s'écria que fuir serait
s'avouer coupable, et qu'il préférait de tout risquer, étant bien sûr
qu'il n'y avait sur lui aucune prise. Je lui demandai s'il n'avait
jamais vu, au moins dans les histoires, d'innocents opprimés, et trop
souvent encore sous nos yeux, par des procès, mais que je ne croyais pas
qu'il en eût vu aucun échapper à des premiers ministres, quand ils y
mettent tout leur pouvoir, encore moins s'ils se trouvent soutenus d'un
prince du sang du caractère et dans la posture où était M. le Duc, et
d'une femme de l'esprit et de l'emportement de M\textsuperscript{me} de
Prie\,; que personne n'ignorait qu'avec de telles parties, si hautement
déclarées et engagées, raison, justice, innocence, évidence n'avaient
plus lieu\,: par conséquent que fuir leur fureur et leur puissance, l'un
et l'autre, n'était rien moins que s'avouer coupable, mais sagesse et
nécessité\,; s'y exposer, folie consommée. Ce raisonnement, qui me
paraissait évident et solide, ne put rien gagner sur Belle-Ile. Il s'en
retourna avec La Fare persuadé, sans être lui-même le moins du monde
ébranlé, malgré ma prédiction réitérée, de laquelle pourtant il ne
s'éloignait pas.

Ils m'apprirent que le roi, avec lequel était M. le duc d'Orléans, etc.,
trouva, en arrivant à Paris, le duc de Tresmes venant en cérémonie
au-devant de lui. La survivance du gouvernement de Paris lui fut donnée
pour son fils aîné, qu'il ne songeait pas à demander. Son fils avait
alors trente ans, et avait eu, dès 1716, la survivance de la charge de
premier gentilhomme de la chambre qu'avait son père. Celle-ci ne nuisit
pas à l'autre. Le premier ministre voulait se faire des amis de ce qui
environnait le roi.

Le 25 novembre, don Patricia Laullez, ambassadeur extraordinaire
d'Espagne, conduit et reçu avec les cérémonies accoutumées, fit au roi
la demande de M\textsuperscript{lle} de Beaujolais pour don Carlos, et
fut ensuite chez M. {[}le Duc{]} et M\textsuperscript{me} la duchesse
d'Orléans. Il fut après traité à dîner avec sa suite, après quoi il alla
chez le cardinal Dubois, où les articles furent signés par lui et par
les commissaires du roi, qui furent le cardinal Dubois, Armenonville,
garde des sceaux, la Houssaye, chancelier de M. le duc d'Orléans,
conseiller d'État, et Dodun, contrôleur général des finances. Laullez
fut ensuite reconduit à Paris, à l'hôtel des ambassadeurs
extraordinaires. Le lendemain il retourna à Versailles, accompagné et
reçu comme la veille, et conduit, sur les cinq heures du soir, dans le
cabinet du roi, où étaient tous les princes et princesses du sang,
debout des deux côtés d'une table, au milieu de laquelle le roi était
dans son fauteuil, sur laquelle le contrat de mariage fut signé par le
roi et tous les princes et princesses du sang sur une colonne, au bas de
laquelle le cardinal Dubois signa, et l'ambassadeur signa seul sur
l'autre colonne\,; après quoi il fut reconduit à Paris.

Le 1er de décembre M\textsuperscript{lle} de Beaujolais partit de Paris
pour se rendre à Madrid, accompagnée, jusqu'à la frontière, de la
duchesse de Duras, qui mena avec elle la duchesse de Fitz-James sa
fille, qui eurent toujours un fauteuil, une soucoupe, le vermeil doré,
etc., avec la princesse. Elle fut servie par les officiers du roi et par
ses équipages, et accompagnée d'un détachement des gardes du corps
jusqu'à la frontière. M. le duc d'Orléans et M. le duc de Chartres la
conduisirent de Paris jusqu'au Bourg-la-Reine. Quelques jours après le
prévôt des marchands, à la tête du corps de là ville de Paris, alla, par
ordre dû roi, complimenter l'ambassadeur d'Espagne, et lui présenter les
présents de la ville.

Enfin la fameuse princesse des Ursins mourut à Rome, où elle s'était, à
la fin, retirée et fixée depuis plus de six ans, aimant mieux y
gouverner la petite cour d'Angleterre que de ne gouverner rien du tout.
Elle avait quatre-vingt-cinq ans, fraîche encore, droite, de la grâce et
des agréments, une santé parfaite jusqu'à la maladie peu longue dont
elle mourut\,; la tête et l'esprit comme à cinquante ans, et fort
honorée à Rome, où elle eut le plaisir de voir les cardinaux del Giudice
et Albéroni l'être fort peu. On a tant et si souvent parlé ici de cette
dame si extraordinaire et si illustre, qu'il n'y a rien à y ajouter.

Madame, dont la santé avait toujours été extrêmement forte et constante,
ne se portait plus bien depuis quelque temps, et se sentait même assez
mal pour être persuadée qu'elle allait tomber dans une maladie dont elle
ne relèverait pas. L'inclination allemande qu'elle avait toujours eue au
dernier point, lui donnait une prédilection extrême pour
M\textsuperscript{me} la duchesse de Lorraine et pour ses enfants,
par-dessus M. le duc d'Orléans et les siens. Elle mourait d'envie de
voir les enfants de M\textsuperscript{me} la duchesse de Lorraine,
qu'elle n'avait jamais vus, et se faisait un plaisir extrême de les voir
à Reims, où M\textsuperscript{me} la duchesse de Lorraine, qui voulait
voir le sacre, les devait amener. Madame, se sentant plus incommodée,
balança fort sur le voyage qui approchait beaucoup, et voulait devancer
le roi à Reims de plusieurs jours pour être plus longtemps avec
M\textsuperscript{me} la duchesse de Lorraine, à qui elle avait donné
rendez-vous à jour marqué et à ses enfants. On a vu ici, à la mort de
Monsieur, qu'elle prit à elle la maréchale de Clerembault, et la feue
comtesse de Beuvron qu'elle avait toujours fort aimées et que Monsieur
avait chassées de chez lui, et qu'il haïssait fort.

La maréchale de Clerembault croyait avoir une grande connaissance de
l'avenir par l'art des petits points\,; et comme, Dieu merci, je ne sais
ce que c'est, je n'expliquerai point cette opération, en laquelle Madame
avait aussi beaucoup de confiance. Elle consulta donc la maréchale sur
le voyage de Reims, qui lui répondit fermement\,: «\,Partez, madame, en
toute sûreté, je me porte bien.\,» C'est qu'elle prétendait avoir vu par
ces petits points qu'elle mourrait avant Madame, qui sur cette confiance
alla à Reims. Elle y fut logée dans la belle abbaye de Saint-Pierre avec
M\textsuperscript{me} la duchesse de Lorraine, où le roi les alla voir
deux fois, et dont une soeur du feu comte de Roucy était abbesse. Madame
vit le sacre et les cérémonies de l'ordre du lendemain dans une tribune
avec M\textsuperscript{me} la duchesse de Lorraine et ses enfants, dans
laquelle le frère du roi de Portugal eut aussi place. Mais au retour du
sacre elle perdit la maréchale de Clerembault, qui mourut à Paris le 27
novembre, dans sa quatre-vingt-neuvième année, ayant jusqu'alors la
santé, la tête, l'esprit et l'usage de tous ses sens comme à quarante
ans. Elle était fille de Chavigny, secrétaire d'État, mort à
quarante-quatre ans, en octobre 1652, dont j'ai parlé à l'entrée de ces
Mémoires, et qui était fils de Bouthillier, surintendant des finances,
mort un an avant lui. La mère de la maréchale était fille unique et
héritière de Jacques Phélypeaux, seigneur de Villesavin et d'Isabelle
Blondeau, que j'ai vue, et fait collation dans sa chambre avec de jeunes
gens de mon âge qui allions voir son arrière-petit-fils, et je la
peindrais encore grande, grasse, l'air sain et frais. Elle nous conta
qu'elle était dans son carrosse avec son mari sur le pont Neuf, lorsque
tout à coup ils entendirent de grands cris, et qu'ils apprirent un
moment après que Henri IV venait d'être tué. Pour revenir à la maréchale
de Clerembault, elle eut plusieurs frères et soeurs, entre autres
l'évêque de Troyes qui, démis et retiré, fut mis dans le conseil de
régence, et duquel il a été souvent parlé ici\,; M\textsuperscript{me}
de Brienne Loménie, femme du secrétaire d'État, morte dès 1664, et la
duchesse de Choiseul, seconde femme sans enfants du dernier duc de
Choiseul, veuve en première noces de Brûlart, premier président du
parlement de Dijon, dont elle eut la duchesse de Luynes, dame d'honneur
de la reine.

La maréchale de Clerembault avait épousé, en 1654, le maréchal de
Clerembault, qui avait été fait maréchal de France dix-huit mois
auparavant. Il eut le gouvernement du Berry, et fut chevalier de l'ordre
en la première grande promotion du feu roi en 1661, et mourut en 1665, à
cinquante-sept ans, ne laissant qu'une fille qui fut religieuse, et deux
fils dont on a parlé ici à l'occasion de leur mort sans alliance. Le
maréchal de Clerembault était homme de qualité, bon homme de guerre, et
avait été mestre de camp général de la cavalerie, fort à la mode sous le
nom de comte de Palluau, avant qu'il prit son nom lorsqu'il devint
maréchal de France. C'était un homme de beaucoup d'esprit, orné,
agréable, plaisant, insinuant et souple, avec beaucoup de manége,
toujours bien avec les ministres, fort au gré du cardinal
Mazarin\footnote{Le comte de Palluau devint maréchal de France en 1652.
  On était alors en pleine Fronde, et les poètes satiriques
  n'épargnèrent pas un général qui était resté fidèle à Mazarin. Blot
  lui décocha le couplet suivant\,: À ce grand maréchal de France, /
  Favori de Son Éminence, / Qui a si bien battu Persan\,; / Palluau, ce
  grand capitaine, / Qui prend un château dans un an, / Et perd trois
  places par semaine.}, et fort aussi au gré du monde et toujours parmi
le meilleur. Sa femme, devenue veuve, fut gouvernante des filles de
Monsieur, et accompagna la reine d'Espagne jusqu'à la frontière, en
qualité de sa dame d'honneur.

C'était une des femmes de son temps qui avait le plus d'esprit, le plus
orné sans qu'il y parût, et qui savait le plus d'anciens faits curieux
de la cour, la plus mesurée et la plus opiniâtrement silencieuse. Elle
en avait contracté l'habitude par avoir été constamment une année
entière sans proférer une seule parole dans sa jeunesse, et se guérit
ainsi d'un grand mal de poitrine. Elle n'avait jamais bu que de l'eau,
et fort peu. Souvent aussi son silence venait de son mépris secret pour
les compagnies où elle se trouvait et pour les discours qu'on y
tenait\,; mais lorsqu'elle était en liberté, elle était charmante, on ne
la pouvait quitter. Je l'ai souvent vue de la sorte entre trois ou
quatre personnes au plus chez la chancelière de Pontchartrain dont elle
était fort amie. C'était un tour, un sel, une finesse, et avec cela un
naturel inimitable. Elle fut allant, venant à la cour en grand habit
presque toujours jusqu'à sa dernière maladie. Fort riche et avare. Par
les chemins et dans les galeries, elle avait toujours un masque de
velours noir. Sans avoir jamais été ni prétendu être belle ni jolie,
elle avait encore le teint parfaitement beau, et elle prétendait que
l'air lui causait des élevures. Elle était l'unique qui en portât, et
quand on la rencontrait et qu'on la saluait, elle ne manquait jamais à
l'ôter pour faire la révérence. Elle aimait fort le jeu, mais le jeu de
commerce et point trop gros, et eût joué volontiers jour et nuit. Je me
suis peut-être trop étendu sur cet article\,: les singularités curieuses
ont fait couler ma plume.

Madame fut d'autant plus touchée de la perte de cette ancienne et intime
amie qu'elle savait que les petits points avaient toujours prédit
qu'elle la survivrait, mais que ce serait de fort peu. En effet, elle la
suivit de fort près. L'hydropisie, qui se déclara tard, fit en très peu
de jours un tel progrès qu'elle se prépara à la mort avec beaucoup de
fermeté et de piété. Elle voulut presque toujours avoir auprès d'elle
l'ancien évêque de Troyes, frère de la maréchale de Clerembault, et lui
dit\,: «\,Monsieur, de Troyes, voilà une étrange partie que nous avons
faite la maréchale et moi.\,» Le roi la vint voir, et elle reçut tous
les sacrements. Elle mourut à Saint-Cloud le 8 de décembre, à quatre
heures du matin, à près de soixante et onze ans. Elle ne voulut point
être ouverte, ni de pompe à Saint-Cloud. Ainsi dès le 10 du même mois,
elle fut portée à Saint-Denis dans un carrosse sans aucun appareil de
deuil, le carrosse précédé, environné et suivi des pages des deux
écuries du roi, des gardes et des suisses de M. le duc d'Orléans, et de
ses valets de pied avec des flambeaux. M\textsuperscript{lle} de
Charolais et les duchesses d'Humières et de Tallard accompagnaient dans
un autre carrosse, où était M\textsuperscript{me} de Châteauthiers, dame
d'atours de Madame, avec M\textsuperscript{me}s de Tavannes et de
Flamarens. Madame tenait en tout beaucoup plus de l'homme que de la
femme. Elle était forte, courageuse, allemande au dernier point,
franche, droite, bonne et bienfaisante, noble et grande en toutes ses
manières, et petite au dernier point sur tout ce qui regardait ce qui
lui était dû. Elle était sauvage, toujours enfermée à écrire, hors les
courts temps de cour chez elle\,; du reste, seule avec ses dames\,;
dure, rude, se prenant aisément d'aversion, et redoutable par les
sorties qu'elle faisait quelquefois, et sur quiconque\,; nulle
complaisance\,; nul tour dans l'esprit, quoiqu'elle {[}ne{]} manquât pas
d'esprit\,; nulle flexibilité, jalouse, comme on l'a dit, jusqu'à la
dernière petitesse, de tout ce qui lui était dû\,; la figure et le
rustre d'un Suisse, capable avec cela d'une amitié tendre et inviolable.
M. le duc d'Orléans l'aimait et la respectait fort. Il ne la quitta
point pendant sa maladie, et lui avait toujours rendu de grands devoirs,
mais il ne se conduisit jamais par elle. Il en fut fort affligé. Je
passai le lendemain de cette mort plusieurs heures seul avec lui à
Versailles, et je le vis pleurer amèrement.

Les ambassadeurs et la cour se présentèrent devant le roi en manteaux
longs et en mantes, ainsi que les princes et les princesses du sang, et
pareillement chez M. {[}le Duc{]} et M\textsuperscript{me} la duchesse
d'Orléans, qui les reçut de même, et M\textsuperscript{me} la duchesse
d'Orléans au lit, après que l'un et l'autre eurent été avec M. le duc de
Chartres, en manteaux et en mantes, saluer le roi, qui après alla voir
M. {[}le Duc{]} et M\textsuperscript{me} la duchesse d'Orléans. Le roi
fut harangué par le parlement et par toutes les autres compagnies,
lesquelles, toutes allèrent saluer M. {[}le Duc{]} et
M\textsuperscript{me} la duchesse d'Orléans. Le roi drapa, parce que
Madame était veuve du grand-père maternel du roi. Cette perte ne fit pas
grande sensation à la cour ni dans le monde. La duchesse de Brancas, sa
dame d'honneur, ne parut à rien, étant déjà attaquée du cancer au sein
dont elle mourut assez longtemps après.

M\textsuperscript{me} de Cani, veuve du fils unique de Chamillart, avec
beaucoup d'enfants, et soeur du duc de Mortemart, s'ennuya enfin de
porter le nom de son mari, et en un tourne-main son mariage se fit avec
le prince de Chalais, grand d'Espagne, qui, ennuyé de l'Espagne où il
n'avait que cette dignité, sans grade militaire qui lui pût faire rien
espérer par delà la médiocre pension qu'il en avait, s'était depuis peu
fixé en France pour toujours, où était son bien et sa famille. Toute
celle de Mortemart parut fort aise de ce mariage. Ce qu'il y eut de
louable, est que les enfants du premier lit n'en ont été que plus
constamment chéris et bien traités en tout de la mère et de son second
mari. Le prince de Robecque, aussi grand d'Espagne, et dégoûté du séjour
et du service d'Espagne, où il était lieutenant général, et fixé en
France avec le même grade, épousa, à Paris, M\textsuperscript{lle} du
Bellay.

L'année finit par le traité de paix conclu à Nystadt entre le czar et la
Suède, qui céda au czar toutes les conquêtes qu'il avait faites sur
elle, ce qui la restreignit au delà de la mer Baltique et lui ôta toute
la considération que les conquêtes de Charles\ldots{}\footnote{Saint-Simon
  n'a pas indiqué de quel Charles il voulait parler. Il s'agit
  probablement ici de Charles X, ou Charles-Gustave, qui régna en Suède
  de 1654 à 1660, et se signala par ses victoires sur les Danois et les
  Polonais.} lui avaient acquise au deçà, et conséquemment toute sa
considération en Allemagne et dans le reste de l'Europe, tellement que
cette monarchie, revenue à son dernier état, se trouva de plus ruinée et
dans le dernier abattement, fruit du prétendu héroïsme de son dernier
monarque\footnote{Passage omis dans les précédentes éditions depuis
  \emph{l'année finit} jusqu'à \emph{son dernier monarque}. Nous n'avons
  pas cru devoir supprimer ce paragraphe, quoiqu'il revienne sur un
  événement dont Saint-Simon a déjà parlé, et qu'il y ait ici une erreur
  de date. Le traité de Nystadt fut signé le 10 septembre 1721 et non à
  la fin de l'année 1722.}.

\hypertarget{chapitre-xviii.}{%
\chapter{CHAPITRE XVIII.}\label{chapitre-xviii.}}

1723

~

{\textsc{Année 1723.}} {\textsc{- Stérilité des récits de cette année\,;
sa cause.}} {\textsc{- Mort de l'abbé de Dangeau.}} {\textsc{- Mort du
prince de Vaudémont\,; du duc de Popoli à Madrid, et sa dépouille.}}
{\textsc{- Mort et caractère de M. Le Hacquais.}} {\textsc{- Obsèques de
Madame à Saint-Denis.}} {\textsc{- Mort, famille, caractère, obsèques de
M\textsuperscript{me} la Princesse.}} {\textsc{- Biron, Lévi et La
Vallière faits et reçus ducs et pairs à la majorité.}} {\textsc{-
Majorité du roi.}} {\textsc{- Lit de justice.}} {\textsc{- Il visite les
princesses belle-fille, filles, même la soeur de feu
M\textsuperscript{me} la Princesse, et point ses petites-filles, quoique
princesses du sang.}} {\textsc{- Conseil de régence éteint.}} {\textsc{-
Forme nouvelle du gouvernement.}} {\textsc{- Survivance de la charge de
secrétaire d'État de La Vrillière à son fils.}} {\textsc{- Mariage
secret du comte de Toulouse avec la marquise de Gondrin.}} {\textsc{-
Fin de la peste de Provence, et le commerce universellement rétabli.}}
{\textsc{- M\textsuperscript{lle} de Beaujolais remise à la frontière
par le duc de Duras au duc d'Ossone, et reçue par Leurs Majestés
Catholiques, etc., à une journée de Madrid, où il se fait de belles
fêtes.}} {\textsc{- Le chevalier d'Orléans, grand prieur de France, et
le comte de Bavière, bâtard de l'électeur, faits grands d'Espagne.}}
{\textsc{- Explication des diverses sortes d'entrées chez le roi, et du
changement et de la nouveauté qui s'y fit.}} {\textsc{- Rétablissement
des rangs et honneurs des bâtards, avec des exceptions peu perceptibles,
dont ils osent n'être pas satisfaits.}} {\textsc{- Cardinal Dubois
éclate sans mesure contre le P. Daubenton.}} {\textsc{- Cause de cet
éclat sans retour.}} {\textsc{- Mort du prince de Courtenay.}}
{\textsc{- Détails des troupes et de la marine rendus aux secrétaires
d'État.}} {\textsc{- Duc du Maine conserve ceux de l'artillerie et des
Suisses, et y travaille chez le cardinal Dubois.}} {\textsc{- Maulevrier
arrivé de Madrid, où Chavigny est chargé des affaires, sans titre.}}
{\textsc{- Mariage de Maulevrier-Colbert avec M\textsuperscript{lle}
d'Estaing, et du comte de Peyre avec M\textsuperscript{lle} de
Gassion.}} {\textsc{- Mort de la princesse de Piémont (palatine
Soultzbach)\,; du duc d'Aumont\,; de Beringhen, premier écuyer du roi\,;
de la marquise d'Alègre\,; de M\textsuperscript{me} de Châteaurenaud et
de M\textsuperscript{me} de Coëtquen, soeur de Noailles\,; du fils aîné
du duc de Lorraine.}} {\textsc{- Cardinal Dubois préside à l'assemblée
du clergé.}} {\textsc{- La Jonchère à la Bastille.}} {\textsc{- Le Blanc
exilé.}} {\textsc{- Breteuil secrétaire d'État de la guerre.}}
{\textsc{- Cause singulière et curieuse de sa fortune.}} {\textsc{- Son
caractère.}}

~

Cette année {[}1723{]}, dont la fin est le terme que j'ai prescrit à ces
Mémoires, n'aura ni la plénitude ni l'abondance des précédentes. J'étais
ulcéré des nouveautés du sacre\,; je voyais s'acheminer le complet
rétablissement de toutes les grandeurs des bâtards, j'avais le coeur
navré de voir le régent à la chaîne de son indigne ministre, et n'osant
rien sans lui ni que par lui\,; l'État en proie à l'intérêt, à
l'avarice, à la folie de ce malheureux sans qu'il y eût aucun remède.
Quelque expérience que j'eusse de l'étonnante faiblesse de M. le duc
d'Orléans, elle avait été sous mes yeux jusqu'au prodige lorsqu'il fit
ce premier ministre après tout ce que je lui avais dit là-dessus, après
ce qu'il m'en avait dit lui-même, enfin de la manière incroyable à qui
ne l'a vu comme moi, dont je l'ai raconté dans la plus exacte vérité. Je
n'approchais plus de ce pauvre prince à tant de grands et utiles talents
enfouis, qu'avec répugnance\,; je ne pouvais m'empêcher de sentir
vivement sur lui ce que les mauvais Israélites se disaient dans le
désert sur la manne\,: \emph{Nauseat anima mea super cibum istum
levissimum}. Je ne daignais plus lui parler. Il s'en apercevait, je
sentais qu'il en était peiné\,; il cherchait à me rapprocher, sans
toutefois oser me parler d'affaires que légèrement et avec contrainte,
quoique sans pouvoir s'en empêcher. Je prenais à peine celle d'y
répondre, et j'y mettais fin tout le plus tôt que je le pouvais\,;
j'abrégeais et je ralentissais mes audiences\,; j'en essuyais les
reproches avec froideur. En effet, qu'aurais je eu à dire ou à discuter
avec un régent qui ne l'était plus, pas même de soi, bien loin de l'être
du royaume, où je voyais tout en désordre.

Le cardinal Dubois, quand il me rencontrait, me faisait presque sa cour.
Il ne savait par où me prendre. Les liens de tous les temps et sans
interruption étaient devenus si forts entre M. le duc d'Orléans et moi,
que le premier ministre, qui les avait sondés plus d'une fois, n'osait
se flatter de les pouvoir rompre. Sa ressource fut d'essayer de me
dégoûter par imposer à son maître une réserve à mon égard qui nous était
à tous deux fort nouvelle, mais qui lui coûtait plus qu'à moi par
l'habitude, et j'oserai dire par l'utilité qu'il avait si souvent
trouvée dans cette confiance, et moi je m'en passais plus que
volontiers, dans le dépit de n'en pouvoir espérer aucun fruit ni pour le
bien de l'État, ni pour l'honneur et l'avantage de M. le duc d'Orléans,
totalement livré à ses plaisirs de Paris, et au dernier abandon à son
ministre. La conviction de mon inutilité parfaite me retira de plus en
plus, sans avoir jamais eu le plus léger soupçon qu'une conduite
différente pût m'être dangereuse, ni que, tout faible et tout abandonné
que fût le régent au cardinal Dubois, celui-ci pût venir à bout de me
faire exiler comme le duc de Noailles et Canillac, ni de me faire donner
des dégoûts à m'en faire prendre le parti. Je demeurai donc dans ma vie
accoutumée, c'est-à-dire ne voyant jamais M. le duc d'Orléans que tête à
tête, mais le voyant peu à peu, toujours plus de loin en plus loin,
froidement, courtement, sans ouvrir aucun propos d'affaires, les
détournant même de sa part quand il en entamait, et y répondant de façon
à les faire promptement tomber. Avec cette conduite et ces vives
sensations, on voit aisément que je ne fus de rien, et que ce que
j'aurai à raconter de cette année sentira moins la curiosité et
l'instruction de bons et de fidèles Mémoires, que la sécheresse et la
stérilité des faits répandus dans des gazettes.

L'abbé de Dangeau mourut au commencement de cette année, à quatre-vingts
ans. Il en a été {[}assez{]} parlé d'avance à l'occasion de la mort de
son frère aîné, pour n'avoir rien à y ajouter. Il n'avait qu'une abbaye
et un joli prieuré à Gournay-sur-Marne, qui lui faisait une très
agréable maison de campagne à la porte de Paris, aussi bon homme et
aussi fade que son frère.

Le prince de Vaudemont mourut presque en même temps, à
quatre-vingt-quatre ans, à Commercy, où il s'était comme retiré depuis
la mort du feu roi, venant rarement et courtement à Paris, et n'allant
guère plus souvent ni plus longuement à Lunéville. Il a tant et si
souvent été parlé de la naissance, de la famille, de la fortune, des
perfidies, des cabales de cet insigne Protée, que je ne m'y étendrai pas
ici. Ses chères nièces lui allaient tenir compagnie tous les ans,
longtemps, surtout depuis que l'aînée, tombée des nues par la mort de
Monseigneur, puis par celle du roi, s'était fait une planche, après le
naufrage, de l'abbaye de Remiremont, qu'elle avait su obtenir fort -peu
après la mort de Monseigneur. La princesse d'Espinoy recueillit
l'immense héritage de ce cher oncle, excepté Commercy, qui revint au duc
de Lorraine, qui renvoya à l'empereur le collier de la Toison, que
Vaudemont avait de Charles II.

Le duc de Popoli, duquel j'ai aussi tant parlé, mourut à Madrid quelques
jours après. Le duc de Bejar eut sa place de majordome-major du prince
des Asturies, et le duc d'Atri, frère du cardinal Acquaviva, eut sa
compagnie italienne des gardes du corps. Le duc de Popoli avait
soixante-douze ans, et il était chevalier du Saint-Esprit et de la
Toison d'or. Ce fut une perte pour la cabale italienne, et un gain pour
les Espagnols et pour les honnêtes gens. Son fils, dont j'ai aussi
beaucoup parlé, trouva un prodigieux argent comptant et force
pierreries, qu'il ne tarda pas à manger, ni à se ruiner ensuite. Il fit
aussitôt après sa couverture de grand d'Espagne.

Un plus honnête homme qu'eux les suivit de près, mais d'une condition si
différente que je n'en parlerais pas ici sans la singularité de ses
vertus\,; et que je l'ai fort connu à Pontchartrain. Il s'appelait Le
Hacquais, et par corruption M. des Aguets, conseiller d'honneur à la
cour des aides, après y avoir été longtemps avocat général avec la plus
grande réputation de droiture et la première d'éloquence, avec une
capacité profonde et une facilité surprenante à parler et à écrire. Il
était plein d'histoire et de belles-lettres, de goût le plus délicat, du
sel le plus fin et du tour le plus singulier et le plus agréable. Il
avait la conversation charmante, naturelle, pleine de traits\,; il était
modeste, poli, respectueux, et jamais ne montrait la moindre érudition.
La galanterie et l'amour de la chasse les avait unis le chancelier de
Pontchartrain et lui dans leur jeunesse\,; leurs coeurs ne s'étaient
jamais désunis depuis. Il était de tous les voyages de Pontchartrain,
aussi aimé de la chancelière, de toute la famille et de tous les amis
qu'il l'était du chancelier, et il était là dans un air de considération
infinie, et y chassait, tant qu'il pouvait, à tirer à pied et à cheval,
et à courre le renard avec le chancelier. Il était extrêmement sobre et
simple en tout. Ses vers galants autrefois, et sur toutes sortes de
sujets, étaient pleins de pensées, de tour, de traits et de justesse. Il
y avait longtemps, quand je le connus à Pontchartrain, qu'il était
convenu fort homme de bien et même pénitent. Ce changement lui avait
tellement fermé la bouche que le chancelier l'appelait son muet, et on y
perdait infiniment. Quand il faisait tant que de dire quelque chose,
c'était toujours avec un sel et une grâce qui ravissait. Je lui disais
souvent que j'avais envie de le battre jusqu'à ce qu'il se mit à parler.
Il ne fut jamais marié, fort solitaire et sauvage depuis sa grande
piété, et mourut avec peu de bien, duquel il ne s'était jamais soucié, à
quatre-vingt-quatre ans, regretté de beaucoup d'amis, et avec une
réputation grande et rare.

Les obsèques de Madame se firent à Saint-Denis, le 13 février.
M\textsuperscript{lle}s de Charolais, de Clermont et de la
Roche-sur-Yon, firent le deuil, menées par M. le duc de Chartres, M. le
duc et M. le comte de Clermont. Les cours supérieures y assistèrent.
L'archevêque d'Albi (Castries) officia, et l'évêque de Clermont
(Massillon) fit l'oraison funèbre, qui fut belle.

M\textsuperscript{me} la Princesse suivit Madame de près. Elle mourut à
Paris, le 23 février, à soixante-quinze ans. Elles étaient filles des
deux frères et fort unies, petites-filles de l'électeur palatin, gendre
de Jacques Ier, roi de la Grande-Bretagne, qui\footnote{Le \emph{qui} se
  rapporte à l'électeur palatin.}, pour s'être voulu faire roi de
Bohème, perdit tous ses États et sa dignité électorale, et mourut
proscrit en Hollande. Son fils aîné fut enfin rétabli, mais dernier
électeur, ce que Madame, qui était sa fille, rie pardonna jamais à la
branche de Bavière. Édouard, frère puîné de l'électeur rétabli, épousa
Anne Gonzague, dite Clèves, dont il eut la princesse de Salm, femme du
gouverneur de l'empereur Joseph, et ministre d'État de l'empereur,
Léopold, M\textsuperscript{me} la Princesse, et la duchesse d'Hanovre ou
de Brunswick, mère de l'impératrice Amélie, épouse de l'empereur Joseph.
Cette Anne Gonzague se rendit illustre par son esprit et sa conduite, et
par sa grande cabale pendant les troubles de la minorité du feu roi,
devint jusqu'à sa mort la plus intime et confidente amie du célèbre
prince de Condé, qu'elle servit plus utilement que personne, de sorte
qu'ils marièrent ensemble leurs enfants. Elle était soeur de la reine
Marie\footnote{Marie de Gonzague-Nevers épousa successivement les deux
  frères Wladislas VII et Jean-Casimir, qui régnèrent en Pologne, le
  premier de 1632 à 1648, et le second de 1648 à 1668.}, deux fois reine
de Pologne, aimée et admirée partout par son esprit, ses talents de
gouvernement et tous les agréments possibles, que la reine mère et le
cardinal de Richelieu empêchèrent Monsieur, Gaston, de l'épouser.

M\textsuperscript{me} la Princesse eut des biens immenses. Elle était
laide, bossue, un peu tortue, et sans esprit, mais douée de beaucoup de
vertu, de piété, de douceur et de patience, dont elle eut à faire un
pénible et continuel usage tant que son mariage dura, qui fut plus de
quarante-cinq ans. Devenue veuve, elle bâtit somptueusement le
Petit-Luxembourg, assez vilain jusqu'alors, l'orna et le meubla de
même\,; mais quand on l'allait voir, on entrait par ce qui s'appelle une
montée, dans une vilaine petite salle à manger, au coin de laquelle
était une porte qui donnait dans un magnifique cabinet, au bout de toute
l'enfilade de l'appartement, qu'on ne voyait jamais. Toutes les
cérémonies dues à son rang furent observées au Petit-Luxembourg, où elle
mourut, mais il n'y fut pas question de la garde de son corps par des
dames. Cette entreprise, tentée précédemment, n'avait pu réussir\,; les
princes du sang enfin s'en étaient dépris. Elle fut portée en cérémonie
aux Carmélites de la rue Saint-Jacques, où elle fut enterrée. Caylus,
évêque d'Auxerre, y fit la cérémonie. J'ai rangé ici cette mort pour ne
pas interrompre ce qui va suivre.

La majorité approchait et mettait bien des gens en mouvement. M. le duc
d'Orléans se laissa entendre qu'il pourrait faire duc et pair le marquis
de Biron, son premier écuyer. Cette notion en réveilla d'autres. Le
prince de Talmont, qui à son mariage avait escroqué le tabouret au feu
roi par surprise, et qui ne pouvait espérer de le transmettre à son
fils, n'oublia rien pour être fait duc et pair. Madame et lui étaient
enfants des deux soeurs, titre qui, joint à sa naissance, le lui faisait
espérer de M. le duc d'Orléans\,: toutefois il n'y put réussir. La
princesse de Conti, dont la passion pour l'élévation de La Vallière son
cousin germain, était extrême, se mit à tourmenter M. le duc d'Orléans,
qui, à ce qu'il me dit, avait donné au fils de La Vallière la survivance
de son gouvernement de Bourbonnais pour être quitte avec la princesse de
Conti, et lui fermer la bouche sur toute autre demande, mais il n'eut
pas la force de résister. Je réussis aussi, quoique avec grande peine,
pour le marquis de Lévi, gendre du feu duc de Chevreuse. Ainsi ces trois
furent déclarés en cet ordre\,: Biron, Lévi et La Vallière. Les deux
premiers, \emph{toto coelo} distants du troisième\footnote{Séparés du
  troisième par toute la distance du ciel à la terre.}, avaient eu
chacun un duché-pairie dans sa maison, et Lévi avait vu éteindre celui
de Ventadour depuis peu d'années. À l'égard de celui de Biron, j'admirai
avec indignation l'effronterie et l'impudence avec laquelle la femme de
Biron osait tirer un titre de prétention de l'extinction du duché-pairie
de Biron. Biron et Lévi passèrent sans grand murmure par leur naissance
et leurs services\,; mais La Vallière qu'on aimait d'ailleurs excita les
clameurs publiques, au point que M. le duc d'Orléans en fut honteux.

Le 19 février, le roi reçut à Versailles les respects de M. le duc
d'Orléans et de toute la cour sur sa majorité, et déclara les trois
nouveaux ducs et pairs. Le lendemain il vint en pompe, après dîner, à
Paris aux Tuileries, et le 22 il alla au parlement tenir son lit de
justice pour la déclaration de sa majorité, et y fit recevoir les trois
nouveaux ducs et pairs. La séance finit par l'enregistrement d'un nouvel
édit contre les duels, qui redevenaient communs. Le 23, le roi reçut aux
Tuileries les harangues des compagnies supérieures et autres corps qui
ont accoutumé d'haranguer. Le 24, il alla voir M\textsuperscript{me} la
Duchesse et les deux filles de M\textsuperscript{me} la Princesse, morte
la veille. On vit avec surprise qu'il alla voir aussi la duchesse de
Brunswick, sa soeur. Ses visites s'y bornèrent\,; elles ne s'étendirent
pas jusqu'aux princes et princesses du sang, petits-enfants de
M\textsuperscript{me} la Princesse. Enfin, le 25, il retourna à
Versailles avec la même pompe qu'il en était venu.

Le conseil de régence prit fin. Le conseil d'État ne fut composé que de
M. le duc d'Orléans, M. le duc de Chartres, M. le Duc, du cardinal
Dubois et de Morville, secrétaire d'État jusqu'alors sans fonction, à
qui le cardinal Dubois remit sa charge de secrétaire d'État avec le
département des affaires étrangères. Maurepas, secrétaire d'État,
jusqu'alors sous la tutelle de La Vrillière, son beau-père, commença à
faire sa charge de secrétaire d'État avec le département de la marine.
La Vrillière demeura comme il était sous le feu roi\,; mais il ne remit
qu'un peu après le détail de Paris et de la maison du roi à son gendre,
qui étaient de son département, et Le Blanc demeura secrétaire d'État
avec le département de la guerre pour ne pas y rester longtemps. Le
conseil des finances, les mêmes, excepté Morville, et de plus
Armenonville, garde des sceaux, Dodun, contrôleur général, et les deux
conseillers d'État au conseil royal des finances. Le maréchal de
Villeroy, chef de ce conseil, était exilé à Lyon. Le conseil des
dépêches\footnote{Conseil de l'administration intérieure, voy. t. Ier,
  p.~446.} était composé de M. le duc d'Orléans, des deux princes du
sang, du cardinal Dubois et des quatre secrétaires d'État. Ainsi tout
cet extérieur, aux princes du sang près, reprit tout celui du temps du
feu roi. On consola La Vrillière de son déchet par la survivance de sa
charge de secrétaire d'État à son fils.

Il y avait assez longtemps que le comte de Toulouse avait pris beaucoup
de goût pour la marquise de Gondrin aux eux de Bourbon, où ils s'étaient
rencontrés et fort vus. Elle était soeur du duc de Noailles qu'il
n'aimait ni n'estimait, et veuve avec deux fils du fils aîné de d'Antin,
avec qui il avait toujours eu beaucoup de commerce et de liaisons de
convenance et de bienséance, parce qu'ils étoient tous deux fils de
M\textsuperscript{me} de Montespan. M\textsuperscript{me} de Gondrin
avait été dame du palais sur la fin de la vie de M\textsuperscript{me}
la Dauphine, jeune, gaie et fort Noailles\,; la gorge fort belle, un
visage agréable, et n'avait point fait parler d'elle. L'affaire fut
conduite au mariage dans le dernier secret. Pour le mieux cacher, le
comte de Toulouse prit le moment de la séance du lit de justice de la
majorité, dont il s'excluait, parce que les bâtards ne traversaient plus
le parquet, et à cause de cela n'allaient point au parlement, ni le
cardinal de Noailles non plus à cause de sa pourpre qui y aurait cédé
aux pairs ecclésiastiques. La maréchale de Noailles alla seule avec sa
fille à l'archevêché, où le comte de Toulouse se rendit en même temps
seul avec d'O, où le cardinal de Noailles leur dit la messe et les maria
dans sa chapelle, au sortir de laquelle chacun s'en retourna comme il
était venu. Rien n'en transpira, et on fut longtemps sans en rien
soupçonner, d'autant que le comte de Toulouse avait toujours paru fort
éloigné de se marier.

En ce même temps la peste qui avait si longtemps désolé la Provence y
fut tout à fait éteinte, et tellement que les barrières furent levées,
le commerce rétabli, et les actions de grâces publiquement célébrées
dans toutes les églises du royaume, et au bout de peu de mois le
commerce entièrement rouvert avec tous les pays étrangers.

M\textsuperscript{lle} de Beaujolais fut remise à la frontière par le
duc de Duras, qui commandait la Guyenne et qui en eut la commission, au
duc d'Ossone, qui avait celle du roi d'Espagne pour la recevoir, et qui
commandait le détachement de la maison du roi d'Espagne envoyé au-devant
d'elle. La duchesse de Duras la remit à la comtesse de Lemos, sa
camarera-mayor, dont j'ai parlé plus d'une fois, et dont la complaisance
d'accepter cette place surprit fort toute la cour d'Espagne. Aucun
Français ni Française ne passa en Espagne avec M\textsuperscript{lle} de
Beaujolais. Elle trouva Leurs Majestés Catholiques, le prince et la
princesse des Asturies à Buytrago, à une journée de Madrid, qui lui
présentèrent don Carlos à la descente de son carrosse. Ils allèrent tous
le lendemain à Madrid, où il y eut beaucoup de fêtes. Le chevalier
d'Orléans, grand prieur de France, y était arrivé sept ou huit jours
auparavant, et il fut fait grand d'Espagne. Bientôt après il fit sa
couverture, et s'en revint aussitôt après avoir rempli l'objet de son
voyage. L'électeur de Bavière, qui avait si bien servi les deux
couronnes, et à qui il en avait coûté si cher, crut, sur cet exemple,
pouvoir demander la même grâce au roi d'Espagne, fils de sa soeur, pour
son bâtard le comte de Bavière, qui était dans le service de France.

M. le duc d'Orléans, qui méprisait tout et qui faisait litière de tout,
avait peu à peu accordé à qui avait voulu, sans choix ni distinction
aucune, les grandes entrées chez le roi, aux uns les grandes, les
premières entrées aux autres, et les avait rendus si nombreux que
c'était un peuple dont la foule ôtait toute distinction, et ne pouvait
qu'importuner beaucoup le roi. Le cardinal Dubois, qui ne
buttait\footnote{Qui ne tendait pas moins.} pas moins à se rendre maître
de l'esprit du roi, qu'il avait fait à dominer M. le duc d'Orléans,
voulut éloigner de tout moyen de familiarité avec le roi tous ceux qu'il
pourrait, et se la procurer en même temps tout entière. Il saisit donc
les premiers moments qui suivirent la majorité pour faire aux entrées le
changement qu'il projetait sous prétexte d'y remettre l'ordre et de
soulager le roi d'une foule importune dans les moments de son
particulier. Pour mieux entendre le manége du cardinal Dubois là-dessus,
il faut expliquer auparavant ce que c'était que les entrées chez le feu
roi, l'ordre qui y était observé, et combien elles étaient précieuses et
rares. Je n'ai fait qu'en dire un mot à l'occasion de celles que le feu
roi lui donna\,: les premières à MM. de Charost, père et fils, et les
grandes, longtemps depuis, aux maréchaux de Boufflers et de Villars.

Il y avait chez le feu roi trois sortes d'entrées fort distinguées, deux
autres fort agréables, une dernière qui était comme entre les mains du
premier gentilhomme de la chambre en année. La première sorte s'appelait
les grandes entrées. Les charges qui les donnaient sont celles de grand
chambellan, des quatre premiers gentilshommes de la chambre en année ou
non, de grand maître de la garde-robe et du maître de la garde-robe en
année. Sans charge elles furent toujours très rares, et une grande
récompense ou un grand effet de faveur\,; je ne les ai vues qu'aux
bâtards et aux maris des bâtardes, même des filles des bâtardes. De gens
de la cour, le duc de Montausier pour avoir été gouverneur de
Monseigneur, le premier maréchal de La Feuillade et le duc de Lauzun,
qui en a joui seul sans charge bien des années jusqu'à la mort du roi.
L'autre sorte d'entrées n'était que par les derrières. Ceux qui les
avaient n'entraient jamais par devant, ni n'en jouissaient dans la
chambre du roi à son lever, à son coucher\,; ou quand ils y voulaient
venir, ils n'entraient qu'avec toute la cour. Ils venaient donc par le
petit degré de derrière qui donnait dans les cabinets du roi, ou par les
portes de derrière des cabinets qui donnaient dans la galerie ou dans le
grand appartement, et entraient ainsi sans être vus dans les cabinets du
roi à toutes heures, hors celles du conseil, ou d'un travail particulier
du roi avec un de ses ministres. C'est ce que n'avaient point les
grandes entrées ni aucune autre. Celles de derrière se trouvaient quand
bon leur semblait dans le cabinet du roi après le lever, où, pendant un
quart d'heure et plus, le roi donnait l'ordre de sa journée, parmi tous
ceux qui avaient des entrées\,; mais l'ordre donné, tout sortait du
cabinet, excepté les entrées des derrières qui demeuraient jusqu'à la
messe, et cela était souvent assez long.

Les soirs, entre le souper et le coucher du roi, ces entrées de derrière
avaient la liberté d'être dans le cabinet où le roi se tenait avec ses
bâtards, ses bâtardes et leurs enfants ou gendres, ou Monseigneur, les
fils de France, M\textsuperscript{me}s les duchesses de Bourgogne et de
Berry\,; et après la mort de M\textsuperscript{me} la duchesse de
Bourgogne, devenue Dauphine, Madame fut enfin admise. Ceux qui avaient
ces entrées étaient les fils de France, les princesses qui viennent
d'être nommées et qui entraient par devant avec le roi. Tout le reste
entrait et sortait par derrière\,: c'étaient les bâtards, les bâtardes,
leurs gendres, petits-gendres et leurs enfants et petits-enfants. À
cette entrée d'après souper M. le Duc, gendre du roi, et M. le prince de
Conti, gendre de M\textsuperscript{me} la Duchesse, et qui ne l'avaient
eue que comme tels à leur mariage, entraient et sortaient seuls par
devant avec le roi. Le reste de ceux qui avaient ces entrées de derrière
ne les avaient que par leurs emplois. C'étaient Mansart, puis d'Antin
qui avaient les bâtiments, Montchevreuil et d'O, comme ayant été
gouverneurs des deux bâtards\,: Chamarande, qui avait eu la survivance,
de son père, de premier valet de chambre. Le reste n'était que des
principaux valets, lesquels avaient aussi les grandes entrées. Ce qui
distinguait ces grandes entrées des premières entrées était le premier
petit lever où les grandes entrées voyaient le roi au lit et sortir de
son lit, avaient toutes les autres entrées excepté celles de derrière,
mais pouvaient aussi entrer à toute heure dans le cabinet du roi, quand
il n'y avait point de travail de ministre, lorsqu'ils avaient quelque
chose à dire au roi de pressé, ce qui n'était pas permis à d'autres. Les
premières entrées avaient, exclusivement aux entrées inférieures, un
second petit lever fort court, et le petit coucher auquel il n'y avait
point de différence ides grandes entrées à celles-ci, qui en sortaient
ensemble.

Longtemps avant la mort du roi, à l'occasion d'une longue goutte qu'il
avait eue, il avait supprimé le grand coucher, c'est-à-dire, que la cour
ne le voyait plus depuis la sortie de son souper. Ainsi tout le coucher
était devenu petit coucher réservé aux grandes entrées et aux premières.
Quand le roi était incommodé, ces grandes entrées avaient leurs
privances et leurs distinctions au-dessus des premières, comme celles-ci
en avaient au-dessus des entrées inférieures, qui en avaient aussi, mais
peu perceptibles sur le reste de la cour. Dans ces cas d'incommodité,
les entrées des derrières entraient par les derrières dans les cabinets,
et de là dans la chambre du roi, en de certains moments rompus, et en
sortaient de même. Ceux qui avaient les premières entrées que j'ai vus,
étaient le maître de la garde-robe qui n'était point en année, le
précepteur et les sous-gouverneurs de Monseigneur et des princes ses
fils, ou qui l'avaient été. Il n'y avait que ceux-là par charge. Des
autres, M. le Prince qui les avait eues seulement au mariage de M. son
fils avec la fille aînée du roi et de M\textsuperscript{me} de
Montespan, le maréchal de Villeroy, comme fils du gouverneur du roi, le
duc de Béthune, lorsqu'il quitta sa compagnie des gardes du corps,
Beringhen, premier écuyer, Tilladet, parce qu'il avait été maître de la
garde-robe avant d'avoir eu les Cent-Suisses\,; enfin, les deux lecteurs
du roi, que je ne compte pas, quoique par charge, parce qu'elles n'ont
rien que ces premières entrées qui les fasse compter pour quelque chose,
et qu'excepté Dangeau qui en acheta une uniquement pour avoir ces
entrées, et qui perça, tous les autres ont été des gens de fort peu de
chose. Viennent après les entrées de la chambre et celles du cabinet.
Toutes les charges chez le roi ont ces deux entrées, et tous les princes
du sang comme tels, ainsi que les cardinaux. Fort peu d'autres gens de
la cour sans charges les pont obtenues.

Celles de la chambre consistent à entrer au lever du roi un moment avant
le reste de la cour, quelquefois pour un instant, quand le roi prenait
un bouillon les jours de médecine, ou de quelque légère incommodité,
privativement au reste de la cour.

Celles du cabinet, qui appartiennent aux charges principales et
secondes, et à fort peu d'autres courtisans, mais aussi aux princes du
sang et aux cardinaux, n'étaient que pour entrer après le lever dans le
cabinet du roi à l'heure qu'il donnait l'ordre pour la journée, et rien
plus.

Enfin la dernière entrée, dont le premier gentilhomme de la chambre en
année disposait, était lorsque le roi allant à la chasse ou se promener,
venait prendre une chaussure et un surtout. L'huissier allait nommer au
premier gentilhomme de la chambre en année les personnes de quelque
distinction qui étaient à la porte et qui désiraient entrer. Le premier
gentilhomme de la chambre ne nommait au roi que celles qu'il voulait
favoriser, qu'il faisait entrer, et de même au retour du roi. C'est ce
qui s'appelait le botter et le débotter. À Marly y entrait qui voulait
indépendamment du premier gentilhomme de la chambre, mais non ailleurs.

On voit ainsi l'ordre de toutes ces entrées, et combien précieuses et
rares étaient les grandes et celles des derrières, même les premières
entrées qui donnaient lieu à faire une cour facile et distinguée, et à
parler au roi à son aise et sans témoins, car les gens de ces entrées
s'écartaient dès que l'un d'eux s'approchait pour parler au roi, qui
était si difficile à accorder des audiences au reste de sa cour.

Le cardinal Dubois, dans son nouveau projet, commença par faire rendre
les brevets des grandes et des premières entrées à ceux qui en avaient
obtenu. Il n'en excepta que le maréchal de Berwick pour les grandes,
qu'il ménageait, pour l'éloigner en lui faisant accepter l'ambassade
d'Espagne, et Belle-Ile pour les premières, qu'il voulait tromper
jusqu'au bout pour le perdre avec Le Blanc, et il fut la dupe de l'un et
de l'autre. Berwick ne fut point en Espagne. Belle-Ile, après un long et
dur séjour à la Bastille, puis en exil à Nevers, revint à la cour faire
la plus prodigieuse fortune, et tous deux conservèrent leurs entrées.
Tous les autres les perdirent, hors le très peu de ceux qui restaient et
qui les avaient du feu roi. Je fus du nombre des supprimés, et M. le duc
d'Orléans le souffrit. Je renvoyai mon brevet dès qu'il me fut
redemandé, sans daigner m'en plaindre, ni en dire un mot au cardinal
Dubois, ni à M. le duc d'Orléans que j'aurais fort embarrassé. Les
entrées, excepté ces deux, demeurèrent donc restreintes aux charges et à
ce si peu d'autres qui les avaient du feu roi. Celles des derrières
furent abolies, en donnant les grandes à d'Antin, à d'O et à Chamarande.
Le cardinal Dubois en inventa de familières qui, du temps du feu roi,
n'étaient que pour Monseigneur et les princes ses fils, Monsieur et M.
le duc d'Orléans, le duc du Maine et le comte de Toulouse. Dubois les
prit pour lui, et, pour faire moins crier, les étendit à tous les
princes du sang, au duc du Maine, à ses deux fils et au comte de
Toulouse. Elles donnèrent droit d'entrée à toute heure où était le roi
quand il ne travaillait pas. Les princes du sang s'en trouvèrent
extrêmement flattés, eux qui n'avaient que celles de la chambre. Jamais
le feu prince de Conti n'en avait eu d'autres avec celles du cabinet. Et
avant que le coucher du roi eût été retranché aux courtisans, j'ai vu
bien des fois M. le Prince assis au-dehors de la porte du cabinet du
roi, entre le souper et le coucher, et assis qui pouvait dans la même
pièce que lui, en attendant le coucher du roi, tandis qu'en sa présence
M. le Duc son fils, comme gendre du roi, entrait dans le cabinet, et
n'en sortait qu'avec le roi, quand il venait se déshabiller pour son
coucher. Ces entrées familières sont demeurées aux princes du sang et
aux bâtards et batardeaux, et il ne sera pas facile désormais de les
leur ôter par un roi qu'une familiarité si grande pourra facilement
gêner et importuner beaucoup.

Tel fut le préparatif du rétablissement des bâtards et des enfants du
duc du Maine dans tous les rangs, honneurs et distinctions dont ils
jouissaient à la mort du roi. C'est ce qui fut fait par une déclaration
du roi enregistrée au parlement, qui n'excepta que le droit de
succession à la couronne, le nom et le titre de prince du sang qui leur
fut de nouveau interdit, et le traversement du parquet, en sorte que
d'ailleurs ils conservèrent en tout et partout l'extérieur de princes du
sang, et en eurent aussi les mêmes entrées. C'était, ce semble, de quoi
être plus que contents, après la dégradation qu'ils avaient, à tous
égards, si justement essuyée. Ils ne le parurent point du tout, et
M\textsuperscript{me} la duchesse d'Orléans encore moins qu'eux. Ils ne
prétendaient à rien moins qu'aux trois points qu'ils tâchèrent d'obtenir
par toutes sortes d'efforts, et à un quatrième qui était une extension
illimitée à leur postérité. Dubois, qui n'osa choquer les princes du
sang en des points si sensibles, n'osa les accorder. Son but était de se
mettre bien avec les uns et les autres, et de les tenir ennemis pour les
opposer et nager ainsi entre eux, appuyé selon l'occasion de ceux qui
lui seraient les plus utiles, en faisant pencher la balance de leur
côté. Nous fîmes nos protestations, dernière ressource des opprimés. Cet
événement acheva de m'éloigner du cardinal et de M. le duc d'Orléans,
auxquels, comme chose très inutile, je ne pris pas la peine d'en dire
une seule parole. Personne de nous ne visita les bâtards sur ce
rétablissement si honteux et si fort à pure perte pour M. le duc
d'Orléans, après tout ce qui s'était passé.

En même temps le cardinal Dubois négociait avec le P. Daubenton, non
seulement le retour des bonnes grâces du roi d'Espagne au maréchal de
Berwick, mais l'agrément de Sa Majesté Catholique pour qu'il allât,
ambassadeur du roi à Madrid. L'impossibilité du succès de cette
entreprise, dont il ne m'avait confié que la moitié, ne l'avait pas
rebuté, quoique je la lui eusse bien clairement exposée, tant il était
pressé de se défaire de ce duc, dont l'estime, l'amitié, la familiarité
pour lui de M. le duc d'Orléans lui était si importune, et duquel il ne
pouvait se délivrer autrement. À l'occasion de la négociation du futur
mariage de M\textsuperscript{lle} de Beaujolais, il avait promis une
grosse abbaye à un frère que le P. Daubenton avait à Paris. Cette abbaye
ne venait point, le cardinal en suspendait le don pour hâter le jésuite
d'obtenir du roi d'Espagne ce qu'il avait si fort à coeur, et payait, en
attendant, son frère d'espérances les plus prochaines. La négociation ne
fut pas longue, le P. Daubenton manda nettement au cardinal qu'il
n'avait pu y réussir, et qu'il n'avait jamais trouvé dans le roi
d'Espagne une inflexibilité si dure ni si arrêtée. Le cardinal entra en
furie, dans le dépit de ne savoir plus comment pouvoir éloigner le duc
de Berwick. Le frère du P. Daubenton se présenta à lui pour insister sur
l'abbaye promise\,; le cardinal l'envoya très salement promener, le
traita comme un nègre, lui chanta pouille du P. Daubenton, lui déclara
qu'il n'avait plus d'abbaye à espérer, lui défendit d'oser jamais
paraître devant lui, et rompit tout commerce avec le P. Daubenton pour
tout le reste de sa vie. On peut juger de l'effet de cette sortie sur un
jésuite accoutumé aux adorations des ministres des plus grandes
puissances, et aux ménagements directs de ces mêmes puissances. On en
verra bientôt les funestes effets.

Je n'ai point su par quelle heureuse fantaisie, car le cardinal Dubois
n'était rien moins que noble et bienfaisant, il avait pris en gré, du
temps de la splendeur de Law, le vieux prince de Courtenay, qui n'avait
pas de quoi vivre. Il lui avait procuré le payement de ses dettes, et
plus de quarante mille livres de rentes au delà. Il n'en jouit que
quelques années\,; il mourut à quatre-vingt-trois ans, en ce temps-ci,
et laissa ce bien à son fils unique qu'il avait eu de {[}Marie{]} de
Lamet\,; il avait eu un aîné tué à vingt-deux ans, sans alliance, étant
mousquetaire au siège de Mons, comme il a été dit ici ailleurs. M. de
Courtenay, après douze ans de veuvage, se remaria, en 1688, à la fille
de Besançon, qu'on appelait M. Duplessis-Besançon, lieutenant général et
gouverneur d'Auxonne, laquelle était veuve de M. Le Brun, président au
grand conseil, dont il laissa une fille mariée au marquis de
Beauffremont en 1712. On a vu ailleurs comment ce prince de Courtenay
perdit la fortune que le cardinal Mazarin avait résolu de lui faire, en
lui donnant une de ses nièces en mariage, et le faisant déclarer prince
du sang. On y a vu aussi ce qu'est devenu son fils, en qui toute cette
maison de Courtenay s'est éteinte, vraiment et légitimement de la maison
royale, sans en avoir jamais pu être reconnu, quoiqu'elle n'en doutât
pas, ni le feu roi non plus.

Fort tôt après la formation des conseils d'État, des finances et des
dépêches, le cardinal Dubois ôta le détail de l'infanterie, de la
cavalerie et des dragons à M. le duc de Chartres, au comte d'Évreux et à
Coigny colonels généraux, et le rendit aux départements du secrétaire
d'État de la guerre. Le comte de Toulouse retint encore quelque peu de
temps celui de la marine\,; mais il le perdit enfin à très peu de chose
près, comme les autres, et le vit passer au secrétaire d'État de la
marine. Pour les Suisses et l'artillerie, tout fut rendu à cet égard, à
peu de chose près, au duc du Maine, comme il l'avait du temps du feu
roi, mais en allant travailler chez le cardinal Dubois sur ces deux
matières.

Maulevrier revint en ce temps-ci d'Espagne, et fut médiocrement reçu. Il
s'en alla tôt après montrer sa Toison dans sa province. Je n'entendis
point parler de lui ni lui de moi, et n'en avons pas ouï parler depuis.
Qui lui aurait dit alors qu'il deviendrait maréchal de France, il en
aurait été pour le moins aussi étonné que le monde le fut quand le bâton
lui fut donné. Chavigny demeura en Espagne sans titre, mais chargé des
affaires en attendant un ambassadeur.

Un autre Maulevrier, mais qui était Colbert, et petit-fils du maréchal
de Tessé, épousa une fille du comte d'Estaing, et le comte de Peyre une
fille de Gassion, petite-fille du garde des sceaux Armenonville.

La princesse de Piémont mourut en couche à Turin, au bout d'un an de
mariage. Elle n'avait pas vingt ans, et était fort belle. Elle était
Palatine-Soultzbach.

Le duc d'Aumont, chevalier de l'ordre, mourut le 6 avril d'apoplexie, à
cinquante-six ans. Il en a été assez parlé ici, suffisamment ailleurs,
pour n'avoir plus rien à en dire. Son fils avait la survivance de sa
charge et de son gouvernement. Beringhen, son beau-frère, ne le survécut
pas d'un mois après une longue maladie. Il était premier écuyer du roi,
et chevalier de l'ordre, et avait soixante et onze ans\,: homme
d'honneur, de fort peu d'esprit, aimé et compté à la cour, estimé et
fort bien avec le feu roi. Son fils avait la survivance de sa charge et
de son petit gouvernement.

La marquise d'Alègre, dont j'ai eu occasion de parler ici quelquefois,
mourut à soixante-cinq ans\,; dévote fort singulière, qui n'était pas
sans esprit et sans vues. Elle avait été belle, on s'en apercevait
encore. On a vu que ce fut elle qui me donna le premier éveil de toute
la conspiration du duc et de la duchesse du Maine, sans rien nommer,
dont son mari était tout du long, qui était fort bête et qui ne s'en
doutait pas.

Deux soeurs du duc de Noailles moururent à un mois l'une de l'autre\,;
M\textsuperscript{me} de Châteaurenaud à trente-quatre ans, et
M\textsuperscript{me} de Coëtquen à quarante-deux ans. On n'avait jamais
fait grand cas de l'une ni de l'autre dans leur famille, ni dans celle
de leurs maris, ni dans le monde.

Le fils aîné du duc de Lorraine mourut de la petite vérole à dix-sept
ans.

Le cardinal Dubois, que l'assemblée du clergé avait élu son premier
président, et qui en fut fort flatté, suivait chaudement l'affaire de La
Jonchère pour perdre Le Blanc qu'il y fit impliquer.
M\textsuperscript{me} de Prie et M. le Duc ne s'y épargnèrent pas. Ce
trésorier avait été mis à la Bastille et fort resserré, où il dit et fit
à peu près ce qu'on voulut. Ainsi, toute l'affection, la confiance, tous
les services publics et secrets que M. le duc d'Orléans avait reçus de
Le Blanc ne purent tenir contre l'impétuosité de M. le Duc et du
cardinal Dubois. Le Blanc eut ordre de donner la démission de sa charge
de secrétaire d'État et de s'en aller sur-le-champ à quinze ou vingt
lieues de Paris, à Doux, terre de Tresnel, son gendre, et sur-le-champ
Breteuil, intendant de Limoges, fut fait secrétaire d'État de la guerre
en sa place.

Cet événement affligea tout le monde. Jamais Le Blanc ne s'était
méconnu. Il était poli jusque avec les moindres, respectueux\,: où il le
devait et où ces messieurs ne le sont guère, obligeant et serviable à
tous, gracieux et payant de raison jusque dans ses refus, expéditif,
diligent, clairvoyant, travailleur fort capable\,; connaissant bien tous
les officiers et tous ceux qui étoient sous sa charge. On peut dire que
ce fut un cri et un deuil public sans ménagement, quoiqu'on sentit
depuis quelque temps que la partie en était faite. Mais la surprise ne
fut pas moins grande et générale de voir Breteuil en sa place, et être
tiré pour cela d'une des dernières et des plus chétives intendances du
royaume, dans un âge qui était encore fort peu avancé, sans avoir jamais
vu ni ouï parler de troupes, de places ni de rien de ce qui appartient à
la guerre, qui n'avait jamais eu ni travail ni application, et qui était
de ces petits-maîtres étourdis de robe, qui ne s'occupait que de son
plaisir. La cause longtemps secrète d'une telle fortune fut précisément
le hasard de sa petite intendance.

Le cardinal Dubois était marié depuis longues années, par conséquent
fort obscurément. Il paya bien sa femme pour se taire quand il eut des
bénéfices\,; mais quand il pointa au grand il s'en trouva fort
embarrassé. Sa bassesse ne lui laissait que les élévations
ecclésiastiques, et il était toujours dans les transes que sa femme ne
l'y fît échouer. Son mariage s'était fait dans le Limousin et célébré
dans une paroisse de village. Nommé à l'archevêché de Cambrai, il prit
le parti d'en faire la confidence à Breteuil et de le conjurer de
n'oublier rien pour enlever les preuves de son mariage avec adresse et
sans bruit.

Dans la posture où Dubois était déjà, Breteuil vit les cieux ouverts
pour lui s'il pouvait réussir à lui rendre un service si délicat et si
important. Il avait de l'esprit et il sut s'en servir. Il s'en retourna
diligemment à Limoges, et, tôt après, sous prétexte d'une légère tournée
pour quelque affaire subite, il s'en alla, suivi de deux ou trois valets
seulement, ajustant son voyage de façon qu'il tomba à une heure de nuit,
dans ce village où le mariage avait été célébré, alla descendre chez le
curé faute d'hôtellerie, lui demanda familièrement la passade comme un
homme que la nuit avait surpris, qui mourait de faim et de soif et qui
ne pouvait aller plus loin. Le bon curé, transporté d'aise d'héberger M.
l'intendant, prépara à la hâte tout ce qu'il put trouver chez lui, et
eut l'honneur de souper tête à tête avec lui, tandis que sa servante
régala les deux valets dont Breteuil se défit ainsi que de la servante
pour demeurer seul avec le curé. Breteuil aimait à boire et y était
expert. Il fit semblant de trouver le souper bon et le vin encore
meilleur. Le curé, charmé de son hôte, ne songea qu'à le reforcer, comme
on dit dans la province\,; le broc était sur la table\,; ils s'en
versaient tour à tour avec une familiarité qui transportait le bon curé.
Breteuil, qui avait son projet, en vint à bout, et enivra le bonhomme à
ne pouvoir se soutenir, ni voir, ni proférer un mot. Quand Breteuil eut,
en cet état, achevé de le bien noyer avec quelques nouvelles lampées, il
profita de ce qu'il en avait tiré dans le premier quart d'heure du
souper. Il lui avait demandé si ses registres étaient en bon ordre, et
depuis quel temps, et sous prétexte de sûreté contre les voleurs, où il
les tenait et où il en gardait les clefs, tellement que dès que Breteuil
se fut bien assuré que le curé ne pouvait plus faire usage d'aucun de
ses sens, il prit ses clefs, ouvrit l'armoire, en tira le registre des
mariages qui contenait l'année dont il avait besoin, en détacha bien
proprement la feuille qu'il cherchait, et malheur aux autres mariages
qui se trouvèrent sur la même feuille, la mit dans sa poche, et rétablit
le registre où il l'avait trouvé, referma l'armoire et remit les clefs
où il les avait prises. Il ne songea plus après ce coup qu'à attendre le
crépuscule du matin pour s'en aller\,; laissa le bon curé cuvant
profondément son vin, et donna quelques pistoles à la servante.

Il s'en alla de là à Brive, chez le notaire, dont il s'était bien
informé, qui avait l'étude et les papiers de celui qui avait fait le
contrat de mariage, s'y enferma avec lui, et de force et d'autorité se
fit remettre la minute du contrat de mariage. Il manda ensuite la femme,
des mains de qui l'abbé Dubois avait su tirer l'expédition de leur
contrat de mariage, la menaça des plus profonds cachots si elle osait
dire jamais une parole de son mariage, et lui promit monts et merveilles
en se taisant. Il l'assura de plus que tout ce qu'elle pourrait dire et
faire serait en pure perte, parce qu'on avait mis ordre à ce qu'elle ne
pût rien prouver, et à se mettre en état, si elle osait branler, de la
faire condamner de calomnie et d'imposture, et la faire raser et pourrir
dans la prison d'un couvent. Breteuil remit les deux importantes pièces
à Dubois, qui l'en récompensa de la charge de secrétaire d'État quelque
temps après.

La femme n'osa souffler. Elle vint à Paris après la mort de son mari. On
lui donna gros sur ce qu'il laissait d'immense. Elle a vécu obscure,
mais fort à son aise, et est morte à Paris plus de vingt ans après le
cardinal Dubois, dont elle n'avait point eu d'enfants. Dubois à qui le
cardinal son frère avait donné sa charge de secrétaire du cabinet du
roi, et la charge des ponts et chaussées qu'avait le feu premier écuyer,
et qui était bon et honnête homme, vécut toujours fort bien avec elle.
Il était assez mauvais médecin de village dans son pays, lorsque son
frère le fit venir à Paris quand il fut secrétaire d'État. Dans la
suite, cette histoire a été sue, et n'a été désavouée ni contredite de
personne.

\hypertarget{chapitre-xix.}{%
\chapter{CHAPITRE XIX.}\label{chapitre-xix.}}

1723

~

{\textsc{Bâtards de Montbéliard.}} {\textsc{- Mezzabarba, légat \emph{a
latere} à la Chine, en arrive à Rome avec le corps du cardinal de
Tournon, et le jésuite portugais Magalhaens.}} {\textsc{- Succès de son
voyage et de son retour.}} {\textsc{- Le roi à Meudon pour la convenance
du cardinal Dubois, dont la santé commence visiblement à s'affaiblir.}}
{\textsc{- Belle-Ile, Conches et Séchelles interrogés.}} {\textsc{- La
Vrillière travaille à se faire duc et pair par une singulière
intrigue.}} {\textsc{- Mort du marquis de Bedmar à Madrid.}} {\textsc{-
Maréchal de Villars grand d'Espagne.}}

~

Ce fut dans ce temps-ci que le conseil aulique jugea à Vienne un procès
dont je ne parle ici que par les efforts qui ont été faits vingt ans
depuis pour revenir à cette affaire par la protection du roi et par la
juridiction du parlement de Paris. Le dernier duc de Montbéliard avait
passé sa vie avec un sérail, et n'avait point laissé d'enfants
légitimes. Entre autres bâtards, il en laissa de deux femmes
différentes, nés pendant la vie de son épouse légitime. Mais il
prétendit les avoir épousées avec la permission de son consistoire, et
les fit considérer comme telles dans son petit État. Toutes les
faussetés et toutes les friponneries les plus redoublées et les plus
entortillées furent employées pour soutenir la validité de ces prétendus
mariages, et pour rendre légitimes, par conséquent, les Sponeck, sortis
de l'une, et les Lespérance, sortis de l'autre. Il fit mieux encore, car
pour mettre ces bâtards d'accord, qui se disputaient le droit à
l'héritage, il maria le frère et la soeur qu'il avait eus de ces deux
différentes maîtresses. Il donna sa prédilection à ces nouveaux mariés,
leur assurant, autant qu'il fut en lui, sa succession\,; les fit
reconnaître à Montbéliard comme les souverains futurs, et mourut bientôt
après, leur laissant beaucoup d'argent comptant et de pierreries.
Sponeck et sa femme se firent prêter serment et reconnaître souverains
par leurs nouveaux sujets, et se mirent en possession de tout le petit
État de Montbéliard. Le duc de Würtemberg, à qui il revenait, faute
d'héritier légitime, les y troubla et s'adressa à l'empereur. Le Sponeck
soutint son prétendu droit, et les Lespérance intervinrent, prétendant
exclure le Sponeck et être seuls légitimes héritiers.

Après bien des débats, les uns et les autres furent déclarés bâtards,
avec défense de porter le nom et les armes de Würtemberg et le titre de
Montbéliard\,; les sujets de ce petit État déliés du serment qu'ils
avaient prêté au Sponeck, obligés à la prêter au duc de Würtemberg
envoyé en possession de tout le Montbéliard\,; et les lettres écrites
par les Sponeck à l'empereur, renvoyées au Sponeck avec les armes de son
cachet et sa signature biffées. Ils intriguèrent pour une révision, et y
furent encore plus maltraités. Le voisinage de ce petit État de
Montbéliard, qui confine à la Franche-Comté, leur fit implorer la
protection du roi pour s'y maintenir. Ils trouvèrent
M\textsuperscript{me} de Carignan, qui disposait fort alors de notre
ministère, laquelle, pour de l'argent, entreprenait tout ce qu'on lui
proposait. Elle les fit écouter, et, contre toute apparence de raison,
renvoyer au parlement de Paris. M. de Würtemberg cria, on le laissa
dire, et la poursuite et l'instruction ne s'en continuèrent pas moins. À
la fin, l'empereur se plaignit, et demanda de quel droit le roi pouvait
prétendre se mêler des affaires domestiques de l'Empire, et quelle
juridiction pouvait avoir le parlement de Paris sur l'État d'un Allemand
naturel, qui se prétendait prince de l'Empire, et dont le procès avait
été jugé par le conseil aulique, tribunal de l'Empire, qui n'en
connaissait point de supérieur à soi, beaucoup moins un tribunal
étranger à l'Empire, tel que le parlement de Paris.

On essaya d'amuser l'empereur, mais il se fâcha si bien qu'on n'osa
passer outre, et le parlement cessa d'y travailler. La chute du garde
des sceaux Chauvelin, et d'autres circonstances qui décréditèrent
M\textsuperscript{me} de Carignan, fit dormir cette affaire. Sponeck et
sa femme, prouvée aussi sa soeur, s'étaient faits catholiques pour
s'acquérir les prêtres et les dévots\,; ils ne bougeaient de
Saint-Sulpice, des jésuites et de tous les lieux de piété en faveur.
C'étaient des saints, malgré l'inceste et le bien d'autrui qu'ils
voulaient s'approprier comme que ce fût. Mais il fallait une grande
protection pour remettre leur affaire en train. Ils la trouvèrent dans
la maison de Rohan, qui avisa qu'en leur faisant gagner leur procès ils
deviendraient conséquemment princes de la maison de Würtemberg, et
qu'ils se déferaient pour rien d'une de leurs filles en la mariant au
fils de cet inceste, en lui obtenant ici le rang de prince étranger. Ils
y mirent tout leur crédit, et parvinrent à leur faire accorder des
commissaires. Tous ces manéges eurent beaucoup de haut et de bas\,; les
commissaires travaillèrent.

Cependant le duc de Würtemberg jeta les hauts cris, l'empereur se fâcha
de nouveau, l'affaire au fond et en la forme était insoutenable\,; on ne
voulut pas se brouiller avec l'empereur pour cette absurdité où le roi
n'avait pas le plus petit intérêt d'État. Ils furent donc condamnés
comme ils l'avaient été à Vienne, avec les mêmes clauses et défenses\,;
et ils furent réduits à obtenir du duc de Würtemberg, au désir des
arrêts du conseil aulique, une légère subsistance comme à des bâtards
qu'il faut nourrir, et eux et les Lespérance, et le roi s'entremit
auprès du duc de Würtemberg pour leur faire donner quelques terres les
plus proches de la Franche-Comté. La douleur des vaincus fut grande, et
celle de leurs protecteurs. Le Sponeck se rompit bientôt le cou en
allant à Versailles, sa femme alla loger chez M\textsuperscript{me} de
Carignan\,; et jusqu'à l'heure que j'écris, a l'audace, malgré tant
d'arrêts, de porter tout publiquement le nom de princesse de
Montbéliard, les armes de Würtemberg pleines à son carrosse, et se
montre ainsi effrontément partout, avec deux tétons gros comme des
timbales, et qui, avec sa dévotion, sont médiocrement couverts. Elle n'a
qu'un fils qui, ne pouvant s'accommoder d'un état si bizarre et si
différent de celui qu'il avait prétendu, s'est retiré dans une
communauté. J'ai poussé ce récit fort au delà des bornes de ces
Mémoires, pour montrer quel bon pays est la France à tous les escrocs,
les aventuriers et les fripons, et jusqu'à quel excès l'impudence y
triomphe.

En voici une autre d'une espèce différente. Le feu pape, irrité de la
désobéissance des jésuites de la Chine, des souffrances et de la mort du
cardinal de Tournon qu'il y avait envoyé son légat \emph{a latere}, y
avait envoyé de nouveau, avec le même caractère et les mêmes pouvoirs,
le prélat Mezzabarba, orné du titre de patriarche d'Alexandrie. Il alla
de Rome à Lisbonne pour y prendre les ordres et les recommandations du
roi de Portugal, pour ne pas dire son attache, sous la protection duquel
les jésuites travaillaient dans ces missions des extrémités de l'orient.
Il fit voile de Lisbonne pour Macao où il fut retenu longtemps avec de
grands respects avant de pouvoir passer à Canton. De Canton, il voulut
aller à Pékin, mais il fallut auparavant s'expliquer avec les jésuites
qui étaient les maîtres de la permission de l'empereur de la Chine, et
qui ne la lui voulurent procurer qu'à bon escient. Il différa tant qu'il
put à s'expliquer, mais il eut affaire à des gens qui en savaient autant
que lui en finesses, et qui pouvaient tout, et lui rien que par eux.
Après bien des ruses employées d'une part pour cacher, de l'autre pour
découvrir, les jésuites en soupçonnèrent assez pour lui fermer tous les
passages.

Mezzabarba avait tout pouvoir\,; mais pour faire exécuter à la lettre
les décrets et les bulles qui condamnaient la conduite des jésuites sur
les rits chinois, et pour prendre toutes les plus juridiques
informations sur ce qui s'était passé entre eux et le cardinal de
Tournon jusqu'à sa mort inclusivement. Ce n'était pas là le compte des
jésuites. Ils n'avaient garde de laisser porter une telle lumière sur
leur conduite avec le précédent légat, encore moins sur la prison où ils
l'avaient enfermé à Canton à son retour de Pékin, et infiniment moins
sur sa mort. Mezzabarba, en attendant la permission de l'empereur de la
Chine pour se rendre à Pékin, voulut commencer à s'informer de ces
derniers faits, et de quelle façon les jésuites se conduisaient à
l'égard des rits chinois depuis les condamnations de Rome. Il n'alla pas
loin là-dessus sans être arrêté. La soumission apparente et les
difficultés de rendre à ces brefs l'obéissance désirée furent d'abord
employées, puis les négociations tentées pour empêcher le légat de
continuer ses informations, et pour le porter à céder à des nécessités
locales inconnues à Rome, et qui ne pouvaient permettre l'exécution des
bulles et des décrets qui les condamnaient. Les promesses de faciliter
son voyage à la cour de l'empereur, et d'y être traité avec les plus
grandes distinctions, furent déployées. On lui fit sentir que le succès
de ce voyage, et le voyage même était entre leurs mains. Mais rien de ce
qui était proposé au légat n'était entre les siennes. Il n'avait de
pouvoir que pour les faire obéir, et il avait les mains liées sur toute
espèce de composition et de suspension. Il en fallut enfin venir à cet
aveu. Les jésuites, hors de toute espérance de retourner cette légation
suivant leurs vues, essayèrent d'un autre moyen. Ce fut de resserrer le
légat et de l'effrayer. Ce moyen eut un plein effet.

Le patriarche, se voyant au même lieu où le cardinal de Tournon avait
cruellement péri entre les mains des mêmes qui lui en montraient de près
la perspective, lâcha pied, et pour sauver sa vie et assurer son retour
en Europe, consentit, non seulement à n'exécuter aucun des ordres dont
il était chargé, et dont l'exécution, qu'il vit absolument impossible,
faisait tout l'objet de sa légation, mais encore d'accorder, contre ses
ordres exprès, par conséquent sans pouvoir, un décret qui suspendit
toute exécution de ceux de Rome, jusqu'à ce que le saint-siège eût été
informé de nouveau. De là, les jésuites prirent occasion d'envoyer avec
lui à Rome le P. Magalhaens, jésuite portugais, pour faire au pape des
représentations nouvelles, en même temps pour être le surveillant du
légat depuis Canton jusqu'à Rome. À ces conditions les jésuites
permirent au légat d'embarquer avec lui le corps du cardinal de Tournon,
et de se sauver ainsi de leurs mains sans avoir passé Canton, et sans y
avoir eu, lors même de sa plus grande liberté, qu'une liberté fort
veillée et fort contrainte. Il débarqua à Lisbonne où, après être
demeuré quelque temps, il arriva en celui-ci à Rome avec le jésuite
Magalhaens et le corps du cardinal de Tournon qui fut déposé à la
Propagande. Mezzabarba y rendit compte de son voyage, et eut plusieurs
longues audiences du pape, où il exposa l'impossibilité qu'il avait
rencontrée à son voyage au delà de Canton, premier port de la Chine à
notre égard, et à réduire les jésuites à aucune obéissance. Il expliqua
ce que, dans le resserrement où ils l'avaient tenu, il avait pu
apprendre de leur conduite, du sort du cardinal de Tournon, enfin du
triste état des missions dans la Chine\,; il ajouta le récit de ses
souffrances, de ses frayeurs\,; et il expliqua comment, en s'opiniâtrant
à l'exécution de ses ordres, il n'y aurait rien avancé que de causer
l'éclat d'une désobéissance nouvelle, et à soi la perte entière de sa
liberté, et vraisemblablement de sa vie, comme il était arrivé au
cardinal de Tournon\,; qu'il n'avait pu échapper et se procurer son
retour pour informer le pape de l'état des choses qu'en achetant cette
grâce par la prévarication dont il s'avouait coupable, mais à laquelle
il avait été forcé par la crainte de ce qui était sous ses yeux, et de
donner directement contre ses ordres une bulle de suspension de
l'exécution des précédentes, jusqu'à ce que le saint-siège, plus
amplement informé, expliquât ce qu'il lui plaisait de décider.

Ce récit, en faveur duquel les faits parlaient, embarrassa et fâcha fort
le pape. La désobéissance et la violence ne pouvaient pas être plus
formelles. Il n'y avait point de distinction à alléguer entre fait et
droit, ni d'explication à demander comme sur la condamnation d'un amas
de propositions \emph{in globo} et d'un autre amas de qualifications
indéterminées. Il n'y avait pas lieu non plus de se récrier contre une
condamnation sans avoir été entendus. La condamnation était claire,
nette, tombait sur des points fixes et précis, longuement soutenus par
les jésuites, et juridiquement discutés par eux et avec eux à Rome. Ils
avaient promis de se soumettre et de se conformer au jugement rendu. Ils
n'en avaient rien fait, leur crédit les avait fait écouter de nouveau,
et de nouveau la tolérance dont il s'agissait avait été condamnée. Ils y
étoient encore revenus sous prétexte qu'on n'entendait point à Rome
l'état véritable de la question, qui dépendait de l'intelligence de la
langue, des moeurs, de l'esprit, des idées et des usages du pays. C'est
ce qui fit résoudre l'envoi de Tournon\,; et ce que Tournon y vit et y
apprit, et ce qu'il tenta d'y faire, et qu'il y fit à la fin, empêcha
son retour et son rapport, et celui de la plupart des ministres de sa
légation.

Quelque bruit et quelque prodigieux scandale qui suivit de tels succès,
les jésuites eurent encore le crédit d'éviter le châtiment, soumis,
respectueux et répandant l'or à Rome dans la même mesure qu'ils en
amassaient à la Chine et au Chili, au Paraguay et dans leurs principales
missions, et à proportion de leur puissance et de leur audace à la
Chine. Ce fut donc pour tirer les éclaircissements locaux qu'ils avaient
bien su empêcher le cardinal de Tournon et la plupart des siens de
rapporter en Europe, et finalement pour faire obéir le saint-siège, que
Mezzabarba y fut envoyé. Il ne se put tirer d'un si dangereux pas qu'en
la manière qu'on vient de voir, directement opposée à ses ordres. Mais
que dire à un homme qui prouve un tel péril pour soi et une telle
inutilité d'y exposer sa vie\,? Aussi ne sut-on qu'y répondre\,; mais la
honte de le voir à Rome en témoigner l'impuissance, par le seul fait
d'être revenu sans exécution, et forcé au contraire à suspendre tout ce
qu'il était chargé de faire exécuter, rendit sa présence si pénible à
supporter, qu'il ne lui en coûta pas seulement le chapeau promis pour le
prix de son voyage, mais l'exil loin de Rome, où il vécut obscurément
plusieurs années, et dans lequel il mourut.

Le pape, la très grande partie du sacré collège et de la cour romaine
voulait faire rendre les plus grands honneurs à la mémoire du cardinal
de Tournon\,; et le peuple, soutenu de plusieurs cardinaux et de
beaucoup de gens considérables, le voulaient faire déclarer martyr. Les
jésuites en furent vivement touchés. Ils sentirent tout le poids du
contre-coup qui tomberait sur eux de ce qui se ferait en l'honneur du
cardinal de Tournon. L'audace, poussée au dernier point de
l'effronterie, leur en para l'affront. Ils insistèrent pour obtenir
qu'après Mezzabarba, leur P. Magalhaens fût écouté à son tour.

Peu occupés de défendre les rits chinois, la désobéissance et les
violences des jésuites de la Chine devant la congrégation de la
Propagande, dont ils n'espéraient rien, ils voulurent aller droit au
pape. Magalhaens y défendit les siens comme il put. Il se flattait peu
de leur parer une condamnation nouvelle. Son grand but fut d'étouffer la
mémoire du cardinal de Tournon et de sauver l'affront insigne des
honneurs qu'on lui préparait. Le pape, gouverné par le cardinal Fabroni,
leur créature et leur pensionnaire, qui les craignait à la Chine, où ils
se moquaient de lui en toute sécurité, et qui s'en servaient si
utilement en Europe, crut mettre tout à couvert en condamnant de nouveau
les rites chinois et les jésuites, leurs protecteurs à la Chine, sous la
plus grande peine, s'ils n'obéissaient pas enfin à ces dernières bulles,
et sous les plus grandes menaces de s'en prendre au général et à la
société en Europe, aux dépens de la mémoire du cardinal de Tournon, qui
fut enfin enterré dans l'église de la Propagande sans aucune pompe.
C'était tout ce que les jésuites s'étaient proposé. Contents au dernier
point de voir tomber par là toute information de ce qui s'était passé à
la Chine, à l'égard de la légation et de la personne du légat, après
tout le bruit qui s'en était fait à Rome, ils se tinrent quittes à bon
marché de la nouvelle condamnation du pape, moyennant que cette énorme
affaire demeurât étouffée, que l'étrange succès de la légation de
Mezzabarba restât tout court sans aucune suite, bien assurés qu'après de
telles leçons données à ces deux légats \emph{a latere}, il ne serait
pas facile de trouver personne qui se voulût charger de pareille
commission, non pas même pour la pourpre, qui n'avait fait qu'avancer la
mort du cardinal de Tournon\,; et qu'à l'égard des condamnations
nouvelles, ils en seraient quittes pour des respects, des promesses
d'obéissance et des soumissions à Rome, et n'en continueraient pas moins
à la Chine à s'en moquer et à les mépriser, comme ils avaient fait
jusqu'alors. C'est en effet comme ils se conduisirent fidèlement à Rome
et à la Chine, sans que Rome ait voulu ou su depuis quel remède y
apporter.

Mais ce qui est incroyable est la manière dont le P. Magalhaens s'y prit
pour conduire l'affaire à cette issue. Ce fut de demander hardiment au
pape de retirer tous les brefs, ou bulles et décrets, qui condamnaient
les rits chinois et la conduite des jésuites à cet égard et à l'égard de
ces condamnations. Il fallait être jésuite pour hasarder une demande si
impudente au pape, en personne, en présence du corps du cardinal de
Tournon, et du légat Mezzabarba, et il ne fallait pas moins qu'être
jésuite pour la faire impunément. Le pape fut encore plus effrayé
qu'indigné de cette audace.

Il crut donc faire un grand coup de politique de les condamner de
nouveau pour ne pas reculer devant ce jésuite, mais d'en adoucir le coup
pour sa compagnie, en supprimant tout honneur à la mémoire du cardinal
de Tournon, et se hâtant de le faire enterrer sans bruit dans l'église
de la Propagande, où il était demeuré en dépôt, en attendant que les
honneurs à rendre à sa mémoire et la pompe de ses obsèques eussent été
résolus, qui furent sacrifiés aux jésuites, avec un scandale dont le
pape ne fut pas peu embarrassé.

Le 11 juin le roi alla demeurer à Meudon. Le prétexte fut de nettoyer le
château de Versailles, la raison fut la commodité du cardinal Dubois.
Flatté au dernier point de présider à l'assemblée du clergé, il voulait
jouir quelquefois de cet honneur. Il désirait aussi se trouver
quelquefois aux assemblées de la compagnie des Indes\,; Meudon le
rapprochait de Paris de plus que la moitié du chemin de Versailles, et
lui épargnait du pavé. Ses débauches lui avaient donné des incommodités
habituelles et douloureuses que le mouvement du carrosse irritait, et
dont il se cachait avec grand soin. Le roi fit à Meudon une revue de sa
maison où l'orgueil du premier ministre voulut se satisfaire\,; il lui
en coûta cher. Il monta à cheval pour y jouir mieux de son triomphe, il
y souffrit cruellement, et rendit son mal si violent qu'il ne put
s'empêcher d'y chercher du secours. Il vit des médecins et des
chirurgiens les plus célèbres, dans le plus grand secret, qui en
augurèrent tous fort mal, et par la réitération des visites et quelques
indiscrétions la chose commença à transpirer. Il ne put continuer
d'aller à Paris qu'une fois ou deux au plus avec grande peine, et
uniquement pour cacher son mal qui ne lui donna presque plus de repos.

En quelque état que fût le cardinal Dubois, ses passions ne l'occupaient
pas moins que si son âge et sa santé lui eussent promis encore quarante
années de vie. Les soins de s'enrichir et de se perpétuer la souveraine
et unique puissance le tourmentaient avec la même vivacité. Il poussait
donc l'affaire de La Jonchère à son gré, sous le prétexte de l'ardeur de
M. le Duc à perdre Le Blanc et Belle-Ile\,; et Belle-Ile s'y trouva
embarrassé par les dépositions de La Jonchère et de ses commis arrêtés
avec lui. Conches, et Séchelles maître des requêtes, fort distingué dans
son métier, ami intime de Le Blanc et de Belle-Ile, y furent aussi
compris. Ils furent tous trois obligés à comparaître devant les
commissaires des malversations, puis devant la chambre de l'Arsenal. Ils
y furent interrogés plusieurs fois. Belle-Ile y déclara qu'allant servir
sous le maréchal de Berwick dans le Guipuscoa et dans la Navarre
espagnole, il avait donné ses billets de banque et ses actions à La
Jonchère pour s'en servir, et lui rendre après en divers temps. Rien
n'était moins répréhensible\,: on ne trouva rien de plus mal dans les
deux autres. Cela piqua, mais ne fit qu'encourager la haine à chercher,
à tâcher, à ne se point rebuter, et à les tenir cependant dans des
filets, mais sans pouvoir encore aller plus loin ni les arrêter.

Une autre pratique s'était élevée depuis quelque temps dans les
ténèbres, avec toute l'adresse et toute l'audace possible. La conduite
de M. le duc d'Orléans persuadait aisément qu'il n'y avait rien, quelque
étrange que fût ce qu'on se proposait, qui fût impossible avec la
protection du cardinal Dubois, et rien encore, pour monstrueux qu'il
fût, qu'on n'arrachât du premier ministre à la recommandation de
l'Angleterre. M\textsuperscript{me} de La Vrillière, au bout de tant
d'années de mariage, ne pouvait se consoler ni s'accoutumer à être
M\textsuperscript{me} de La Vrillière. Elle le faisait sentir souvent à
son mari. Il était glorieux autant et plus qu'il osait l'être\,; les
fonctions que je lui avais procurées pendant la régence, qui l'y avaient
rendu nécessaire à tout le monde, l'avaient achevé de gâter\,; lui et sa
femme n'imaginèrent rien moins que de se faire duc et pair\,; et voici
comment ils s'y prirent. La comtesse de Mailly, mère de
M\textsuperscript{me} de La Vrillière, était Saint-Hermine, et de
Saintonge. Elle avait originairement beaucoup de parents calvinistes qui
s'étaient retirés en divers temps dans les États de la maison de
Brunswick, où des alliances de plusieurs d'eux avec les Olbreuse, de
même pays qu'eux ou fort voisins, leur avaient fait espérer, puis
obtenir la protection de la duchesse de Zell, de laquelle il a été parlé
ailleurs. Personne n'ignorait le crédit qu'avait eu la baronne de
Platten sur l'électeur d'Hanovre qui l'avait fait comtesse, et qu'elle
en conservait encore quelques restes, quoique depuis longtemps une autre
maîtresse l'eût supplantée, que l'électeur avait même attirée et élevée
en dignité en Angleterre, depuis que lui-même y eut été prendre
possession de la couronne de la Grande-Bretagne, à la mort de la reine
Anne.

Schaub, ce Suisse dont ce prince s'était si longtemps servi à Vienne, ce
drôle si intrigant, si rusé, si délié, si Anglais, si autrichien, si
ennemi de la France, si confident du ministère de Londres, que nous
avons si souvent rencontré dans ce qui a été donné ici, d'après M. de
Torcy, sur les affaires étrangères, ce Schaub était ici chargé du vrai
secret entre le ministère Anglais et le cardinal Dubois, sur lequel il
avait su usurper tout pouvoir. Aussi était-il fort cultivé dans notre
cour. M. et M\textsuperscript{me} de La Vrillière l'avaient fort attiré
chez eux par cette raison, et Schaub, qui était fort entrant, et avide
d'écumer partout où il pouvait espérer quelque récolte, s'y était rendu
extrêmement familier. Pour s'amuser ou autrement, il s'avisa de tourner
autour de M\textsuperscript{me} de La Vrillière. Il la voyait encore
coquette au dernier point, et n'ignorait pas qu'elle n'avait jamais été
cruelle. La dame s'en aperçut bientôt, elle ne s'en offensa pas, et fit
si bien qu'elle le rendit amoureux tout de bon\,; car elle était encore
jolie. Alors elle le jugea un instrument propre à la servir, et son mari
et elle lui firent confidence de leurs vues et de leur besoin de la
protection du roi d'Angleterre. Schaub, qui avait les siennes, fut
charmé d'une ouverture qui l'y conduisait, et se mit à digérer le
projet. Ils surent que la comtesse de Platten avait une fille belle et
bien faite, d'âge sortable pour leur fils, mais sans aucun bien, comme
toutes les Allemandes, et dès lors ils ne songèrent plus qu'à ce mariage
pour se procurer l'intercession du roi d'Angleterre, laquelle ne lui
coûtant rien, il ne la refuserait pas à son ancienne maîtresse pour
l'établissement de sa fille. Les parents calvinistes de la comtesse de
Mailly, retirés et depuis longtemps établis dans les États de la maison
de Brunswick, se mirent en campagne pour faire la proposition de ce
mariage\,; ils furent écoutés. M\textsuperscript{me} de Platten se
serait bien gardée de prendre une fille de La Vrillière qui aurait
exclus son fils et sa postérité des chapitres protestants pour des
siècles, comme des chapitres catholiques\,; mais sa fille à donner au
fils de La Vrillière n'avait pas le même inconvénient.

L'affaire réglée donna lieu à Schaub de jouer son personnage. Il sonda
le cardinal Dubois sur son attachement pour le roi d'Angleterre et pour
ses ministres principaux. Il en reçut toutes les protestations d'un
homme qui leur devait son chapeau, par conséquent le premier ministère,
auquel, sans le chapeau, il n'aurait pu atteindre, et qui l'avait mis en
état de recevoir une pension de quarante mille livres sterling de
l'Angleterre, qui passait par les mains de Schaub depuis qu'il était en
France, et qui était depuis longtemps au fait des liaisons intimes, ou
plutôt de la dépendance entière de Dubois du ministère Anglais. Quand sa
matière fut bien préparée, il lui parla du mariage, du crédit que la
comtesse de Platten conservait très solide sur le roi d'Angleterre, sur
ses liaisons intimes avec ses principaux ministres allemands et Anglais,
de l'embarras où se trouvait la comtesse de Platten de donner sa fille à
un homme qui, de l'état que ses pères avaient toujours exercé, quelque
honorable et distingué qu'il fût en France, n'oserait penser à sa fille
s'il était Allemand\,; que ce mariage toutefois convenait extrêmement à
M. le duc d'Orléans et à Son Éminence, parce que ce serait un lien de
plus avec le roi d'Angleterre et avec ses ministres, un moyen certain
d'être toujours bien et sûrement informés de leurs intentions, et de les
faire entrer dans celles de Son Altesse Royale et de Son Éminence\,;
qu'il croyait rendre un service essentiel à l'un et à l'autre de ménager
cette affaire\,; mais qu'elle était désormais entre les mains de Son
Éminence pour lever la seule difficulté qui l'arrêtait, en rendant le
fils de La Vrillière capable d'y prétendre, et en comblant d'aise et de
reconnaissance la comtesse de Platten, et avec elle le roi d'Angleterre
et ses ministres les plus confidents, en faisant pour La Vrillière la
seule chose dont il fût susceptible, et que méritaient si fort les
grands services rendus à l'État depuis si longtemps, par tant de grands
ministres ses pères, ou de son même nom.

Dubois, qui, par ce qu'il était né, et par la politique qu'il s'était
faite et qu'il avait inspirée de longue main à son maître, voulait tout
confondre et tout anéantir, prêta une oreille favorable à Schaub, et ne
fut point effarouché de la proposition qu'il lui fit enfin de faire La
Vrillière duc et pair. Il servait l'Angleterre suivant son propre
goût\,; il s'en assurait de plus en plus son énorme pension par une
complaisance qui, bien loin de lui coûter, se trouvait dans l'unisson de
son goût et de sa politique. Il ne laissa pas, pour se mieux faire
valoir, d'en représenter les difficultés à Schaub, mais en lui laissant
la liberté de lui en parler, et l'espérance de pouvoir réussir.

Soit de concert avec le premier ministre, soit de pure hardiesse, tant à
son égard même qu'à celui de M. le duc d'Orléans, Schaub revint à la
charge et dit au cardinal qu'il ne s'était pas trompé lorsqu'il l'avait
assuré que cette affaire serait extrêmement agréable au roi d'Angleterre
et à ses plus confidents ministres, que jusqu'alors il n'avait parlé à
Son Excellence que de lui-même, mais qu'il venait d'être chargé de lui
recommander la chose au nom du roi d'Angleterre qui la désirait avec
passion, et de la part de ses ministres qui lui demandaient cette grâce
comme le gage de leur amitié, et qu'il avait le même ordre du roi
d'Angleterre d'en parler de sa part à M. le duc d'Orléans. Le cardinal
lui accorda toute liberté de le faire, et lui promit d'y préparer M. le
duc d'Orléans et d'agir de son mieux auprès de lui pour lever, s'il
pouvait, les difficultés qui se rencontreraient. Pour le faire court, M.
le duc d'Orléans trouva la proposition extrêmement ridicule\,; mais sans
cesser de la trouver telle, il fut entraîné. La Vrillière, en
conséquence, parla au cardinal Dubois, et de son aveu à M. le duc
d'Orléans. Il en fut assez bien reçu, et si transporté de joie, lui et
sa femme, que le secret transpira.

Le duc de Berwick en fut averti des premiers\,; il en parla à M. le duc
d'Orléans avec toute la force et la dignité possible, et l'embarrassa
étrangement. Il me vint trouver aussitôt après à Meudon, où la cour ne
vint que quelque temps après, et m'apprit cette belle intrigue\,; le
clou qu'il avait taché d'y mettre aussitôt, et m'exhorta à parler, de
mon côté, à M. le duc d'Orléans.

Je ne me fis pas beaucoup prier sur une affaire de cette nature, et
j'allai dès le lendemain à Versailles chez M. le duc d'Orléans. Il
rougit et montra un embarras extrême au premier mot que je lui en dis.
Je vis un homme entraîné dans la fange, qui en sentait toute la
puanteur, et qui n'osait ni s'en montrer barbouillé ni s'en nettoyer,
dans la soumission sous laquelle il commençait secrètement à gémir. Je
lui demandai où il avait vu ou lu faire un duc et pair de robe ou de
plume, et donner la plus haute récompense qui fût en la main de nos
rois, et le comble de ce à quoi pouvait et devait prétendre la plus
ancienne et la plus haute noblesse, à un greffier du roi, dont la
famille en avait toujours exercé la profession depuis qu'elle s'était
fait connaître pour la première fois sous Henri IV, sans avoir jamais
porté les armes, qui est l'unique profession de la noblesse. Cet exorde
me conduisit loin, et mit M. le duc d'Orléans aux abois. Il voulut se
défendre sur la vive intercession du roi d'Angleterre, et sur la
position où il était avec lui. Je lui répondis que je ne pouvais
présumer qu'il espérât me faire recevoir cette raison comme sérieuse\,;
qu'il connaissait très bien Schaub, et que c'était lui-même qui m'avait
appris que c'était un insigne fripon, un audacieux menteur, plein
d'esprit, d'adresse, de souplesses, singulièrement faux et hardi à
controuver tout ce qui lui faisait besoin, et de génie ennemi de la
France\,; qu'étant tel par le portrait que Son Altesse Royale m'en avait
souvent fait, j'étais fort éloigné de penser que Son Altesse Royale crût
sur une si périlleuse parole que le roi d'Angleterre ni ses ministres
s'intéressassent à lui faire faire ce qui était sans aucun exemple, pour
mieux marier la fille d'une maîtresse abandonnée depuis si longtemps, du
crédit de laquelle nous n'avions jamais ouï parler pendant huit ans de
sa régence, et qu'il avait été question sans cesse de manier et de
s'aider du roi d'Angleterre\,; que par conséquent il m'était clair qu'il
était bien persuadé que le roi d'Angleterre ne prenait pas la moindre
part aux imaginations de La Vrillière, ni pas un de ses ministres\,; que
cet intérêt, présenté par Schaub comme véritable et vif, n'était que
l'effet de son adresse et de son amour pour M\textsuperscript{me} de La
Vrillière, saisi par Son Altesse Royale pour prétexte et pour excuse de
ce qu'il voyait énorme et sans exemple, à quoi néanmoins il se laissait
entraîner. J'ajoutai que, quand il serait certain que l'intercession de
l'Angleterre serait vraie et vive, je le suppliais de me dire s'il était
bon d'accoutumer les grandes puissances étrangères à s'ingérer des
grâces et de l'intérieur de la cour\,; s'il ne prévoyait pas quelle
tentation il préparait à la fidélité des ministres du roi et de ses
successeurs par l'exemple de La Vrillière\,; si lui-même oserait
hasarder de demander au roi d'Angleterre, pour un Anglais ou un
Hanovrien, une pareille élévation dans sa cour, et s'il connaissait
aucun exemple semblable de puissance à puissance dans toute l'Europe,
avec toutefois la seule exception d'occasions singulières, qui avaient
quelquefois procuré la Jarretière à des Français, mais des Français qui
n'étaient pas de l'état de La Vrillière, tels, par exemple, que l'amiral
Chabot, le connétable Anne et le maréchal de Montmorency, son fils aîné,
le maréchal de Saint-André, qui, en naissance, en établissements, et par
eux-mêmes étaient de fort grands personnages\,; et dans des temps
postérieurs les ducs de Chevreuse-Lorraine et de La Valette, sans parler
du duc de Lauzun qui l'avait eue dans Paris de la reconnaissance, d'un
roi détrôné\,; et de plus encore, quelle comparaison, surtout en France,
entre la Jarretière et la dignité de duc et pair\,? Je n'oubliai pas
l'abus des grandesses françaises\,; mais je lui fis remarquer leur
nouveauté, leur cause entre des rois, grand-père et petit-fils, ou neveu
et oncle de même maison, et qui encore n'avaient jamais produit de ducs
et pairs de France en Espagne, et l'échange de fort peu de colliers du
Saint-Esprit contre beaucoup de colliers de la Toison d'or.

Ces raisons, qui prévenaient toute réplique, mirent M. le duc d'Orléans
à non plus. Il se promenait la tête basse dans son cabinet, et ne savait
que dire. Le projet était de cacher dans le plus profond secret cet
ouvrage de ténèbres, et que personne n'en pût avoir le vent que par la
déclaration de La Vrillière duc et pair. Berwick et moi le
déconcertions, et M. le duc d'Orléans découvert, se voyait incontinent
exposé à la multitude des représentations, des demandes de la même
grâce, sur un tel exemple, et qui ne se pourvoient refuser, et en grand
nombre, enfin au cri public, qu'il redoutait toujours. Je continuai mes
instances et mes raisonnements sur un si beau canevas, et je le quittai
au bout d'une heure sans savoir ce qui en serait. J'allai de là rendre
au duc de Berwick ce que je venais de faire. Nous conclûmes de revenir
sans cesse à la charge par nous et par d'autres, que lui, qui habitait
Versailles, se chargea de lui lâcher, et de rendre la chose publique
pour exciter le cri public. Ce cri devint si grand et si universel qu'il
arrêta le prince et le cardinal, et qu'il étourdit jusqu'à l'audace de
La Vrillière et de sa femme, et jusqu'à l'impudence de Schaub.

Le public farcit cette ambition de ridicules, et ce ne fut pas ce qui
contint le moins M. le duc d'Orléans. La figure de La Vrillière n'était
pas commune, il était un peu gros et singulièrement petit\,; il était
vif, et ses mouvements tenaient de la marionnette. Quoiqu'on ne se fasse
pas, et que ces défauts n'influent que sur le corps, ils donnent beau
champ au ridicule. M. le prince de Conti allait disant tout haut qu'il
avait envoyé prendre les mesures du petit fauteuil de polichinelle pour
en faire faire un dessus pour La Vrillière quand il serait duc et pair,
et qu'il le viendrait voir. Enfin on en dit de toutes les façons.

Ce vacarme et ces dérisions arrêtèrent pour un temps. M. et
M\textsuperscript{me} de La Vrillière, et Schaub lui-même étaient
déconcertés. Ils avaient bien prévu l'extrême danger d'être découverts
plus tôt que par là déclaration même. Ce malheur arrivé, ils prirent le
parti de laisser ralentir l'orage, de continuer après de presser leur
affaire sourdement, et de la faire déclarer quand on ne s'y attendrait
plus. Ils y furent encore trompés. Tant de gens considérables avaient
intérêt de la traverser, ou de s'en servir pour être élevés au même
honneur, qu'ils furent éclairés de trop près. La Vrillière, peut-être
informé de ce que j'avais dit à M. le duc d'Orléans, qui rendait tout au
cardinal Dubois, de qui Schaub pouvait l'avoir su, me vint trouver à
Meudon pour me demander en grâce de ne le point traverser auprès de M.
le duc d'Orléans\,; et, pour tâcher à me tenir de court, m'assura que
non seulement il en avait parole de lui et du cardinal Dubois, mais que
l'un et l'autre l'avaient donnée au roi d'Angleterre\,; qu'ainsi c'était
une affaire faite, qui n'attendait plus qu'une prompte déclaration\,;
que ce qu'il me demandait était donc moins la crainte de la retarder,
puisque enfin ils s'étaient mis dans la nécessité de la finir, que pour
n'avoir pas la douleur, après toute l'amitié que je lui avais témoignée
toute ma vie, de me trouver opposé à son bonheur.

La vérité est que je me fusse passé bien volontiers de cette visite. Je
ne me voulais pas brouiller avec un homme que j'avais si grandement
obligé en tant de façons, parce que je lui avais des obligations
précédentes, et qui me devait tout ce qu'il était et tout ce qu'il
prétendait devenir\,; je ne voulais ni m'engager, ni mentir, ni donner
prise. Je battis donc la campagne sur l'ancienne amitié\,; je lui avouai
mon éloignement des érections nouvelles, qui toujours en amenaient
d'autres, et augmentaient un nombre déjà trop grand\,; que lui-même ne
l'ignorait pas, avec qui je m'en étais plaint souvent\,; qu'à chose
promise et à lui et au roi d'Angleterre, et qui n'attendait plus que la
déclaration, ce serait peine perdue de travailler contre\,; que, de
plus, il était trop à portée de l'intérieur pour n'avoir pas remarqué
que depuis longtemps je battais de plus en plus en retraite\,; puis
force propos polis, qui ne signifiaient rien. Il fut content ou fit
semblant de l'être, mais j'eus lieu de croire que ce fut le dernier, par
ce qui arriva sept ou huit jours après à l'abbé de Saint-Simon, qui tout
de suite vint me le conter à Meudon.

Il alla chez La Vrillière, à Versailles, lui parler d'une affaire. Après
y avoir répondu honnêtement\,: «\,Voyez-vous, lui dit-{[}il{]} ce tiroir
de mon bureau\,? il y a dedans la liste de tous ceux qui se sont opposés
à mon affaire, et de tous ces beaux messieurs qui en ont tenu de si
jolis discours. Elle se fera malgré eux et leurs dents, et sans que je
m'en remue. Ce n'est plus mon affaire, c'est celle du roi d'Angleterre,
qui l'a entreprise, qui en a la parole positive, qui prétend se la faire
tenir\,; et nous verrons si on aimera mieux rompre avec lui et avoir la
guerre. Si cela arrive, j'en serai fâché, mais je m'en lave les mains.
Il faudra s'en prendre à ces messieurs les opposants et autres beaux
discoureurs, desquels tous j'ai la liste que je n'oublierai jamais, et
qui, je vous le promets, me le payeront tôt ou tard plus cher qu'au
marché.\,» La menace était bien indiscrète, et le \emph{plus cher qu'au
marché} bien bourgeois\,; mais, pour en suivre le style, c'est que le
hareng sent toujours la caque. L'abbé de Saint-Simon sourit, n'osant
rire tout à fait, et lui applaudit sur ce qu'il fallait éviter la guerre
avec l'Angleterre pour si peu de chose\,; qu'il ne croyait pas qu'il pût
y avoir de choix là-dessus, et se moqua doucement de lui, avec toutes
les politesses qui le laissèrent fort content. L'abbé de Saint-Simon ne
fut pas le seul dépositaire de cette confidence.

La Vrillière crut faire taire le monde en persuadant que son affaire
était sûre, et qu'il n'y craignait plus d'oppositions. Il eut la folie
de débiter la guerre comme inévitable avec l'Angleterre si on ne lui
tenait pas la parole qu'on avait donnée à cette couronne sur ce qui le
regardait, et de s'excuser de se trouver la cause innocente de la guerre
si elle s'embarquait à son occasion sur une affaire dont il ne se mêlait
plus, parce qu'elle n'était plus la sienne depuis qu'elle était devenue
celle du roi d'Angleterre. Ces propos, qui sentaient par trop les
petites-maisons, remirent dans les conversations de tout le monde son
oncle paternel et son frère aîné, enfermés depuis longtemps, et lui
donnèrent un grand ridicule. Le déchaînement public accrocha si bien son
affaire qu'elle gagna le temps que la cour vint à Meudon, que la santé
du cardinal le rendit presque invisible, même à Schaub, suspendit toute
affaire. Cet état du cardinal aboutit promptement à la mort, et M. le
duc d'Orléans délivré d'avoir à compter avec lui, aima mieux compter
avec le monde. Schaub et La Vrillière demeurèrent éconduits.

Le marquis de Bedmar, dont j'ai souvent parlé pendant mon ambassade
d'Espagne, mourut à Madrid, à soixante et onze ans, laissant de soi une
estime et un regret général. Il avait servi toute sa vie en Flandre, où
montant par tous les degrés, il y était devenu gouverneur général des
Pays-Bas espagnols par \emph{interim}, en l'absence de l'électeur de
Bavière, et gouverneur de Bruxelles, enfin général des armées des deux
couronnes, en pleine égalité avec nos maréchaux de France généraux des
armées de Flandre. Il s'y conduisit si bien qu'il en acquit l'affection
du roi, qui lui donna l'ordre du Saint-Esprit, lui procura la grandesse,
puis la vice-royauté de Sicile. De retour en Espagne, il y fut ministre
d'État et chef du conseil des ordres et du conseil de guerre, avec une
grande considération. J'en ai donné ailleurs la maison, la famille, et
le caractère. J'ai admiré cent fois en Espagne comment cet homme, si
fait pour le grand monde, qui en avait un si long usage, et qui pendant
tant d'années avait vécu si publiquement et si splendidement, avait pu,
de retour en Espagne, en reprendre la vie commune des seigneurs
espagnols, manger seul son \emph{puchero}\footnote{Pot au feu.}, et
achever sa vie dans une solitude presque continuelle, interrompue
seulement par quelques visites plus de bienséance que de société, et par
quelques fonctions.

On fut surpris en même temps d'apprendre que le maréchal de Villars
était fait grand d'Espagne, sans l'avoir jamais servie que dans
l'affaire de Cellamare et du duc du Maine, et sans qu'on ait jamais su
comment il avait obtenu cette grâce, que M. le duc d'Orléans lui permit
d'accepter, parce qu'il permettait tout. Le maréchal avait essayé
d'obtenir de la cour de Vienne, où il était fort connu pour y avoir été
longtemps en deux fois envoyé extraordinaire du feu roi, un titre de
prince de l'Empire\,; mais il n'y put parvenir. Le maréchal voulait
toutes les dignités, tous les honneurs, toutes les richesses, et il en
fut comblé sans en être rassasié ni ennobli.

\end{document}
